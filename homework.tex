\documentclass[UTF8,a4paper,10pt]{article}
\usepackage{ctex}
\usepackage[top=1in, bottom=1in, left=1in, right=1in]{geometry}
\usepackage{amsmath}
\usepackage{amsfonts}
\usepackage{amsthm}
\usepackage{amssymb}
\usepackage{mathtools}
\usepackage{bm}
\usepackage{extarrows}
\usepackage{enumerate}
\newcommand{\Mod}[1]{\ (\mathrm{mod}\ #1)}
\newcommand{\ud}[1]{\mathrm{d}#1}
\newcommand{\e}{\mathrm e}
\setlength\abovedisplayskip{2pt}
\setlength\belowdisplayskip{2pt}
\pagestyle{empty}
\begin{document}
\section*{Chapter 1}
PP. 28 {\bf Exercise} (4), (6), (8).
\begin{enumerate}
	\kaishu
	\item[(4)] We have distributed two hundred balls into one hundred boxes with
	      the restrictions that no box got more than one hundred balls, and each
	      box got at least one. Prove that it is possible to find some boxes that
	      together contain exactly one hundred balls.

	      题意: 把~$200$~个球分给~$100$~个箱子, 每个箱子里至少有~$1$~个球, 至多有~$100$
	      ~个球. 证明存在若干个箱子中的球数之和恰好是~$100$.
	      \begin{proof}
		      如果每个箱子中装有相同个数的球, 那么每个箱子中有~$2$~个球, 任意~$50$~个箱子中球数之和都是~$100$.
		      如果每个箱子中装有球数不全相同, 适当地排列这~$100$~个箱子, 使得前两个箱子中球数不同. 用~$p_i$~记第~$i$~个箱子中所装球的数量, $p_i\in\mathbf{Z}_{+}, 1\leqslant p_i\leqslant 100, i=1,2,\cdots,100$. 考虑~$S_n=\sum_{k=1}^{n} p_k$:
		      \begin{enumerate}[(1)]
			      \item 若存在~$S_i, S_j, 1\leqslant i,j\leqslant 100$~使得~$S_i\equiv S_j\Mod{100}$, 不妨设~$i<j$, 则~$S_j-S_i\equiv 0\Mod{100}$. 又显然~$S_j-S_i=p_{i+1}+p_{i+2}+\cdots+p_{j}\leqslant p_1+\cdots+p_{100}=200$, 故~$a_{i+1}+a_{i+2}+\cdots+a_{j}=100$;
			      \item 若~$S_n$~两两模~$100$~不同地方法余, 考虑~$p_2, S_1, S_2, \cdots, S_{100}$~这~$101$~个正整数, 由\emph{鸽笼原理}可知必有两个数同余. 由~$S_1, S_2,\cdots, S_{100}$~两两不同余, 故存在~$1\leqslant k\leqslant n$~使得~$S_k\equiv a_2\Mod{100}$. 同~$(1)$~有~$S_k-a_2=a_1+a_3+a_4+\cdots+a_k=100$.
		      \end{enumerate}
	      \end{proof}
	\item[(6)]
	      \begin{enumerate}[(a)]
		      \item The set $M$ consists of nine positive integers, none of which has a prime divisor larger than six. Prove that $M$ has two elements whose product is the square of an integer.
		      \item (Some knowledge of linear algebra and abstract algebra required.) The set $A$ consists of $n+1$ positive integers, none of which has a prime divisor that is larger than the $n$th smallest prime number. Prove that there exists a non-empty subset $B\subseteq A$ so that the product of the elements of $B$ is a perfect square.
	      \end{enumerate}

	      题意: $M$~含有~$9$~个正整数, 其素因子不大于~$6$, 证明存在~$M$~中的两个数之积是某个整数的平方. 进一步, 含有~$n+1$~个正整数的集合~$A$~的素因子不大于第~$n$~个素数, 存在~$A$~的子集~$B$~中的元素之积是完全平方数.
	      \begin{proof}
		      \begin{enumerate}[(a)]
			      \item 对~$M$~中的数作素分解, 可以写为~$2^i3^j5^k, i,j,k\in\mathbf{Z}_{+}\cup\{0\}$. 按照~$i,j,k$~的奇偶性可以将这~$9$~个数分为~$8$~类. 由\emph{鸽笼原理}, 存在两个数在同一类中. 由于这两个数的素分解中, $2,3,5$~对应的次数的奇偶性相同, 故对于这两个数之积, 其素分解的相应因子次数均为偶数次, 故是一个整数的平方.
			      \item 用~$p_i$~表示第~$i$~小的素数, 即~$p_1=2, p2=3, \cdots$, 对~$A$~中的数作素分解, $A\ni a=p_1^{a_1}\cdots p_n^{a_n}$, 并用有序数组~$(a_1,a_2,\cdots,a_n)$~表示~$a$. 对于每个~$a_i$, 对其做模~$2$~同余, 可以得到一个只取~$0,1$~的~$n$~元有序数对, 以下仍用~$(a_1,a_2,\cdots,a_n)$~表示. 并且~$a$~是完全平方数当且仅当~$a=(0,0,\cdots,0)\coloneqq\mathbf{0}$.

			            基于以上讨论, 集合~$A$~中的元素可以看作是线性空间~$F_2^n$~中的向量, 其中~$F_2$~是只含有~$0,1$~的集合. 由于~$n$~维线性空间中的~$n+1$~个向量一定线性相关. 因此存在一个线性相关的向量组, 其对应的数集记为~$B'=\{b_1,b_2,\cdots,b_m\}\subseteq A, 0<m\leqslant n+1$,  使得
			            \[
				            k_1b_1+k_2b_2+\cdots+k_mb_m=\mathbf{0},\  k_1, k_2,\cdots, k_m\;\text{不全为~$0$.}
			            \]
			            注意~$k_i=0$~或~$1, i=1,2,\cdots,m$, 记满足~$k_i=1$~的~$b_i$~全体为~$B\subseteq B'\subseteq A$, 则显然~$\sum\limits_{b\in B} b=\mathbf{0}$, 即~$B$~中元素之积是完全平方数.
		      \end{enumerate}
	      \end{proof}
	\item[(8)] The sum of one hundred given real numbers is zero. Prove that at least 99 of the pairwise sums of these hundred numbers are nonnegative. Is this result the best possible one?

	      题意: 给定一百个实数, 其和为零. 证明其中至少有~$99$~对非负和.
	      \begin{proof}
		      设给定的~$100$~个实数为~$a_1<a_2<\cdots<a_{99}<a_{100}$.
		      \begin{enumerate}[(a)]
			      \item 如果~$a_{50}+a_{99}\geqslant 0$, 那么
			            \[
				            0\leqslant a_{50}+a_{99}\leqslant a_{51}+a_{99} \leqslant\cdots\leqslant a_{98}+a_{99}\leqslant a_{100}+a_{99},
			            \]
			            是~$50$~对非负和; 对于任意的~$i\geqslant 50$, 有
			            \[
				            0\leqslant a_i+a_{99}\leqslant a_i+a_{100},
			            \]
			            这给出了~$a_{50}+a_{100}, a_{51}+a_{100}, \cdots, a_{98}+a_{100}$~这~$48$~对非负和.
			      \item 如果~$a_{50}+a_{99}<0$, 那么由于
			            \[
				            a_1+a_2+\cdots+a_{49}+a_{51}+\cdots+a_{98}+a_{100}>0,
			            \]
			            \[
				            0>a_{50}+a_{99}\geqslant a_{49}+a_{98}\geqslant a_{48}+a_{87}\geqslant\cdots\geqslant a_{2}+a_{51},
			            \]
			            所以~$a_1+a_{100}>0$, 故
			            \[
				            0<a_1+a_{100}\leqslant a_2+a_{100}\leqslant\cdots\leqslant a_{99}+a_{100};
			            \]
			            是~$99$~对非负和.
		      \end{enumerate}
	      \end{proof}
\end{enumerate}

\section*{Chapter 2}
PP. 28 {\bf Exercise} (1), (3).
\kaishu
\begin{enumerate}
	\item[(1)] Let $p(k)$ be a polynomial of degree $d$. Prove that $q(n) =\sum_{k=1}^n p(k)$ is a polynomial of degree $d + 1$. Prove that this polynomial $q$ satisfies $q(0) = 0$.
	      \begin{proof}
		      对于~$d$~作\emph{第二数学归纳法}.

		      当~$d=0$~时, $p(n)$~是常数~$c$, 有~$q(n)=\sum_{k=1}^n c=nc$, 是一个~$1$~次多项式.

		      假设命题对于次数小于~$d$~的多项式均成立. 为了证明对于次数为~$d$~时的多项式成立, 只需要证明~$p(n)=n^d$~成立.
		      \begin{enumerate}[(i)]
			      \item \emph{断言}*: 存在~$d+1$~次多项式~$z_d(n)$~使得~$z_d(n+1)-z_d(n)=n^d, n\in\mathbf{N}_{+}$, $z(0)=0$.

			            使用\emph{第二数学归纳法}. $d=1$~时, 取~$z_1(n)=n$, 显然有~$z_1(n+1)-z_1(n)=1$. 假设对于所有~$k<d, k\in\mathbf{N}_{+}$~均有~断言*~成立. 对于~$n^d$, 显然有~$(n+1)^{d+1}-n^{d+1}=(n+1)^d+(n+1)^{d-1}n+\cdots+n^d$~是一个~$d$~次多项式, 对于等号右边的~$d$~次多项式中次数小于~$d$~的项利用归纳假设化简, 等式两边同时减去第~$i$~个等式的若干倍后再首~$1$~化, 就得到~$n^d$, 和相应的~$z_d(n)$.
			      \item 由于~$d+1$~次多项式~$z_d(n)$~存在, 故
			            \[
				            \begin{split}
					            1^d+2^d+\cdots+n^d=&(z_d(1)-z_d(0))+(z_d(2)-z_d(1))+\cdots+(z_d(n)-z_d(n-1))\\
					            =&z_d(n)-z_d(0)\\
					            =&z_d(n)
				            \end{split}
			            \]
			            即命题对于~$n^d$~成立.
			      \item 对于任意的~$p(n)=a_dn^d+a_{d-1}n^{d-1}+\cdots+a_1n+a_0\ (a_d\neq 0)$, 由归纳假设和~(ii),
			            \[
				            q(n)=\sum_{k=1}^n p(k)=a_d\sum_{k=1}^n k^d + a_{d-1}\sum_{k=1}^n k^{d-1}+\cdots+a_0\sum_{k=1}^n 1
			            \]
			            是一个~$d+1$~次多项式, 且~$q(0)=0$. 即命题对于~$d$~次多项式也成立.
		      \end{enumerate}
		      由第二数学归纳法即知命题成立.
	      \end{proof}
	\item[(3)] At a tennis tournament, there were $2^n$ participants, and any two of them played against each other exactly one time. Prove that we can find $n+1$ players that can form a line in which everybody has defeated all the players who are behind him in the line.

	      题意: $2^n$~个人参加的网球比赛, 任意两个人之间赛一场, 证明存在~$n+1$~个人可以排成一列, 其中前面的人胜后面的人.
	      \begin{proof}
		      \emph{数学归纳法}. $n=1$~时, 显然成立. 假设对于~$n$, 命题都成立. 证明~$n+1$~时的情形. 一个~$2^{n+1}$~场比赛的胜者~$A$~必须至少赢得~$2^n$~场. 在被~$A$~击败的~$2^n$~个人中, 由归纳假设可以找出~$n+1$~个人排成一列使得前面的人战胜后面的人, 再把~$A$~排在这一列队伍的最前面, 就得到了符合题意得~$n+2$~个人的队伍.
	      \end{proof}
\end{enumerate}

\section*{Chapter 3}
PP. 50. {\bf Exercise} (6), (7), (17).
\begin{enumerate}
	\item[(6)] How many five-digit positive integers are there with middle digit 6 that are divisible by three?

	      题意: $5$~位正整数, 中间一位是~$6$, 可以被~$3$~整除.
	      \begin{proof}
		      $5$~位正整数, 中间一位是~$6$, 可以被~$3$~整除当且仅当去掉中间一位后的~$4$~位正整数可以被~$3$~整除. 而在~$1000$~到~$9999$~中共有~$3000$~个数可以被~$3$~整除.
	      \end{proof}

	\item[(7)] How many five-digit positive integers are there that contain the digit 9 and are divisible by three?

	      题意: $5$~位正整数, 含有~$9$, 可以被~$3$~整除.
	      \begin{proof}
		      类似于上一题, $10000$~到~$99999$~这~$90000$~个数中共有~$30000$~个可以被~$3$~整除, 再去掉其中不含~$9$~就可以了. 从高到低, 第一位有~$1$~到~$8$~$8$~种取法, 第二、三、四位有~$0$~到~$8$~$9$~种取法, 而第五位无论前四位如何取, 总在~$\{0,3,6\},\{1,4,7\},\{2,5,8\}$~三者中任居其一. 故共有~$8\cdot 9^3\cdot 3=17496$~种取法, 因此共有~$30000-17496=12504$~种取法.
	      \end{proof}

	\item[(17)] Let $k\geqslant 1$, and let $b_1, b_2, \cdots , b_k$ be positive integers with sum less than
	      $n$, where $n$ is a positive integer. Prove that then
	      $b_1!b_2!\cdots b_k! < n!$
	      holds. Can you make that statement stronger?
	      \begin{proof}
		      记~$b_{k+1}=n-\sum_{i=1}^k$~, 则由\emph{定理 3.5}知道, $n+1$~种物体, 第~$i$~种有~$b_i$~个, 的线性序有
		      \[T=\dfrac{n!}{b_1!b_2!\cdots b_kb_{k+1}}\]
		      种排法.
	      \end{proof}
\end{enumerate}

\section*{Chapter 4}
PP. 92. {\bf Exercise} (5), (6), (16).
\begin{enumerate}
	\item[(5)] Prove that for integers $0 \leqslant k \leqslant n-1$,
	      \[
		      \sum_{j=0}^k \binom{n}{j}=\sum_{j=0}^{k} \binom{n-1-j}{k-j} 2^j.
	      \]
	      \begin{proof}
		      等号左边表示\emph{至多含有~$k$~个~$1$~的长度为~$n$~的~$0-1$~序列的个数.}

		      注意长为~$n$~的~$0-1$~序列至多含有~$k$~个~$1$, 则至少含有~$n-k$~个~$0$.

		      接下来按照第~$n-k$~个~$0$~的位置不同来计算至多含有~$k$~个~$1$~的长度为~$n$~的~$0-1$~序列的个数. 如果这个~$0$~出现在第~$(n-j)$~位, $0\leqslant j\leqslant k$, 那么这个~$0$~的左侧就有~$n-k-1$~个~$0$, 有~$\displaystyle\binom{n-j-1}{n-k-1}=\binom{n-1-j}{k-j}$~种取法; 这个~$0$~的右边的~$j$~位上可以有任意多的~$0$, 所以有~$2^j$~种取法. 因此等号右边也表示\emph{至多含有~$k$~个~$1$~的长度为~$n$~的~$0-1$~序列的个数}. 于是等式成立.
	      \end{proof}

	\item[(6)] A heap consists of $n$ stones. We split the heap into two smaller heaps, neither of which are empty. Denote $p_1$ the product of the number of stones in each of these two heaps. Now take any of the two small heaps, and do likewise. Let $p_2$ be the product of the number of stones in each of the two smaller heaps just obtained. Continue this procedure until each heap consists of one stone only. This will clearly take $n-1$ steps, where a step is the splitting of one heap. For what sequence of splits will the sum $p_1 + p_2 + \cdots + p_{n-1}$ be maximal? When is that sum minimal?

	      题意: 分一堆有~$n$~个石头的堆, 每次选择一个石头数目大于~$1$~的堆分为不为零的两份, 用~$p_i$~表示分成的两份的石头数目的乘积. 如何分使得~$p_1+p_2+\cdots+p_{n-1}$~最大? 如何分使得其最小?
	      \begin{proof}
		      事实上, $p_1+p_2+\cdots+p_{n-1}$~是一个常值~$\displaystyle\binom{n}{2}$.

		      \emph{数学归纳法}. $n=2$~时显然为~$p_1=1=\displaystyle\binom{2}{2}$. 假设对于数量小于~$n$~的石堆, 命题成立, 则对于一堆~$n$~个石头的石堆, 第一步将其分为两部分, 分别为~$k$~个和~$n-k$~个, 由归纳假设
		      \[
			      p_1+p_2+\cdots+p_{n-1}=k(n-k)+\binom{k}{2}+\binom{n-k}{2}=\binom{n}{2}.
		      \]
		      故命题恒成立.
	      \end{proof}

	\item[(16)] Let $p$ be a prime number, and let $x > 1$ be any positive integer. Consider a wheel with p spokes shown in Figure 4.2.
	      \begin{enumerate}[(a)]
		      \item We have paints of $x$ different colors. How many ways are there to color the spokes if we want to use at least two colors?
		      \item How many ways are there to do the same if we do not consider two paint jobs different if one can be obtained from the other by rotation?
		      \item What theorem of number theory does this prove?
	      \end{enumerate}

	      题意: 用~$x$~种颜色给~$p$~个辐条上色, 关于上色方法的问题.
	      \begin{proof}
		      \begin{enumerate}[(a)]
			      \item 每个车轴都有~$x$~种上色方法, 一共有~$x^p$~种, 这其中包含了~$x$~种全部上同一种颜色的方法, 因此一共有~$x^p-x$~种方法上色.
			      \item 车轮转~$p$~次之后就回到一开始的图像, 因此同一种图像一共出现了~$p$~次, 所以考虑到旋转, 一共有~$\dfrac{x^p-x}{p}$~种上色方法.
			      \item 因为上色方法一定是整数, 因此这说明~$p\mid x^p-x$.
		      \end{enumerate}
	      \end{proof}
\end{enumerate}

\section*{Chapter 5}
PP. 105. {\bf Exercise} (1), (4), (5)
\begin{enumerate}
	\item[(1)] Find a formula for $S(n, 3)$.

	      题意: $S(n,3)$~是把~$[n]$~分为~$3$~个非空子集的方法数.
	      \begin{proof}
		      假设~$n\geqslant 3$.

		      首先确定满射~$f\colon [n]\to[3]$~的个数. $f\colon [n]\to[3]$~一共有~$3^n$~个, 其中像集中元素个数为~$1$~的映射有~$3$~个, 像集中元素个数为~$2$~的映射有~$3\cdot 2\cdot(2^{n-1}-1)$~个, 于是满射一共有~$3^n- 3\cdot 2\cdot(2^{n-1}-1)-3$~个.

		      由\emph{推论~5.9}可知

		      \[
			      S(n,3)=\dfrac{3^{n-1}+1}{2}-2^{n-1}.
		      \]
	      \end{proof}
	\item[(4)]
	      \begin{enumerate}[(a)]
		      \item Let $h(n)$ be the number of ways to place any number (including
		            zero) non-attacking rooks on the Ferrers shape of the “staircase”
		            partition $(n-1, n-2, \cdots , 1)$. Prove that $h(n) = B(n)$.
		      \item In how many ways can we place $k$ non-attacking rooks on this
		            Ferrers shape?
	      \end{enumerate}
	      \begin{proof}

	      \end{proof}

	\item[(5)] Let $m$ and $n$ be positive integers so that $m \geqslant n$. Prove that the
	      Stirling numbers of the second kind satisfy the recurrence relation
	      \[ S(m, n) = \sum_{i=1}^{m} S(m-i, n-1)n^{i-1} .\]

	      \begin{proof}
		      用~$\pi$~表示把~$[m]$~分割成~$n$~块的分割.

		      等式左边是\emph{此类分割的数目}.
		      用~$m-1$~表示把~$\pi$~的一部分~$[i]$~分为~$n-1$~块的分割方法, 记为~$\pi_i$, 则~$\pi_i$~有~$S(m-i,n-1)$~种可能的方法. 从~$m-i$~的定义可知~$~m-i+i$~是最后一块, 因此~$m-i+2,m-i+3,\cdots,m$~可以在任意一块中, 有~$n^{i-1}$~种可能的选择方法, 于是右边也是\emph{此类分割的数目}.
	      \end{proof}
\end{enumerate}

\section*{Chapter 8}
PP. 174. {\bf Exercise} (1), (8), (9).
\begin{enumerate}
	\item[(1)] Find an explicit formula for $a_k$ if $a_0 = 0$ and $a_{k+1} = a_k +2^k$ for $k\geqslant 0$.
	      \begin{proof}
		      记~$a_k$~的生成函数为~$A(x)=\sum_{k\geqslant 0} a_kx^k$. 在递归式两边同时乘以~$x^{k+1}$~并求和, 有
		      \[
			      \sum_{k\geqslant 0} a_{k+1}x^{k+1}=\sum_{k\geqslant 0} a_kx^k+x\sum_{k\geqslant 0}(2x)^k.
		      \]
		      故~$A(x)=xA(x)+\dfrac{x}{1-2x}$, 即~$A(x)=\dfrac{x}{(1-x)(1-2x)}=x(1+x+x^2+\cdots)(1+2x+4x^2+\cdots)$, 从而~$a_k=\sum_{i=0}^{k-1} 2^i=2^k-1$.
	      \end{proof}
	\item[(8)] Find a simple, closed form for the generating function of the sequence
	      defined by $a_n = n^2$.
	      \begin{proof}
		      由于~$\dfrac{1}{(1-x)^3}=\sum_{n\geqslant 0}\displaystyle\binom{n+2}{2}x^n$. 即
		      \[
			      \dfrac{x^2}{(1-x)^3}=\sum_{k\geqslant 0}\binom{n+2}{2}x^{n+2}=\sum_{k\geqslant 2}\binom{n}{2}x^n.
		      \]
		      由于~$\dfrac{1}{(1-x)^2}=\sum_{n\geqslant 1}nx^{n-1}$. 即~$\dfrac{x}{(1-x)^2}=\sum_{n\geqslant 1}nx^n$. 再注意~$n^2=2\displaystyle\binom{n}{2}+n$, 于是
		      \[
			      \sum_{n\geqslant 0}n^2x^n=2\sum_{n\geqslant 2}\binom{n}{2}x^n+\sum_{n\geqslant 1} nx^n=\dfrac{x(x+1)}{(1-x)^3}.
		      \]
	      \end{proof}
	\item[(9)] Let $f(n)$ be the number of subsets of $[n]$ in which the distance of any
	      two elements is at least three. Find the generating function of $f(n)$.
	      \begin{proof}
		      按照~$n$~是否属于子集分类: 若~$n$~在子集中, 那么~$n-1,n-2$~均不在子集中, 子集的选法和~$f(n-3)$~相同; 若~$n$~不在子集中, 则子集的选法和~$f(n-1)$~相同.

		      综上~$f(n)=f(n-1)+f(n-3), n\geqslant 3$, 又显然有~$f(0)=1, f(1)=2, f(2)=3$, 则同乘以~$x^n$~后递加, 容易得到生成函数~$F(x)=\dfrac{1+x+x^2}{1-x-x^3}$.
	      \end{proof}
\end{enumerate}
\end{document}