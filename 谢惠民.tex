% Latex utf-8
\documentclass[UTF8,a4paper,10pt,twoside]{book}
\usepackage{CTEX}
\usepackage[top=1in, bottom=1in, left=1in, right=1in]{geometry}
\usepackage{tablists}
\usepackage{amsmath}
\usepackage{amsfonts}
\usepackage{amsthm}
\usepackage{amssymb}
\usepackage{bm}
\usepackage{extarrows}
\usepackage{enumerate}
\usepackage{titlesec}
\usepackage{graphicx}
\newcommand{\ud}[1]{\mathrm{d}#1}
\newcommand{\e}{\mathrm e}
\DeclareMathOperator{\arccot}{arccot}
\DeclareMathOperator{\arcsec}{arcsec}
\DeclareMathOperator{\arccsc}{arccsc}
\DeclareMathOperator{\sech}{sech}
\DeclareMathOperator{\csch}{csch}
\DeclareMathOperator{\arsinh}{arsinh}
\DeclareMathOperator{\arcosh}{arcosh}
\pagestyle{empty}
\begin{document}
\setlength\abovedisplayskip{3pt}
\setlength\belowdisplayskip{3pt}
\begin{titlepage}
	\begin{center}
		\vfill
		% Upper part of the page
		{\Huge \bfseries 数学分析题习课讲义}\\[1.5cm]
		% Title
		{\LARGE \bfseries 参考答案}\\[0.4cm]
		\vfill
	\end{center}

\end{titlepage}
\renewcommand{\proofname}{\bf 证明}

\setcounter{chapter}{1}
\chapter{数列极限}
\section{数列极限的基本概念}
\subsection{思考题 pp.13.}
\kaishu
\begin{enumerate}
	\item 数列收敛有很多等价定义. 例如:
	      \begin{enumerate}[(1)]
		      \item 数列~$\{a_n\}$~收敛于~$a$~$\Longleftrightarrow$~$\forall\varepsilon>0,\exists N\in\mathbf{N}_{+},\forall n\geqslant N$, 成立~$|a_n-a|<\varepsilon$;
		      \item 数列~$\{a_n\}$~收敛于~$a$~$\Longleftrightarrow$~$\forall m\in\mathbf{N}_{+},\exists N\in\mathbf{N}_{+},\forall n>N$, 成立~$|a_n-a|<1/m$;\footnote{有些像级数的~Weierstrass-M~判别法, 事实上也可以用~Cauchy~收敛准则给出一个和~Weierstrass-M~判别法类似的证明. 本条是所有二分法/三分法证明的基础.}
		      \item 数列~$\{a_n\}$~收敛于~$a$~$\Longleftrightarrow$~$\forall\varepsilon>0,\exists N\in\mathbf{N}_{+},\forall n>N$, 成立~$|a_n-a|<K\varepsilon$. 其中~$K$~是一个与~$\varepsilon$~和~$n$~无关的正常数.
	      \end{enumerate}
	      试证明以上定义与数列收敛等价.

	      \begin{proof}
		      \begin{enumerate}[(1)]
			      \item $\Rightarrow$ 取~$N=N_0+1$. $\Leftarrow$ 显然.
			      \item $\Rightarrow$ 取~$\varepsilon=1/m, m\in\mathbf{N}_{+}$. $\Leftarrow$ 由于~$\lim\limits_{m\to\infty} 1/m=0$, 故存在~$M\in\mathbf{N}_{+}$, 当~$m>M$~时, $1/m<\varepsilon$. 选定~$m$, 使用定义, 存在$N_0\in\mathbf{N}_{+}$, $\forall n>N$, 有~$|a_n-a|<1/m<\varepsilon$.
			      \item $\Rightarrow$ 取~$K=1$. $\Leftarrow$ 取~$\varepsilon'=\varepsilon/K$, 则~$\exists N\in\mathbf{N}_{+}, \forall n>N, |a_n-a|<K\varepsilon'=\varepsilon$.\qedhere
		      \end{enumerate}
	      \end{proof}

	\item 问: 在数列收敛的定义中, $N$~是否是~$\varepsilon$~的函数?
	      \begin{proof}[\bf 答]
		      否. 对于任意的~$\varepsilon$, 存在一个~$N_0\in\mathbf{N}_{+}$, 使得当~$n>N_0$~时都有~$|a_n-a|<\varepsilon$, 而~$\forall N>N_0$~都可以是符合定义的~$N$, 即每一个~$\varepsilon$~都可以对应无穷多个~$N$, 故不是.\qedhere
	      \end{proof}
	\item 判断: 若~$\{a_n\}$~收敛, 则有~$\lim\limits_{n\to\infty} (a_{n+1}-a_n)=0$~和~$\lim\limits_{n\to\infty} a_{n+1}/a_n=1$.
	      \begin{proof}[\bf 答]
		      $\lim\limits_{n\to\infty} (a_{n+1}-a_n)=0$. 对于任意给定的~$\varepsilon>0$, 存在~$N>0$, 当~$n>N$时有~$|a_n-a|<\varepsilon/2$, 从而~$|a_{n+1}-a|<\varepsilon/2$, 于是对于~$n>N$,
		      \[
			      |a_{n+1}-a_n|\leqslant|a_{n+1}-a|+|a_n-a|<\varepsilon/2+\varepsilon/2=\varepsilon.
		      \]

		      $\lim\limits_{n\to\infty} a_{n+1}/a_n=1$. 举一反例~$\{(-1)^n 1/n\}$, 显然~$\lim\limits_{n\to\infty} (-1)^n1/n=0$, 但
		      \[
			      \lim\limits_{n\to\infty}\dfrac{(-1)^{n+1} 1/(n+1)}{(-1)^n 1/n}=\lim\limits_{n\to\infty} -1\cdot\dfrac{n}{n+1}=-1.\qedhere
		      \]
	      \end{proof}
	\item 设收敛数列~$\{a_n\}$~的每一项都是整数, 问: 该数列有什么特殊性质?
	      \begin{proof}[\bf 答]
		      从某一项开始后每一项均相同. 取~$\varepsilon=1/2$, 则存在~$N\in\mathbf{N}_{+}$, 使对~$n>N$~有~$|a_{n+1}-a_n|<1/2$, 注意到~$a_n\in\mathbf{Z}, n\in\mathbf{N}_{+}$, 知~$a_{n+1}=a_n, \forall n>N$.\qedhere
	      \end{proof}
	\item 问: 收敛数列是否一定是单调数列? 无穷小量是否一定是单调数列?
	      \begin{proof}[\bf 答]
		      均不一定. 如分别取~$\{a+(-1)^n 1/m\}$(收敛但不单调)~和~$\{(-1)^n 1/n\}$(无穷小量但不单调). \qedhere
	      \end{proof}
	\item \footnote{原本的6题中, 一个很小很小的量显然不是一个无穷小量, 注意无穷小量是一个趋于零的极限过程即可.}问: 正无穷大量数列是否一定单调增加? 无界数列是否一定为无穷大量?
	      \begin{proof}[\bf 答]
		      均不一定. 如分别取~$\{n+2\sin{n}\}$(正无穷大量但不单调)~和~$\{n\cdot\sin{n}\}$(无界但非无穷大). \qedhere
	      \end{proof}
	\item 问: 如果数列~$\{a_n\}$~收敛于~$a$, 那么绝对值~$|a_n-a|$~是否随着~$n$~的增加而单调减少趋于~$0$?
	      \begin{proof}[\bf 答]
		      不一定. 如取~$\{a_n\}$~为形如
		      \[
			      1,1/2,1/3,1/6,1/4,1/8,1/12,\cdots,1/n,1/2n,\cdots,1/n(n-1),1/(n+1),\cdots
		      \]
		      的数列, 由于~$1/n$~和~$1/(n+1)$~之间的所有项都严格小于~$1/(n+1)$, 于是~$\{a_n\}$~的上控数列\footnote{请结合数列的上下极限部分.}~$\{\overline{a_n}\}$~为~$1,1/2,1/3,1/4,1/4,\cdots$, 其中~$1/n$~连续出现了~$n-3$~次($n\geqslant3$), 显然~$\lim\limits_{n\to\infty}\overline{a_n}=0$. 而全为正项的数列~$\{a_n\}$~有一个子列~$\{1/n\}$~收敛于~$0$, 故
		      \[
			      \varliminf\limits_{n\to\infty} a_n=\varlimsup\limits_{n\to\infty} a_n=0.
		      \]
		      即~$\lim\limits_{n\to\inf}a_n=0$, 但显然~$\{|a_n|\}$~并不单调.\qedhere
	      \end{proof}
	\item 判断: 非负数列的极限是非负数, 正数列的极限是整数.
	      \begin{proof}[\bf 答]
		      非负数列的极限是非负数. 反证法. 假设非负数列~$\{a_n\}$~的极限为~$A<0$, 则存在~$N\in\mathbf{N}_{+}$, 当~$n>N$~时有~$|a_n-A|<-A/2$, 即当~$n>N$~时有~$3A/2<a_n<A/2<0$, 与~$\{a_n\}$~非负矛盾.

		      正数列的极限不一定为正数, 如取~$\{1/n\}$, 其极限为~$0$.\qedhere
	      \end{proof}
\end{enumerate}

\subsection{练习题 pp.17.}
\begin{enumerate}
	\item 按极限定义证明:
	      \begin{tabenum}[(1)]
		      \tabenumitem $\lim\limits_{n\to\infty} \dfrac{3n^2}{n^2-4}=3;$
		      \tabenumitem $\lim\limits_{n\to\infty} \dfrac{\sin{n}}{n}=0;$\\
		      \tabenumitem $\lim\limits_{n\to\infty} (1+n)^{\frac{1}{n}}=1;$
		      \tabenumitem $\lim\limits_{n\to\infty} \dfrac{a^n}{n!}=0.$
	      \end{tabenum}
	      \begin{proof}
		      对于任何~$\varepsilon>0$,
		      \begin{enumerate}[(1)]
			      \item 取~$N=[\sqrt{12/\varepsilon+4}]+1$, 当~$n>N$~时, $|\dfrac{3n^2}{n^2-4}-3|=\dfrac{12}{n^2-4}<\varepsilon;$
			      \item 取~$N=[1/\varepsilon]$, 当~$n>N$~时, $|\dfrac{\sin{n}}{n}\leqslant\dfrac{1}{n}<\varepsilon;$
			      \item 由于~$(1+n)^{\frac{1}{n}}>1, \forall n\in\mathbf{N}_{+}$, 故令~$y_n=(1+n)^{\frac{1}{n}}-1>0$, 有$n+1=(1+y_n)^n\geqslant \dfrac{n(n-1)}{2}y_n^2$, 即
			            \[
				            \sqrt[n]{n+1}-1=y_n\leqslant\sqrt{\dfrac{2(n+1)}{n(n-1)}}.
			            \]
			            又由~$\lim\limits_{n\to\infty}\dfrac{2(n+1)}{n(n-1)}$, 故存在~$N\in\mathbf{N}_{+}$, 使当~$n>N$~时有~$\dfrac{2(n+1)}{n(n-1)}<\varepsilon<1$, 故当~$n>N$~时有
			            \[
				            \sqrt[n]{n+1}-1=y_n\leqslant\sqrt{\dfrac{2(n+1)}{n(n-1)}}<\sqrt{\varepsilon}<\varepsilon;
			            \]
			      \item 若~$0<a\leqslant 1$, 显然取~$N=[\varepsilon]+1$, 当~$n>N$~时
			            \[
				            \dfrac{a^n}{n!}\leqslant\dfrac{1}{n}<\varepsilon.
			            \]
			            若~$a>1$, 则存在~$k\in\mathbf{N}_{+}$~使得~$k<a<k+1$, 于是
			            \[
				            \dfrac{a^n}{n!}=\dfrac{a\cdot a\cdots a\cdot a\cdot a\cdots a\cdot a}{n\cdot(n-1)\cdots(k+1)k(k-1)\cdots2\cdot1}\leqslant\dfrac{a}{n}\dfrac{a\cdots a}{a\cdots a}\cdot\dfrac{a}{k}\dfrac{a}{k-1}\cdots\dfrac{a}{2}\dfrac{a}{1}.
			            \]
			            注意上式中最后一项是一常数, 可记为~$K$, 取~$N=[aK/\varepsilon]+1$, 当~$n>N$~时有~$\dfrac{a^n}{n!}<\varepsilon.$\qedhere
		      \end{enumerate}
	      \end{proof}
	\item 设~$a_n\geqslant 0, n\in\mathbf{N}_{+}$, 数列~$\{a_n\}$~收敛于~$a$, 则~$\lim\limits_{n\to\infty}\sqrt{a_n}=\sqrt{a}$.
	      \begin{proof}
		      $|\sqrt{a_n}-\sqrt{a}|=\dfrac{|a_n-a|}{\sqrt{a_n}+\sqrt{a}}\leqslant\dfrac{|a_n-a|}{\sqrt{a}}$. $\forall\varepsilon>0$, 由~$\lim\limits_{n\to\infty} a_n=a$, $\exists N\in\mathbf{N}_{+}$, 当~$n>N$~时有~$|a_n-a|\leqslant\sqrt{a}\varepsilon$. 故当~$n>N$~时, ~$|\sqrt{a_n}-\sqrt{a}|\leqslant\dfrac{|a_n-a|}{\sqrt{a}}<\varepsilon$, 即~$\lim\limits_{n\to\infty}\sqrt{a_n}=\sqrt{a}$.\qedhere
	      \end{proof}
	\item 若~$\lim\limits_{n\to\infty} a_n=a$, 则~$\lim\limits_{n\to\infty}|a_n|=|a|$. 反之如何?
	      \begin{proof}
		      $\forall\varepsilon>0$, 由~$\lim\limits_{n\to\infty} a_n=a$, $\exists N\in\mathbf{N}_{+}$, 当~$n>N$~时有~$|a_n-a|<\varepsilon$. 故当~$n>N$~时, $||a_n|-|a||\leqslant|a_n-a|<\varepsilon$, 即~$\lim\limits_{n\to\infty}|a_n|=|a|$.\qedhere
	      \end{proof}
	\item \footnote{关于原先的~$5$~题, 完全可以使用相应函数极限的定义加上~Heine~定理证明, 并且本质没有任何不同.} 设~$a>1$, 证明~$\lim\limits_{n\to\infty}\dfrac{\log_a{n}}{n}=0$. (可以利用已知的极限~$\lim\limits_{n\to\infty}\sqrt[n]{n}=1$.)
	      \begin{proof}
		      \[
			      \lim\limits_{n\to\infty}\dfrac{\log_a{n}}{n}=\lim\limits_{n\to\infty}\log_a{n^{1/n}}=\lim\limits_{n\to\infty}\log_a{1}=0.
		      \]
		      其中第二个等号用到了~$\log_a{x}$~的连续性.\qedhere
	      \end{proof}
\end{enumerate}

\section{收敛数列的基本性质}
\subsection{思考题 pp.18.}
\begin{enumerate}
	\item	设~$\{a_n\}$~收敛而~$\{b_n\}$~发散, 问: $\{a_n+b_n\}$~和~$\{a_nb_n\}$~的敛散性如何?
	      \begin{proof}
		      $\{a_n+b_n\}$~发散. 反证法. 假设~$\lim\limits_{n\to\infty} a_n=a, \lim\limits_{n\to\infty} (a_n+b_n)=A$, 则对于~$\forall\varepsilon>0$, $\exists N_1, N_2\in\mathbf{N}_{+}$, 当~$n>N_1$~时, $|(a_n+b_n)-A|<\varepsilon/2$; 当~$n>N_2$~时, $|a_n-a|<\varepsilon/2$. 令~$N=\max\{N_1,N_2\}$, 则当~$n>N$~时有
		      \[
			      |b_n-(A-a)|=|[(a_n+b_n)-A]-(a_n-a)|\leqslant|(a_n+b_n)-A|+|a_n-a|<\varepsilon/2+\varepsilon/2=\varepsilon.
		      \]
		      即~$\lim\limits_{n\to\infty} b_n=A-a$, 与~$\{b_n\}$~发散矛盾.

		      $\{a_nb_n\}$~可能发散也可能收敛. 如取~$a_n=1/n, b_n=n\sin{n}$, 则~$a_nb_n=\sin{n}$, $\{a_nb_n\}$~发散; 取~$a_n=1/n, b_n=(-1)^n$, 则~$a_nb_n=(-1)^n1/n$, $\{a_nb_n\}$~收敛.\qedhere
	      \end{proof}
	\item 设~$\{a_n\}$~和~$\{b_n\}$~都发散, 问: $\{a_n+b_n\}$~和~$\{a_nb_n\}$~的敛散性如何?
	      \begin{proof}
		      $\{a_n+b_n\}$~可能发散也可能收敛. 如取~$a_n=(-1)^n, b_n=(-1)^{n+1}$, 则~$a_n+b_n=0$, $\{a_n+b_n\}$~收敛; 取~$a_n=b_n=(-1)^n$, 则~$a_n+b_n=(-1)^n\cdot 2$, $\{a_n+b_n\}$~发散.

		      $\{a_nb_n\}$~可能发散也可能收敛. 如取~$a_n=b_n=(-1)^n$, 则~$a_nb_n=1$, $\{a_nb_n\}$~收敛; 取~$a_n=(-1)^n, b_n=n$, 则~$a_nb_n=(-1)^n\cdot n$, $\{a_nb_n\}$~发散.\qedhere
	      \end{proof}
	\item 设~$a_n\leqslant b_n\leqslant c_n, n\in\mathbf{N}_{+}$, 已知~$\lim\limits_{n\to\infty} (c_n-a_n)=0$, 问: 数列~$\{b_n\}$~是否收敛?
	      \begin{proof}
		      $\{b_n\}$~不一定收敛. 取一反例, $a_n=n, b_n=n+1/2n, c_n=n+1/n, n\in\mathbf{H}_{+}$, 则~$\lim\limits_{n\to\infty} (c_n-a_n)=\lim\limits_{n\to\infty} 1/n=0$, 但显然~$\{b_n\}$~发散.\qedhere
	      \end{proof}
	\item 找出下列运算中的错误:
	      \[
		      \lim\limits_{n\to\infty} \left(\dfrac{1}{n+1}+\cdots+\dfrac{1}{2n}\right)=\lim\limits_{n\to\infty}\dfrac{1}{n+1}+\cdots+\lim\limits_{n\to\infty}\dfrac{1}{2n}=0.
	      \]
	      \begin{proof}
		      问题在于第二个等号, 极限的四则运算法则之对于有限次的加减乘除(除法要求分母的数列不为零)成立, 对于可列次的四则运算没有意义.\qedhere
	      \end{proof}
	\item 设已知~$\{a_n\}$~收敛于~$a$, 又对每个~$n$~有~$b<a_n<c$, 问: 是否成立~$b<a<c$?
	      \begin{proof}
		      不一定成立. 如取~$b=0, c=1, a_n=1/n, n\in\mathbf{N}_{+}$, 则有~$b<a_n<c, \forall n\in\mathbf{H}_{+}$, 但~$a=\lim\limits_{n\to\infty} a_n=0$, 故~$a=c$.\qedhere
	      \end{proof}
	\item 设已知~$\{a_n\}$~收敛于~$a$, 又有~$b\leqslant a\leqslant c$, 问: 是否存在~$N$, 使得当~$n>N$~时成立~$b\leqslant a\leqslant c$?
	      \begin{proof}
		      两次应用数列极限的保序性, 所得的正整数分别记为~$N_1$~和~$N_2$, 则取~$N=\max\{N_1,N_2\}$, 当~$n>N$~时就有~$b_n\leqslant a_n\leqslant c_n$.\qedhere
	      \end{proof}
	\item 设已知~$\lim\limits_{n\to\infty} a_n=0$, 问是否有~$\lim\limits_{n\to\infty} (a_1a_2\cdots a_n)=0$? 又问: 反之如何?
	      \begin{proof}\footnote{结合无穷级数的相关知识可以给出另一证明. 记~$u_n=a_1\cdots a_n$, 由无穷级数的~d'Alembert~比值判别法, $\lim\limits_{n\to\infty} \dfrac{u_{n+1}}{u_n}=\lim\limits_{n\to\infty} a_{n+1}=0$, 有无穷级数~$\sum\limits_{n=1}^{\infty}u_n$~收敛, 故~$\lim\limits_{n\to\infty} u_n=0$.}
		      对于~$\varepsilon_0=1$, 由~$\lim\limits_{n\to\infty} a_n=0$~知存在~$N\in\mathbf{N}_{+}$~使得当~$n>N$~时有~$|a_n|<1$, 记~$K=|a_1a_2\cdots a_N|$. 对于~$\forall 0<\varepsilon<1$, $\exists N'\in\mathbf{N}_{+}$, 当~$n>N'$~时有~$|a_n|<\varepsilon/K$. 因此对于~$n>\max\{N,N'\}$, $|a_1a_2\cdots a_n|=K|a_{N+1}\cdots a_n|\leqslant K|a_n|<K\cdot\varepsilon/K=\varepsilon$, 即~$\lim\limits_{n\to\infty} (a_1a_2\cdots a_n)=0$.\qedhere
	      \end{proof}
\end{enumerate}

\subsection{练习题 pp.25.}
\begin{enumerate}
	\item	证明: $\{a_n\}$~收敛的充分必要条件是~$\{a_{2k}\}$~和~$\{a_{2k-1}\}$~收敛于同一极限.
	      \begin{proof}
		      必要性. 设~$\lim\limits_{n\to\infty} a_n=a$, 则对于~$\forall\varepsilon>0$, $\exists N\in\mathbf{N}_{+}$, 当~$n>N$~时, $|a_n-a|<\varepsilon$. 当~$k>N$~时, $2k>2k-1>N$, 故当~$k>N$~时, $|a_{2k}-a|<\varepsilon, |a_{2k-1}-a|<\varepsilon$. 即~$\lim\limits_{n\to\infty} a_{2k}=\lim\limits_{n\to\infty} a_{2k-1}=a$.

		      充分性. 设~$\lim\limits_{n\to\infty} a_{2k}=\lim\limits_{n\to\infty} a_{2k-1}=a$, 则对于~$\forall\varepsilon>0$, $\exists K_1\in\mathbf{N}_{+}$, 当~$k>K_1$~时, $|a_{2k}-a|<\varepsilon$; $\exists K_2\in\mathbf{N}_{+}$, 当~$k>K_2$~时, $|a_{2k-1}-a|<\varepsilon$. 取~$N=\max\{K_1,K_2\}$, 则当~$n>N$~时, $|a_n-a|<\varepsilon$.\qedhere
	      \end{proof}
	\item 以下是可以应用夹逼定理的几个题.
	      \begin{enumerate}[(1)]
		      \item 给定~$p$~个正数~$a_1,a_2,\cdots,a_p$, 求~$\lim\limits_{n\to\infty} \sqrt[n]{a_1^n+a_2^n+\cdots+a_p^n}$;
		      \item 设~$x_n=\dfrac{1}{\sqrt{n^2+1}}+\dfrac{1}{\sqrt{n^2+2}}+\cdots+\dfrac{1}{\sqrt{(n+1)^2}}, n\in\mathbf{N}_{+}$, 求~$\lim\limits_{n\to\infty} x_n$;
		      \item 设~$a_n=\left(1+\dfrac{1}{2}+\cdots+\dfrac{1}{n}\right)^\frac{1}{n}, n\in\mathbf{N}_{+}$, 求~$\lim\limits_{n\to\infty} a_n$;
		      \item 设~$\{a_n\}$~为正数列, 并且已知它收敛于~$a>0$, 证明~$\lim\limits_{n\to\infty} \sqrt[n]{a_n}=1$.
	      \end{enumerate}
	      \begin{proof}
		      \begin{enumerate}[(1)]
			      \item $\max\limits_{1\leqslant k\leqslant p} a_k\leqslant\sqrt[n]{a_1^n+a_2^n+\cdots+a_p^n}\leqslant\sqrt[n]{n\max\limits_{1\leqslant k\leqslant p} a_k^n}=\sqrt[n]{n}\max\limits_{1\leqslant k\leqslant p} a_k\to\max\limits_{1\leqslant k\leqslant p} a_k (n\to\infty)$, 故~$\lim\limits_{n\to\infty} \sqrt[n]{a_1^n+a_2^n+\cdots+a_p^n}=\max\limits_{1\leqslant k\leqslant p} a_k$;
			      \item $\dfrac{2n+1}{\sqrt{(n+1)^2}}\leqslant x_n\leqslant \dfrac{2n+1}{\sqrt{n^2+1}}$, $\lim\limits_{n\to\infty}\dfrac{2n+1}{\sqrt{(n+1)^2}}=\lim\limits_{n\to\infty}\dfrac{2n+1}{\sqrt{n^2+1}}=2$, 故~$\lim\limits_{n\to\infty} x_n=2$;
			      \item $\sqrt[n]{n\cdot 1/n}\leqslant a_n\leqslant \dfrac[n]{n}\to 1(n\to\infty)$, 故~$\lim\limits_{n\to\infty} a_n=1$;
			      \item 取~$\varepsilon=a/2>0$, 由~$\lim\limits_{n\to\infty} a_n=a$, $\exists N\in\mathbf{N}_{+}$, 当~$n>N$~时, $|a_n-a|<a/2$, 即当~$n>N$~时~$a/2<a_n<3a/2$. 同时开~$n$~次根号, 有~$\sqrt[n]{a/2}<\sqrt[n]{a}<\sqrt[n]{3a/2}, \forall n>N$. 由于~$\lim\limits_{n\to\infty}\sqrt[n]{a/2}=\lim\limits_{n\to\infty}\sqrt[n]{3a/2}=1$, 故~$\lim\limits_{n\to\infty} \sqrt[n]{a_n}=1$.\qedhere
		      \end{enumerate}
	      \end{proof}
	\item 求以下极限:
	      \begin{enumerate}[(1)]
		      \item $\lim\limits_{n\to\infty} (1+x)(1+x^2)\cdots(1+x^{2^n})$, 其中~$|x|<1$;
		      \item $\lim\limits_{n\to\infty} \left(1-\dfrac{1}{2^2}\right)\left(1-\dfrac{1}{3^2}\right)\cdots\left(1-\dfrac{1}{n^2}\right)$;
		      \item $\lim\limits_{n\to\infty} \left(1-\dfrac{1}{1+2}\right)\left(1-\dfrac{1}{1+2+3}\right)\cdots\left(1-\dfrac{1}{1+2+\cdots+n}\right)$;
		      \item $\lim\limits_{n\to\infty} \left(\dfrac{1}{1\cdot2\cdot3}+\dfrac{1}{2\cdot3\cdot4}+\cdots+\dfrac{1}{n(n+1)(n+2)}\right)$;
		      \item $\lim\limits_{n\to\infty} \sum\limits_{k=1}^{n}\dfrac{1}{k(k+1)\cdots(k+\nu)}$, 其中~$\nu\in\mathbf{N}_{+}, \nu>1$.\\
		            (最后两个题是$\lim\limits_{n\to\infty} \left(\dfrac{1}{1\cdot2}+\dfrac{1}{2\cdot3}+\cdots+\dfrac{1}{n(n+1)}\right)$~的推广.)
	      \end{enumerate}
	      \begin{proof}
		      \begin{enumerate}[(1)]
			      \item $\begin{aligned}[t]
					            (1+x)(1+x^2)\cdots(1+x^{2^n})= & \dfrac{(1+x)(1-x)(1+x^2)\cdots(1+x^{2^n})}{1-x}                   \\
					            =                              & \dfrac{(1-x^2)(1+x^2)\cdots(1+x^{2^n})}{1-x}                      \\
					            =                              & \cdots=\dfrac{1-x^{2^{n+1}}}{1-x}\to\dfrac{1}{1-x} \ (n\to\infty)
				            \end{aligned}$
			      \item $\begin{aligned}[t]
					            \left(1-\dfrac{1}{2^2}\right)\left(1-\dfrac{1}{3^2}\right)\cdots\left(1-\dfrac{1}{n^2}\right)= & \left(1-\dfrac{1}{2}\right)\left(1+\dfrac{1}{2}\right)\left(1-\dfrac{1}{3}\right)\left(1+\dfrac{1}{3}\right)\cdots\left(1-\dfrac{1}{n}\right)\left(1+\dfrac{1}{n}\right) \\
					            =                                                                                              & \dfrac{1}{2}\cdot\dfrac{3}{2}\cdot\dfrac{2}{3}\cdot\dfrac{4}{3}\cdots\dfrac{n-1}{n}\cdot\dfrac{n+1}{n}                                                                   \\
					            =                                                                                              & \dfrac{1}{2}\dfrac{n+1}{n}\to\dfrac{1}{2} \ (n\to\infty)
				            \end{aligned}$
			      \item $\begin{aligned}[t]
					            \left(1-\dfrac{1}{1+2}\right)\left(1-\dfrac{1}{1+2+3}\right)\cdots\left(1-\dfrac{1}{1+2+\cdots+n}\right)= & \dfrac{2}{1+2}\cdot\dfrac{2+3}{1+2+3}\cdots\dfrac{2+\cdots+n}{1+2+\cdots n}                     \\
					            =                                                                                                         & \dfrac{1\cdot4}{2\cdot3}\cdot\dfrac{2\cdot5}{3\cdot4}\cdot\cdots\cdot\dfrac{(n-1)(n+2)}{n(n+1)} \\
					            =                                                                                                         & \dfrac{2!(n-1)!(n+2)!}{3!n!(n+1)!}                                                              \\
					            =                                                                                                         & \dfrac{n+2}{3n}\to\dfrac{1}{3} \ (n\to\infty)
				            \end{aligned}$
			      \item $\begin{aligned}[t]
					              & \dfrac{1}{1\cdot2\cdot3}+\dfrac{1}{2\cdot3\cdot4}+\cdots+\dfrac{1}{n(n+1)(n+2)}                                                                                                            \\
					            = & \dfrac{1}{2}\left[\left(\dfrac{1}{1\cdot2}-\dfrac{1}{2\cdot3}\right)+\left(\dfrac{1}{2\cdot3}-\dfrac{1}{3\cdot4}\right)+\cdots+\left(\dfrac{1}{n(n+1)}-\dfrac{1}{(n+1)(n+2)}\right)\right] \\
					            = & \dfrac{1}{4}-\dfrac{1}{2(n+1)(n+2)}\to\dfrac{1}{4} \ (n\to\infty)
				            \end{aligned}$
			      \item $\begin{aligned}[t]
					            \sum\limits_{k=1}^{n}\dfrac{1}{k(k+1)\cdots(k+\nu)}= & \sum\limits_{k=1}^n\dfrac{1}{\nu}\left(\dfrac{1}{k(k+1)\cdots(k+\nu-1)}-\dfrac{1}{(k+1)(k+2)\cdots(k+\nu)}\right)                            \\
					            =                                                    & \dfrac{1}{\nu}\left(\dfrac{1}{1\cdot2\cdot\cdots\cdot\nu}-\dfrac{1}{(n+1)(n+2)\cdots(n+\nu)}\right)\to\dfrac{1}{\nu\cdot\nu!} \ (n\to\infty)
				            \end{aligned}$\qedhere
		      \end{enumerate}
	      \end{proof}
	\item 设~$s_n=a+3a^2+\cdots+(2n-1)a^n$, $|a|<1$, 求~$\{a_m\}$~的极限.\\
	      (试计算~$s_n-as_n$.)
	      \begin{proof}
		      \begin{equation*}
			      \begin{split}
				      S_n   &= a+3a^2+\cdots+(2n-1)a^n;              \\
				      aS_n  &= \qquad\; a^2+\cdots+(2n-3)a^n+(2n-1)a^{n+1}.
			      \end{split}
		      \end{equation*}
		      上面两式相减, 有
		      \[
			      (1-a)S_n=a+2(a^2+a^3+\cdots+a^n)-(2n-1)a^{n+1}=\dfrac{a(1+a)}{1-a}-(2n-1)a^{n+1}-\dfrac{2a^{n+1}}{1-a}.
		      \]
		      故
		      \[
			      S_n=\dfrac{a(1+a)}{(1-a)^2}-\dfrac{1}{1-a}\left((2n-1)a^{n+1}+\dfrac{2a^{n+1}}{1-a}\right)\to\dfrac{a(1+a)}{(1-a)^2}\ (n\to\infty).\qedhere
		      \]
	      \end{proof}
	\item 设正数列~$\{x_n\}$~收敛, 极限大于~$0$, 证明: 这个数列有正下界, 但在数列中不一定有最小数.
	      \begin{proof}
		      设~$\lim\limits_{n\to\infty} x_n=A>0$, 则对~$\varepsilon=A/2>0$, $\exists N\in\mathbf{N}_{+}$, 当~$n>N$~时, $|x_n-A|<A/2$, 即当~$n>N$~时, $x_n>A/2$, 记~$M=\max\{x_1,x_2,\cdots,x_N, A/2\}$, 则~$x_n\geqslant M, \forall n\in\mathbf{N}_{+}$, 即~$M$~是~$\{x_n\}$~的一个正的下界.

		      举一个无最小数的例子: $x_n=1+1/n, n\in\mathbf{N}_{+}$.\qedhere
	      \end{proof}
	\item 证明: 若有~$\lim\limits_{n\to\infty} a_n=+\infty$, 则在数列~$\{a_n\}$~中一定有最小数.
	      \begin{proof}
		      任取~$k\in\mathbf{N}_{+}$, 对于~$a_k$, $\exists N\in\mathbf{N}$, 当~$n>N$~时有~$a_n>a_k$. 取~$a=\min\{a_1,a_2,\cdots,a_N,a_k\}$, 则~$a\leqslant a_n, \forall n\in\mathbf{N}_{+}$, 同时~$a$~是~$\{a_n\}$~中的某一项, 故~$a$~是~$\{a_n\}$~中的最小数.\qedhere
	      \end{proof}
	\item 证明: 无界数列至少有一个子列是确定符号的无穷大量.
	      \begin{proof}
		      设~$\{x_n\}$~是无界数列, 不妨设其无上界, 即对任意~$M>0$, $\exists n\in\mathbf{N}_{+}$~使得~$x_n>M$.

		      对于~$M_1=1$, $\exists n_1\in\mathbf{N}_{+}$, 使得~$x_{n_1}>1$;

		      对于~$M_2=2$, $\exists n_2\in\mathbf{N}_{+}, n_2>n_1$, 使得~$x_{n_2}>2$, 断言这样的~$n_2$~是可以找到的, 否则~$\forall n>n_1$, $x_n\leqslant M_2$, 与~$\{x_n\}$~无界矛盾;

		      假设已经找出了~$x_{n_k}$, 使得~$x_{n_k}>x_{n_{k-1}}$, $x_{n_{k}}>M_k=k$, 则对于~$M_{k+1}=k+1$, $\exists n_{k+1}\in\mathbf{N}_{+}, n_{k+1}>n_{k}$, 使得~$x_{n_{k+1}}>k+1$, 断言这样的~$n_{k+1}$~是可以找到的, 否则~$\forall n>n_k$, $x_n\leqslant M_{k+1}$, 与~$\{x_n\}$~无界矛盾. 由数学归纳法可知找出了数列~$\{x_{n_{k}}\}$~使得~$n_1<n_2<\cdots<n_k<n_{k+1}<\cdots$, $x_{n_k}>k, k\in\mathbf{N}_{+}$. 这说明~$\{x_{n_k}\}$~是~$\{x_n\}$~的子列, 并且~$\{x_{n_k}\}$~是正的无穷大量. 同理若~$\{x_n\}$~无下界时可找到一个子列是负的无穷大量.\qedhere
	      \end{proof}
	\item 证明: 数列~$\{\tan{n}\}$~发散.
	      \begin{proof}
		      \begin{equation*}
			      \begin{split}
				      |\tan(n+1)-\tan{n}|=&\left|\dfrac{\sin(n+1)}{\cos(n+1)}-\dfrac{\sin{n}}{\cos{n}}\right|\\
				      =&\left|\dfrac{\sin(n+1)\cos{n}-\cos(n+1)\sin{n}}{\cos(n+1)\cos{n}}\right|\\
				      =&\left|\dfrac{\sin{1}}{\cos(n+1)\cos{n}}\right|\\
				      \geqslant&\sin{1},\ \forall n\in\mathbf{N}_{+}.
			      \end{split}
		      \end{equation*}
		      这说明~$\exists\varepsilon_0=\sin{1}$, $\forall N\in\mathbf{N}_{+}$, $\exists n>N$~使得~$|\tan(n+1)-\tan{n}|\geqslant\sin{1}>0$. 由~Cauchy~收敛准则知, $\{\tan{n}\}$~发散.\qedhere
	      \end{proof}
	\item 设数列~$\{S_n\}$~的定义为
	      \[
		      S_n=1+\dfrac{1}{2^p}+\dfrac{1}{3^p}+\cdots+\dfrac{1}{n^p}, n\in\mathbf{N}_{+}.
	      \]
	      证明: $\{S_n\}$~在以下两种情况均发散: (1)$p\leqslant 0$; (2)$0<p<1$.
	      \begin{proof}
		      当~$p\leqslant 0$~时, $S_{n+1}-S_n=\dfrac{1}{n^p}=\begin{cases}1, & p=0;\\ n^{-p}, & p<0.\end{cases}$ 由~Cauchy~收敛准则知~$\{S_n\}$~发散.\\

		      当~$0<p<1$~时, 对于~$\forall n\in\mathbf{N}_{+}$, $\exists k\in\mathbf{N}_{+}$~使得~$2^k<n<2^{k+1}$, 故
		      \begin{equation*}
			      \begin{split}
				      S_n\geqslant & S_{2^k}=1+\dfrac{1}{2^p}+\dfrac{1}{3^p}+\cdots+\dfrac{1}{2^{kp}}                                                                                         \\
				      \geqslant        & 1+\dfrac{1}{2^p}+\left(\dfrac{1}{3^p}+\dfrac{1}{4^p}\right)\cdots+\left(\dfrac{1}{(2^{k-1}+1)^p}+\dfrac{1}{(2^{k-1}+2)^p}\cdots+\dfrac{1}{2^{kp}}\right) \\
				      \geqslant        & 1+\dfrac{1}{2^p}+\dfrac{2}{4^p}+\cdots+\dfrac{2^{k-1}}{2^{kp}}                                                                                           \\
				      =                & 1+\dfrac{1}{2}(2^{1-p}+2^{2(1-p)}+\cdots+2^{k(1-p)})                                                                                                     \\
				      =                & 1+\dfrac{2^{1-p}}{2}\dfrac{2^{k(1-p)}-1}{2^{1-p}-1}\to+\infty\ (n\to\infty)
			      \end{split}
		      \end{equation*}
		      故~$\{S_n\}$~发散.\qedhere
	      \end{proof}
\end{enumerate}

\section{单调数列}
\subsection{练习题 pp.30.}
\begin{enumerate}
	\item 若~$\{x_n\}$~单调, 则~$\{|x_n|\}$~至少从某项开始后单调. 又问: 反之如何?
	      \begin{proof}
		      不妨设~$\{x_n\}$~单增.

		      若~$x_n\leqslant 0, n=1,2,\cdots$, 则~$\{|x_n|\}$~是单调递减数列;

		      若~$\exists n_0$~使得~$x_{n_0}>0$, 则在集合~$\{x_1, x_2,\cdots,x_{n_0}\}$~中必可找出~$n_1$~使得~$x_{n_1}<0<x_{n_1+1}$, 于是~$x_n>0, \forall n>n_1$, 又由于~$\{x_n\}$~单调递增, 知~$\{|x_n|\}$~从~$n_1$~项后单调递增.

		      反之不成立. 举一反例, $x_n=(-1)^nn, n\in\mathbf{N}_{+}$, 则易知~$\{|x_n\}$~单调递增, 但~$\{x_n\}$~在任意项之后都不单调.\qedhere
	      \end{proof}
	\item 设~$\{a_n\}$~单调增加, ~$\{b_n\}$~单调减少, 且有~$\lim\limits_{n\to\infty} (a_n-b_n)=0$. 证明: $\{a_n\}$~和~$\{b_n\}$~都收敛, 且极限相等.
	      \begin{proof}
		      $\{a_n-b_n\}$~收敛从而有界, 即~$\exists M>0, \forall n\in\mathbf{N}_{+}, |a_n-b_n|\leqslant M$. 特别有~$a_n\leqslant b_n+M, \forall n\in\mathbf{N}_{+}$. 由于~$\{b_n\}$~单调减少, $a_n\leqslant b_1+M, \forall n\in\mathbf{N}_{+}$, 即~$\{a_n\}$~单调增加有上界~$b_1+M$, 故~$\{a_n`\}$~收敛. 同理可知~$\{b_n\}$~单调减少有下界~$a_1-M$, 故~$\{b_n\}$~也收敛. 由极限的四则运算法则, $\lim\limits_{n\to\infty} a_n-\lim\limits_{n\to\infty} b_n=\lim\limits_{n\to\infty} (a_n-b_n)=0$, 故~$\lim\limits_{n\to\infty} a_n=\lim\limits_{n\to\infty} b_n$.\qedhere
	      \end{proof}
	\item 按照极限的定义证明: 单调增加有上界的数列的极限不小于数列的任何一项, 单调减少有下界的数列的极限不大于数列极限的任何一项.
	      \begin{proof}
		      设~$\{x_n\}$~是单调增加有上界的数列, 由单调有界原理, ~$\{x_n\}$~收敛, 记~$\lim\limits_{n\to\infty} x_n=a$. 若~$\exists n_0\in\mathbf{N}_{+}$~使得~$x_{n_0}>a$, 则由~$\{x_n\}$~单调增加知~$\forall n\in\mathbf{N}_{+}, n>n_0$, $x_n\geqslant x_{n_0}>a$. 对于~$\varepsilon_0=\dfrac{x_{n_0}-a}{2}>0$, $x_n-a\geqslant x_{n_0}-a>\varepsilon$, 这与~$\lim\limits_{n\to\infty} x_n=a$~矛盾.

		      设~$\{y_n\}$~是单调减少有下界的数列, 则~$\{-y_n\}$~是单调增加有上界的数列, 由单调有界原理, 若~$\lim\limits_{n\to\infty} y_n=b$, 则~$\lim\limits_{n\to\infty} -y_n=-b$. 由前可知, $\forall n\in\mathbf{N}_{+}, -y_n\leqslant-b$, 即~$\forall n\in\mathbf{N}_{+}, y_n\geqslant b$.\qedhere
	      \end{proof}
	\item 设~$x_n=\dfrac{2}{3}\cdot\dfrac{3}{5}\cdots\dfrac{n+1}{2n+1}, n\in\mathbf{N}_{+}$, 求数列~$\{x_n\}$~的极限.
	      \begin{proof}
		      易知~$\forall n\in\mathbf{N}_{+}, x_n>0$, 且~$\dfrac{x_n}{x_{n-1}}=\dfrac{n+1}{2n+1}<1, \forall n\in\mathbf{N}_{+}$. $\{x_n\}$~单调递减有下界, 故极限存在, 设为~$a$, 在递推式~$x_n=\dfrac{n+1}{2n+1}x_{n-1}$~两边令~$n\to\infty$, 有~$a=a/2$, 故~$a=0$~或~$1$. 由~3~题可知, $a$~不大于~$\{x_n\}$~的任意一项, 故~$\lim\limits_{n\to\infty} x_n=a$.\qedhere
	      \end{proof}
	\item 设~$a_n=\dfrac{10}{1}\cdot\dfrac{11}{3}\cdots\dfrac{n+9}{2n-1}, n\in\mathbf{N}_{+}$, 求数列~$\{a_n\}$~的极限.
	      \begin{proof}
		      易知~$\forall n\in\mathbf{N}_{+}, a_n>0$, 且~$\dfrac{x_n}{x_{n-1}}=\dfrac{n+9}{2n-1}<1, \forall n\in\mathbf{N}_{+}, n>10$. $\{x_n\}$~从第~$11$~项起单调递减有下界, 故极限存在, 设为~$a$, 在递推式~$x_n=\dfrac{n+9}{2n-1}x_{n-1}$~两边令~$n\to\infty$, 有~$a=a/2$, 故~$a=0$~或~$1$. 同上题推理有~$\lim\limits_{n\to\infty} a_n=0$.\qedhere
	      \end{proof}
	\item 在例题~2.2.6~的基础上证明: 当~$p>1$~时数列~$\{S_n\}$~收敛, 其中
	      \[
		      S_n=1+\dfrac{1}{2^p}+\dfrac{1}{3^p}+\cdots+\dfrac{1}{n^p}, n\in\mathbf{N}_{+}.
	      \]
	      \begin{proof}
		      对于~$\forall n\in\mathbf{N}_{+}$, $\exists k\in\mathbf{N}_{+}$~使得~$2^{k-1}<n<2^{k}$, 故
		      \begin{equation*}
			      \begin{split}
				      S_n\leqslant & S_{2^k-1}=1+\dfrac{1}{2^p}+\dfrac{1}{3^p}+\cdots+\dfrac{1}{(2^n-1)^p}                                                                                         \\
				      \leqslant        & 1+\left(\dfrac{1}{2^p}+\dfrac{1}{2^p}\right)+\cdots+\left(\dfrac{1}{(2^{k-1})^p}+\dfrac{1}{(2^{k-1})^p}\cdots+\dfrac{1}{(2^{k-1})^p}\right) \\
				      =                & 1+2^{1-p}+2^{2(1-p)}+\cdots+2^{(k-1)(1-p)}                                                                                                    \\
				      =                & \dfrac{1-2^{k(1-p)}}{1-2^{1-p}}\leqslant \dfrac{1}{1-2^{1-p}}
			      \end{split}
		      \end{equation*}
		      这表明~$\{S_n\}$~有界, 又显然~$\{S_n\}$~单调递增, 故由单调有界原理知~$\{S_n\}$~收敛.\qedhere
	      \end{proof}
	\item 设~$0<x_0<\dfrac{\pi}{2}, x_n=\sin{x_{n-1}}, n\in\mathbf{N}_{+}$, 证明: $\{x_n\}$~收敛, 并求其极限.
	      \begin{proof}
		      $\sin{x}<x, \forall x\in(0,\dfrac{\pi}{2}$, 故由数学归纳法易知~$x_n=\sin{x_{n-1}}<x_{n-1}, n=1, 2,\cdots$, 即~$\{x_n\}$~单调递减; 又由~$x_0>0$~易知~$x_n>0, n=1,2,\cdots$, 即~$0$~是~$\{x_n\}$~的下界. 由单调有界原理, $\{x_n\}$~收敛. 设~$\lim\limits_{n\to\infty} x_n=\xi$, 在~$x_n=\sin{x_{n-1}}$~两边令~$n\to\infty$, 注意~$\sin{x}$~是其定义域上的连续函数, 由~Heine~定理及极限的保序性, $\xi=\sin{\xi}, \xi\in[0,\pi/2]$, 故~$\xi=0$.\qedhere
	      \end{proof}
	\item 设~$a_n=\left[\dfrac{(2n-1)!!}{(2n)!!}\right]^2, n\in\mathbf{N}_{+}$, 证明: $\{a_n\}$~收敛于~$0$.\\
	      (观察~$a_n=\left(\dfrac{1\cdot3}{2\cdot2}\right)\left(\dfrac{3\cdot5}{4\cdot4}\right)\cdots\left(\dfrac{(2n-3)(2n-1)}{(2n-2)(2n-2)}\right)\left(\dfrac{2n-1}{(2n)^2}\right)$.)
	      \begin{proof}
		      \[
			      0\leqslant a_n=\left(\dfrac{1\cdot3}{2\cdot2}\right)\left(\dfrac{3\cdot5}{4\cdot4}\right)\cdots\left(\dfrac{(2n-3)(2n-1)}{(2n-2)(2n-2)}\right)\left(\dfrac{2n-1}{(2n)^2}\right).
		      \]
		      由平均值不等式可知~$(2n-3)(2n-1)\leqslant\left(\dfrac{(2n-3)+(2n-1)}{2}\right)^2=(2n-2)^2$, 即~$\dfrac{(2n-3)(2n-1)}{(2n-2)(2n-2)}\leqslant 1$, 于是~$0\leqslant a_n\leqslant\dfrac{2n-1}{4n^2}\to 0\ (n\to\infty)$. 故由夹逼定理知~$\lim\limits_{n\to\infty} a_n=0$.\qedhere
	      \end{proof}
	\item 设~$a_n=\left[\dfrac{(2n)!!}{(2n-1)!!}\right]^2\cdot\dfrac{1}{2n+1}, n\in\mathbf{N}_{+}$, 证明: $\{a_n\}$~收敛.\\
	      (方法与上一题类似. 在学了积分学后将于命题~$11.4.1$~中求出上述数列的极限为~$\dfrac{\pi}{2}$. 这就是~Wallis~公式.)
	      \begin{proof}
		      \begin{equation*}
			      \begin{split}
				      a_n=&\left[\dfrac{(2n)!!}{(2n-1)!!}\right]^2\cdot\dfrac{1}{2n+1}=\left(\dfrac{2\cdot4\cdot6\cdots2n}{1\cdot3\cdot5\cdots(2n-1)}\right)^2\cdot\dfrac{1}{2n+1}\\
				      =&\left(\dfrac{2}{1^2}\right)\left(\dfrac{2\cdot4}{3^2}\right)\left(\dfrac{4\cdot6}{5^2}\right)\cdots\left(\dfrac{(2n-2)(2n)}{(2n-1)^2}\right)\cdot\dfrac{2n}{2n+1}\leqslant2.
			      \end{split}
		      \end{equation*}
		      其中用到了基本不等式~$(2n-2)(2n)\leqslant\left(\dfrac{(2n-2)+(2n)}{2}\right)^2=(2n-1)^2$, 即~$\dfrac{(2n-2)(2n)}{(2n-1)^2}\leqslant 1$, 于是~$\{a_n\}$~有上界; 又由
		      \[
			      \dfrac{a_n}{a_{n-1}}=\left(\dfrac{2n}{2n-1}\right)^2\dfrac{2n-1}{2n+1}=\dfrac{4n^2}{4n^2-1}\geqslant 1.
		      \]
		      故~$\{a_n\}$~单调增加. 由单调有界原理知~$\{a_n\}$~收敛.\qedhere

	      \end{proof}
	\item 下列数列中, 哪些是单调的?
	      \begin{tabenum}[(1)]
		      \item $\left\{\dfrac{1}{1+n^2}\right\}$; \item $\{\sin{n}\}$;	\item $\{\sqrt[n]{n!}\}$.
	      \end{tabenum}
	      \begin{proof}
		      \begin{enumerate}[(1)]
			      \item $\dfrac{a_n}{a_{n-1}}=\dfrac{1+(n-1)^2}{1+n^2}\leqslant 1$, 故~$\{a_n\}$~单调减少;
			      \item 由于~$\{\sin{n}\}$~有界, 若其单调, 则~$\{\sin{n}\}$~收敛, 而已知其发散, 故不单调;
			      \item 由于~$n!<(n+1)^n$, 故~$(n!)^{n+1}<(n!)^n(n+1)^n$, 不等式两边开~$n(n+1)$~次根号, 就有
			            \[
				            \sqrt[n]{n!}<\sqrt[n+1]{n!\cdot (n+1)}=\sqrt[n+1]{(n+1)!},
			            \]
			            故~$\{\sqrt[n]{n!}\}$~单调增加.\qedhere
		      \end{enumerate}
	      \end{proof}
	\item 证明: 单调数列~$\{a_n\}$~收敛的充分必要条件是它有一个收敛子列.
	      \begin{proof}
		      必要性. 若~$\{a_n\}$~收敛, 则其任意子列~$\{a_{n_k}\}$~均收敛.

		      充分性. 不妨设~$\{a_n\}$~单调增加, 则其任意子列也单调增加. 设~$\lim\limits_{n\to\infty} x_{n_k}=a$. 则由~3~可知, $\forall k\in\mathbf{N}_{+}, a_{n_k}\leqslant a$. 若~$\{a_n\}$~无上界, 则存在~$n_0\in\mathbf{N}_{+}$~使得~$x_{n_0}>a$, 从而对于充分大的~$k\in\mathbf{N}_{+}$, 有~$a_{n_k}\geqslant a_{n_0}>a$. 这与~$a_{n_k}\leqslant a, \forall k\in\mathbf{N}_{+}$~矛盾. 故~$\{a_n\}$~有上界. 从而由单调有界原理, $\{a_n\}$~收敛.
	      \end{proof}
	\item 对每个自然数~$n$, 用~$x_n$~表示方程~$x+x^2+\cdots+x^n=1$~在闭区间~$[0,1]$~中的根.\footnote{事实上, 这里需要使用函数的单调性及连续性证明方程的根在闭区间~$[0,1]$~中存在唯一.} 求~$\lim\limits_{n\to\infty} x_n$.
	      \begin{proof}
		      令~$f_n(x)=x+x^2+\cdots+x^n$, 则~$f_n'(x)=1+2x+\cdots+nx^{n-1}>0, \forall x>0$. 注意~$f_n(0)=0, f_n(1)=n\geqslant 1, \forall n\in\mathbf{N}_{+}$, 故由~$f_n(x)$~的单调性及连续函数的介值定理知, $f_n(x)$~的零点在~$[0,1]$~上存在唯一.

		      \[
			      f_n(x_n)=f_{n+1}(x_{n+1})=f_n(x_{n+1})+x_{n+1}^{n+1}\geqslant f_n(x_{n+1}), \forall n\in\mathbf{N}_{+},
		      \]
		      由~$f_n(x)$~的单调性易知~$x_n\geqslant x_{n+1}, n\in\mathbf{N}_{+}$, 故~$\{x_n\}$~单调有界, 从而收敛. 记~$\lim\limits_{n\to\infty} x_n=\xi$, 在~$1=f_n(x_n)=x_n+x_n^2+\cdots+x_n^n=\dfrac{x_n(1-x_n^n)}{1-x_n}$~的两侧取极限, 有~$1=2\xi-\lim\limits_{n\to\infty} x_n^{n+1}$. 注意到~$0\leqslant x_n^{n+1}\leqslant x_2^{n+1}\to 0\ (n\to\infty)$, 故~$\xi=1/2$, 即~$\lim\limits_{n\to\infty} x_n=1/2$.\qedhere
	      \end{proof}
\end{enumerate}

\section{Cauchy~命题与~Stolz~定理}
\subsection{思考题 pp.35.}
若在这三个命题的条件中将极限值~$l$~改为不带符号的无穷大量~$\infty$, 则结论不成立. 请读者举出反例.

\subsection{练习题 pp.37.}
\begin{enumerate}
	\item 设~$\lim\limits_{n\to\infty} x_n=+\infty$, 证明: $\lim\limits_{n\to\infty} \dfrac{x_1+x_2+\cdots+x_n}{n}=+\infty$.
	      \begin{proof}
		      对于~$\forall M>0$, 由~$\lim\limits_{n\to\infty} x_n=+\infty$, 知存在~$N_1\in\mathbf{N}_{+}$~使当~$n>N_1$~时有~$x_n>3M$, 于是
		      \begin{equation*}
			      \begin{split}
				      \dfrac{x_1+x_2+\cdots+x_n}{n}=&\dfrac{x_1+x_2+\cdots+x_{N_1}}{n}+\dfrac{x_{N_1+1}+x_{N_1+2}+\cdots+x_n}{n}\\
				      >&\dfrac{x_1+x_2+\cdots+x_{N_1}}{n}+\dfrac{n-N_1}{n}\cdot 3M
			      \end{split}
		      \end{equation*}
		      由于~$\dfrac{x_1+x_2+\cdots+x_{N_1}}{n}\to 0, \dfrac{n-N_1}{n}\to 1\ (n\to\infty)$ 故存在~$N_2\mathbf{N}_{+}$~使当~$n>N_2$~时, $\dfrac{x_1+x_2+\cdots+x_{N_1}}{n}>-M/2$~且~$\dfrac{n-N_1}{n}>1/2$. 于是当~$n>N=\max\{N_1,N_2\}$~时,
		      \[
			      \dfrac{x_1+x_2+\cdots+x_n}{n}>3M/2-M/2=M.
		      \]
		      即~$\lim\limits_{n\to\infty} \dfrac{x_1+x_2+\cdots+x_n}{n}=+\infty$.\qedhere
	      \end{proof}
	\item 设~$\{x_n\}$~单调增加, $\lim\limits_{n\to\infty} \dfrac{x_1+x_2+\cdots+x_n}{n}=a$, 证明: $\{x_n\}$~收敛于~$a$.
	      \begin{proof}
		      断言~$\forall n\in\mathbf{N}_{+}, x_n\leqslant a$. 否则存在~$n_0\mathbf{N}_{+}$~使得~$x_{n_0}>a$, 不妨设~$x_{n_0-1}\leqslant a<x_{n_0}$, 则
		      \begin{equation*}
			      \begin{split}
				      \dfrac{x_1+x_2+\cdots+x_n}{n}-a=&\dfrac{(x_1-a)+(x_2-a)+(x_n-a)}{n}\\
				      \geqslant& \dfrac{(x_1-a)+(x_2-a)+\cdots+(x_{n_0-1}-a)}{n}+\dfrac{(n-n_0+1)(x_{n_0}-a)}{n}\\
				      =& x_{n_0}-a+\dfrac{(x_1-a)+(x_2-a)+\cdots+(x_{n_0-1}-a)-(n_0-1)(x_{n_0}-a)}{n}\\
				      \to& x_{n_0}-a>0\ (n\to\infty)
			      \end{split}
		      \end{equation*}
		      与~$\lim\limits_{n\to\infty} \dfrac{x_1+x_2+\cdots+x_n}{n}=0$~矛盾. 故~$\dfrac{x_1+x_2+\cdots+x_n}{n}\leqslant x_n\leqslant a, \forall n\in\mathbf{N}_{+}$, 由夹逼定理知~$\lim\limits_{n\to\infty} x_n=a$.\qedhere
	      \end{proof}
	\item 设~$\{a_{2k-1}\}$~收敛于~$a$, $\{a_{2k-1}\}$~收敛于~$b$, 其~$a\neq b$, 求~$\lim\limits_{n\to\infty} \dfrac{a_1+a_2+\cdots+a_n}{n}$.\\
	      (注意: 虽然数列~$\{a_n\}$~发散, 但前~$n$~项的算术平均值所组成的数列仍可以有极限.\footnote{这里可以和级数的~Ces\`aro~求和结合起来看.} 一个典型例子就是~$\{(-1)^n\}$.)
	      \begin{proof}
		      记~$y_n=\dfrac{a_1+a_2+\cdots+a_n}{n}$, 则由~Cauchy~命题, 有
		      \begin{equation*}
			      \begin{split}
				      y_{2n}&=\dfrac{1}{2}\left(\dfrac{a_1+a_3+\cdots+a_{2n-1}}{n}+\dfrac{a_2+a_4+\cdots+a_2n}{n}\right)\to\dfrac{a+b}{2},\\
				      y_{2n+1}&=\dfrac{n+1}{2n+1}\left(\dfrac{a_1+a_3+\cdots+a_{2n+1}}{n+1}+\dfrac{n}{n+1}\dfrac{a_2+a_4+\cdots+a_{2n}}{n}\right)\to\dfrac{a+b}{2},\ n\to\infty.
			      \end{split}
		      \end{equation*}
		      即~$\lim\limits_{n\to\infty} y_{2n}=\lim\limits_{n\to\infty} y_{2n+1}=\dfrac{a+b}{2}$, 由~pp.25. 1.~知~$\lim\limits_{n\to\infty} {a_1+a_2+\cdots+a_n}{n}=\dfrac{a+b}{2}$.\qedhere
	      \end{proof}
	\item 若~$\lim\limits_{n\to\infty} (a_n-a_{n-1})=d$, 证明: $\lim\limits_{n\to\infty} \dfrac{a_n}{n}=d$.\\
	      (本题可以说是~Cauchy~命题的另一种形式, 也很有用.)
	      \begin{proof}
		      定义\footnote{这里定义的合理性在于任意改变数列的有限项, 数列的敛散性不变, 并且若其收敛, 其极限值不变.}~$a_0=0$, 记~$y_n=a_n-a_{n-1}, n=1,2,\cdots$, 则由~Cauchy~命题可知~$\lim\limits_{n\to\infty} \dfrac{y_1+y_2+\cdots+y_n}{n}=d$, 即~$\lim\limits_{n\to\infty} \dfrac{a_n}{n}=d$.\qedhere
	      \end{proof}
	\item 设~$\{a_n\}$~为正数列, 且收敛于~$A$, 证明: $\lim\limits_{n\to\infty} (a_1a_2\cdots a_n)^\frac{1}{n}=A$.\\
	      (本题与~Cauchy~命题的关系是明显的.)
	      \begin{proof}
		      由基本不等式知,
		      \[
			      \dfrac{n}{\frac{1}{a_1}+\frac{1}{a_2}+\cdots+\frac{1}{a_n}}\leqslant(a_1a_2\cdots a_n)^{\frac{1}{n}}\leqslant\dfrac{a_1+a_2+\cdots+a_n}{n}, \forall n\in\mathbf{N}_{+}.
		      \]
		      若~$A=0$, 则~$0\leqslant(a_1a_2\cdots a_n)^{\frac{1}{n}}\leqslant\dfrac{a_1+a_2+\cdots+a_n}{n}$, 由~Cauchy~命题知, $\lim\limits_{n\to\infty} \dfrac{a_1+a_2+\cdots+a_n}{n}=0$. 故由夹逼定理知~$\lim\limits_{n\to\infty} (a_1a_2\cdots a_n)^\frac{1}{n}=0$.

		      若~$A>0$, 则~$\lim\limits_{n\to\infty} \dfrac{1}{a_n}=\dfrac{1}{A}$, 由~Cauchy~命题知
		      \[
			      \lim\limits_{n\to\infty} \dfrac{n}{\frac{1}{a_1}+\frac{1}{a_2}+\cdots+\frac{1}{a_n}}=\lim\limits_{n\to\infty} \dfrac{1}{\dfrac{\frac{1}{a_1}+\frac{1}{a_2}+\cdots+\frac{1}{a_n}}{n}}=\dfrac{1}{\frac{1}{A}}=A,
		      \]
		      \[
			      \lim\limits_{n\to\infty} \dfrac{a_1+a_2+\cdots+a_n}{n}=A.
		      \]
		      故由夹逼定理知, $\lim\limits_{n\to\infty} (a_1a_2\cdots a_n)^\frac{1}{n}=A$.\qedhere
	      \end{proof}
	\item 设~$\{a_n\}$~为正数列, 且存在极限~$\lim\limits_{n\to\infty} \dfrac{a_{n+1}}{a_n}=l$, 证明: $\lim\limits_{n\to\infty} \sqrt[n]{a_n}=l$.\\
	      (本题对类型为~$\{\sqrt[n]{a_n}\}$~的极限问题很有用, 可以说是例题~2.1.2~的一个发展. 这个结果在无穷级数的研究中也很重要.\footnote{参见正项级数的比值判别法(d'Alembert)和根值判别法(Cauchy), 我们有: 前者有效时后者一定有效, 但反之不成立, 如~$\sum_{n=1}^{\infty} 2^{-n-(-1)^n}$.})
	      \begin{proof}
		      定义~$a_0=0$, 则~$\sqrt[n]{a_n}=\sqrt[n]{\dfrac{a_n}{a_{n-1}}\dfrac{a_{n-1}}{a_{n-2}}\cdots\dfrac{a_1}{a_0}}, \forall n\in\mathbf{N}_{+}$. $\{a_n\}$~是正数列, 故~$\left\{\dfrac{a_n}{a_{n-a}}\right\}$~也是正数列. 由~5~可知
		      \[
			      \lim\limits_{n\to\infty} \sqrt[n]{a_n}=\lim\limits_{n\to\infty} \sqrt[n]{\dfrac{a_n}{a_{n-1}}\dfrac{a_{n-1}}{a_{n-2}}\cdots\dfrac{a_1}{a_0}}=\lim\limits_{n\to\infty} \dfrac{a_n}{a_{n-1}}=l.\qedhere
		      \]
	      \end{proof}
	\item 设~$\lim\limits_{n\to\infty} (x_n-x_{n-2})=0$, 证明: $\lim\limits_{n\to\infty} \dfrac{x_n}{n}=0$.
	      \begin{proof}
		      定义~$x_{-1}=x_0=0$, 并记~$a_n=x_n-x_{n-2}, n=1,2,\cdots$. 由~$\lim\limits_{n\to\infty} a_n=0$~知~$\lim\limits_{n\to\infty} a_{2n}=\lim\limits_{n\to\infty} a_{2n-1}=0$. 由~Cauchy~命题知,
		      \[
			      \lim\limits_{n\to\infty} \dfrac{x_{2n}}{2n}=\lim\limits_{n\to\infty} \dfrac{n}{2n}\dfrac{x_{2n}}{n}=\lim\limits_{n\to\infty} \dfrac{n}{2n}\dfrac{a_2+a_4+\cdots+a_{2n}}{n}=0.
		      \]
		      同理有
		      \[
			      \lim\limits_{n\to\infty} \dfrac{x_{2n-1}}{2n-1}=\lim\limits_{n\to\infty} \dfrac{n}{2n-1}\dfrac{x_{2n-1}}{n}=\lim\limits_{n\to\infty} \dfrac{n}{2n-1}\dfrac{a_1+a_3+\cdots+a_{2n-1}}{n}=0.
		      \]
		      故由~pp.25. 1.~知~$\lim\limits_{n\to\infty} \dfrac{x_n}{n}=0$.\qedhere
	      \end{proof}
	\item 设~$\lim\limits_{n\to\infty} (x_n-x_{n-2})=0$, 证明: $\lim\limits_{n\to\infty} \dfrac{x_n-x_{n-1}}{n}=0$.
	      \begin{proof}
		      定义~$x_{-1}=x_0=0$, 并记~$a_n=x_n-x_{n-2}, n=1,2,\cdots$. 由~Cauchy~命题知, $\lim\limits_{n\to\infty} \dfrac{x_n+x_{n-1}}{n}=\lim\limits_{n\to\infty} \dfrac{a_1+a_2+\cdots+a_n}{n}=\lim\limits_{n\to\infty} a_n=0$. 由~7~知, $\lim\limits_{n\to\infty} \dfrac{x_n}{n}=0$. 故
		      \[
			      \lim\limits_{n\to\infty} \dfrac{x_n-x_{n-1}}{n}=\lim\limits_{n\to\infty} \dfrac{x_n+x_{n+1}}{n}-\dfrac{2(n-1)}{n}\dfrac{x_{n-1}}{n-1}=0.\qedhere
		      \]
	      \end{proof}
	\item 设数列~$\{a_n\}$~满足条件~$0<a_1<1$~和~$a_{n+1}=a_n(1-a_n)$, 证明: $\lim\limits_{n\to\infty} na_n=1$.
	      \begin{proof}
		      由于~$0<a_1<1, a_{n+1}=a_n(1-a_n)$, 故
		      \[
			      0<a_2=a_1(1-a_1)\leqslant\left(\dfrac{a_1+(1-a_1)}{2}\right)^2=\dfrac{1}{4}<1,
		      \]
		      归纳地可以得到~$\forall n\in\mathbf{N}_{+}, 0<a_n<1$. 由~$\dfrac{a_{n+1}}{a_n}=1-a_n<1$, 知~$\{a_n\}$~单调减少有下界, 故其收敛. 设~$\lim\limits_{n\to\infty} a_n=a$, 在递推式~$a_{n+1}=a_n(1-a_n)$~两侧令~$n\to\infty$, 就有~$a=a(1-a)$, 故~$\lim\limits_{n\to\infty} a_n=0$.

		      又由~$a_{n+1}=a_n(1-a_n)$, 同时取倒数就有
		      \[
			      \dfrac{1}{a_{n+1}}=\dfrac{1}{a_n(1-a_n)}=\dfrac{1}{a_n}+\dfrac{1}{1-a_n},
		      \]
		      故~$\lim\limits_{n\to\infty} \dfrac{1}{a_{n+1}}-\dfrac{1}{a_n}=\lim\limits_{n\to\infty} \dfrac{1}{1-a_n}=1$. 由~Cauchy~命题可知~$\lim\limits_{n\to\infty} \dfrac{1}{n}\sum\limits_{k=1}^{n-1}\left(\dfrac{1}{a_{k+1}}-\dfrac{1}{a_k}\right)=1$, 即~$\lim\limits_{n\to\infty}\left(\dfrac{1}{na_n}-\dfrac{1}{na_1}\right)=1$. 于是就有~$\lim\limits_{n\to\infty} \dfrac{1}{na_n}=1$, 故~$\lim\limits_{n\to\infty} na_n=1$.\qedhere
	      \end{proof}
	\item 若~$\lim\limits_{n\to\infty} a_n=\alpha$, $\lim\limits_{n\to\infty} b_n=\beta$, 证明: $\lim\limits_{n\to\infty} \dfrac{a_1b_n+a_2b_{n-1}+\cdots+a_nb_1}{n}=\alpha\beta$.
	      \begin{proof}
		      \footnote{本题的证明方法可以用于一切类似~Teoplitz~定理的极限证明, 事实上~Teoplitz~定理也能类似给出证明.}
		      当~$\beta=0$~时, 由于~$\lim\limits_{n\to\infty} a_n=\alpha$, 故存在~$M>0$~使得~$|a_n|\leqslant M, n=1,2,\cdots$; 对于~$\forall\varepsilon>0$, 由~$\lim\limits_{n\to\infty} b_n=0$, $\exists N_1\in\mathbf{N}_{+}$~使得当~$n>N_1$~时, $|b_n|\leqslant\varepsilon/2M$, 于是
		      \[
			      \left|\dfrac{a_1b_n+a_2b_{n-1}+\cdots+a_nb_1}{n}\right|\leqslant M\cdot\dfrac{|b_1|+|b_2|+\cdots+|b_{N_1}|}{n}+M\cdot\dfrac{n-N_1}{n}\dfrac{\varepsilon}{2M}.
		      \]
		      对于常数~$M'=M\cdot(|b_1|+|b_2|+\cdots+|b_{N_1}|)$, 存在~$N_2\in\mathbf{N}_{+}$~使得当~$n>N_2$~时, $\dfrac{M'}{n}<\varepsilon/2$. 故当~$n>\max\{N_1,N_2\}$~时,
		      \[
			      \left|\dfrac{a_1b_n+a_2b_{n-1}+\cdots+a_nb_1}{n}\right||\leqslant M\cdot\dfrac{|b_1|+|b_2|+\cdots+|b_{N_1}|}{n}+M\cdot\dfrac{n-N_1}{n}\dfrac{\varepsilon}{2M}<\varepsilon/2+\varepsilon/2=\varepsilon.
		      \]
		      即~$\lim\limits_{n\to\infty} \dfrac{a_1b_n+a_2b_{n-1}+\cdots+a_nb_1}{n}=0$.

		      当~$\beta\neq 0$~时, 由~$\lim\limits_{n\to\infty} b_n=\beta$~知~$\lim\limits_{n\to\infty} (b_n-\beta)=0$. 故有
		      \[
			      \lim\limits_{n\to\infty} \dfrac{a_1(b_n-\beta)+a_2(b_{n-1}-\beta)+\cdots+a_n(b_1-\beta)}{n}=0.
		      \]
		      即
		      \begin{equation*}
			      \begin{split}
				      \lim\limits_{n\to\infty} \dfrac{a_1b_n+a_2b_{n-1}+\cdots+1_nb_1}{n}&= \lim\limits_{n\to\infty} \dfrac{a_1(b_n-\beta)+a_2(b_{n-1}-\beta)+\cdots+a_n(b_1-\beta)}{n}\\
				      &+\lim\limits_{n\to\infty} \beta\dfrac{a_1+a_2+\cdots+a_n}{n}\\
				      &=0+\beta\alpha=\alpha\beta.\qedhere
			      \end{split}
		      \end{equation*}
	      \end{proof}
\end{enumerate}

\section{自然对数的底~$\e$~和~Euler~常数~$\gamma$}
\subsection{练习题 pp.45.}
\begin{enumerate}
	\item 计算下列极限:
	      \begin{tabenum}[(1)]
		      \item $\lim\limits_{n\to\infty} \left(1-\dfrac{1}{n}\right)^n$;
		      \item $\lim\limits_{n\to\infty} \left(1+\dfrac{1}{2n}\right)^n$;\\
		      \item $\lim\limits_{n\to\infty} \left(1+\dfrac{2}{n}\right)^n$;
		      \item \footnote{原题为~$\lim\limits_{n\to\infty} \left(1+\dfrac{1}{n}\right)^{n^2}$, 显然也可以用本题的方法计算, 但结果为一个无穷大量.}$\lim\limits_{n\to\infty} \left(1-\dfrac{1}{n}\right)^{n^2}$;\\
		      \item $\lim\limits_{n\to\infty} \left(1+\dfrac{1}{n^2}\right)^n$;
		      \item $\lim\limits_{n\to\infty} \left(1+\dfrac{1}{n}+\dfrac{1}{n^2}\right)^n$.
	      \end{tabenum}
	      (在计算中可以应用~2.1.5~小节的题~5~中有关连续性的结果. 但是要请读者注意, 在现阶段如下的做法是缺乏依据的 (以题~(3)~为例):
	      \[
		      \lim\limits_{n\to\infty} \left(1+\dfrac{2}{n}\right)^n=\lim\limits_{n\to\infty} \left[\left(1+\dfrac{2}{n}\right)^{\frac{n}{2}}\right]^2=\e^2.)
	      \]
	      \begin{proof}
		      \hspace{1em}
		      \begin{enumerate}[(1)]
			      \item $\lim\limits_{n\to\infty} \left(\dfrac{n}{n-1}\right)^n=\lim\limits_{n\to\infty} \dfrac{n}{n-1}\left(1+\dfrac{1}{n-1}\right)^{n-1}=\e$, 故~$\lim\limits_{n\to\infty} \left(1-\dfrac{1}{n}\right)^n=\dfrac{1}{\e}$;
			      \item $\lim\limits_{n\to\infty} \left(1+\dfrac{1}{2n}\right)^n=\lim\limits_{n\to\infty} \sqrt{\left(1+\dfrac{1}{2n}\right)^{2n}}=\sqrt{\e}$;
			      \item $
				            \begin{aligned}[t]
					            \lim\limits_{n\to\infty} \left(1+\dfrac{2}{n}\right)^n= & \lim\limits_{n\to\infty} \left(1+\dfrac{1}{n+1}\right)^n\left(1+\dfrac{1}{n}\right)^n                           \\
					            =                                                       & \lim\limits_{n\to\infty} \dfrac{n+1}{n+2}\left(1+\dfrac{1}{n+1}\right)^{n+1}\left(1+\dfrac{1}{n}\right)^n=\e^2;
				            \end{aligned}
			            $
			      \item $\lim\limits_{n\to\infty} \left(1-\dfrac{1}{n}\right)^{n^2}=\lim\limits_{n\to\infty} \left(\dfrac{n}{n-1}\right)^{-n^2}=\lim\limits_{n\to\infty} \left(1+\dfrac{1}{n-1}\right)^{-(n-1)\frac{n^2}{n-1}}=0$;
			      \item $\lim\limits_{n\to\infty} \left(1+\dfrac{1}{n^2}\right)^n=\lim\limits_{n\to\infty} \left(1+\dfrac{1}{n^2}\right)^{n^2\cdot \frac{1}{n}}=\e^0=1$;
			      \item $\lim\limits_{n\to\infty} \left(1+\dfrac{1}{n}+\dfrac{1}{n^2}\right)^n=\lim\limits_{n\to\infty} \left(1+\dfrac{1}{n}+\dfrac{1}{n^2}\right)^{\frac{n^2}{n+1}\cdot\frac{n+1}{n}}=\e$.\qedhere
		      \end{enumerate}
	      \end{proof}

	\item 设~$x\in\mathbf{N}_{+}$, 证明: $\dfrac{k}{n+k}<\ln\left(1+\dfrac{k}{n}\right)<\dfrac{k}{n}$.
	      \begin{proof}
		      由~pp.38 {\bf 命题 2.5.1}~中的不等式~$\left(1+\dfrac{1}{n}\right)^n<\e<\left(1+\dfrac{1}{n}\right)^{n+1}$~两边取对数, 可以得到不等式
		      \[
			      \dfrac{1}{n+1}<\ln\left(1+\dfrac{1}{n}\right)<\dfrac{1}{n}.
		      \]
		      注意到
		      \[
			      \begin{split}
				      \ln\left(1+\dfrac{k}{n}\right)=&\ln\left(\dfrac{n+k}{n}\right)\\
				      =&\ln\left(\dfrac{n+k}{n+k-1}\cdot\dfrac{n+k-1}{n+k-1}\cdots\dfrac{n+1}{n}\right)\\
				      =&\ln\left(1+\dfrac{1}{n+k-1}\right)+\ln\left(1+\dfrac{1}{n+k-2}\right)+\cdots+\ln\left(1+\dfrac{1}{n}\right),
			      \end{split}
		      \]
		      就有
		      \[
			      \dfrac{k}{n+k}<\dfrac{1}{n+k}+\dfrac{1}{n+k-1}+\cdots+\dfrac{1}{n+1}<\ln\left(1+\dfrac{k}{n}\right)<\dfrac{1}{n+k-1}+\dfrac{1}{n+k-2}+\cdots+\dfrac{1}{n}<\dfrac{k}{n}.\qedhere
		      \]
	      \end{proof}
	\item 求~$\lim\limits_{n\to\infty} \left(1+\dfrac{1}{n^2}\right)\left(1+\dfrac{2}{n^2}\right)\cdots\left(1+\dfrac{n}{n^2}\right)$.
	      \begin{proof}
		      记~$x_n=\left(1+\dfrac{1}{n^2}\right)\left(1+\dfrac{2}{n^2}\right)\cdots\left(1+\dfrac{n}{n^2}\right), y_n=\ln{x_n}, n\in\mathbf{N}_{+}$. 则由上题有
		      \[
			      \begin{split}
				      y_n<&\dfrac{1}{n^2}+\dfrac{2}{n^2}+\cdots+\dfrac{n}{n^2}=\dfrac{n(n+1)}{2n^2}\to\dfrac{1}{2} \ (n\to\infty),\\
				      y_n>&\dfrac{1}{n^2+1}+\dfrac{2}{n^2+2}+\cdots+\dfrac{n}{n^2+n}>\dfrac{1}{n^2+n}+\dfrac{2}{n^2+n}+\cdots+\dfrac{n}{n^2+n}=\dfrac{1}{2}.
			      \end{split}
		      \]
		      故~$\lim\limits_{n\to\infty} y_n=\dfrac{1}{2}$, 即~$\lim\limits_{n\to\infty} x_n=\sqrt{\e}$.\qedhere
	      \end{proof}
	\item \footnote{本题的结果与~Heine~定理结合就给出了一个~$\lim\limits_{x\to+\infty} \left(1+\dfrac{1}{x}\right)^{x}=\e$~的证明.} 设~$\{p_n\}$~是正数列, 且~$p_n\to+\infty$. 计算~$\lim\limits_{n\to\infty} \left(1+\dfrac{1}{p_n}\right)^{p_n}$.
	      \begin{proof}
		      对于任意~$n\in\mathbf{N}_{+}$, 有~$[p_n]\leqslant p_n<[p_n]+1$, $\dfrac{1}{[p_n]+1}\leqslant\dfrac{1}{[p_n]}$, 因此
		      \[
			      \left(1+\dfrac{1}{[p_n]+1}\right)^{[p_n]}<\left(1+\dfrac{1}{p_n}\right)^{p_n}<\left(1+\dfrac{1}{[p_n]}\right)^{[p_n]+1}
		      \]

		      由于~$\lim\limits_{n\to\infty} \left(1+\dfrac{1}{n+1}\right)^n=\lim\limits_{n\to\infty} \left(1+\dfrac{1}{n}\right)^{n+1}=\e$, 即对于任意给定的~$\varepsilon>0$, 存在~$N\in\mathbf{N}_{+}$, 使得当~$n>N$~时有
		      \[
			      \left|\left(1+\dfrac{1}{n+1}\right)^n-\e\right|<\varepsilon \ \text{且}\ \left|\left(1+\dfrac{1}{n}\right)^{n+1}-\e\right|<\varepsilon.
		      \]
		      特别地, 当~$n>N$~时有
		      \[
			      \left(1+\dfrac{1}{n+1}\right)^n>\e-\varepsilon \ \text{且}\ \left(1+\dfrac{1}{n}\right)^{n+1}<\e+\varepsilon.
		      \]
		      由于~$\lim\limits_{n\to\infty} p_n=+\infty$, 对于~$N>0$, 存在~$M\in\mathbf{N}_{+}$~使得~$n>M$~时, $[p_n]>N$, 于是当~$n>M$~时, 就有
		      \[
			      \e-\varepsilon<\left(1+\dfrac{1}{[p_n]+1}\right)^{[p_n]}<\left(1+\dfrac{1}{p_n}\right)^{p_n}<\left(1+\dfrac{1}{[p_n]}\right)^{[p_n]+1}<\e+\varepsilon.
		      \]
		      即当~$n>M$~时~$\left|\left(1+\dfrac{1}{p_n}\right)^{p_n}-e\right|<\varepsilon$, 所以~$\lim\limits_{n\to\infty} \left(1+\dfrac{1}{p_n}\right)^{p_n}=\e$. \qedhere
	      \end{proof}
	\item 求~$\lim\limits_{n\to\infty} \dfrac{n!2^n}{n^n}$.\footnote{事实上, 对于通项带有~$a^n$~项的数列, 可以尽情利用~Cauchy~根值判别法或者~d'Alembert~比值判别法. 如果记~$a_n=\dfrac{n!2^n}{n^n}$, 则~$\lim\limits_{n\to\infty} \sqrt[n]{a_n}=\lim\limits_{n\to\infty} \dfrac{2\sqrt[n]{n!}}{n}=\dfrac{2}{\e}<1$, 故~$\sum\limits_{n=1}^{\infty}a_n<+\infty$, 从而~$\lim\limits_{n\to\infty} a_n=0$.}
	      \begin{proof}

	      \end{proof}
	\item 求极限~$\lim\limits_{n\to\infty} \dfrac{\ln{n}}{1+\dfrac{1}{2}+\cdots+\dfrac{1}{n}}$.
	      \begin{proof}
		      由~Stolz~定理, 有
		      \[
			      \lim\limits_{n\to\infty} \dfrac{\ln{n}}{1+\dfrac{1}{2}+\cdots+\dfrac{1}{n}}=\lim\limits_{n\to\infty} \dfrac{\ln{n+1}-\ln{n}}{\dfrac{1}{n+1}}=\lim\limits_{n\to\infty} \ln\left(1+\dfrac{1}{n}\right)^{n+1}=\e.
		      \]
	      \end{proof}

	\item 证明: $\left(\dfrac{n+1}{\e}\right)^n<n!<\e\left(\dfrac{n+1}{\e}\right)^{n+1}$.\\
	      (\footnote{只需应用本题及夹逼准则即可.}由此又可以得到~$\lim\limits_{n\to\infty} \dfrac{n}{\sqrt[n]{n!}}=\e$.)
	      \begin{proof}
		      数学归纳法.
		      \begin{enumerate}[(1)]
			      \item $n=1$~时, 由于~$2<\e<4$, 故有~$\dfrac{2}{\e}<1<\dfrac{4}{\e}=\e\cdot\left(\dfrac{2}{\e}\right)^2$, 即~$n=1$~时成立.
			      \item 假设对于~$n$~时成立, 即
			            \[
				            \left(\dfrac{n+1}{\e}\right)^n<n!<\e\left(\dfrac{n+1}{\e}\right)^{n+1}.
			            \]
			            则对于~$n+1$~时,
			            \[
				            \begin{split}
					            (n+1)!&<(n+1)\cdot\e\left(\dfrac{n+1}{\e}\right)^{n+1}
					            =\e^2\left(\dfrac{n+2}{\e}\right)^{n+2}\left(\dfrac{n+1}{n+2}\right)^{n+2}<\e\left(\dfrac{n+2}{\e}\right)^{n+2},\\
					            &>(n+1)\cdot\left(\dfrac{n+1}{\e}\right)^{n}
					            =\e\left(\dfrac{n+2}{\e}\right)^{n+1}\left(\dfrac{n+1}{n+2}\right)^{n+1}>\left(\dfrac{n+2}{\e}\right)^{n+1}.
				            \end{split}
			            \]
			            这里用到了
			            \[
				            \left(\dfrac{n+2}{n+1}\right)^{n+1}<\e<\left(\dfrac{n+2}{n+1}\right)^{n+2}.
			            \]
		      \end{enumerate}
		      由数学归纳法可知
		      \[
			      \left(\dfrac{n+1}{\e}\right)^n<n!<\e\left(\dfrac{n+1}{\e}\right)^{n+1}.
		      \]
		      对于~$n\in\mathbf{N}_{+}$~恒成立.\qedhere
	      \end{proof}

	\item 设~$S_n=1+2^2+3^3+\cdots+n^n, n\in\mathbf{N}_{+}$. 证明: 对~$n\geqslant 2$~成立不等式
	      \[
		      n^n\left(1+\dfrac{1}{4(n-1)}\right)\leqslant S_n\leqslant n^n\left(1+\dfrac{2}{\e(n-1)}\right).
	      \]
	      \begin{proof}
		      数学归纳法.
		      \begin{enumerate}[(1)]
			      \item $n=2$~时, $2^2\left(1+\dfrac{1}{4(2-1)}\right)=5=S_2<2^2\left(1+\dfrac{2}{\e(2-1)}\right)$, 即~$n=2$~时成立.
			      \item 假设对于~$n$~时成立, 即
			            \[
				            n^n\left(1+\dfrac{1}{4(n-1)}\right)\leqslant S_n\leqslant n^n\left(1+\dfrac{2}{\e(n-1)}\right).
			            \]
			            则对于~$n+1$~时,
			            \[
				            \begin{split}
					            S_{n+1}=S_n+(n+1)^{n+1}<&n^n\left(1+\dfrac{2}{e(n-1)}\right)+(n+1)^{n+1}\\
					            =&(n+1)^{n+1}\left(1+\dfrac{n^n}{(n+1)^{n+1}}\left(1+\dfrac{2}{\e(n-1)}\right)\right)\\
					            <&(n+1)^{n+1}\left(1+\dfrac{2}{n}\left(\dfrac{n}{n+1}\right)^{n+1}\right)\\
					            <&(n+1)^{n+1}\left(1+\dfrac{2}{\e n}\right),\\
					            S_{n+1}=S_n+(n+1)^{n+1}>&n^n\left(1+\dfrac{1}{4(n-1)}\right)+(n+1)^{n+1}\\
					            =&(n+1)^{n+1}\left(1+\dfrac{n^n}{(n+1)^{n+1}}\left(1+\dfrac{1}{4(n-1)}\right)\right)\\
					            \geqslant&(n+1)^{n+1}\left(1+\dfrac{1}{n}\left(\dfrac{n}{n+1}\right)^{n+1}\right)\\
					            \geqslant&(n+1)^{n+1}\left(1+\dfrac{1}{4n}\right).
				            \end{split}
			            \]
			            这里用到了
			            \[
				            \e<\left(\dfrac{n}{n+1}\right)^{n+1}\leqslant 4.
			            \]
		      \end{enumerate}
		      由数学归纳法可知
		      \[
			      n^n\left(1+\dfrac{1}{4(n-1)}\right)\leqslant S_n\leqslant n^n\left(1+\dfrac{2}{\e(n-1)}\right)
		      \]
		      对于~$n\geqslant 2$~恒成立.\qedhere
	      \end{proof}
	\item 设有~$a_1=1,a_n=n(a_{n-1}+1), n=2,3,\cdots$, 又设~$x_n=\prod\limits_{k=1}^n\left(1+\dfrac{1}{a_k}\right), n\in\mathbf{N}_{+}$, 求数列~$\{x_n\}$~的极限.
	      \begin{proof}
		      对于~$n\in\mathbf{N}_{+}$, 有
		      \[
			      \begin{split}
				      x_n=& \prod_{k=1}^n \left(1+\dfrac{1}{a_k}\right)=\dfrac{a_1+1}{a_1}\cdot\dfrac{a_2+1}{a_2}\cdot\cdots\cdot\dfrac{a_n+1}{a_n}\\
				      =&\dfrac{a_2}{2a_1}\cdot\dfrac{a_3}{3a_2}\cdot\cdots\cdot\dfrac{a_{n+1}}{(n+1)a_n}\\
				      =&\dfrac{a_{n+1}}{(n+1)!}\\
				      =&\dfrac{(n+1)(a_n+1)}{(n+1)!}\\
				      =&\dfrac{1}{n!}+\dfrac{a_n}{n!}\\
				      \cdots&\\
				      =&\dfrac{1}{n!}+\cdots+\dfrac{1}{2!}+\dfrac{a_2}{2!}\\
				      =&\dfrac{1}{n!}+\cdots+\dfrac{1}{2!}+\dfrac{2(a_1+1)}{2!}\\
				      =&\dfrac{1}{n!}+\cdots+\dfrac{1}{2!}+1+1.
			      \end{split}
		      \]
		      即~$x_n=\sum\limits_{k=0}^{n}\dfrac{1}{k!}, n\in\mathbf{N}_{+}$. 故~$\lim\limits_{n\to\infty} x_n=\e$.\qedhere
	      \end{proof}
\end{enumerate}

\section{由迭代生成的数列}
\subsection{练习题 pp.52}
在以下各题中均可使用几何方法, 或做出几何解释.
\begin{enumerate}
	\item
	      \begin{enumerate}[(1)]
		      \item 设~$a>0, x_1=\sqrt{a}, x_{n+1}=\sqrt{a+x_n}, n\in\mathbf{N}_{+}$, 求~$\lim\limits_{n\to\infty} x_n$;
		      \item 设~$a>0, x_1=\sqrt{a}, x_{n+1}=\sqrt{ax_n}, n\in\mathbf{N}_{+}$, 求~$\lim\limits_{n\to\infty} x_n$.
	      \end{enumerate}
	      (这两题外形相似, 都可用本节方法解决. 但题~(2)~有更简单的直接解法.)

	\item 设~$A>0, 0<b_0<A^{-1}, b_{n+1}=b_n(2-Ab_n),n\in\mathbf{N}_{+}$. 证明~$\lim\limits_{n\to\infty} b_n=A^{-1}$.
\end{enumerate}

\section{对于教学的建议}
\subsection{第一组参考题}
\begin{enumerate}
	\item 设~$\{a_{2k-1}\}, \{a_{2k}\}, \{a_{3k}\}$~都收敛, 证明: $\{a_n\}$~收敛.
	      \begin{proof}
		      取子数列~$\{a_{3k}\}$~奇数项和偶数项所排成的子列~$\{a_{6k-3}\}$~和~$\{a_{6k}\}$, 它们均为同一收敛数列的子列, 故均收敛且极限相等. 注意到, $\{a_{6k-3}\}$~也是收敛数列~$\{a_{2k-1}\}$~的一个子列, $\{a_{6k}\}$~也是收敛数列~$\{a_{2k}\}$~的一个子列, 从而
		      \[
			      \lim\limits_{k\to\infty} a_{2k-1}=\lim\limits_{k\to\infty} a_{6k-3}=\lim\limits_{k\to\infty} a_{6k}=\lim\limits_{k\to\infty} a_{2k}.
		      \]
		      即~$\{a_{2k-1}\}, \{a_{2k}\}$~收敛到相同的极限, 故由~pp.25 1~知~$\{a_n\}$~收敛. \qedhere
	      \end{proof}

	\item 设~$\{a_n\}$~有界, 且满足条件~$a_n\leqslant a_{n+2}, a_n\leqslant a_{n+3}, n\in\mathbf{N}_{+}$, 证明: $\{a_n\}$~收敛.
	      \begin{proof}
		      由题意, 有
		      \[
			      a_1\leqslant a_3\leqslant\cdots\leqslant a_{2k-1}\leqslant\cdots;
		      \]
		      \[
			      a_2\leqslant a_4\leqslant\cdots\leqslant a_{2k}\leqslant\cdots;
		      \]
		      \[
			      a_3\leqslant a_6\leqslant\cdots\leqslant a_{3k}\leqslant\cdots.
		      \]
		      即数列~$\{a_{2k-1}\}, \{a_{2k}\}, \{a_{3k}\}$~均单调, 注意到~$\{a_n\}$~有界, 故均收敛. 由~1.~知~$\{a_n\}$~收敛.\qedhere
	      \end{proof}

	\item 设~$\{a_n+a_{n+1}\}$~和~$\{a_n+a_{n+2}\}$~都收敛, 证明: $\{a_n\}$~收敛.
	      \begin{proof}
		      令~$\lim\limits_{n\to\infty} (a_n+a_{n+1})=A$, $\lim\limits_{n\to\infty} (a_n+a_{n+2})=B$. 则由极限的四则运算有,
		      \[
			      \lim\limits_{n\to\infty} (a_{n+1}-a_{n})=\lim\limits_{n\to\infty} (a_{n+2}-a_{n+1})=B-A,
		      \]
		      \[
			      \lim\limits_{n\to\infty} a_n=\lim\limits_{n\to\infty} a_{n+1}=\dfrac{B-A+A}{2}=\dfrac{B}{2}.\qedhere
		      \]
	      \end{proof}

	\item 设数列~$\{a_n\}$~收敛于~$0$, 有存在极限~$\lim\limits_{n\to\infty} \left|\dfrac{a_{n+1}}{a_n}\right|=a$. 证明: $a\leqslant 1$.
	      \begin{proof}
		      反证法. 若~$a>1$, 则对于~$\varepsilon=\dfrac{a-1}{2}>0$, 由于~$\lim\limits_{n\to\infty} \left|\dfrac{a_{n+1}}{a_n}\right|=a$, 则存在~$N_1\in\mathbf{N}_{+}$~使当~$n>N_1$~时有
		      \[
			      \left|\dfrac{a_{n+1}}{a_n}\right|>a-\varepsilon=\dfrac{1+a}{2}>1.
		      \]
		      要使极限~$\lim\limits_{n\to\infty} \left|\dfrac{a_{n+1}}{a_n}\right|$~存在, 则对于~$N_1>0$, 存在~$N>N_1, N\in\mathbf{N}_{+}$~使得~$a_{N}\neq 0$. 因此,
		      \[
			      |a_n|=\left|a_N\cdot\dfrac{a_{N+1}}{a_{N}}\cdots\dfrac{a_n}{a_{n-1}}\right|>|a_N|\cdot\left(\dfrac{1+a}{2}\right)^{n-N}\to+\infty.
		      \]
		      与~$\lim\limits_{n\to\infty} a_n=0$~矛盾. \qedhere
	      \end{proof}

	\item $a_n=\displaystyle\sum\limits_{k=1}^n\left(\sqrt{1+\dfrac{k}{n^2}}-1\right), n\in\mathbf{N}_{+}$. 计算~$\lim\limits_{n\to\infty} a_n$.
	      \begin{proof}
		      对~$k=1,\cdots, n$, 有~$\sqrt{1+\dfrac{k}{n^2}}-1=\dfrac{k}{n^2\left(\sqrt{1+\dfrac{k}{n^2}}+1\right)}$, 并且
		      \[
			      \sum_{k=1}^{n} \dfrac{k}{n^2\left(\sqrt{1+\dfrac{n}{n^2}}+1\right)}\leqslant \sum_{k=1}^{n} \dfrac{k}{n^2\left(\sqrt{1+\dfrac{k}{n^2}}+1\right)}\leqslant \sum_{k=1}^{n} \dfrac{k}{n^2\left(\sqrt{1+\dfrac{1}{n^2}}+1\right)}.
		      \]
		      注意当~$n\to\infty$~时,
		      \[
			      \sum_{k=1}^{n} \dfrac{k}{n^2\left(\sqrt{1+\dfrac{n}{n^2}}+1\right)}=\dfrac{n(n+1)}{2n^2\left(\sqrt{1+\dfrac{1}{n}}+1\right)}\to\dfrac{1}{4};
		      \]
		      \[
			      \sum_{k=1}^{n} \dfrac{k}{n^2\left(\sqrt{1+\dfrac{1}{n^2}}+1\right)}=\dfrac{n(n+1)}{2n^2\left(\sqrt{1+\dfrac{1}{n^2}}+1\right)}\to\dfrac{1}{4}.
		      \]
		      故由夹逼准则知~$\lim\limits_{n\to\infty} a_n=\dfrac{1}{4}$. \qedhere
	      \end{proof}

	\item 用~$p(n)$~表示能整除~$n$~的素数的个数, 证明: $\lim\limits_{n\to\infty} \dfrac{p(n)}{n}=0$.
	      \begin{proof}
		      由于~$\forall p$~是素数, $p\geqslant 2$, 对于~$n$~的素因子分解~$n=p_1^{r(1)}p_2^{r(2)}\cdots p_{p(n)}^{r(p(n))}$, 显然有~$n\geqslant 2^{p(n)}$, 故
		      \[
			      0\leqslant\dfrac{p(n)}{n}\leqslant\dfrac{\ln{n}}{n\ln{2}}\to 0\ (n\to\infty).
		      \]
		      故由夹逼准则知~$\lim\limits_{n\to\infty} \dfrac{p(n)}{n}=0$.\qedhere
	      \end{proof}

	\item 设~$a_0, a_1, \cdots ,a_p$~是~$p+1$~个给定的数, 且满足条件~$a_0+a_1+\cdots+a_p=0$. 求~$\lim\limits_{n\to\infty} (a_0\sqrt{n}+a_1\sqrt{n+1}+\cdots+a_p\sqrt{n+p})$.
	      \begin{proof}
		      由~$a_0+a_1+\cdots+a_p=0$~知~$a_0=-(a_1+\cdots+a_p)$, 故
		      \[
			      \begin{split}
				      a_0\sqrt{n}+a_1\sqrt{n+1}+\cdots+a_p\sqrt{n+p}=&a_1(\sqrt{n+1}-\sqrt{n})+\cdots+a_p(\sqrt{n+p}-{n})\\
				      =&a_1\dfrac{1}{\sqrt{n+1}+\sqrt{n}}+\cdots+a_p+\dfrac{p}{\sqrt{n+p}+\sqrt{n}}\\
				      \to&0 \ (n\to\infty).
			      \end{split}
		      \]
	      \end{proof}

	\item 证明: 当~$0<k<1$~时, $\lim\limits_{n\to\infty} [(1+n)^k-n^k]=0$.

	\item
	      \begin{enumerate}[(1)]
		      \item 设~$\{x_n\}$~收敛. 令~$y_n=n(x_n-x_{n-1}), n\in\mathbf{N}_{+}$, 问~$\{y_n\}$~是否收敛?
		      \item 在上一小题中, 若~$\{y_n\}$~也收敛, 证明: $\{y_n\}$~收敛于零.
	      \end{enumerate}

	\item
	      \begin{enumerate}[(1)]
		      \item 设正数列~$\{a_n\}$~满足条件~$\lim\limits_{n\to\infty} \dfrac{a_n}{a_{n+1}}=0$, 证明: $\{a_n\}$~是无穷大量.
		      \item 设正数列~$\{a_n\}$~满足条件~$\lim\limits_{n\to\infty} \dfrac{a_n}{a_{n+1}+a_{n+2}}=0$, 证明: $\{a_n\}$~无界.
	      \end{enumerate}

	\item 证明: $\left(\dfrac{n}{3}\right)^n<n!<\left(\dfrac{n}{2}\right)^n$, 其中右边的不等式当~$n\geqslant 6$~时成立.

	\item 证明: $\left(\dfrac{n}{\e}\right)^n<n!<\e\left(\dfrac{n}{2}\right)^n$.

	\item (对于命题~2.5.4~的改进) 证明:
	      \begin{enumerate}[(1)]
		      \item  当~$n\geqslant 2$~时成立
		            \[
			            1+1+\dfrac{1}{2!}+\cdots+\dfrac{1}{n!}+\dfrac{1}{n!n}=3-\dfrac{1}{2!1\cdot 2}-cdots-\dfrac{1}{n!(n-1)n};
		            \]
		      \item $\e=3-\lim\limits_{n\to\infty} \displaystyle\sum_{k=0}^n\dfrac{1}{(k+2)!(k+1)(k+2)}$;
		      \item 用~$\displaystyle \sum\limits_{k=0}^n\dfrac{1}{k!}+\dfrac{1}{n!n}$~计算~$\e$~要比不加上最后一项好得多.
	      \end{enumerate}

	\item 设~$a_n=1+\dfrac{1}{\sqrt{2}}+\cdots+\dfrac{1}{\sqrt{n}}-2\sqrt{n}, n\in\mathbf{N}_{+}$, 证明: $\{a_n\}$~收敛.

	\item 设已知存在极限~$\lim\limits_{n\to\infty} \dfrac{a_1+a_2+\cdots+a_n}{n}$, 证明: $\lim\limits_{n\to\infty} \dfrac{a_n}{n}=0$.

	\item 证明: $\lim\limits_{n\to\infty} (n!)^{1/n^2}=1$.

	\item 设对每个~$n$~有~$x_n<1$~和~$(1-x_n)x_{n+1}\geqslant \dfrac{1}{4}$, 证明~$\{x_n\}$~收敛, 并求其极限.

	\item 设~$a_1=b, a_2=c$, 在~$n\geqslant 3$~时~$a_n$~由~$a_n=\dfrac{a_{n-1}+a_{n-2}}{2}$~定义. 求~$\lim\limits_{n\to\infty} a_n$.

	\item 设~$a,b,c$~是三个给定的实数. 令~$a_1=a, b_1=b, c_1=c$, 并以递推公式定义
	      \[
		      a_{n+1}=\dfrac{b_n+c_n}{2}, b_{n+1}=\dfrac{c_n+a_n}{2}, c_n=\dfrac{a_n+b_n}{2}, n\in\mathbf{N}_{+}.
	      \]
	      求这三个数列的极限.

	\item
	      \begin{enumerate}[(1)]
		      \item 设~$a_1>b_1>0, a_{n+1}=\dfrac{2a_nb_n}{a_n+b_n}, b_{n+1}=\sqrt{a_{n+1}b_n}, n\in\mathbf{N}_{+}$, 证明: $\{a_n\}$~和~$\{b_n\}$~收敛于同一极限.
		      \item 在~$a_1=2\sqrt{3}, b_1=3$~时, 证明上述极限等于单位圆的半周长~$\pi$. (这里可以利用极限~$\lim\limits_{n\to\infty} n\sin{\dfrac{\pi}{n}}=\pi$.)
	      \end{enumerate}
\end{enumerate}
\subsection{第二组参考题}
\begin{enumerate}
	\item 设~$a_n=\sqrt{1+\sqrt{2+\cdots+\sqrt{n}}}, n\in\mathbf{N}_{+}$, 证明: $\{a_n\}$~收敛.

	\item 证明: 对于每个自然数~$n$~成立不等式~$\left(1+\dfrac{1}{n}\right)^n>\sum\limits_{k=0}^n\dfrac{1}{k!}-\dfrac{\e}{2n}$.
\end{enumerate}

\chapter{实数系的基本定理}
\section{确界的概念和确界存在定理}
\subsection{练习题 pp.69.}
\begin{enumerate}
	\item 试证明确界的唯一性.

	\item 设对每个~$x\in A$~成立~$x<a$. 问: 在~$\sup{A}<a$~和~$\sup{A}\leqslant a$~中哪个是对的?

	\item 设数集~$A$~以~$\beta$~为上界, 又有数列~$\{x_n\}\subset A$~和~$\lim\limits_{n\to\infty} x_n=\beta$. 证明: $\beta=\sup{A}$.

	\item 求下列数集的上确界和下确界:
	      \begin{tabenum}[(1)]
		      \tabenumitem $\{x\in\mathbf{Q}|x>0\}$;
		      \tabenumitem $\{y|y=x^2,x\in(-\dfrac{1}{2},1)\}$;\\
		      \tabenumitem $\left\{\left(1+\dfrac{1}{n}\right)^n | n\in\mathbf{N}_{+}\right\}$;
		      \tabenumitem $\{n\e^{-n}|n\in\mathbf{N}_{+}\}$;\\
		      \tabenumitem $\{\arctan{x}|x\in(-\infty,\infty)\}$;
		      \tabenumitem $\{(-1)^n+\dfrac{1}{n}(-1)^{n+1}|n\in\mathbf{N}_{+}\}$;\\
		      \tabenumitem $\{1+n\sin{\dfrac{n\pi}{2}}|n\in\mathbf{N}_{+}\}$.
	      \end{tabenum}

	\item 证明:
	      \begin{enumerate}[(1)]
		      \item $\sup\{x_n+y_n\}\leqslant\sup\{x_n\}+\sup\{y_n\}$;
		      \item $\int\{x_n+y_n\}\geqslant\inf\{x_n\}+\inf\{x_n\}$.
	      \end{enumerate}

	\item 设有两个数集~$A$~和~$B$, 且对数集~$A$~中的任何一个数~$x$~和数集中的任何一个数~$y$~成立不等式~$x\leqslant y$. 证明: $\sup{A}\leqslant\inf{B}$.

	\item 设数集~$A$~有上界, 数集~$B=\{x+c|x\in A\}$, 其中~$c$~是一个常数. 证明:
	      \[
		      \sup{B}=\sup{A}+c, \inf{B}=\inf{A}+c.
	      \]

	\item 设~$A,B$~是两个有上界的数集, 又有数集~$C\subset\{x+y|x\in A, y\in B\}$, 则~$\sup{C}\leqslant\sup{A}+\sup{B}$. 举出严格成立不等号的例子.

	\item 设~$A,B$~是两个有上界的数集, 又有数集~$C\supset\{x+y|x\in A, y\in B\}$, 则~$\sup{C}\geqslant\sup{A}+\sup{B}$. 举出严格成立不等号的例子.\\

	      (合并以上两题可见, 当且仅当~$C=\{x+y|x\in A, y\in B\}$~时成立~$\sup{C}=\sup{A}+\sup{B}$.)
\end{enumerate}

\section{闭区间套定理}

\section{凝聚定理}

\section{Cauchy~收敛准则}

\section{覆盖定理}

\section{数列的上极限和下极限}

\section{对于教学的建议}

\end{document}
