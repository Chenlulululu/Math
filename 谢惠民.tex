%Latex utf-8
\documentclass[UTF8,a4paper,10pt,twoside]{book}
\usepackage{CTEX}
\usepackage[top=1in, bottom=1in, left=1in, right=1in]{geometry}
\usepackage{tablists}
\usepackage{amsmath}
\usepackage{amsfonts}
\usepackage{amsthm}
\usepackage{amssymb}
\usepackage{bm}
\usepackage{extarrows}
\usepackage{enumerate}
\usepackage{titlesec}
\usepackage{graphicx}
\newcommand{\ud}[1]{\mathrm{d}#1}
\pagestyle{empty}
\begin{document}
\setlength\abovedisplayskip{2pt}
\setlength\belowdisplayskip{2pt}
\begin{titlepage}

	\begin{center}
		\vfill
		% Upper part of the page
		{\Huge \bfseries 数学分析习题课讲义}\\[1.5cm]
		% Title
		{\LARGE \bfseries 参考答案}\\[0.4cm]
		\vfill
	\end{center}

\end{titlepage}
\renewcommand{\proofname}{\bf 证明}

\setcounter{chapter}{1}
\chapter{数列极限}
\section{数列极限的基本概念}
\subsection{思考题}
\kaishu
\begin{enumerate}
	\item 数列收敛有很多等价定义. 例如:
	      \begin{enumerate}[(1)]
		      \item 数列~$\{a_n\}$~收敛于~$a$~$\Longleftrightarrow$~$\forall\varepsilon>0,\exists N\in\mathbf{N}_{+},\forall n\geqslant N$, 成立~$|a_n-a|<\varepsilon$;
		      \item 数列~$\{a_n\}$~收敛于~$a$~$\Longleftrightarrow$~$\forall m\in\mathbf{N}_{+},\exists N\in\mathbf{N}_{+},\forall n>N$, 成立~$|a_n-a|<1/m$;\footnote{有些像级数的~Weierstrass-M~判别法, 事实上也可以用~Cauchy~收敛准则给出一个和~Weierstrass-M~判别法类似的证明. 本条是所有二分法/三分法证明的基础.}
		      \item 数列~$\{a_n\}$~收敛于~$a$~$\Longleftrightarrow$~$\forall\varepsilon>0,\exists N\in\mathbf{N}_{+},\forall n>N$, 成立~$|a_n-a|<K\varepsilon$. 其中~$K$~是一个与~$\varepsilon$~和~$n$~无关的正常数.
	      \end{enumerate}
	      试证明以上定义与数列收敛等价.

	      \begin{proof}
		      \begin{enumerate}[(1)]
			      \item $\Rightarrow$ 取~$N=N_0+1$. $\Leftarrow$ 显然.
			      \item $\Rightarrow$ 取~$\varepsilon=1/m, m\in\mathbf{N}_{+}$. $\Leftarrow$ 由于~$\lim\limits_{m\to\infty} 1/m=0$, 故存在~$M\in\mathbf{N}_{+}$, 当~$m>M$~时, $1/m<\varepsilon$. 选定~$m$, 使用定义, 存在$N_0\in\mathbf{N}_{+}$, $\forall n>N$, 有~$|a_n-a|<1/m<\varepsilon$.
			      \item $\Rightarrow$ 取~$K=1$. $\Leftarrow$ 取~$\varepsilon'=\varepsilon/K$, 则~$\exists N\in\mathbf{N}_{+}, \forall n>N, |a_n-a|<K\varepsilon'=\varepsilon$.\qedhere
		      \end{enumerate}
	      \end{proof}

	\item 问: 在数列收敛的定义中, $N$~是否是~$\varepsilon$~的函数?
	      \begin{proof}[\bf 答]
		      否. 对于任意的~$\varepsilon$, 存在一个~$N_0\in\mathbf{N}_{+}$, 使得当~$n>N_0$~时都有~$|a_n-a|<\varepsilon$, 而~$\forall N>N_0$~都可以是符合定义的~$N$, 即每一个~$\varepsilon$~都可以对应无穷多个~$N$, 故不是.\qedhere
	      \end{proof}
	\item 判断: 若~$\{a_n\}$~收敛, 则有~$\lim\limits_{n\to\infty} (a_{n+1}-a_n)=0$~和~$\lim\limits_{n\to\infty} a_{n+1}/a_n=1$.
	      \begin{proof}[\bf 答]
		      $\lim\limits_{n\to\infty} (a_{n+1}-a_n)=0$. 对于任意给定的~$\varepsilon>0$, 存在~$N>0$, 当~$n>N$时有~$|a_n-a|<\varepsilon/2$, 从而~$|a_{n+1}-a|<\varepsilon/2$, 于是对于~$n>N$,
		      \[|a_{n+1}-a_n|\leqslant|a_{n+1}-a|+|a_n-a|<\varepsilon/2+\varepsilon/2=\varepsilon.\]

		      $\lim\limits_{n\to\infty} a_{n+1}/a_n=1$. 举一反例~$\{(-1)^n 1/n\}$, 显然~$\lim\limits_{n\to\infty} (-1)^n1/n=0$, 但
		      \[\lim\limits_{n\to\infty}\dfrac{(-1)^{n+1} 1/(n+1)}{(-1)^n 1/n}=\lim\limits_{n\to\infty} -1\cdot\dfrac{n}{n+1}=-1.\]\qedhere
	      \end{proof}
	\item 设收敛数列~$\{a_n\}$~的每一项都是整数, 问: 该数列有什么特殊性质?
	      \begin{proof}[\bf 答]
		      从某一项开始后每一项均相同. 取~$\varepsilon=1/2$, 则存在~$N\in\mathbf{N}_{+}$, 使对~$n>N$~有~$|a_{n+1}-a_n|<1/2$, 注意到~$a_n\in\mathbf{Z}, n\in\mathbf{N}_{+}$, 知~$a_{n+1}=a_n, \forall n>N$.\qedhere
	      \end{proof}
	\item 问: 收敛数列是否一定是单调数列? 无穷小量是否一定是单调数列?
	      \begin{proof}[\bf 答]
		      均不一定. 如分别取~$\{a+(-1)^n 1/m\}$(收敛但不单调)~和~$\{(-1)^n 1/n\}$(无穷小量但不单调). \qedhere
	      \end{proof}
	\item \footnote{原本的6题中, 一个很小很小的量显然不是一个无穷小量, 注意无穷小量是一个趋于零的极限过程即可.}问: 正无穷大量数列是否一定单调增加? 无界数列是否一定为无穷大量?
	      \begin{proof}[\bf 答]
		      均不一定. 如分别取~$\{n+2\sin{n}\}$(正无穷大量但不单调)~和~$\{n\cdot\sin{n}\}$(无界但非无穷大). \qedhere
	      \end{proof}
	\item 问: 如果数列~$\{a_n\}$~收敛于~$a$, 那么绝对值~$|a_n-a|$~是否随着~$n$~的增加而单调减少趋于~$0$?
	      \begin{proof}[\bf 答]
		      不一定. 如取~$\{a_n\}$~为形如
		      \[1,1/2,1/3,1/6,1/4,1/8,1/12,\cdots,1/n,1/2n,\cdots,1/n(n-1),1/(n+1),\cdots\]
		      的数列, 由于~$1/n$~和~$1/(n+1)$~之间的所有项都严格小于~$1/(n+1)$, 于是~$\{a_n\}$~的上控数列\footnote{请结合数列的上下极限部分.}~$\{\overline{a_n}\}$~为~$1,1/2,1/3,1/4,1/4,\cdots$, 其中~$1/n$~连续出现了~$n-3$~次($n\geqslant3$), 显然~$\lim\limits_{n\to\infty}\overline{a_n}=0$. 而全为正项的数列~$\{a_n\}$~有一个子列~$\{1/n\}$~收敛于~$0$, 故
		      \[\varliminf\limits_{n\to\infty} a_n=\varlimsup\limits_{n\to\infty} a_n=0.\]
		      即~$\lim\limits_{n\to\inf}a_n=0$, 但显然~$\{|a_n|\}$~并不单调.\qedhere
	      \end{proof}
	\item 判断: 非负数列的极限是非负数, 正数列的极限是整数.
	      \begin{proof}[\bf 答]
		      非负数列的极限是非负数. 反证法. 假设非负数列~$\{a_n\}$~的极限为~$A<0$, 则存在~$N\in\mathbf{N}_{+}$, 当~$n>N$~时有~$|a_n-A|<-A/2$, 即当~$n>N$~时有~$3A/2<a_n<A/2<0$, 与~$\{a_n\}$~非负矛盾.

		      正数列的极限不一定为正数, 如取~$\{1/n\}$, 其极限为~$0$.\qedhere
	      \end{proof}
\end{enumerate}

\subsection{练习题}
\begin{enumerate}
	\item 按极限定义证明:
	      \begin{tabenum}[(1)]
		      \tabenumitem $\lim\limits_{n\to\infty} \dfrac{3n^2}{n^2-4}=3;$
		      \tabenumitem $\lim\limits_{n\to\infty} \dfrac{\sin{n}}{n}=0;$\\
		      \tabenumitem $\lim\limits_{n\to\infty} (1+n)^{\frac{1}{n}}=1;$
		      \tabenumitem $\lim\limits_{n\to\infty} \dfrac{a^n}{n!}=0.$
	      \end{tabenum}
	      \begin{proof}
		      对于任何~$\varepsilon>0$,
		      \begin{enumerate}[(1)]
			      \item 取~$N=[\sqrt{12/\varepsilon+4}]+1$, 当~$n>N$~时, $|\dfrac{3n^2}{n^2-4}-3|=\dfrac{12}{n^2-4}<\varepsilon;$
			      \item 取~$N=[1/\varepsilon]$, 当~$n>N$~时, $|\dfrac{\sin{n}}{n}\leqslant\dfrac{1}{n}<\varepsilon;$
			      \item 由于~$(1+n)^{\frac{1}{n}}>1, \forall n\in\mathbf{N}_{+}$, 故令~$y_n=(1+n)^{\frac{1}{n}}-1>0$, 有$n+1=(1+y_n)^n\geqslant \dfrac{n(n-1)}{2}y_n^2$, 即
			            \[\sqrt[n]{n+1}-1=y_n\leqslant\sqrt{\dfrac{2(n+1)}{n(n-1)}}.\]
			            又由~$\lim\limits_{n\to\infty}\dfrac{2(n+1)}{n(n-1)}$, 故存在~$N\in\mathbf{N}_{+}$, 使当~$n>N$~时有~$\dfrac{2(n+1)}{n(n-1)}<\varepsilon<1$, 故当~$n>N$~时有
			            \[\sqrt[n]{n+1}-1=y_n\leqslant\sqrt{\dfrac{2(n+1)}{n(n-1)}}<\sqrt{\varepsilon}<\varepsilon;\]
			      \item 若~$0<a\leqslant 1$, 显然取~$N=[\varepsilon]+1$, 当~$n>N$~时
			            \[\dfrac{a^n}{n!}\leqslant\dfrac{1}{n}<\varepsilon.\]
			            若~$a>1$, 则存在~$k\in\mathbf{N}_{+}$~使得~$k<a<k+1$, 于是
			            \[\dfrac{a^n}{n!}=\dfrac{a\cdot a\cdots a\cdot a\cdot a\cdots a\cdot a}{n\cdot(n-1)\cdots(k+1)k(k-1)\cdots2\cdot1}\leqslant\dfrac{a}{n}\dfrac{a\cdots a}{a\cdots a}\cdot\dfrac{a}{k}\dfrac{a}{k-1}\cdots\dfrac{a}{2}\dfrac{a}{1}.\]
			            注意上式中最后一项是一常数, 可记为~$K$, 取~$N=[aK/\varepsilon]+1$, 当~$n>N$~时有~$\dfrac{a^n}{n!}<\varepsilon.$\qedhere
		      \end{enumerate}
	      \end{proof}
	\item 设~$a_n\geqslant 0, n\in\mathbf{N}_{+}$, 数列~$\{a_n\}$~收敛于~$a$, 则~$\lim\limits_{n\to\infty}\sqrt{a_n}=\sqrt{a}$.
	      \begin{proof}
		      $|\sqrt{a_n}-\sqrt{a}|=\dfrac{|a_n-a|}{\sqrt{a_n}+\sqrt{a}}\leqslant\dfrac{|a_n-a|}{\sqrt{a}}$. $\forall\varepsilon>0$, 由~$\lim\limits_{n\to\infty} a_n=a$, $\exists N\in\mathbf{N}_{+}$, 当~$n>N$~时有~$|a_n-a|\leqslant\sqrt{a}\varepsilon$. 故当~$n>N$~时, ~$|\sqrt{a_n}-\sqrt{a}|\leqslant\dfrac{|a_n-a|}{\sqrt{a}}<\varepsilon$, 即~$\lim\limits_{n\to\infty}\sqrt{a_n}=\sqrt{a}$.\qedhere
	      \end{proof}
	\item 若~$\lim\limits_{n\to\infty} a_n=a$, 则~$\lim\limits_{n\to\infty}|a_n|=|a|$. 反之如何?
	      \begin{proof}
		      $\forall\varepsilon>0$, 由~$\lim\limits_{n\to\infty} a_n=a$, $\exists N\in\mathbf{N}_{+}$, 当~$n>N$~时有~$|a_n-a|<\varepsilon$. 故当~$n>N$~时, $||a_n|-|a||\leqslant|a_n-a|<\varepsilon$, 即~$\lim\limits_{n\to\infty}|a_n|=|a|$.\qedhere
	      \end{proof}
	\item \footnote{关于原先的~$5$~题, 完全可以使用相应函数极限的定义加上~Heine~定理证明, 并且本质没有任何不同.} 设~$a>1$, 证明~$\lim\limits_{n\to\infty}\dfrac{\log_a{n}}{n}=0$. (可以利用已知的极限~$\lim\limits_{n\to\infty}\sqrt[n]{n}=1$.)
	      \begin{proof}
		      \[ \lim\limits_{n\to\infty}\dfrac{\log_a{n}}{n}=\lim\limits_{n\to\infty}\log_a{n^{1/n}}=\lim\limits_{n\to\infty}\log_a{1}=0.\]
		      其中第二个等号用到了~$\log_a{x}$~的连续性.\qedhere
	      \end{proof}
\end{enumerate}

\section{收敛数列的基本性质}
\subsection{思考题}
\begin{enumerate}
	\item	设~$\{a_n\}$~收敛而~$\{b_n\}$~发散, 问: $\{a_n+b_n\}$~和~$\{a_nb_n\}$~的敛散性如何?
	      \begin{proof}
		      $\{a_n+b_n\}$~发散. 反证法. 假设~$\lim\limits_{n\to\infty} a_n=a, \lim\limits_{n\to\infty} (a_n+b_n)=A$, 则对于~$\forall\varepsilon>0$, $\exists N_1, N_2\in\mathbf{N}_{+}$, 当~$n>N_1$~时, $|(a_n+b_n)-A|<\varepsilon/2$; 当~$n>N_2$~时, $|a_n-a|<\varepsilon/2$. 令~$N=\max\{N_1,N_2\}$, 则当~$n>N$~时有
		      \[|b_n-(A-a)|=|[(a_n+b_n)-A]-(a_n-a)|\leqslant|(a_n+b_n)-A|+|a_n-a|<\varepsilon/2+\varepsilon/2=\varepsilon.\]
		      即~$\lim\limits_{n\to\infty} b_n=A-a$, 与~$\{b_n\}$~发散矛盾.

		      $\{a_nb_n\}$~可能发散也可能收敛. 如取~$a_n=1/n, b_n=n\sin{n}$, 则~$a_nb_n=\sin{n}$, $\{a_nb_n\}$~发散; 取~$a_n=1/n, b_n=(-1)^n$, 则~$a_nb_n=(-1)^n1/n$, $\{a_nb_n\}$~收敛.\qedhere
	      \end{proof}
	\item 设~$\{a_n\}$~和~$\{b_n\}$~都发散, 问: $\{a_n+b_n\}$~和~$\{a_nb_n\}$~的敛散性如何?
	      \begin{proof}
		      $\{a_n+b_n\}$~可能发散也可能收敛. 如取~$a_n=(-1)^n, b_n=(-1)^{n+1}$, 则~$a_n+b_n=0$, $\{a_n+b_n\}$~收敛; 取~$a_n=b_n=(-1)^n$, 则~$a_n+b_n=(-1)^n\cdot 2$, $\{a_n+b_n\}$~发散.

		      $\{a_nb_n\}$~可能发散也可能收敛. 如取~$a_n=b_n=(-1)^n$, 则~$a_nb_n=1$, $\{a_nb_n\}$~收敛; 取~$a_n=(-1)^n, b_n=n$, 则~$a_nb_n=(-1)^n\cdot n$, $\{a_nb_n\}$~发散.\qedhere
	      \end{proof}
	\item 设~$a_n\leqslant b_n\leqslant c_n, n\in\mathbf{N}_{+}$, 已知~$\lim\limits_{n\to\infty} (c_n-a_n)=0$, 问: 数列~$\{b_n\}$~是否收敛?
	      \begin{proof}
		      $\{b_n\}$~不一定收敛. 取一反例, $a_n=n, b_n=n+1/2n, c_n=n+1/n, n\in\mathbf{H}_{+}$, 则~$\lim\limits_{n\to\infty} (c_n-a_n)=\lim\limits_{n\to\infty} 1/n=0$, 但显然~$\{b_n\}$~发散.\qedhere
	      \end{proof}
	\item 找出下列运算中的错误:
	      \[\lim\limits_{n\to\infty} \left(\dfrac{1}{n+1}+\cdots+\dfrac{1}{2n}\right)=\lim\limits_{n\to\infty}\dfrac{1}{n+1}+\cdots+\lim\limits_{n\to\infty}\dfrac{1}{2n}=0.\]
	      \begin{proof}
		      问题在于第二个等号, 极限的四则运算法则之对于有限次的加减乘除(除法要求分母的数列不为零)成立, 对于可列次的四则运算没有意义.\qedhere
	      \end{proof}
	\item 设已知~$\{a_n\}$~收敛于~$a$, 又对每个~$n$~有~$b<a_n<c$, 问: 是否成立~$b<a<c$?
	      \begin{proof}
		      不一定成立. 如取~$b=0, c=1, a_n=1/n, n\in\mathbf{N}_{+}$, 则有~$b<a_n<c, \forall n\in\mathbf{H}_{+}$, 但~$a=\lim\limits_{n\to\infty} a_n=0$, 故~$a=c$.\qedhere
	      \end{proof}
	\item 设已知~$\{a_n\}$~收敛于~$a$, 又有~$b\leqslant a\leqslant c$, 问: 是否存在~$N$, 使得当~$n>N$~时成立~$b\leqslant a\leqslant c$?
	      \begin{proof}
		      两次应用数列极限的保序性, 所得的正整数分别记为~$N_1$~和~$N_2$, 则取~$N=\max\{N_1,N_2\}$, 当~$n>N$~时就有~$b_n\leqslant a_n\leqslant c_n$.\qedhere
	      \end{proof}
	\item 设已知~$\lim\limits_{n\to\infty} a_n=0$, 问是否有~$\lim\limits_{n\to\infty} (a_1a_2\cdots a_n)=0$? 又问: 反之如何?
	      \begin{proof}\footnote{结合无穷级数的相关知识可以给出另一证明. 记~$u_n=a_1\cdots a_n$, 由无穷级数的~d'Alembert~比值判别法, $\lim\limits_{n\to\infty} \dfrac{u_{n+1}}{u_n}=\lim\limits_{n\to\infty} a_{n+1}=0$, 有无穷级数~$\sum\limits_{n=1}^{\infty}u_n$~收敛, 故~$\lim\limits_{n\to\infty} u_n=0$.}
		      对于~$\varepsilon_0=1$, 由~$\lim\limits_{n\to\infty} a_n=0$~知存在~$N\in\mathbf{N}_{+}$~使得当~$n>N$~时有~$|a_n|<1$, 记~$K=|a_1a_2\cdots a_N|$. 对于~$\forall 0<\varepsilon<1$, $\exists N'\in\mathbf{N}_{+}$, 当~$n>N'$~时有~$|a_n|<\varepsilon/K$. 因此对于~$n>\max\{N,N'\}$, $|a_1a_2\cdots a_n|=K|a_{N+1}\cdots a_n|\leqslant K|a_n|<K\cdot\varepsilon/K=\varepsilon$, 即~$\lim\limits_{n\to\infty} (a_1a_2\cdots a_n)=0$.\qedhere
	      \end{proof}
\end{enumerate}

\subsection{练习题}
\begin{enumerate}
	\item	证明: $\{a_n\}$~收敛的充分必要条件是~$\{a_{2k}\}$~和~$\{a_{2k-1}\}$~收敛于同一极限.
	      \begin{proof}
		      必要性. 设~$\lim\limits_{n\to\infty} a_n=a$, 则对于~$\forall\varepsilon>0$, $\exists N\in\mathbf{N}_{+}$, 当~$n>N$~时, $|a_n-a|<\varepsilon$. 当~$k>N$~时, $2k>2k-1>N$, 故当~$k>N$~时, $|a_{2k}-a|<\varepsilon, |a_{2k-1}-a|<\varepsilon$. 即~$\lim\limits_{n\to\infty} a_{2k}=\lim\limits_{n\to\infty} a_{2k-1}=a$.

		      充分性. 设~$\lim\limits_{n\to\infty} a_{2k}=\lim\limits_{n\to\infty} a_{2k-1}=a$, 则对于~$\forall\varepsilon>0$, $\exists K_1\in\mathbf{N}_{+}$, 当~$k>K_1$~时, $|a_{2k}-a|<\varepsilon$; $\exists K_2\in\mathbf{N}_{+}$, 当~$k>K_2$~时, $|a_{2k-1}-a|<\varepsilon$. 取~$N=\max\{K_1,K_2\}$, 则当~$n>N$~时, $|a_n-a|<\varepsilon$.\qedhere
	      \end{proof}
	\item 以下是可以应用夹逼定理的几个题.
	      \begin{enumerate}[(1)]
		      \item 给定~$p$~个正数~$a_1,a_2,\cdots,a_p$, 求~$\lim\limits_{n\to\infty} \sqrt[n]{a_1^n+a_2^n+\cdots+a_p^n}$;
		      \item 设~$x_n=\dfrac{1}{\sqrt{n^2+1}}+\dfrac{1}{\sqrt{n^2+2}}+\cdots+\dfrac{1}{\sqrt{(n+1)^2}}, n\in\mathbf{N}_{+}$, 求~$\lim\limits_{n\to\infty} x_n$;
		      \item 设~$a_n=\left(1+\dfrac{1}{2}+\cdots+\dfrac{1}{n}\right)^\frac{1}{n}, n\in\mathbf{N}_{+}$, 求~$\lim\limits_{n\to\infty} a_n$;
		      \item 设~$\{a_n\}$~为正数列, 并且已知它收敛于~$a>0$, 证明~$\lim\limits_{n\to\infty} \sqrt[n]{a_n}=1$.
	      \end{enumerate}
	      \begin{proof}
		      \begin{enumerate}[(1)]
			      \item $\max\limits_{1\leqslant k\leqslant p} a_k\leqslant\sqrt[n]{a_1^n+a_2^n+\cdots+a_p^n}\leqslant\sqrt[n]{n\max\limits_{1\leqslant k\leqslant p} a_k^n}=\sqrt[n]{n}\max\limits_{1\leqslant k\leqslant p} a_k\to\max\limits_{1\leqslant k\leqslant p} a_k (n\to\infty)$, 故~$\lim\limits_{n\to\infty} \sqrt[n]{a_1^n+a_2^n+\cdots+a_p^n}=\max\limits_{1\leqslant k\leqslant p} a_k$;
			      \item $\dfrac{2n+1}{\sqrt{(n+1)^2}}\leqslant x_n\leqslant \dfrac{2n+1}{\sqrt{n^2+1}}$, $\lim\limits_{n\to\infty}\dfrac{2n+1}{\sqrt{(n+1)^2}}=\lim\limits_{n\to\infty}\dfrac{2n+1}{\sqrt{n^2+1}}=2$, 故~$\lim\limits_{n\to\infty} x_n=2$;
			      \item $\sqrt[n]{n\cdot 1/n}\leqslant a_n\leqslant \dfrac[n]{n}\to 1(n\to\infty)$, 故~$\lim\limits_{n\to\infty} a_n=1$;
			      \item 取~$\varepsilon=a/2>0$, 由~$\lim\limits_{n\to\infty} a_n=a$, $\exists N\in\mathbf{N}_{+}$, 当~$n>N$~时, $|a_n-a|<a/2$, 即当~$n>N$~时~$a/2<a_n<3a/2$. 同时开~$n$~次根号, 有~$\sqrt[n]{a/2}<\sqrt[n]{a}<\sqrt[n]{3a/2}, \forall n>N$. 由于~$\lim\limits_{n\to\infty}\sqrt[n]{a/2}=\lim\limits_{n\to\infty}\sqrt[n]{3a/2}=1$, 故~$\lim\limits_{n\to\infty} \sqrt[n]{a_n}=1$.\qedhere
		      \end{enumerate}
	      \end{proof}
	\item 求以下极限:
	      \begin{enumerate}[(1)]
		      \item $\lim\limits_{n\to\infty} (1+x)(1+x^2)\cdots(1+x^{2^n})$, 其中~$|x|<1$;
		      \item $\lim\limits_{n\to\infty} \left(1-\dfrac{1}{2^2}\right)\left(1-\dfrac{1}{3^2}\right)\cdots\left(1-\dfrac{1}{n^2}\right)$;
		      \item $\lim\limits_{n\to\infty} \left(1-\dfrac{1}{1+2}\right)\left(1-\dfrac{1}{1+2+3}\right)\cdots\left(1-\dfrac{1}{1+2+\cdots+n}\right)$;
		      \item $\lim\limits_{n\to\infty} \left(\dfrac{1}{1\cdot2\cdot3}+\dfrac{1}{2\cdot3\cdot4}+\cdots+\dfrac{1}{n(n+1)(n+2)}\right)$;
		      \item $\lim\limits_{n\to\infty} \sum\limits_{k=1}^{n}\dfrac{1}{k(k+1)\cdots(k+\nu)}$, 其中~$\nu\in\mathbf{N}_{+}, \nu>1$.\\
		            (最后两个题是$\lim\limits_{n\to\infty} \left(\dfrac{1}{1\cdot2}+\dfrac{1}{2\cdot3}+\cdots+\dfrac{1}{n(n+1)}\right)$~的推广.)
	      \end{enumerate}
	      \begin{proof}
		      \begin{enumerate}[(1)]
			      \item $\begin{aligned}[t]
					            (1+x)(1+x^2)\cdots(1+x^{2^n})= & \dfrac{(1+x)(1-x)(1+x^2)\cdots(1+x^{2^n})}{1-x}                   \\
					            =                              & \dfrac{(1-x^2)(1+x^2)\cdots(1+x^{2^n})}{1-x}                      \\
					            =                              & \cdots=\dfrac{1-x^{2^{n+1}}}{1-x}\to\dfrac{1}{1-x} \ (n\to\infty)
				            \end{aligned}$
			      \item $\begin{aligned}[t]
					            \left(1-\dfrac{1}{2^2}\right)\left(1-\dfrac{1}{3^2}\right)\cdots\left(1-\dfrac{1}{n^2}\right)= & \left(1-\dfrac{1}{2}\right)\left(1+\dfrac{1}{2}\right)\left(1-\dfrac{1}{3}\right)\left(1+\dfrac{1}{3}\right)\cdots\left(1-\dfrac{1}{n}\right)\left(1+\dfrac{1}{n}\right) \\
					            =                                                                                              & \dfrac{1}{2}\cdot\dfrac{3}{2}\cdot\dfrac{2}{3}\cdot\dfrac{4}{3}\cdots\dfrac{n-1}{n}\cdot\dfrac{n+1}{n}                                                                   \\
					            =                                                                                              & \dfrac{1}{2}\dfrac{n+1}{n}\to\dfrac{1}{2} \ (n\to\infty)
				            \end{aligned}$
			      \item $\begin{aligned}[t]
					            \left(1-\dfrac{1}{1+2}\right)\left(1-\dfrac{1}{1+2+3}\right)\cdots\left(1-\dfrac{1}{1+2+\cdots+n}\right)= & \dfrac{2}{1+2}\cdot\dfrac{2+3}{1+2+3}\cdots\dfrac{2+\cdots+n}{1+2+\cdots n}                     \\
					            =                                                                                                         & \dfrac{1\cdot4}{2\cdot3}\cdot\dfrac{2\cdot5}{3\cdot4}\cdot\cdots\cdot\dfrac{(n-1)(n+2)}{n(n+1)} \\
					            =                                                                                                         & \dfrac{2!(n-1)!(n+2)!}{3!n!(n+1)!}                                                              \\
					            =                                                                                                         & \dfrac{n+2}{3n}\to\dfrac{1}{3} \ (n\to\infty)
				            \end{aligned}$
			      \item $\begin{aligned}[t]
					              & \dfrac{1}{1\cdot2\cdot3}+\dfrac{1}{2\cdot3\cdot4}+\cdots+\dfrac{1}{n(n+1)(n+2)}                                                                                                            \\
					            = & \dfrac{1}{2}\left[\left(\dfrac{1}{1\cdot2}-\dfrac{1}{2\cdot3}\right)+\left(\dfrac{1}{2\cdot3}-\dfrac{1}{3\cdot4}\right)+\cdots+\left(\dfrac{1}{n(n+1)}-\dfrac{1}{(n+1)(n+2)}\right)\right] \\
					            = & \dfrac{1}{4}-\dfrac{1}{2(n+1)(n+2)}\to\dfrac{1}{4} \ (n\to\infty)
				            \end{aligned}$
			      \item $\begin{aligned}[t]
					            \sum\limits_{k=1}^{n}\dfrac{1}{k(k+1)\cdots(k+\nu)}= & \sum\limits_{k=1}^n\dfrac{1}{\nu}\left(\dfrac{1}{k(k+1)\cdots(k+\nu-1)}-\dfrac{1}{(k+1)(k+2)\cdots(k+\nu)}\right)                                    \\
					            =                                                    & \dfrac{1}{\nu}\left(\dfrac{1}{1\cdot2\cdot\cdots\cdot\nu}-\dfrac{1}{(n+1)(n+2)\cdots(n+\nu)}\right)\to\dfrac{1}{\nu\cdot\nu!} \ (n\to\infty)\qedhere
				            \end{aligned}$
		      \end{enumerate}
	      \end{proof}
	\item 设~$s_n=a+3a^2+\cdots+(2n-1)a^n$, $|a|<1$, 求~$\{a_m\}$~的极限.\\
	      (试计算~$s_n-as_n$.)
	      \begin{proof}
		      \begin{equation*}
			      \begin{split}
				      S_n   &= a+3a^2+\cdots+(2n-1)a^n;              \\
				      aS_n  &= \qquad\; a^2+\cdots+(2n-3)a^n+(2n-1)a^{n+1}.
			      \end{split}
		      \end{equation*}
		      上面两式相减, 有
		      \[	(1-a)S_n=a+2(a^2+a^3+\cdots+a^n)-(2n-1)a^{n+1}=\dfrac{a(1+a)}{1-a}-(2n-1)a^{n+1}-\dfrac{2a^{n+1}}{1-a}.\]
		      故
		      \[S_n=\dfrac{a(1+a)}{(1-a)^2}-\dfrac{1}{1-a}\left((2n-1)a^{n+1}+\dfrac{2a^{n+1}}{1-a}\right)\to\dfrac{a(1+a)}{(1-a)^2}\ (n\to\infty).\]\qedhere
	      \end{proof}
	\item 设正数列~$\{x_n\}$~收敛, 极限大于~$0$, 证明: 这个数列有正下界, 但在数列中不一定有最小数.
	      \begin{proof}
		      设~$\lim\limits_{n\to\infty} x_n=A>0$, 则对~$\varepsilon=A/2>0$, $\exists N\in\mathbf{N}_{+}$, 当~$n>N$~时, $|x_n-A|<A/2$, 即当~$n>N$~时, $x_n>A/2$, 记~$M=\max\{x_1,x_2,\cdots,x_N, A/2\}$, 则~$x_n\geqslant M, \forall n\in\mathbf{N}_{+}$, 即~$M$~是~$\{x_n\}$~的一个正的下界.

		      举一个无最小数的例子: $x_n=1+1/n, n\in\mathbf{N}_{+}$.\qedhere
	      \end{proof}
	\item 证明: 若有~$\lim\limits_{n\to\infty} a_n=+\infty$, 则在数列~$\{a_n\}$~中一定有最小数.
	      \begin{proof}
		      任取~$k\in\mathbf{N}_{+}$, 对于~$a_k$, $\exists N\in\mathbf{N}$, 当~$n>N$~时有~$a_n>a_k$. 取~$a=\min\{a_1,a_2,\cdots,a_N,a_k\}$, 则~$a\leqslant a_n, \forall n\in\mathbf{N}_{+}$, 同时~$a$~是~$\{a_n\}$~中的某一项, 故~$a$~是~$\{a_n\}$~中的最小数.\qedhere
	      \end{proof}
	\item 证明: 无界数列至少有一个子列是确定符号的无穷大量.
	      \begin{proof}
		      设~$\{x_n\}$~是无界数列, 不妨设其无上界, 即对任意~$M>0$, $\exists n\in\mathbf{N}_{+}$~使得~$x_n>M$.

		      对于~$M_1=1$, $\exists n_1\in\mathbf{N}_{+}$, 使得~$x_{n_1}>1$;

		      对于~$M_2=2$, $\exists n_2\in\mathbf{N}_{+}, n_2>n_1$, 使得~$x_{n_2}>2$, 断言这样的~$n_2$~是可以找到的, 否则~$\forall n>n_1$, $x_n\leqslant M_2$, 与~$\{x_n\}$~无界矛盾;

		      假设已经找出了~$x_{n_k}$, 使得~$x_{n_k}>x_{n_{k-1}}$, $x_{n_{k}}>M_k=k$, 则对于~$M_{k+1}=k+1$, $\exists n_{k+1}\in\mathbf{N}_{+}, n_{k+1}>n_{k}$, 使得~$x_{n_{k+1}}>k+1$, 断言这样的~$n_{k+1}$~是可以找到的, 否则~$\forall n>n_k$, $x_n\leqslant M_{k+1}$, 与~$\{x_n\}$~无界矛盾. 由数学归纳法可知找出了数列~$\{x_{n_{k}}\}$~使得~$n_1<n_2<\cdots<n_k<n_{k+1}<\cdots$, $x_{n_k}>k, k\in\mathbf{N}_{+}$. 这说明~$\{x_{n_k}\}$~是~$\{x_n\}$~的子列, 并且~$\{x_{n_k}\}$~是正的无穷大量. 同理若~$\{x_n\}$~无下界时可找到一个子列是负的无穷大量.\qedhere
	      \end{proof}
	\item 证明: 数列~$\{\tan{n}\}$~发散.
	\item 设数列~$\{S_n\}$~的定义为
	      \[S_n=1+\dfrac{1}{2^p}+\dfrac{1}{3^p}+\cdots+\dfrac{1}{n^p}, n\in\mathbf{N}_{+}.\]
	      证明: $\{S_n\}$~在以下两种情况均发散: (1)$p\leqslant 0$; (2)$0<p<1$.
\end{enumerate}
\end{document}