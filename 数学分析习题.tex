%pdf latex
\documentclass[UTF8,a4paper,20pt]{article}
\usepackage[hyperref]{CTEX}
\hypersetup{linkcolor=black}
\usepackage{amsmath}
\usepackage{amsthm}
\usepackage{amsfonts}
\usepackage[linkcolor=black]{hyperref}
\usepackage{bm}
\usepackage{amssymb}
\usepackage{multicol}
\usepackage{extarrows}
\usepackage[top=1in, bottom=1in, left=1in, right=1in]{geometry}
\usepackage{enumerate}
\renewcommand{\proofname}{\bf 证明}
\begin{document}
\newcommand{\e}{\mathrm{e}}
\newcommand{\ud}[1]{\mathrm{d}#1}
\renewcommand{\emptyset}{\varnothing}
\pagestyle{empty}
\tableofcontents
\clearpage
\pagestyle{plain}

\section{实数和一元数列的极限}\setcounter{page}{1}
\subsection{数列和数列的定义}
\begin{enumerate}
\item {\large\kaishu 根据数列极限的$\varepsilon$-$N$定义证明以下极限:}
	\begin{enumerate}[(1)]
	\item $\lim\limits_{n \to \infty}\dfrac{n}{2n-1}=\dfrac{1}{2};$
	\item $\lim\limits_{n \to \infty}\dfrac{2\sqrt{n}-1}{3\sqrt{n}+1}=\dfrac{2}{3};$
	\item $\lim\limits_{n \to \infty}\dfrac{2^n}{n!}=0;$
	\item $\lim\limits_{n \to \infty}\dfrac{n}{a^n}=0\quad(a>1);$
	\item $\lim\limits_{n \to \infty}\dfrac{n^2}{a^n}=0\quad(a>1).$
	\end{enumerate}

\item {\large\kaishu 设$x_n\leqslant a\leqslant y_n, n=1,2,\cdots,$ 且$\lim\limits_{n \to \infty}\left(y_n-x_n)\right)=0,$ 证明: $\lim\limits_{n\to \infty} x_n=\lim\limits_{n\to \infty} y_n = a.$}

{\heiti 证明} 由$x_n\leqslant a\leqslant y_n, n=1,2,\cdots$ 有 $0\leqslant a-x_n\leqslant y_n-x_n$, 故$|a-x_n|\leqslant |y_n-x_n|$; 同理有 $0 \geqslant a-y_n \geqslant x_n-y_n$, 故$|a-y_n|\leqslant|y_n-x_n|$. 由$\lim\limits_{n \to \infty}\left(y_n-x_n)\right)=0$可知对于任意固定的$\varepsilon>0$, 存在正整数$N$使得当$n>N$时,
\[
	|a-x_n|\leqslant|y_n-x_n|<\varepsilon,\quad
	|a-y_n|\leqslant|y_n-x_n|<\varepsilon,
\]
故$\lim\limits_{n \to \infty}x_n=\lim\limits_{n \to \infty}y_n=a.$

\item {\large\kaishu 设存在常数$0<\lambda<1$使得$|x_{n+1}|\leqslant\lambda|x_n|, n=1,2,\cdots,$ 证明: $\lim\limits_{n\to \infty}x_n=0.$}

{\heiti 证明} $|x_n|\leqslant\lambda|x_{n-1}|\leqslant\lambda^2|x_{n-2}|\leqslant\cdots\leqslant\lambda^{n-1}|x_1|$, 对于任意给定的$\varepsilon >0$, 要使$|x_n|\leqslant\varepsilon$, 只需要$n>\dfrac{\ln \varepsilon-\ln|x_1|}{\ln \lambda}+1$即可, 于是对于上面给定的$\varepsilon$, 取$N=\left[\dfrac{\ln \varepsilon-\ln|x_1|}{\ln \lambda}\right]+2$, 则当$n>N$时,
\[
	|x_n|\leqslant\lambda^{n-1}|x_1|<\varepsilon
\]
即$\lim\limits_{n\to \infty}x_n=0.$

\item {\large\kaishu 已知$\lim\limits_{n\to \infty}x_n=a$, 证明:}
	\begin{enumerate}[(1)]
	\item $\lim\limits_{n \to \infty}|x_n|=|a|;$
	\item $\lim\limits_{n \to \infty}\sqrt{x_n}=\sqrt{a};$
	\item $\lim\limits_{n \to \infty}\sqrt[3]{x_n}=\sqrt[3]{a}.$
	\end{enumerate}

{\heiti 证明}
由$\lim\limits_{n\to \infty}x_n=a$, 知对于任意固定的$\varepsilon>0$, 存在正整数$N$使得当$n>N$时有$|x_n-a|<\varepsilon$, 因此对于上面给定的$\varepsilon$和$N$, 由三角不等式
\[
||x_n|-|a||\leqslant|x_n-a|<\varepsilon,
\]
因此$\lim\limits_{n \to \infty}|x_n|=|a|;$\\
同理可知对于任意固定的$\varepsilon>0$, 存在正整数$N$使得当$n>N$时有$|x_n-a|<\sqrt{a}\varepsilon$, 因此对于上面给定的$\varepsilon$和$N$, 
\[
|\sqrt{x_n}-\sqrt{a}|=\dfrac{|x_n-a|}{\sqrt{x_n}+\sqrt{a}}\leqslant\dfrac{|x_n-a|}{\sqrt{a}}<\varepsilon,
\]
因此$\lim\limits_{n \to \infty}\sqrt{x_n}=\sqrt{a};$\\
同理可知对于任意固定的$\varepsilon>0$, 存在正整数$N$使得当$n>N$时有$|x_n-a|<\sqrt[3]{a^2}\varepsilon$, 因此对于上面给定的$\varepsilon$和$N$, 
\[
|\sqrt[3]{x_n}-\sqrt[3]{a}|=\dfrac{|x_n-a|}{\sqrt[3]{x_n^2}+\sqrt[3]{ax_n}+\sqrt[3]{a^2}}\leqslant\dfrac{|x_n-a|}{\sqrt[3]{a^2}}<\varepsilon,
\]
因此$\lim\limits_{n \to \infty}\sqrt[3]{x_n}=\sqrt[3]{a}.$

\item {\large\kaishu 证明: $\lim\limits_{n \to \infty} x_n=a$ 的充要条件是 $\lim\limits_{n \to \infty}x_{2n}=a$且$\lim\limits_{n \to \infty}x_{2n-1}=a$.}

{\heiti 证明}\\
``$\Rightarrow$''  由$\lim\limits_{n \to \infty} x_n=a$可知对于任意固定的$\varepsilon>0$, 存在正整数$N$使得当$n>N$时有$|x_n-a|<\varepsilon$, 故$|x_n-a|<\varepsilon(n>2N)$且$|x_n-a|<\varepsilon(n>2N-1)$,即对于上面给定的$\varepsilon$和$N$, 当$n>N$时, $2n>2n-1>n>N$, $|x_{2n}-a|<\varepsilon$且$|x-{2n-1}-a|<\varepsilon$, 即$\lim\limits_{n \to \infty}x_{2n}=\lim\limits_{n \to \infty}x_{2n-1}=a.$\\
``$\Leftarrow$''   由$\lim\limits_{n \to \infty}x_{2n}=a$知对任意给定的$\varepsilon>0$, 存在正整数$N_1$使得当$n>N_1$时有$|x_{2n}-a|<\varepsilon$; 同理, 由$\lim\limits_{n \to \infty}x_{2n-1}=a$知存在正整数$N_2$使得当$n>N_2$时有$|x_{2n-1}-a|<\varepsilon$. 令$N=\max\{2N_1, 2N_2-1\}$, 则当$n>N$时有$|x_n-a|<\varepsilon$, 即$\lim\limits_{n \to \infty} x_n=a$. 

\item {\large\kaishu 已知$|b|<a$, 证明: $\lim\limits_{n \to \infty} \sqrt[n]{a^n+b^n}=a.$}

{\heiti 证明}\\
由$|b|<a$, 故$\dfrac{a}{|b|}>1$, 于是
\[
|\sqrt[n]{a^n+b^n}-a|=|b|\left(\sqrt[n]{\left(\dfrac{a}{|b|}\right)^n+1}-\dfrac{a}{|b|}\right)
\]

\item {\large\kaishu 已知 $\lim\limits_{n \to \infty} x_n=a$, 证明: $\lim\limits_{n \to \infty} \dfrac{[nx_n]}{n}=a.$}

{\heiti 证明}\\
由Gauss函数性质, $nx_n\leqslant[nx_n]\leqslant nx_n+1$, 即$x_n\leqslant\dfrac{[nx_n]}{n}\leqslant x_n+\dfrac{1}{n}$. 
\[
\left|\dfrac{[nx_n]}{n}-a\right|\leqslant \left|\dfrac{[nx_n]}{n}-x_n\right|+|x_n-a|\leqslant \dfrac{1}{n}+|x_n-a|,
\]
对于任意给定的$\varepsilon>0$, 由$\lim\limits_{n \to \infty} x_n=a$知存在正整数$N_1$使得当$n>N_1$时有$|x_n-a|<\dfrac{\varepsilon}{2}$, 由$\lim\limits_{n \to \infty} x_n=a$知存在正整数$N_2$使得当$n>N_2$时有$\dfrac{1}{n}<\dfrac{\varepsilon}{2}$. 令$N=\max\{N_1, N_2\}$, 则当$n>N$时有
\[
\left|\dfrac{[nx_n]}{n}-a\right|\leqslant \dfrac{1}{n}+|x_n-a|<\dfrac{\varepsilon}{2}+\dfrac{\varepsilon}{2}=\varepsilon, 
\]
即$\lim\limits_{n \to \infty} \dfrac{[nx_n]}{n}=a.$

\item {\large\kaishu 已知 $x_n \neq 0, n=1,2,\cdots,$ 且 $\lim\limits_{n \to \infty} \left| \dfrac{x_{n+1}}{x_n} \right| =l<1$, 证明: $\lim\limits_{n \to \infty}x_n=0.$}

{\heiti 证明}\\
\[
|x_n|=\left|x_n \dfrac{x_{n-1}}{x_{n-1}} \dfrac{x_{n-2}}{x_{n-2}} \cdots \dfrac{x_1}{x_1}\right|\leqslant\left|\dfrac{x_n}{x_{n-1}}\right| \left|\dfrac{x_{n-1}}{x_{n-2}}\right| \cdots \left|\dfrac{x_2}{x_1}\right| |x_1|,
\]
对于任意给定的$\xi<\dfrac{1-l}{2}$, 由$\lim\limits_{n \to \infty} \left| \dfrac{x_{n+1}}{x_n} \right| =l$知存在正整数$M$使得当$n>M$时, $\left|\left|\dfrac{x_{n+1}}{x_n}\right|-l\right|<\xi$, 从而$\left|\dfrac{x_{n+1}}{x_n}\right|<l+\xi<1$. 于是当$n>M$时, 
\[
|x_n|\leqslant\left|\dfrac{x_n}{x_{n-1}}\right| \left|\dfrac{x_{n-1}}{x_{n-2}}\right| \cdots \left|\dfrac{x_{M+1}}{x_M}\right|\left|\dfrac{x_{M}}{x_{M-1}}\right|\cdots\left|\dfrac{x_2}{x_1}\right| |x_1|<(l+\xi)^{n-M}\left|\dfrac{x_{N}}{x_{N-1}}\right|\cdots\left|\dfrac{x_2}{x_1}\right| |x_1|=(l+\xi)^{n-M}\cdot C,
\]
对于任意给定的$\varepsilon>0$, 要使上式小于$\varepsilon$, 只需要$n>\dfrac{\ln\varepsilon-\ln C}{\ln(l+\xi)}+M$, 于是令$N=\left[\dfrac{\ln\varepsilon-\ln C}{\ln(l+\xi)}\right]+M+1$, 则当$n>N$时, $|x_n|<\varepsilon$, 即$\lim\limits_{n \to \infty}x_n=0.$

\item {\large\kaishu 已知 $\lim\limits_{n \to \infty}|x_n|=l<1$, 证明: $\lim\limits_{n \to \infty} x_n^n=0.$}

{\heiti 证明}\\
对于任意给定的$\xi<\dfrac{1-l}{2}$, 由$\lim\limits_{n \to \infty}|x_n|=l$知存在正整数$M$使得当$n>M$时, $||x_n|-l|<\xi$, 即$|x_n|<l+\xi<1$. 于是当$n>M$时, 
\[
|x_n^n|< (l+\xi)^n, 
\]
对于任意给定的$\varepsilon>0$, 要使$|x_n^n|<\varepsilon$, 只需要$n>\dfrac{\ln\varepsilon}{\ln(l+\xi)}$, 于是令$N=\max\left\{\left[\dfrac{\ln\varepsilon}{\ln(l+\xi)}\right]+1, M\right\}$, 则当$n>N$时, $|x_n^n|< (l+\xi)^n<\varepsilon$, 即$\lim\limits_{n \to \infty} x_n^n=0.$

\item {\large\kaishu 已知 $\lim\limits_{n \to \infty}x_n=a$, 证明: $\lim\limits_{n \to \infty}\dfrac{x_1+2x_2+\cdots+nx_n}{1+2+\cdots+n}=a.$}

{\heiti 证明}\\
\begin{equation*}
\begin{split}
\left|\dfrac{x_1+2x_2+\cdots+nx_n}{1+2+\cdots+n}-a\right|&=\left|\dfrac{(x_1-a)+2(x_2-a)+\cdots+n(x_n-a)}{1+2+\cdots+n}\right|\\&\leqslant\dfrac{|x_1-a|+2|x_2-a|+\cdots+n|x_n-a|}{1+2+\cdots+n}
\end{split}
\end{equation*}
对于任意的$\varepsilon>0$, 由$\lim\limits_{n \to \infty}x_n=a$知存在正整数$N1$使得当$n>N$时有$|x_n-a|<\dfrac{\varepsilon}{2}$, 从而当$n>N$时,
\begin{equation*}
\begin{split}
\left|\dfrac{x_1+2x_2+\cdots+nx_n}{1+2+\cdots+n}-a\right|&\leqslant\dfrac{|x_1-a|+2|x_2-a|+\cdots+N|x_N-a|+(N+1)|x_{N+1}-a|+\cdots+n|x_n-a|}{1+2+\cdots+n}\\
&<\dfrac{|x_1-a|+2|x_2-a|+\cdots+N|x_N-a|+[(N+1)+n]\dfrac{n-N}{2}\cdot\dfrac{\varepsilon}{2}}{1+2+\cdots+n},
\end{split}
\end{equation*}
取定$N$后, 由于$\lim\limits_{n \to \infty}\dfrac{|x_1-a|+2|x_2-a|+\cdots+N|x_N-a|}{1+2+\cdots+n}=0$, 存在正整数$N'>N$使得当$n>N'$时,
\[
\dfrac{|x_1-a|+2|x_2-a|+\cdots+N|x_N-a|}{1+2+\cdots+n}<\dfrac{\varepsilon}{2}.
\]
于是当$n>N'$时就有
\[
\left|\dfrac{x_1+2x_2+\cdots+nx_n}{1+2+\cdots+n}-a\right|<\dfrac{|x_1-a|+2|x_2-a|+\cdots+N|x_N-a|}{1+2+\cdots+n}+\dfrac{[(N+1)+n]\dfrac{n-N}{2}}{1+2+\cdots+n}\cdot\dfrac{\varepsilon}{2}<\dfrac{\varepsilon}{2}+\dfrac{\varepsilon}{2}=\varepsilon.
\]
所以$\lim\limits_{n \to \infty}\dfrac{x_1+2x_2+\cdots+nx_n}{1+2+\cdots+n}=a.$

\item {\large\kaishu 已知 $\lim\limits_{n \to \infty}(x_{n+1}-x_n)=a$, 证明: $\lim\limits_{n \to \infty}\dfrac{x_n}{n}=a.$}

{\heiti 证明}\\
\[
\left|\dfrac{x_n}{n}-a\right|\leqslant\dfrac{|(x_n-x_{n-1})-a|+\cdots+|(x_2-x_1)-a|+|x_1-a|}{n},
\]
对于任意的$\varepsilon>0$, 由$\lim\limits_{n \to \infty}(x_{n+1}-x_n)=a$知存在正整数$N$使得当$n>N$时有$|(x_{n+1}-x_n)-a|<\dfrac{\varepsilon}{2}$, 于是
\begin{equation*}
\begin{split}
\left|\dfrac{x_n}{n}-a\right|&\leqslant\dfrac{|(x_n-x_{n-1})-a|+\cdots+|(x_2-x_1)-a|+|x_1-a|}{n}\\&<\dfrac{(n-N)\cdot\dfrac{\varepsilon}{2}+|(x_N-x_{N-1})-a|+\cdots+|(x_2-x_1)-a|+|x_1-a|}{n},
\end{split}
\end{equation*}
取定$N$后, 由于$\lim\limits_{n \to \infty}\dfrac{|(x_N-x_{N-1})-a|+\cdots+|(x_2-x_1)-a|+|x_1-a|}{n}=0$, 存在正整数$N'>N$使得当$n>N'$时,
\[
\dfrac{|(x_N-x_{N-1})-a|+\cdots+|(x_2-x_1)-a|+|x_1-a|}{n}<\dfrac{\varepsilon}{2}.
\]
于是当$n>N'$时就有
\begin{equation*}
\begin{split}
\left|\dfrac{x_n}{n}-a\right|&<\dfrac{|(x_N-x_{N-1})-a|+\cdots+|(x_2-x_1)-a|+|x_1-a|}{n}+(1-\dfrac{N}{n})\cdot\dfrac{\varepsilon}{2}\\&<\dfrac{\varepsilon}{2}+\dfrac{\varepsilon}{2}=\varepsilon.
\end{split}
\end{equation*}
所以$\lim\limits_{n \to \infty}\dfrac{x_n}{n}=a.$
\end{enumerate}

\clearpage
\subsection{收敛数列的性质}
\begin{enumerate}
\item {\large\kaishu 根据数列极限的四则运算和已经掌握的极限求以下极限:}
	\begin{enumerate}[(1)]
	\item $\lim\limits_{n \to \infty}\dfrac{3n^3+2n^2-n-2}{2n^3-3n+1};$
	\item $\lim\limits_{n \to \infty}\dfrac{2\sqrt[3]{n}-\sqrt{n}+1}{5\sqrt[3]{n}+3\sqrt{n}-2};$
	\item $\lim\limits_{n \to \infty}\dfrac{3^n+(-2)^n}{5^n+(-3)^n};$
	\item $\lim\limits_{n \to \infty}\dfrac{3^nn^2a+2^nn^3b}{3^nn^2+2^nn^3};$
	\item $\lim\limits_{n \to \infty}(\sqrt{n+1}-\sqrt{n-1});$
	\item $\lim\limits_{n \to \infty}\sqrt[3]{n^2}(\sqrt[3]{n+1}-\sqrt[3]{n});$
	\item $\lim\limits_{n \to \infty}\left(\dfrac{1}{1\cdot 2}+\dfrac{1}{2\cdot 3}+\cdots+\dfrac{1}{n(n+1)}\right);$
	\item $\lim\limits_{n \to \infty}\dfrac{1}{n^2}[1+3+\cdots+(2n-1)];$
	\item $\lim\limits_{n \to \infty}\dfrac{1+a+a^2+\cdots+a^n}{1+b+b^2+\cdots+b^n} (|a|<1, |b|<1);$
	\item $\lim\limits_{n \to \infty}\left(1-\dfrac{1}{2^2}\right)\left(1-\dfrac{1}{3^2}\right)\cdots\left(1-\dfrac{1}{n^2}\right);$
	\item $\lim\limits_{n \to \infty}\left(1+\dfrac{1}{2}\right)\left(1+\dfrac{1}{4}\right)\cdots\left(1+\dfrac{1}{2^{2n}}\right);$
	\item $\lim\limits_{n \to \infty}\left(\dfrac{1}{1\cdot 2\cdot 3}+\dfrac{1}{2\cdot 3\cdot 4}+\cdots+\dfrac{1}{n(n+1)(n+2)}\right).$
	\end{enumerate}
{\heiti 解} 
	\begin{enumerate}[(1)]
	\item $\dfrac{3n^3+2n^2-n-2}{2n^3-3n+1}=\dfrac{3+\dfrac{2}{n}-\dfrac{1}{n^2}-\dfrac{2}{n^3}}{2-\dfrac{3}{n^2}+\dfrac{1}{n^3}}\to \dfrac{3}{2}\quad(n \to \infty).$
	\item $\dfrac{2\sqrt[3]{n}-\sqrt{n}+1}{5\sqrt[3]{n}+3\sqrt{n}-2}=\dfrac{\dfrac{2}{\sqrt[3]{n^2}}-1+\dfrac{1}{\sqrt{n}}}{\dfrac{5}{\sqrt[3]{n^2}}+3-\dfrac{2}{\sqrt{n}}}\to -\dfrac{1}{3}\quad(n\to\infty).$
	\item $\dfrac{3^n+(-2)^n}{5^n+(-3)^n}=\left(\dfrac{3}{5}\right)^n\cdot\dfrac{1+\left(-\dfrac{2}{3}\right)^n}{1+\left(-\dfrac{3}{5}\right)^n}\to 0\quad(n\to\infty).$
	\item $\dfrac{3^nn^2a+2^nn^3b}{3^nn^2+2^nn^3}=\dfrac{3^nn^2a\left(1+n\left(\dfrac{2}{3}\right)^n\dfrac{b}{a}\right)}{3^nn^2\left(1+\left(\dfrac{2}{3}\right)^n\right)}=a\cdot\dfrac{1+n\left(\dfrac{2}{3}\right)^n\dfrac{b}{a}}{1+n\left(\dfrac{2}{3}\right)^n}\to a\quad(n\to\infty).$\\
其中用到了$n^kq^n\to 0\quad(n\to \infty, k\in\mathbb{N}^{*},0<q<1)$.
	\item $\sqrt{n+1}-\sqrt{n-1}=\dfrac{2}{\sqrt{n+1}-\sqrt{n-1}}=\dfrac{\dfrac{2}{\sqrt{n-1}}}{1+\sqrt{1+\dfrac{2}{n-1}}}\to 0 \quad(n\to\infty).$
	\item $\sqrt[3]{n^2}(\sqrt[3]{n+1}-\sqrt[3]{n})=\dfrac{\sqrt[3]{n^2}}{\sqrt[3]{(n+1)^2}+\sqrt[3]{n(n+1)}+\sqrt[3]{n^2}}=\dfrac{1}{\sqrt[3]{\left(\dfrac{n+1}{n}\right)^2}+\sqrt[3]{\dfrac{n+1}{n}}+1}\to \dfrac{1}{3}$.
	\item $\dfrac{1}{1\cdot 2}+\dfrac{1}{2\cdot 3}+\cdots+\dfrac{1}{n(n+1)}=\left(1-\dfrac{1}{2}\right)+\left(\dfrac{1}{2}-\dfrac{1}{3}\right)+\cdots+\left(\dfrac{1}{n}-\dfrac{1}{n+1}\right)=1-\dfrac{1}{n+1}\to 1$.
	\item $\dfrac{1}{n^2}[1+3+\cdots+(2n-1)]=\dfrac{n[1+(2n-1)]}{2n^2}=1.$
	\item $\dfrac{1+a+a^2+\cdots+a^n}{1+b+b^2+\cdots+b^n}=\dfrac{\dfrac{1-a^{n+1}}{1-a}}{\dfrac{1-b^{n+1}}{1-b}}=\dfrac{1-b}{1-a}\cdot\dfrac{1-a^{n+1}}{1-b^{n+1}}\to\dfrac{1-b}{1-a}\quad(n\to\infty, |a|<1, |b|<1).$
	\item $\left(1-\dfrac{1}{2^2}\right)\left(1-\dfrac{1}{3^2}\right)\cdots\left(1-\dfrac{1}{n^2}\right)=\dfrac{1}{2}\dfrac{3}{2}\dfrac{2}{3}\dfrac{4}{3}\cdots\dfrac{n-1}{n}\dfrac{n+1}{n}=\dfrac{1}{2}\left(1+\dfrac{1}{n}\right)\to \dfrac{1}{2}\quad(n\to\infty).$
	\item 
$\left(1+\dfrac{1}{2}\right)\left(1+\dfrac{1}{4}\right)\cdots\left(1+\dfrac{1}{2^{2n}}\right)=\dfrac{\left(1-\dfrac{1}{2}\right)\left(1+\dfrac{1}{2}\right)\left(1+\dfrac{1}{4}\right)\cdots\left(1+\dfrac{1}{2^{2n}}\right)}{\left(1-\dfrac{1}{2}\right)}\\
	=\dfrac{\left(1-\dfrac{1}{4}\right)\left(1+\dfrac{1}{4}\right)\cdots\left(1+\dfrac{1}{2^{2n}}\right)}{\left(1-\dfrac{1}{2}\right)}\\
	=\cdots\\
	=\dfrac{1-\dfrac{1}{2^{2n+1}}}{1-\dfrac{1}{2}}=2-\dfrac{1}{2^{2n}}\to 2\quad(n\to\infty).$
	\item $\dfrac{1}{1\cdot 2\cdot 3}+\dfrac{1}{2\cdot 3\cdot 4}+\cdots+\dfrac{1}{n(n+1)(n+2)}$\\$=\dfrac{1}{2}\left[\left(\dfrac{1}{1\cdot2}-\dfrac{1}{2\cdot3}\right)+\left(\dfrac{1}{2\cdot3}-\dfrac{1}{3\cdot4}\right)+\cdots+\left(\dfrac{1}{n(n+1)}-\dfrac{1}{(n+1)(n+2)}\right)\right]$\\$=\dfrac{1}{4}-\dfrac{1}{2(n+1)(n+2)}\to\dfrac{1}{4}\quad(n\to\infty).$
	
	\end{enumerate}

\item {\large\kaishu 利用两边夹法则求下列极限:}
	\begin{enumerate}[(1)]
	\item $\lim\limits_{n\to\infty}\sqrt{1-\dfrac{1}{n}}$;
	\item $\lim\limits_{n\to\infty}\dfrac{\sin n!}{\sqrt{n}}$;
	\item $\lim\limits_{n\to\infty}\left(1-\dfrac{1}{\sqrt[n]{2}}\right){\mathrm e}^{\cos^2n}$;
	\item $\lim\limits_{n\to\infty}\sqrt[n]{n\log_2n}$;
	\item $\lim\limits_{n\to\infty}\left(\dfrac{1}{\sqrt{n^2+1}}+\dfrac{1}{\sqrt{n^2+2}}+\cdots+\dfrac{1}{\sqrt{n^2+n}}\right)$;
	\item $\lim\limits_{n\to\infty}\left(\dfrac{1}{\sqrt{n^2+1}}+\dfrac{1}{\sqrt{n^2+2}}+\cdots+\dfrac{1}{\sqrt{(n+1)^2}}\right)$;
	\item $\lim\limits_{n\to\infty}\dfrac{1}{\sqrt[n]{n!}}$;
	\item $\lim\limits_{n\to\infty}\sqrt[n^2]{n!}$;
	\item $\lim\limits_{n\to\infty}\sum\limits_{k=2}^n\left[\dfrac{1}{\sqrt[k]{n^k+1}}+\dfrac{1}{\sqrt[k]{n^k+1}}\right]$;
	\item $\lim\limits_{n\to\infty}\sin{\pi\sqrt{n^2+1}}$;
	\item $\lim\limits_{n\to\infty}(-1)^n\sin{\pi\sqrt{n^n+n}}$;
	\item $\lim\limits_{n\to\infty}\underbrace{\sin\sin\cdots\sin}_{n} x$;
	\end{enumerate}
{\heiti 解} 
	\begin{enumerate}[(1)]
	\item $1-\dfrac{1}{n}\leqslant\sqrt{1-\dfrac{1}{n}}\leqslant 1, \quad\Rightarrow\quad\sqrt{1-\dfrac{1}{n}}\to 1.$
	\item $-\dfrac{1}{\sqrt{n}}\leqslant\dfrac{\sin{n!}}{\sqrt{n}}\leqslant\dfrac{1}{\sqrt{n}}, \quad\Rightarrow\quad\dfrac{\sin{n!}}{\sqrt{n}}\to 0.$
	\item $-1\leqslant\cos{n}\leqslant{1}, \quad\Rightarrow\quad 0\leqslant\cos^2{n}\leqslant 1, \quad\Rightarrow\quad 1-\dfrac{1}{\sqrt[n]{2}}\leqslant\left(1-\dfrac{1}{\sqrt[n]{2}}\right){\mathrm e}^{\cos^2{n}}\leqslant\left(1-\dfrac{1}{\sqrt[n]{2}}\right){\mathrm e}, \quad\Rightarrow\quad\left(1-\dfrac{1}{\sqrt[n]{2}}\right){\mathrm e}^{\cos^2{n}}\to 0.$
	\item $\log_2{n}\leqslant n, \forall n \in \mathbf{N}, \quad\Rightarrow\quad \sqrt[n]{n}\leqslant\sqrt[n]{n\log_2{n}}\leqslant\sqrt[n]{n^2}=\left(\sqrt[n]{n}\right)^2, \quad\Rightarrow\quad \sqrt[n]{n\log_2^n}\to 1.$
	\item $\dfrac{1}{\sqrt{n^2+1}}+\dfrac{1}{\sqrt{n^2+2}}+\cdots+\dfrac{1}{\sqrt{n^2+n}}\leqslant\dfrac{n}{\sqrt{n^2+1}}=\sqrt{\dfrac{1}{1+\dfrac{1}{n^2}}}\to 1,$\\
		  $\dfrac{1}{\sqrt{n^2+1}}+\dfrac{1}{\sqrt{n^2+2}}+\cdots+\dfrac{1}{\sqrt{n^2+n}}\geqslant\dfrac{n}{\sqrt{n^2+n}}=\sqrt{\dfrac{1}{1+\dfrac{1}{n}}}\to 1,$\\
		  $\Rightarrow\quad \dfrac{1}{\sqrt{n^2+1}}+\dfrac{1}{\sqrt{n^2+2}}+\cdots+\dfrac{1}{\sqrt{n^2+n}}\to 1.$
	\item $\dfrac{1}{\sqrt{n^2+1}}+\dfrac{1}{\sqrt{n^2+2}}+\cdots+\dfrac{1}{\sqrt{(n+1)^2}}\leqslant\dfrac{2n+1}{\sqrt{n^2+1}}=\sqrt{\dfrac{(2n+1)^2}{n^2+1}}=\sqrt{\dfrac{4n^2+4n+1}{n^2+1}}\to2,$\\
		  $\dfrac{1}{\sqrt{n^2+1}}+\dfrac{1}{\sqrt{n^2+2}}+\cdots+\dfrac{1}{\sqrt{(n+1)^2}}\geqslant\dfrac{2n+1}{(n+1)^2}=\dfrac{2n+1}{n+1}\to 2,$\\
		  $\Rightarrow\quad \dfrac{1}{\sqrt{n^2+1}}+\dfrac{1}{\sqrt{n^2+2}}+\cdots+\dfrac{1}{\sqrt{(n+1)^2}}\to 2.$
	\item $0<\dfrac{1}{\sqrt[n]{n!}}\leqslant\dfrac{1}{\sqrt[n]{n^n}}=\dfrac{1}{n}\to 0, \quad\Rightarrow\quad \dfrac{1}{\sqrt[n]{n!}}\to 0.$
	\item $1\leqslant\sqrt[n^2]{n!}\leqslant\sqrt[n^2]{n^n}=((n^2)^{\frac{1}{n}})^{\frac{1}{n}}=\sqrt[n]{n}\to 1,\quad\Rightarrow\quad \sqrt[n^2]{n!}\to 1.$   -
	\item 
	\end{enumerate}

\item 已知$\lim\limits_{n\to\infty}x_n=a$, $\lim\limits_{n\to\infty}y_n=b$. 证明:
\[ \lim\limits_{n\to\infty}\min\{x_n,y_n\}=\min\{a,b\}, \lim\limits_{n\to\infty}\max\{x_n,y_n\}=\max\{a,b\}. \]
{\heiti 证明}\\
若$a=b$, 显然成立. 若$a\neq b$, 不妨设$a>b$, 断言存在$N$使得当$n>N$时有$x_n>y_n$.事实上, 取$\varepsilon=\dfrac{a-b}{2}$, 由$\lim\limits_{n\to\infty}x_n=a$, $\lim\limits_{n\to\infty}y_n=b$知分别存在$N_1,N_2$使得
\[|x_n-a|<\dfrac{a-b}{2}(\text{当$n>N_1$时}),\quad |y_n-b|<\dfrac{a-b}{2}(\text{当$n>N_2$时}),\]
则当$n>\max\{N_1,N_2\}$时
\[x_n>a-\dfrac{a-b}{2}=\dfrac{a+b}{2}=b+\dfrac{a-b}{2}>y_n.\]
于是当$n>N$时, $\max\{x_n, y_n\}=x_n$, $\min\{x_n, y_n\}=y_n$. 由$\lim\limits_{n\to\infty}x_n=a$, $\lim\limits_{n\to\infty}y_n=b$知对于任意给定的$\varepsilon>0$, 分别存在$N_1',N_2'$使得
\[|x_n-a|<\dfrac{a-b}{2}(\text{当$n>N_1'$时}),\quad |y_n-b|<\dfrac{a-b}{2}(\text{当$n>N_2'$时}),\]
于是分别当$n>\max\{N,N_1'\}$和$n>\max\{N,N_2'\}$时
\[ \left|\max\{x_n,y_n\}-\max\{a,b\}\right|=|x_n-a|<\varepsilon, \quad \left|\min\{x_n,y_n\}-\min\{a,b\}\right|=|y_n-b|<\varepsilon.\]
即$\lim\limits_{n\to\infty}\min\{x_n,y_n\}=\min\{a,b\}, \lim\limits_{n\to\infty}\max\{x_n,y_n\}=\max\{a,b\}.$

\item 设$a_1,a_2,\cdots,a_m\in\mathbf{R}^+$, 求$\lim\limits_{n\to\infty}\sqrt[n]{a_1^n+a_2^n+\cdots+a_m^n}$.\\
{\heiti 证明}\\
记$\max\{a_1,a_2,\cdots,a_m\}=a$, 显然有$a\leqslant\sqrt[n]{a_1^n+a_2^n+\cdots+a_m^n}\leqslant\sqrt[n]{ma^n}=a\sqrt[n]{m}\to a$, 由两边夹法则知$\lim\limits_{n\to\infty}\sqrt[n]{a_1^n+a_2^n+\cdots+a_m^n}=a$.

\item 已知$x_n>0, n=1,2,\cdots$, 且$\lim\limits_{n\to\infty}x_n=a>0$. 证明: $\lim\limits_{n\to\infty}\sqrt[n]{x_n}=1$.\\
{\heiti 证明}\\
取$\varepsilon=\dfrac{a}{2}>0$, 由$\lim\limits_{n\to\infty}x_n=a$知存在$N>0$使得对于一切$n>N$有$|x_n-a|<\dfrac{a}{2}$, 即
\[\dfrac{a}{2}<x_n<\dfrac{3a}{2}\]
不等式两边同时开$n$次根号, 由于$x_n>0,n=1,2,\cdots$, 有
\[\sqrt[n]{\dfrac{a}{2}}<\sqrt[n]{x_n}<\sqrt[n]{\dfrac{3a}{2}},\]
又因为$\lim\limits_{n\to\infty}\sqrt[n]{\dfrac{a}{2}}=\lim\limits_{n\to\infty}\sqrt[n]{\dfrac{3a}{2}}=1$, 由两边夹法则知$\lim\limits_{n\to\infty}\sqrt[n]{x_n}=1$.

\item 已知$\lim\limits_{n\to\infty}x_n=a$, 证明:
	\begin{enumerate}[(1)]
	\item 对任意$b>0$有$\lim\limits_{n\to\infty}b^{x_n}=b^a$;
	\item 对任意$b>0$有$\lim\limits_{n\to\infty}\log_bx_n=\log_ba$(假定$x_n>0,n=1,2,\cdots$, 且$a>0$);
	\end{enumerate}
{\heiti 证明}
	\begin{enumerate}[(1)]
	\item 对于任意给定的$\delta_1=1>0$, 由于$\lim\limits_{n\to\infty}x_n=a$, 知存在$N\in\mathbb{N}$使得当$n>N$时有$|x_n-a|<\delta$
	\item 对任意$b>0$有$\lim\limits_{n\to\infty}\log_bx_n=\log_ba$(假定$x_n>0,n=1,2,\cdots$, 且$a>0$);
	\end{enumerate}
\item 已知$x_n>0, n=1,2,\cdots$, 且$\lim\limits_{n\to\infty}\dfrac{x_{n+1}}{x_n}=a$. 证明: $\lim\limits_{n\to\infty}\sqrt[n]{x_n}=a$.\\
{\heiti 证明}\\
	对于任意给定的$\varepsilon>0$, 由于$\lim\limits_{n\to\infty}\dfrac{x_{n+1}}{x_n}=a$, 知存在$N>0$使得当$n>N$时有$|\dfrac{x_{n+1}}{x_n}-a|<\varepsilon$, 即
\begin{equation*}
	\begin{split}
	x_n(a-\varepsilon)<&x_{n+1}<x_n(a+\varepsilon),\\
	x_{n-1}(a-\varepsilon)<&x_n<x_{n-1}(a+\varepsilon),\\
	&\cdots\\
	x_N(a-\varepsilon)<&x_{N+1}<x_N(a+\varepsilon),\\
	\end{split}
\end{equation*}
将以上不等式反复迭代, 有
\[x_N(a-\varepsilon)^{n-N}<x_n<x_N(a+\varepsilon)^{n-N},\]  
记$A=\dfrac{x_N}{(a-\varepsilon)^N}, B=\dfrac{x_N}{(a+\varepsilon)^N}$, 则
\[A(a-\varepsilon)^n<x_n<B(a+\varepsilon)^n, \forall n>N.\]
同时开$n$次根号, 得
\[\sqrt[n]{A}(a-\varepsilon)<\sqrt[n]{x_n}<\sqrt[n]{B}(a+\varepsilon).\]
由于$A,B$均为常数, 故$\lim\limits_{n\to\infty}\sqrt[n]{A}=\lim\limits_{n\to\infty}\sqrt[n]{B}=1$, 由极限保序性知有
\[a-\varepsilon<\sqrt[n]{x_n}<a+\varepsilon, \forall n>N.\]
即$\lim\limits_{n\to\infty}\sqrt[n]{x_n}=a$.

\item 求下列极限:
	\begin{enumerate}[(1)]
	\item $\lim\limits_{n\to\infty}\sum\limits_{k=1}^n\dfrac{1}{k(k+m)}(m\text{是正整数})$;
	\item $\lim\limits_{n\to\infty}\sum\limits_{k=1}^n\dfrac{2^kk}{(k+2)!}$;
	\item $\lim\limits_{n\to\infty}\sum\limits_{k=1}^n\dfrac{k^3+6k^2+11k+5}{(k+3)!}$;
	\item $\lim\limits_{n\to\infty}\dfrac{1}{n^3}\left[1\cdot 3+2\cdot 4+\cdots+n(n+2)\right]$;
	\item $\lim\limits_{n\to\infty}\left(\dfrac{2^3-1}{2^3+1}\dfrac{3^3-1}{3^3+1}\cdots\dfrac{n^3-1}{n^3+1}\right)$.
	\end{enumerate}
{\heiti 解}
	\begin{enumerate}[(1)]	
	\item $\sum\limits_{k=1}^n\dfrac{1}{k(k+m)}=\dfrac{1}{m}\sum\limits_{k=1}^n\left(\dfrac{1}{k}-\dfrac{1}{k+m}\right)=\dfrac{1}{m}\sum\limits_{k=1}^m\left(\dfrac{1}{k}-\dfrac{1}{n+k}\right)\to\dfrac{1}{m}\sum\limits_{k=1}^m\dfrac{1}{k}.$
	\item 
    \item $\sum\limits_{k=1}^n\dfrac{k^3+6k^2+11k+5}{(k+3)!}=\sum\limits_{k=1}^n\dfrac{(k+3)^3-3(k+3)^2+2(k+3)-1}{(k+3)!}\\
			=\sum\limits_{k=1}^n\left[\dfrac{(k+3)^2}{(k+2)!}-\dfrac{3(k+3)}{(k+2)!}+\dfrac{2}{(k+2)!}-\dfrac{1}{(k+3)!}\right]\\
			=\sum\limits_{k=1}^n\left[\dfrac{(k+2)^2+2(k+2)+1}{(k+2)!}-\dfrac{3(k+2)+3}{(k+2)!}+\dfrac{2}{(k+2)!}-\dfrac{1}{(k+3)!}\right]\\
			=\sum\limits_{k=1}^n\left[\dfrac{1}{k!}-\dfrac{1}{(k+3)!}\right]\\
			=1+\dfrac{1}{2!}+\dfrac{1}{3!}-\dfrac{1}{(n+1)!}-\dfrac{1}{(n+2)!}-\dfrac{1}{(n+3)!}\to\dfrac{5}{3}.$
	\item $\dfrac{1}{n^3}\left[1\cdot 3+2\cdot 4+\cdots+n(n+2)\right]=\dfrac{1}{n^3}\sum\limits_{k=1}^n(k^2+2k)\\
			=\dfrac{1}{n^3}\left[\dfrac{n(n+1)(2n+1)}{6}+\dfrac{n(n+1)}{2}\right]\\
			=\dfrac{n^3+3n^2+2n}{3n^3}\to \dfrac{1}{3}.$
	\item $\dfrac{2^3-1}{2^3+1}\dfrac{3^3-1}{3^3+1}\cdots\dfrac{n^3-1}{n^3+1}=\dfrac{(2-1)(2^2+2+1)}{(2+1)(2^2-2+1)}\dfrac{(3-1)(3^2+3+1)}{(3+1)(3^2-3+1)}\cdots\dfrac{(n-1)(n^2+n-1)}{(n+1)(n^2+n-1)}\\
			=\left(\dfrac{1}{3}\dfrac{2}{4}\dfrac{3}{5}\cdots\dfrac{n-3}{n-1}\dfrac{n-2}{n}\dfrac{n-1}{n+1}\right)\left(\dfrac{7}{3}\dfrac{13}{7}\cdots\dfrac{n^2+n+1}{n^2-n+1}\right)\\
            =\dfrac{1\cdot2}{n(n+1)}\dfrac{n^2+n+1}{3}=\dfrac{2n^2+2n+2}{3n^2+3n}\to\dfrac{2}{3}.$
	\end{enumerate}

\item 证明: 极限$\lim\limits_{n\to\infty}\sin{n}$不存在.\\
{\heiti 证明}\\
假设$\lim\limits_{n\to\infty}\sin{n}=a$, 在等式
\[\sin(n+1)-\sin(n-1)=2\sin{1}\cos{n}\]
两边取极限($n\to\infty$), 有$0=2\sin{1}\lim\limits_{n\to\infty}\cos{n}$, 即$\lim\limits_{n\to\infty}\cos{n}=0$. 再在等式
\[\sin^2{n}+\cos^2{n}=1\]
两边取极限有$\lim\limits_{n\to\infty}\cos{n}=1$, 矛盾. 

\item 已知$\lim\limits_{n\to\infty}x_n=a$, 证明:
	\begin{enumerate}[(1)]
	\item $\lim\limits_{n\to\infty}\dfrac{x_n+2x_{n-1}+\cdots+nx_1}{n(n+1)}=\dfrac{a}{2}$;
	\item $\lim\limits_{n\to\infty}\dfrac{C_n^0+C_n^1x_1+C_n^2x_2+\cdots+C_n^{n-1}x_{n-1}+C_n^n x_n}{2}=a$;
	\item $\lim\limits_{n\to\infty}(x_n+\lambda x_{n-1}+ \lambda^2 x_{n-2}+\cdots+\lambda^{n-1}x_1)=a$;
	\end{enumerate}
{\heiti 证明}\\
	三个极限采用相同的模式证明:
	\begin{enumerate}[1$^{\circ}$]
	\item {\heiti 当$a=0$时:} 由$\lim\limits_{n\to\infty}x_n=0$ 知对于任意给定的$\varepsilon>0$, 存在 $N_0>0$, 使得当$n>N_0$时有$|x_n|<\dfrac{\varepsilon}{2}$, 并记$\hat{x}=\max\{|x_1|, |x_2|, \cdots, |x_{N_0}|\}$, 当$n>N_0$时,
	\begin{equation*}
	\begin{split}
	\left|\dfrac{x_n+2x_{n-1}+\cdots+nx_1}{n(n+1)}\right|&\leqslant\left|\dfrac{x_n+\cdots+(n-N_0)x_{N_0+1}}{n(n+1)}\right|+\left|\dfrac{(n-N_0+1)x_{N_0}+\cdots+nx_1}{n(n+1)}\right|\\
	&<\dfrac{[1+(n-N_0)](n-N_0)}{n(n+1)}\dfrac{\varepsilon}{2}+\dfrac{[(n-N_0+1)+n]N_0}{n(n+1)}\hat{x}\\
	&=\dfrac{n^2-2nN_0+N_0^2-N_0}{n^2+n}\dfrac{\varepsilon}{2}+\dfrac{2nN_0-N_0^2+N_0}{n^2+n}\hat{x}\}.
	\end{split}
	\end{equation*}
	由于$\lim\limits_{n\to\infty}\dfrac{2nN_0-N_0^2+N_0}{n^2+n}=0$, 故对于上面给定的$\varepsilon$, 存在$N_1>0$使得当$n>N_1$时有
\[\dfrac{2nN_0-N_0^2+N_0}{n^2+n}\hat{x}<\dfrac{\varepsilon}{2},\] 
于是当$n>\max\{N_0,N_1\}$时
	\[\left|\dfrac{x_n+2x_{n-1}+\cdots+nx_1}{n(n+1)}\right|<\dfrac{n^2-2nN_0+N_0^2-N_0}{n^2+n}\dfrac{\varepsilon}{2}+\dfrac{2nN_0-N_0^2+N_0}{n^2+n}\hat{x}<\dfrac{\varepsilon}{2}+\dfrac{\varepsilon}{2}=\varepsilon.\]
即$\lim\limits_{n\to\infty}\dfrac{x_n+2x_{n-1}+\cdots+nx_1}{n(n+1)}=0$.

	\begin{equation*}
	\left|\dfrac{C_n^0+C_n^1x_1+\cdots+C_n^n x_n}{2^n}\right|\leqslant\dfrac{C_n^0+C_n^1|x_1|+\cdots+C_n^{N_0}|x_{N_0}|}{2^n}+\dfrac{C_n^{N_0+1}+C_n^{N_0+2}+\cdots+C_n^n}{2^n}\dfrac{\varepsilon}{2}.
	\end{equation*}
对于任意的$1\leqslant k\leqslant N_0$, $0<\dfrac{C_n^k}{2^n}<\dfrac{n^{k+1}}{2^n}\to 0$, 于是右边第一项是有限$N_0$个无穷小量的和, 还是无穷小量. 于是对于上面给定的$\varepsilon$, 存在$N_2>0$使得当$n>N_2$时有
\[\dfrac{C_n^0+C_n^1+\cdots+C_n^{N_0}}{2^n}\hat{x}<\dfrac{\varepsilon}{2},\]
于是当$n>\max\{N_0,N_2\}$时
	\[\left|\dfrac{C_n^0+C_n^1x_1+\cdots+C_n^n x_n}{2^n}\right|\leqslant\dfrac{C_n^0+C_n^1+\cdots+C_n^{N_0}}{2^n}\hat{x}+\dfrac{C_n^{N+1}+C_n^{N+2}+\cdots+C_n^n}{2^n}\dfrac{\varepsilon}{2}<\dfrac{\varepsilon}{2}+\dfrac{\varepsilon}{2}=\varepsilon.\]
即$\lim\limits_{n\to\infty}\dfrac{C_n^0+C_n^1x_1+C_n^2x_2+\cdots+C_n^{n-1}x_{n-1}+C_n^n x_n}{2}=0$.

	\begin{equation*}
	\left|x_n+\lambda x_{n-1}+ \lambda^2 x_{n-2}+\cdots+\lambda^{n-1}x_1\right|\leqslant\left|1+\cdots+\lambda^{n-N_0-1}\right|\dfrac{\varepsilon}{2}+|\lambda^{n-N_0}x_{N_0}+\cdots+\lambda^{n-1}x_1|
	\end{equation*}
对于任意的$1\leqslant k\leqslant N_0$, $\lim\limits_{n\to\infty}\lambda^{n-k}=0$, 于是邮编第二项是有限$N_0$个无穷小量之和, 是一个无穷小量. 于是对于上面给定的$\varepsilon$, 存在$N_3>0$使得当$n>N_3$时有
\[\lambda^{n-N_0}+\cdots+\lambda^{n-1}\hat{x}<\dfrac{\varepsilon}{2}\]
\[\left|x_n+\lambda x_{n-1}+ \lambda^2 x_{n-2}+\cdots+\lambda^{n-1}x_1\right|<(1+\cdots+\lambda^{n-N_0-1})\dfrac{\varepsilon}{2}+(\lambda^{n-N_0}+\cdots+\lambda^{n-1})\hat{x}<\dfrac{\varepsilon}{2}+\dfrac{\varepsilon}{2}=\varepsilon\]
	\item {\heiti 当$a\neq 0$时:} 由$\lim\limits_{n\to\infty}x_n=a$, 知$\lim\limits_{n\to\infty}x_n-a=0$.

 由(1)知有
\[\lim\limits_{n\to\infty}\dfrac{(x_n-a)+2(x_{n-1}-a)+\cdots+n(x_1-a)}{n(n+1)}=0.\]
故
\[\lim\limits_{n\to\infty}\dfrac{x_n+2x_{n-1}+\cdots+nx_1}{n(n+1)}=\lim\limits_{n\to\infty}\left[\dfrac{(x_n-a)+2(x_{n-1}-a)+\cdots+n(x_1-a)}{n(n+1)}+\dfrac{a}{2}\right]=0+\dfrac{a}{2}=\dfrac{a}{2}.\]

由(2)知有
\[\lim\limits_{n\to\infty}\dfrac{C_n^0+C_n^1(x_1-a)+\cdots+C_n^n(x_n-a)}{2}=0.\]
故
\[\lim\limits_{n\to\infty}\dfrac{C_n^0+C_n^1x_1+\cdots+C_n^n x_n}{2}=\lim\limits_{n\to\infty}\left[\dfrac{C_n^0+C_n^1(x_1-a)+\cdots+C_n^n(x_n-a)}{2}+a\right]=0+a=a.\]

由(3)知有
\[\lim\limits_{n\to\infty}[(x_n-a)+\lambda (x_{n-1}-a)+ \cdots+\lambda^{n-1}(x_1-a)]=0.\]
故
\[\lim\limits_{n\to\infty}(x_n+\lambda x_{n-1}+\cdots+\lambda^{n-1}x_1)=\lim\limits_{n\to\infty}\left[(x_n-a)+\cdots+\lambda^{n-1}(x_1-a)+\dfrac{a}{1-\lambda}\right]=0+\dfrac{a}{1-\lambda}=\dfrac{a}{1-\lambda}.\]
	\end{enumerate}

\item 设$\lim\limits_{n\to\infty}x_n=a$. 又设$\{p_n\}$是正数列, 满足
\[ \lim\limits_{n\to\infty}\dfrac{p_n}{p_1+p_2+\cdots+p_n}=0.\]
证明: $\lim\limits_{n\to\infty}\dfrac{p_1x_n+p_2x_{n-1}+\cdots+p_nx_1}{p_1+p_2+\cdots+p_n}=a$.\\
{\heiti 证明}
\begin{enumerate}[1$^{\circ}$]
\item 当$a=0$时. 由于$\lim\limits_{n\to\infty}x_n=0$, 对于任意给定的$\varepsilon>0$, 存在相应的$N_1$使得当$n>N_1$时有$|x_n|<\dfrac{\varepsilon}{2}$. 于是
\[ \left|\dfrac{p_1x_n+p_2x_{n-1}+\cdots+p_nx_1}{p_1+p_2+\cdots+p_n}\right|\leqslant \dfrac{p_1+p_2+\cdots+p_{n-N_1}}{p_1+p_2+\cdots+p_n}\dfrac{\varepsilon}{2}+\dfrac{p_{n-N_1+1}|x_{N_1}|+\cdots+p_n|x_1|}{p_1+p_2+\cdots+p_n}\]
由于$\lim\limits_{n\to\infty}\dfrac{p_n}{p_1+\cdots+p_n}=0$是无穷小量, 故$\dfrac{p_{n-N_1+1}+\cdots+p_n}{p_1+p_2+\cdots+p_n}$是无穷小量. 对于上面给定的$\varepsilon$, 存在相应的$N_2$使得当$n>N_2$时有$\dfrac{p_{n-N_1+1}+\cdots+p_n}{p_1+p_2+\cdots+p_n}\max\{|x_1|,\cdots,|x_{N_1}|\}<\dfrac{\varepsilon}{2}$, 于是
\[ \left|\dfrac{p_1x_n+\cdots+p_nx_1}{p_1+\cdots+p_n}\right|<\dfrac{p_1+\cdots+p_{n-N_1}}{p_1+\cdots+p_n}\dfrac{\varepsilon}{2}+\dfrac{p_{n-N_1+1}+\cdots+p_n}{p_1+\cdots+p_n}\max\{|x_1|,\cdots,|x_{N_1}|\}<\dfrac{\varepsilon}{2}+\dfrac{\varepsilon}{2}=\varepsilon.\]
即$\lim\limits_{n\to\infty}\dfrac{p_1x_n+p_2x_{n-1}+\cdots+p_nx_1}{p_1+p_2+\cdots+p_n}=0$. 
\item 当$a\neq 0$时. 由于$\lim\limits_{n\to\infty}x_n=0$, 故$\lim\limits_{n\to\infty}x_n-a=0$, 由1$^{\circ}$可知
\[\lim\limits_{n\to\infty}\dfrac{p_1(x_n-a)+p_2(x_{n-1}-a)+\cdots+p_n(x_1-a)}{p_1+p_2+\cdots+p_n}=0\]
于是
\[\lim\limits_{n\to\infty}\dfrac{p_1x_n+\cdots+p_nx_1}{p_1+\cdots+p_n}=\lim\limits_{n\to\infty}\left[\dfrac{p_1(x_n-a)\cdots+p_n(x_1-a)}{p_1+\cdots+p_n}+\dfrac{p_1a+\cdots+p_na}{p_1+\cdots+p_n}\right]=0+a=a.\]
\end{enumerate}

\item 已知$\lim\limits_{n\to\infty}x_n=a$且$\lim\limits_{n\to\infty}y_n=b$, 证明:
\[\lim\limits_{n\to\infty}\dfrac{x_1y_n+x_2y_{n-1}+\cdots+x_{n-1}y_2+x_ny_1}{n}=ab\]
{\heiti 证明}\\
\begin{enumerate}[1$^{\circ}$]
\item 当$b=0$时. 由于$\{x_n\}$收敛因而有界, 从而存在$M>0$使得$|x_n|\leqslant M$. 于是
\[ \left|\dfrac{x_1y_n+\cdots+x_ny_1}{n}\right|\leqslant M\dfrac{|y_1|+|y_2|+\cdots+|y_n|}{n}\to 0\]
这里收敛的断言参见$\lim\limits_{n\to\infty}\dfrac{x_1+x_2+\cdots+x_n}{n}=\lim\limits_{n\to\infty}x_n$的证明. 
\item 当$b\neq 0$时. 则有$\lim\limits_{n\to\infty}y_n-b=0$, 由1$^{\circ}$的结论可知
\[\lim\limits_{n\to\infty}\dfrac{x_1(y_n-b)+x_2(y_{n-1}-b)+\cdots+x_n(y_1-b)}{n}=0.\]
从而
\begin{equation*}
\begin{split}
\lim\limits_{n\to\infty}\dfrac{x_1y_n+x_2y_{n-1}+\cdots+x_ny_1}{n}&=\lim\limits_{n\to\infty}\left[\dfrac{x_1(y_n-b)+\cdots+x_n(y_1-b)}{n}+b\dfrac{x_1+\cdots x_n}{n}\right]\\&=0+ab=ab.
\end{split}
\end{equation*}
\end{enumerate}
\end{enumerate}

\clearpage
\subsection{趋向无穷的数列和三个记号}
\begin{enumerate}[1.]
\item 证明: 
	\begin{enumerate}[(1)]
	\item $\lim\limits_{n\to\infty}\left(1+\dfrac{1}{3}+\dfrac{1}{5}+\cdots+\dfrac{1}{2n-1}\right)=+\infty$;
	\item $\lim\limits_{n\to\infty}\left(\dfrac{1}{a+b}+\dfrac{1}{2a+b}+\dfrac{1}{3a+b}+\cdots+\dfrac{1}{na+b}\right)=+\infty$, 其中$a>0,b>0$;8
	\item $\lim\limits_{n\to\infty}\left(\dfrac{1}{\sqrt[4]{1^3\cdot2}}+\dfrac{1}{\sqrt[4]{2^3\cdot 3}}+\dfrac{1}{\sqrt[4]{3^3\cdot 4}}+\cdots+\dfrac{1}{\sqrt[4]{n^3\cdot(n+1)}}\right)=+\infty$.
	\end{enumerate}

{\heiti 证明}
	\begin{enumerate}[(1)]
	\item \[1+\dfrac{1}{3}+\dfrac{1}{5}+\cdots+\dfrac{1}{2n-1}\geqslant\dfrac{1}{2}+\dfrac{1}{4}+\cdots+\dfrac{1}{2n}=\dfrac{1}{2}\left(1+\dfrac{1}{2}+\cdots+\dfrac{1}{n}\right)\to+\infty,\] 
于是$\lim\limits_{n\to\infty}\left(1+\dfrac{1}{3}+\dfrac{1}{5}+\cdots+\dfrac{1}{2n-1}\right)=+\infty$.
	\item \[\dfrac{1}{a+b}+\dfrac{1}{2a+b}+\cdots+\dfrac{1}{na+b}\geqslant\dfrac{1}{a+b}+\dfrac{1}{2(a+b)}+\cdots+\dfrac{1}{n(a+b)}=\frac{1}{a+b}\left(1+\dfrac{1}{2}+\cdots+\dfrac{1}{n}\right)\to+\infty,\]
于是$\lim\limits_{n\to\infty}\left(\dfrac{1}{a+b}+\dfrac{1}{2a+b}+\dfrac{1}{3a+b}+\cdots+\dfrac{1}{na+b}\right)=+\infty$.
	\item \[\dfrac{1}{\sqrt[4]{1^3\cdot2}}+\cdots+\dfrac{1}{\sqrt[4]{n^3\cdot(n+1)}}\geqslant\dfrac{1}{2}+\dfrac{1}{3}+\cdots+\dfrac{1}{n+1}\to +\infty,\]
于是$\lim\limits_{n\to\infty}\left(\dfrac{1}{\sqrt[4]{1^3\cdot2}}+\dfrac{1}{\sqrt[4]{2^3\cdot 3}}+\dfrac{1}{\sqrt[4]{3^3\cdot 4}}+\cdots+\dfrac{1}{\sqrt[4]{n^3\cdot(n+1)}}\right)=+\infty$.
	\end{enumerate}
\end{enumerate}

\clearpage
\section{一元函数的极限和连续性}
\subsection{函数的极限}
\begin{enumerate}[1.]
\item 应用$\varepsilon-\delta$语言证明以下极限:

	\begin{enumerate}[(1)]
	\item $\lim\limits_{x\to2}x^2=4$;
	\item $\lim\limits_{x\to1}\dfrac{x}{2x^2+1}=\dfrac{1}{3}$;
	\item $\lim\limits_{x\to1}\dfrac{x^2-1}{2x^2-x-1}=\dfrac{2}{3}$;
	\item $\lim\limits_{x\to1}\dfrac{x+2}{2\sqrt{x}-1}=3$;
	\item $\lim\limits_{x\to\frac{\pi}{2}}(2x-\pi)\cos\dfrac{x-\pi}{2x-\pi}=0$.
	\end{enumerate}

\item 设$n$为正整数. 应用$\varepsilon-\delta$语言证明以下极限:
	\begin{enumerate}[(1)]
	\item $\lim\limits_{x\to x_0} x^n=x_0^n$;
	\item $\lim\limits_{x\to x_0} x^{\frac{1}{n}}=x_0^{\frac{1}{n}}(x_0>0)$.
	\end{enumerate}
{\heiti 证明}\\
当$0<|x-x_0|<|x_0|$时, $x,x_0$同号, 于是有
	\begin{equation*}
	\begin{split}
	|x^n-x_0^n|\leqslant&|x-x_0||x^{n-1}+x^{n-2}x_0+\cdots+xx_0^{n-2}+x_0^{n-1}|\\
	\leqslant&|x-x_0|(C_{n-1}^0x^{n-1}+C_{n-1}^1x^{n-2}x_0+\cdots+C_{n-1}^{n-1}x_0^{n-1})\\
	=&|x-x_0||x+x_0|^{n-1}\\
	\leqslant&|x-x_0||2x_0+1|^{n-1}
	\end{split}
	\end{equation*}
	取$\delta=\min\{|x_0|,1,\dfrac{\varepsilon}{|2x_0+1|^{n-1}}\}$, 则当$0<|x-x_0|<\delta$时, $|x-x_0|\leqslant|x-x_0||2x_0+1|^{n-1}<\varepsilon$. 即$\lim\limits_{x\to x_0}x^n=x_0^n$.

同理, 有
	\begin{equation*}
	|\sqrt[n]{x}-\sqrt[n]{x_0}|=\dfrac{|x-x_0|}{\sqrt[n]{x^{n-1}}+\sqrt[n]{x^{n-2}x_0}+\cdots+\sqrt[n]{x_0^{n-1}}}\leqslant\dfrac{|x-x_0|}{(\sqrt[n]{x_0})^{n-1}}.
	\end{equation*}
	取$\delta=\min\{x_0,(\sqrt[n]{x_0})^{n-1}\varepsilon\}$, 则当$0<|x-x_0|<\delta$时, $|\sqrt[n]{x}-\sqrt[n]{x_0}|<\varepsilon$. 即$\lim\limits_{x\to x_0} x^{\frac{1}{n}}=x_0^{\frac{1}{n}}$.

\item 已知$\lim\limits_{x\to x_0} f(x)=a$. 应用$\varepsilon-\delta$语言证明以下结论: 
	\begin{enumerate}[(1)]
	\item $\lim\limits_{x\to x_0}f^2(x)\mathrm{sgn} f(x)=a^2\mathrm{sgn} a$;
	\item $\lim\limits_{x\to x_0}\sqrt[3]{f(x)}=\sqrt[3]{a}$.
	\end{enumerate}
{\heiti 证明}\\
	\begin{enumerate}[(1)]
	\item 若$a=0$, 则由$\lim\limits_{x\to x_0}f(x)=0$知对于任意给定的$\varepsilon>0$, 存在相应的$\delta>0$使得当$0<|x-x_0|<\delta$时有$|f(x)|<\sqrt{\varepsilon}$, 从而
\[ |f^2\mathrm{sgn}f(x)|\leqslant|f^2|\leqslant|f|^2<(\sqrt{\varepsilon})^2=\varepsilon. \]
	若$a\neq 0$, 有
\[ |f^2\mathrm{sgn}f(x)-a^2\mathrm{sgn}a|=|(f^2-a^2)\mathrm{sgn}f(x)+a^2(\mathrm{sgn}f(x)-\mathrm{sgn}a)|\leqslant|f^2-a^2|+a^2|\mathrm{sgn}f(x)-\mathrm{sgn}a|. \]
由于$\lim\limits_{x\to x_0}f(x)=a$, 若$a>0$, 存在$\delta_{+}>0$使得当$0<|x-x_0|<\delta_{+}$时, $f(x)>\dfrac{a}{2}>0$, 从而有$\mathrm{sgn}f(x)=\mathrm{sgn}a$, 同理当$a<0$时存在$\delta_{-}>0$使得当$0<|x-x_0|<\delta_{-}$时, $f(x)<\dfrac{a}{2}<0$, 从而也有$\mathrm{sgn}f(x)=\mathrm{sgn}a$. 而由于
	\[ |f^2-a^2|=|f^2-fa+fa-a^2|\leqslant|f||f-a|+|f-a||a|\]
由于$\lim\limits_{x\to x_0}f(x)=a$, 对于任意给定的$\varepsilon>0$, 存在$\delta_{M}>0$使得当$0<|x-x_0|<\delta_{M}$时, $|f(x)|\leqslant M$ 并且$|f(x)-a|<\dfrac{\varepsilon}{M+|a|}$. 从而对于上面给定的$\varepsilon$, 取$\delta=\min\{\delta_{+},\delta_{-},\delta_{M}\}$, 则当$0<|x-x_0|<\delta$时, 
	\[|f^2\mathrm{sgn}f(x)-a^2\mathrm{sgn}a|\leqslant|f^2-a^2|<\varepsilon.\]
即$\lim\limits_{x\to x_0}f^2(x)\mathrm{sgn} f(x)=a^2\mathrm{sgn} a$.
	
	\item 若$a=0$, 由于$\lim\limits_{x\to x_0}f(x)=0$知对于任意给定的$\varepsilon>0$, 存在相应的$\delta>0$使得当$0<|x-x_0|<\delta$时有$|f(x)|<\varepsilon^3$, 从而
\[ |\sqrt[3]{f(x)}|\leqslant\sqrt[3]{|f(x)|}<\sqrt[3]{\varepsilon}=\varepsilon.\]
	若$a\neq 0$, 由上题(2)即可得证. 从而$\lim\limits_{x\to x_0}\sqrt[3]{f(x)}=\sqrt[3]{a}$
	\end{enumerate}

\item 应用函数极限的四则运算规律求以下极限($m,n$均表示自然数):
	\begin{enumerate}[(1)]
	\item $\lim\limits_{x\to 0} \dfrac{(x-1)^3-2x-1}{x^3+x-2}$;
	\item $\lim\limits_{x\to 1} \dfrac{x^3-1}{x^2-3x+2}$;
	\item $\lim\limits_{x\to 0} \dfrac{(1+x)(1+2x)(1+3x)-1}{2x^3+x^2}$;
	\item $\lim\limits_{x\to -1} \dfrac{(2+x)^4-(5+4x)}{(x+1)^2(x^2+2x+3)}$;
	\item $\lim\limits_{x\to 1} \dfrac{x^3-3x+2}{x^4-4x+3}$;
	\item $\lim\limits_{x\to -1} \dfrac{x^3-3x-2}{x^5-2x-1}$;
	\item $\lim\limits_{x\to 4} \dfrac{\sqrt{1+2x}-3}{\sqrt{x}-2}$;
	\item $\lim\limits_{x\to -1} \dfrac{2-\sqrt{x+5}}{1+\sqrt[3]{x}}$;
	\item $\lim\limits_{x\to 1} \dfrac{\sqrt{2-x}-\sqrt{x}}{\sqrt[3]{2-x}-\sqrt[3]{x}}$;
	\item $\lim\limits_{x\to a} \dfrac{\sqrt{x-a}+\sqrt{x}-\sqrt{a}}{\sqrt{x^2-a^2}}$;
	\item $\lim\limits_{x\to 1} \dfrac{x+x^2+\cdots+x^n-n}{x-1}$;
	\item $\lim\limits_{x\to 1} \dfrac{x^m-1}{x^n-1}$;
	\item $\lim\limits_{x\to 1} \dfrac{x^{n+1}-(n-1)x+n}{(x-1)^2}$;
	\item $\lim\limits_{x\to 1} \left(\dfrac{m}{x^m-1}-\dfrac{n}{x^n-1}\right)$;
	\end{enumerate}
{\heiti 解}
	\begin{enumerate}[(1)]
	\item $\lim\limits_{x\to 0} \dfrac{(x-1)^3-2x-1}{x^3+x-2}=\lim\limits_{x\to 0} \dfrac{x^3-3x^2+x-2}{x^3+x-2}= \dfrac{-2}{-2}=1$.
	\item $\lim\limits_{x\to 1} \dfrac{x^3-1}{x^2+3x+2}=\lim\limits_{x\to 1} \dfrac{(x-1)(x^2+x+1)}{(x-1)(x+2)}=\lim\limits_{x\to 1} \dfrac{x^2+x+1}{x-2}=\dfrac{3}{-1}=-3$.
	\item $\lim\limits_{x\to 0} \dfrac{(1+x)(1+2x)(1-3x)-1}{2x^3+x^2}=\lim\limits_{x\to 0} -\dfrac{6x^3+7x^2}{2x^3+x^2}=\lim\limits_{x\to 0} -\dfrac{6x+7}{2x+1}=-7$.
	\item $\lim\limits_{x\to -1} \dfrac{(2+x)^4-(5+4x)}{(x+1)^2(x^2+2x+3)}=\lim\limits_{x\to -1} \dfrac{(x+1)^4+4(x+1)^3+6(x+1)^2}{(x+1)^2(x^2+2x+3)}=\lim\limits_{x\to-1} \dfrac{(x+1)^2+4(x+1)+6}{x^2+2x+3}= \dfrac{6}{2}=3$.
	\item $\lim\limits_{x\to 1} \dfrac{x^3-3x+2}{x^4-4x+3}=\lim\limits_{x\to 1} \dfrac{(x-1)^3+3(x-1)^2}{(x-1)^4+4(x-1)^3+6(x-1)^2}=\lim\limits_{x\to 1} \dfrac{(x-1)+3}{(x-1)^2+4(x-1)+6}=\dfrac{3}{6}=\dfrac{1}{2}$.
	\item $\lim\limits_{x\to -1} \dfrac{x^3-3x-2}{x^5-2x-1}=\lim\limits_{x\to -1} \dfrac{(x+1)^3-3(x+1)^2}{(x+1)^4-5(x+1)^3+10(x+1)^2-10(x+1)+3}=\dfrac{0}{3}=0$.
	\item $\lim\limits_{x\to 4} \dfrac{\sqrt{1+2x}-3}{\sqrt{x}-2}=\lim\limits_{x\to 4} \dfrac{(1+2x-9)(\sqrt{x}+2)}{(x-4)(\sqrt{1+2x}+3)}=\lim\limits_{x\to 4} \dfrac{2(\sqrt{x}+2)}{\sqrt{1+2x}+3}=\dfrac{4}{3}$.
	\item $\lim\limits_{x\to -1} \dfrac{2-\sqrt{x+5}}{1+\sqrt[3]{x}}=\lim\limits_{x\to -1} \dfrac{[4-(x+5)](1-\sqrt[3]{x}+\sqrt[3]{x^2})}{(1+x)(2+\sqrt{x+5})}=-\lim\limits_{x\to -1} \dfrac{1-\sqrt[3]{x}+\sqrt[3]{x^2}}{2+\sqrt{x+5}}= -\dfrac{3}{4}$.
	\item $\lim\limits_{x\to 1} \dfrac{\sqrt{2-x}-\sqrt{x}}{\sqrt[3]{2-x}-\sqrt[3]{x}}=\lim\limits_{x\to 1} \dfrac{(2-x)-x}{(2-x)-x}\dfrac{\sqrt[3]{(2-x)^2}+\sqrt[3]{x(2-x)}+\sqrt[3]{x^2}}{\sqrt{2-x}+\sqrt{x}}=\dfrac{3}{2}$.
	\item $\lim\limits_{x\to a} \dfrac{\sqrt{x-a}+\sqrt{x}-\sqrt{a}}{\sqrt{x^2-a^2}}=\lim\limits_{x\to a}\left(\dfrac{1}{\sqrt{x+a}}+\dfrac{x-a}{(x+a)\sqrt{x-a}}\right)=\lim\limits_{x\to a}\left(\dfrac{1}{\sqrt{x+a}}+\dfrac{\sqrt{x-a}}{x+a}\right)=\dfrac{1}{\sqrt{2a}}$.
	\item $\lim\limits_{x\to 1} \dfrac{x+x^2+\cdots+x^n-n}{x-1}=\lim\limits_{x\to 1}\left(\dfrac{x-1}{x-1}+\dfrac{x^2-1}{x-1}+\cdots+\dfrac{x^n-1}{x-1}\right)=1+2+\cdots+n=\dfrac{n(n+1)}{2}$.
	\item $\lim\limits_{x\to 1} \dfrac{x^m-1}{x^n-1}=\lim\limits_{x\to 1}\dfrac{x-1}{x-1}\dfrac{x^{m-1}+\cdots+x+1}{x^{n-1}+\cdots+x+1}=\dfrac{m}{n}$.
	\item $\lim\limits_{x\to 1} \dfrac{x^{n+1}-(n-1)x+n}{(x-1)^2}=\lim\limits_{x\to 1} \dfrac{(x^{n+1}-1)-(n+1)x+(n+1)}{(x-1)^2} =\lim\limits_{x\to 1} \dfrac{(x^n+\cdots+x+1)-(n+1)}{x-1}=\lim\limits_{x\to 1} \dfrac{(x^n+\cdots+x^2+x)-n}{x-1}=\dfrac{n(n+1)}{2}$.
	\item 
	\end{enumerate}

\item 已知$\lim\limits_{x\to x_0}f(x)=a, \lim\limits_{x\to x_0}g(x)=b$. 证明: 
\[\lim\limits_{x\to x_0}\max\{f(x), g(x)\}=\max\{a,b\}, \lim\limits_{x\to x_0}\min\{f(x),g(x)\}=\min\{a,b\}.\]
{\heiti 证明}
显然当$a=b$时成立, 当$a\neq b$时, 不妨设$a>b$. 由于$\lim\limits_{x\to x_0}f(x)=a$, $\lim\limits_{x\to x_0}g(x)=b$, 断言存在$\delta>0$, 使得当$0<|x-x_0|<\delta$时有$f(x)>g(x)$: 事实上, 对于$\varepsilon=\dfrac{a-b}{2}>0$, 由于$\lim\limits_{x\to x_0}f(x)=a$, 存在$\delta_1>0$使得当$0<|x-x_0|<\delta_1$时有$|f(x)-a|<\dfrac{a-b}{2}$, 即$f(x)>a-\dfrac{a-b}{2}=\dfrac{a+b}{2}$; 另一方面, 由于$\lim\limits_{x\to x_0}g(x)=b$, 存在$\delta_2>0$使得当$0<|x-x_0|<\delta_2$时有$|g(x)-b|<\dfrac{a-b}{2}$, 即$g(x)<b+\dfrac{a-b}{2}=\dfrac{a+b}{2}$. 从而令$\delta=\min\{\delta_1,\delta_2\}$, 则当$0<|x-x_0|<\delta$时有$g(x)<\dfrac{a+b}{2}<f(x)$. 于是在$x_0$的$\delta$领域内, 
\[\max\{f(x),g(x)\}=f(x), \min\{f(x),g(x)\}=g(x), \max\{a,b\}=a, \min\{a,b\}=b. \]
从而由于$\lim\limits_{x\to x_0}f(x)=a$, $\lim\limits_{x\to x_0}g(x)=b$, 知
\begin{equation*}
\begin{split}
\lim\limits_{x\to x_0}\max\{f(x), g(x)\}&=\lim\limits_{x\to x_0} f(x)=a=\max\{a,b\},\\ 
\lim\limits_{x\to x_0}\min\{f(x),g(x)\}&=\lim\limits_{x\to x_0} g(x)=b=\min\{a,b\}.
\end{split}
\end{equation*}

\end{enumerate}

\clearpage
\subsection{函数的极限(续)}
\begin{enumerate}[1.]
\item 求以下单侧极限:
	\begin{multicols}{2}
	\begin{enumerate}[(1)]
	\item $\lim\limits_{x\to 0^+} \dfrac{x-1}{3x-2\sqrt{x}-1}$;
	\item $\lim\limits_{x\to 1^-} \dfrac{\sqrt[3]{1-x}-\sqrt[4]{1-x}}{\sqrt[3]{1-x}+3\sqrt[4]{1-x}}$;
	\item $\lim\limits_{x\to 1^+} \dfrac{[3x]}{x+2}$;
	\item $\lim\limits_{x\to 1^-} \dfrac{[3x]}{x+2}$;
	\item $\lim\limits_{x\to 2^+} \dfrac{[x]^2-1}{x^2-1}$;
	\item $\lim\limits_{x\to 2^-} \dfrac{[x]^2-1}{x^2-1}$;
	\item $\lim\limits_{x\to 2^+} \arctan \dfrac{\sqrt{x-1}}{x-2}$;
	\item $\lim\limits_{x\to 2^-} \arctan \dfrac{\sqrt{x-1}}{x-2}$;
	\item $\lim\limits_{x\to 0^+} \dfrac{1}{1+2^{\frac{1}{x}}}$;
	\item $\lim\limits_{x\to 0^-} \dfrac{1}{1+2^{\frac{1}{x}}}$.
	\end{enumerate}
	\end{multicols}
{\heiti 解}
	\begin{enumerate}[(1)]
	\item $\lim\limits_{x\to 0^+} \dfrac{x-1}{3x-2\sqrt{x}-1}=\lim\limits_{x\to 0^+} \dfrac{(\sqrt{x}+1)(\sqrt{x}-1)}{(3\sqrt{x}+1)(\sqrt{x}-1)}=\lim\limits_{x\to 0^+} \dfrac{\sqrt{x}+1}{3\sqrt{x}+1}=\dfrac{1}{1}=1$.
	\item $\lim\limits_{x\to 1^-} \dfrac{\sqrt[3]{1-x}-\sqrt[4]{1-x}}{\sqrt[3]{1-x}+3\sqrt[4]{1-x}}\xlongequal{y=\sqrt[12]{1-x}}\lim\limits_{y\to 0^+}\dfrac{y^4-y^3}{y^4+3y^3}=\lim\limits_{y\to 0^+}\dfrac{y-1}{y+3}=-\dfrac{1}{3}$. 
	\item $\lim\limits_{x\to 1^+} \dfrac{[3x]}{x+2}=1$.
	\item $\lim\limits_{x\to 1^-} \dfrac{[3x]}{x+2}=\dfrac{2}{3}$.
	\item $\lim\limits_{x\to 2^+} \dfrac{[x]^2-1}{x^2-1}=1$.
	\item $\lim\limits_{x\to 2^-} \dfrac{[x]^2-1}{x^2-1}=0$.
	\item $\lim\limits_{x\to 2^+} \arctan \dfrac{\sqrt{x-1}}{x-2}=\dfrac{\pi}{2}$.
	\item $\lim\limits_{x\to 2^-} \arctan \dfrac{\sqrt{x-1}}{x-2}=-\dfrac{\pi}{2}$.
  	\item $\lim\limits_{x\to 0^+} \dfrac{1}{1+2^{\frac{1}{x}}}=0$.
	\item $\lim\limits_{x\to 0^-} \dfrac{1}{1+2^{\frac{1}{x}}}=1$.
	\end{enumerate}


\item 求以下无穷远处的极限:
	\begin{multicols}{2}
	\begin{enumerate}[(1)]
	\item $\lim\limits_{x\to +\infty}(\sqrt{(x+a)(x+b)}-x)$;
	\item $\lim\limits_{x\to +\infty}(\sqrt[3]{x^3+3x^2}-\sqrt{x^-2x})$;
	\item $\lim\limits_{x\to +\infty}(\sqrt{x+\sqrt{x+\sqrt{x}}}-\sqrt{x-\sqrt{x+\sqrt{x}}})$;
	\item $\lim\limits_{x\to\infty} \dfrac{(x+\sqrt{x^2-2x})^n+(x-\sqrt{x^2-2x})^n}{\sqrt[3]{x^{3n}+1}+\sqrt[3]{x^3n-1}}$;
	\item $\lim\limits_{x\to\infty} x^{\frac{1}{3}}[(x+1)^{\frac{2}{3}}-(x-1)^{\frac{2}{3}}]$;
	\item $\lim\limits_{x\to\infty} \arcsin{\dfrac{1-x}{1+x}}$;
	\item $\lim\limits_{x\to+\infty}(\sin{\sqrt{x+1}}-\sin{\sqrt{x}})$;
	\item $\lim\limits_{x\to+\infty}\arccos{(\sqrt{x^2+x}-x)}$;
	\end{enumerate}
	\end{multicols}
{\heiti 解}
	\begin{enumerate}[(1)]
	\item 
		\begin{flalign*}
		&\lim\limits_{x\to +\infty}(\sqrt{(x+a)(x+b)}-x)=\lim\limits_{x\to +\infty} \dfrac{(x+a)(x+b)-x^2}{\sqrt{(x+a)(x+b)}+x}& \\
		&=\lim\limits_{x\to +\infty} \dfrac{(a+b)x+ab}{\sqrt{(x+a)(x+b)}+x}=\lim\limits_{x\to +\infty} \dfrac{(a+b)+\dfrac{ab}{x}}{\sqrt{1+\dfrac{a+b}{x}+\dfrac{ab}{x^2}}+1}=a+b.& 
		\end{flalign*}
	\item 
		\begin{flalign*}
		&\lim\limits_{x\to +\infty}(\sqrt[3]{x^3+3x^2}-\sqrt{x^-2x})=\lim\limits_{x\to+\infty} \left( \dfrac{3x^2}{(\sqrt[3]{x^3+3x^2})^2+x\sqrt[3]{x^3+3x^2}+x^2}+\dfrac{2x}{\sqrt{x^2-2x}+x}\right)& \\
		&=\lim\limits_{x\to +\infty} \dfrac{3}{\sqrt[3]{\left(\dfrac{x^3+3x^2}{x^3}\right)^2}+\sqrt[3]{\left(\dfrac{x^3+3x^2}{x^3}\right)}+1}+\lim\limits_{x\to+\infty} \dfrac{2}{\sqrt{\dfrac{x^2-2x}{x^2}}+1}=1+1=2&
		\end{flalign*}
	\item 
		
	\end{enumerate}

\item 根据极限~$\lim\limits_{x\to 0} \dfrac{\sin{x}}{x}=1$~求以下极限:
	\begin{multicols}{2}
	\begin{enumerate}[(1)]
	\item $\lim\limits_{x\to 0} \dfrac{\sin{ax}}{\sin{bx}} (a,b\neq 0)$;
	\item $\lim\limits_{x\to 0} \dfrac{\tan{ax}}{\tan{bx}} (a,b\neq 0)$;
	\item $\lim\limits_{x\to 0} \dfrac{\sin{3x}-\sin{2x}}{\sin{5x}}$;
	\item $\lim\limits_{x\to 0} \dfrac{1-\cos{x}}{x^2}$;
	\item $\lim\limits_{x\to 0} \dfrac{\tan{x}-\sin{x}}{\sin^3{x}}$;
	\item $\lim\limits_{x\to 0} \dfrac{\cos{3x}-\cos{2x}}{\sin{x^2}}$;
	\item $\lim\limits_{x\to \frac{\pi}{4}} \tan{2x}tan{(\dfrac{\pi}{4}-x)}$;
	\item $\lim\limits_{x\to 1} (1-x)\tan{\dfrac{\pi x}{2}}$;
	\item $\lim\limits_{x\to 0} \dfrac{\sqrt{1-\cos{x^2}}}{1-\cos{x}}$;
	\item $\lim\limits_{x\to 0^+} \dfrac{1-\sqrt{\cos{x}}}{(1-\cos{\sqrt{x}})^2}$;
	\item $\lim\limits_{x\to 0} \dfrac{\arctan{x}}{\arcsin{x}}$;
	\item $\lim\limits_{n\to\infty} \cos{\dfrac{x}{2}}\cos{\dfrac{x}{4}}\cdots\cos{\dfrac{x}{2^n}}$.
	\end{enumerate}
	\end{multicols}

\item 根据极限~$\lim\limits_{x\to0}(1+x)^{\frac{1}{x}}=\e$~求以下极限 (必要时需作变量变换并使用对数法): 
	\begin{multicols}{2}
	\begin{enumerate}[(1)]
	\item $\lim\limits_{x\to 0}(1-2x)^{\frac{1}{x}}$; 
	\item $\lim\limits_{x\to 0}(\e^x+x)^{\frac{1}{x}}$; 
	\item $\lim\limits_{x\to\infty}\left(\dfrac{x+a}{x-a}\right)^x$; 
	\item $\lim\limits_{x\to\infty}\left(\dfrac{x^2+1}{x^2-1}\right)^x$; 
	\item $\lim\limits_{x\to 0} \dfrac{\ln(x^2+\e^x)}{\ln(x^3+\e^{2x})}$; 
	\item $\lim\limits_{x\to 0} \dfrac{\ln(1+x\e^x)}{\ln(x+\sqrt{1+x^2})}$; 
	\item $\lim\limits_{x\to 0}(2\e^x-1)^{\frac{1}{x}}$; 
	\item $\lim\limits_{x\to 0}(2\e^{\frac{x}{1+x}}-1)^{\frac{1+x^2}{x}}$; 
	\item $\lim\limits_{x\to -\infty} \dfrac{\ln(1+3^x)}{\ln(1+2^x)}$; 
	\item $\lim\limits_{x\to +\infty} \dfrac{\ln(1+3^x)}{\ln(1+2^x)}$; 
	\item $\lim\limits_{x\to +\infty} \ln(1+2^x)\ln(1+\dfrac{3}{x})$; 
	\item $\lim\limits_{x\to 1} (1-x)\log_x2$.
	\end{enumerate}
	\end{multicols} 

\item 求以下综合类型的极限: 
	\begin{multicols}{2}
	\begin{enumerate}[(1)]
	\item 
	\end{enumerate}
	\end{multicols}
\item 设函数$f$在$(a,+\infty)$上单增. 证明:
	\begin{enumerate}[(1)]
	\item 如果存在数列$\{x_n\}$使$\lim\limits_{n\to\infty}x_n=+\infty$且$\lim\limits_{n\to\infty}f(x_n)=a$, 则$\lim\limits_{x\to+\infty}f(x)=a$;
	\item 如果$f$严格单增, $\lim\limits_{x\to+\infty}f(x)=a$且$\lim\limits_{n\to\infty}f(x_n)=a$, 则$\lim\limits_{n\to\infty}x_n=+\infty$.
	\end{enumerate}

{\heiti 证明}
	若存在这样的数列, 由于$\lim\limits_{n\to\infty}x_n=+\infty$, 于是从有限多项后数列的每一项都落在区间$(a,+\infty)$上, 由于数列自身的性质, 这两个数列在敛散性上完全相同, 因此不加区分地也将这个去掉前面有限多项的数列记为$\{x_n\}$.
	\begin{enumerate}[(1)]
	\item  断言对于任意的$n\in\mathbb{N}^+$, $f(x_n)\leqslant a$: 若存在$N_0\in\mathbb{N}^+$, $f(x_{N_0})>a$, 则对于$\varepsilon=f(x_{N_0})-a>0$, 由于$f(x_n)\to a (n\to\infty)$知存在$N_1\in\mathbb{N}^+$, 使得当$n>N_1$时有$|f(x_n)-a|<\varepsilon$; 又由于$x_n\to +\infty (n\to\infty)$, 则对于$x_{N_0}$, 存在$N_2\in\mathbb{N}^+$, 使得当$n>N_2$时有$x_n>x_{N_0}$, 由于函数$f$在$(a,+\infty)$上单增, 当$n>N_2$时有$f(x_n)\geqslant f(x_{N_0})$, 进而当$n>\max\{N_1,N_2\}$时, $f(x_n)-a\geqslant f(x_{N_0})-a =\varepsilon$, 这与$\lim\limits_{n\to\infty}f(x_n)=a$矛盾. 
		
		对于任意$x>a$, 由$\lim\limits_{n\to\infty}x_n=+\infty$知存在$N_x\in\mathbb{N}^+$, 使得当$n>N_x$时有$x_n>x$, 故$f(x)\leqslant f(x_n)\leqslant a$. 对于任意$\varepsilon>0$, 由$\lim\limits_{n\to\infty}f(x_n)=a$, 知存在$N>0$使得当$n>N$时, $f(x_n)>a-\varepsilon$, 则当$x>x_{N+1}$时, $f(x)\geqslant f(x_{N+1})>a-\varepsilon$. 故对于上面的$\varepsilon>0$ 取$M=\max\{a,x_{N+1}\}$, 则当$x>M$时就有
	\[ a-\varepsilon<f(x_{N+1})\leqslant f(x)\leqslant f(x_{N_x+1}) \leqslant a < a+\varepsilon. \]
即$\lim\limits_{x\to+\infty} f(x)=a$.
	   
	\item 断言$f(x)<a$: 首先证明$f(x)\leqslant a$, 否则存在$f(x_0)>a$, 对于$\varepsilon=f(x_0)-a>0$, 当$x>x_0$时, $f(x)>f(x_0)$, 即$f(x)-a>f(x_0)-a=\varepsilon$, 这与$\lim\limits_{x\to+\infty}f(x)=a$矛盾; 在证明$f(x)\neq a$, 否则存在$f(x_1)=a$, 当$x>x_1$时, 有$f(x)>f(x_1)=a$, 这与前面已证明的$f(x)\leqslant a$矛盾. 

	对于任意$M>0$, $f(M)<a$, 取$\varepsilon=a-f(M)>0$, 由$\lim\limits_{n\to\infty}f(x_n)=a$, 知存在$N>0$使得当$n>N$时, $f(x_n)>a-\varepsilon=f(M)$. 由于$f$在$(a,+\infty)$上严格单调递增, $f$的反函数$f^{-1}$存在, 并且$f^{-1}$也是$(a,+\infty)$上的严格单调递增函数. 上式左右取反函数, 当$n>N$时, 
	\[ x_n=f^{-1}\circ f(x_n)> f^{-1}\circ f(M)=M,\]
即$\lim\limits_{n\to\infty}x_n=+\infty$.
	\end{enumerate}

\item 定义在区间~$(a,+\infty)$~上的函数~$f$~称为是{\bf 渐进~$T$~周期的}, 其中~$T$~是正常数, 如果存在~$T$~周期函数~$g$~使成立
	\[ \lim\limits_{x\to+\infty} (f(x)-g(x))=0.\]
证明: ~$f$~是渐进~$T$~周期函数的充要条件是成立
	\[ \lim\limits_{m,n\to\infty} [f(x+mT)-f(x+nT)]=0, \quad \forall x>a. \]
 
{\heiti 证明}\\
	必要性: ~$f$~是渐进~$T$~周期函数, 故存在~$T$~周期函数~$g$~使成立
	\[ \lim\limits_{x\to+\infty} (f(x)-g(x))=0.\]
	对于任意~$\varepsilon>0$, 存在相应的~$M>0$, 使得当~$x>M$~时, $|f(x)-g(x)|<\dfrac{\varepsilon}{2}$. 对于给定的~$\varepsilon>0$, 当~$m,n>N=\dfrac{M}{T}$~时, $x+mT,x+nT>M$, 
	{\setlength\abovedisplayskip{1pt}
	\setlength\belowdisplayskip{1pt}
	\begin{equation*}
		\begin{split}
			&|f(x+mT)-f(x+nT)|\\
			=&|[f(x+mT)-g(x)]+[g(x)-f(x+nT)]|\\
			\leqslant & |f(x+mT)-g(x)|+|f(x+nT)-g(x)|\\
			=&|f(x+mT)-g(x+mT)|+|f(x+nT)-g(x+nT)|\\
			<&\dfrac{\varepsilon}{2}+\dfrac{\varepsilon}{2}\\
			=&\varepsilon.
		\end{split}
	\end{equation*}
	}
	即$\lim\limits_{m,n\to\infty} [f(x+mT)-f(x+nT)]=0, \quad \forall x>a.$ 其中第二个``=''是由于~$g(x)$~是~$T$~周期函数, 故~$g(x)=g(x+mT)=g(x+nT)$.

	充分性: ~$\lim\limits_{m,n\to\infty} [f(x+mT)-f(x+nT)]=0, \quad \forall x>a$. 故对于任意的~$\varepsilon>0$, 存在相应~$\bar{N}_x>0$, 当~$m,n>\bar{N}_x$~时,
	\[ |f(x+mT)-f(x+nT)|<\varepsilon, \quad \forall x>a.\]
故由Cauchy收敛准则知对于~$\forall x>a$, 数列~$\{f(x+nT)\}$~收敛, 做映射
	{\setlength\abovedisplayskip{1pt}
	\setlength\belowdisplayskip{1pt}
	\begin{equation*}
		\begin{split}
			g: (a,+\infty) \quad &\mapsto \quad \mathbb{R}\\
			   x \quad &\to \quad g(x)=\lim\limits_{n\to\infty} f(x+nT)
		\end{split}
	\end{equation*}
	}
而~$g(x+T)=\lim\limits_{n\to\infty} f[(x+T)+nT]=\lim\limits_{n\to\infty} f[x+(n+1)T] =\lim\limits_{n\to\infty}f(x+nT)=g(x)$, 即~$g(x)$~是~$(a,+\infty)$~上的~$T$~周期函数. 

\item 证明{\bf Cauchy定理}: 设函数~$f$~定义在区间~$(a,+\infty)$~上, 并在每个有穷区间~$(a,b)$~上有界. 则当等式右端的极限存在时, 成立
	\begin{enumerate}[(1)]
	\item $\lim\limits_{x\to+\infty}  \dfrac{f(x)}{x}=\lim\limits_{x\to+\infty}[f(x+1)-f(x)]$;	
	\item $\lim\limits_{x\to+\infty} [f(x)]^{\frac{1}{x}}=\lim\limits_{x\to+\infty} \dfrac{f(x+1)}{f(x)}$.
	\end{enumerate}

\item 设函数~$f$~定义在区间~$(a,\infty)$~上, 并且在每个有穷区间~$(a,b)$~上有界. 又设
\[ \lim\limits_{x\to+\infty} \dfrac{f(x+1)-f(x)}{x^p}=c,\]
其中~$p$~为正常数. 证明:
\[ \lim\limits_{x\to+\infty} \dfrac{f(x)}{x^{p+1}}=\dfrac{c}{p+1}.\]
\end{enumerate}

\clearpage
\subsection{函数的连续性}
\begin{enumerate}[1.]
\item 已知~$f(x)$~,~$g(x)$~和~$h(x)$~都在区间~$I$~上连续. 证明下列函数也在区间~$I$~上连续:
	\begin{enumerate}[(1)]
	\item $a(x)=|f(x)|$;
	\item $m(x)=\min\{f(x),g(x)\}$;
	\item $M(x)=\max\{f(x),g(x)\}$;
	\item 函数~$u(x)$, 其定义是对每个$x\in I$, $u(x)$~的值等于~$f(x),g(x),h(x)$~三个数中位于另外两个中间的那个数;
	\item $f_c(x)=\begin{cases} f(x), &\text{当~$|f(x)|\leqslant c$,}\\ c, &\text{当~$f(x)>c$,}\\ -c, &\text{当~$f(x)<-c$.}\end{cases}$
	\end{enumerate}

\item 设~$f$~是定义在开区间~$I$~上的函数, $x_0$~是~$I$~中一点, $f$~在~$x_0$~点附近有界. 对充分小的~$\delta>0$, 令
\[ \omega_{f,x_0}(\delta)=\sup_{x,y\in B_{\delta}(x_0)}|f(x)-f(y)|,\]
这里~$B_{\delta}(x_0)=(x_0-\delta,x_0+\delta)$. $\omega_{f,x_0}(\delta)$~叫做~$f$~在~$B_{\delta}(x_0)$~上的{\bf 振幅}. 证明: 函数~$f$~在~$x_0$~点连续的充要条件是~$\lim\limits_{\delta\to0^+}\omega_{f,x_0}(\delta)=0$.

\item 设~$S$~ 是$\mathbb{R}$的非空子集. 定义点$x\in\mathbb{R}$到$S$的距离为
\[ d(x,S)=\mathrm{dist}(x,S)=\inf\{|x-y|:y\in S\}.\]
证明: 对$\mathbb{R}$的任意非空子集$S$, 函数$x\mapsto d(x,S)$都是$\mathbb{R}$上的连续函数.

\item 设~$f$~是定义在区间~$[a,+\infty)$~上的连续函数. 对每个~$x\geqslant a$, 令
\[ m(x)=\inf_{a\leqslant t\leqslant x}{f(t)}, \qquad M(x)=\sup_{a\leqslant t\leqslant x}{f(t)}.\]
证明: 函数~$m(x)$~和~$M(x)$~都在区间~$[a,+\infty)$~上连续.

\item 证明: 非常数的连续周期函数必有最小正周期.

\item 证明: 单调函数最多只有第一类间断点.

\item 定义在开区间~$I$~上的函数~$f$~称为{\bf 凸函数}, 如果对任意~$x,y\in I$~和任意~$0<\theta<1$~都成立不等式
\[ f(\theta x+(1-\theta)y)\leqslant \theta f(x)+(1-\theta)f(y).\]
 在几何上, 这意味着如果~$A,B,C$~是曲线~$y=f(x)$~上的三个点并且~$B$~位于~$A$~和~$C$~之间, 则~$B$~位于弦~$AC$~上或~$AC$~的下方. 证明: 凸函数都是连续函数. 

\item 设~$f$~是区间~$I$~上的连续函数, 满足以下条件: 对任意~$x,y\in I$~都成立不等式
\[ f\left(\dfrac{x+y}{2}\right)\leqslant\dfrac{f(x)+f(y)}{2}. \]
证明: $f$~是区间~$I$~上的凸函数. 

\item 设~$I$~是一个闭区间, 即~$I$~是四种区间~$[a,b], [a,+\infty), (-\infty,b], (-\infty,+\infty)$~之一. 又设~$f$~是定义在~$I$~上的函数, 满足以下两个条件:
	\begin{enumerate}[(1)]
	\item $f$~的值域含于~$I$, 即~$f$~把区间~$I$~映照进~$I$;
	\item 存在常数~$0<\lambda<1$~使对任意~$x,y\in I$~都成立~$|f(x)-f(y)|\leqslant\lambda|x-y|$.
	\end{enumerate}
任取~$x_0\in I$, 按以下递推公式构作数列~$\{x_n\}$:
\[ x_n=f(x_{n-1}), \quad n=1,2,\cdots. \]
证明: 数列~$\{x_n\}$~收敛, 并且其极限~$\bar{x}=\lim\limits_{n\to\infty}x_n$~是方程~$f(x)=x$~在~$I$~中的唯一根.
\end{enumerate}

\clearpage
\subsection{连续函数的性质}
\begin{enumerate}[1.]
\item 举例说明:
	\begin{enumerate}[(1)]
	\item 开区间上的连续函数不一定有界;
	\item 不连续的函数, 即使定义在闭区间上, 也不一定有界;
	\item 开区间上的连续函数, 即使有界, 也不一定达到最大值和最小值; 
	\item 不连续的函数, 即使定义在闭区间上且有界, 也不一定达到最大值和最小值;
	\item 不连续的函数, 即使定义在闭区间上且变号, 也不一定有零点.
	\end{enumerate}

\item 设~$I$~是开区间, $f$~是~$I$~上的连续函数. 令~$m=\inf{x\in I}{f(x)}, M=\sup_{x\in I}{f(x)}$. 规定当~$f$~无下界时~$m=-\infty$, 当~$f$~无上界时~$M=+\infty$. 证明: 对任意~$m<c<M$, 必存在~$\xi\in I$~使~$f(\xi)=c$. 

\item 证明: 奇数次的实系数代数方程必有实数根. \\
{\heiti 证明}\\
设
\[f(x)=a_nx^n+a_{n-1}x^{n-1}+\cdots+a_1x+a_0 \quad\text{($a_n\neq 0$, $n$~为奇数)}.\]
则有
\[ \lim\limits_{x\to+\infty}\dfrac{f(x)}{a_nx^n}=\lim\limits_{x\to+\infty}(1+\dfrac{a_{n-1}}{a_n}\dfrac{1}{x}+\cdots+\dfrac{a_1}{a_n}\dfrac{1}{x^{n-1}}+\dfrac{a_0}{a_n}\dfrac{1}{x^n})=1,\]
故~$\lim\limits_{x\to+\infty} f(x)=\lim\limits_{x\to+\infty} a_nx^n$. 因此当~$a_n>0$时, 由于~$n$~为奇数,  $\lim\limits_{x\to+\infty}f(x)=+\infty$, $\lim\limits_{x\to-\infty}f(x)=-\infty$. 于是对于任意~$M>0$, 存在~$x_M>0$~使得当~$x\geqslant x_M$~时成立~$f(x)>M>0$; 对于任意~$N<0$, 存在~$x_N<0$~使得当~$x\leqslant x_N$~时成立$f(x)<N<0$. 在区间~$[x_N,x_M]$~上使用零值定理, 则存在~$\xi\in(x_N,x_M)$~使得~$f(\xi)=0$, 于是~$\xi$~就是奇数次实系数代数方程~$f(x)=0$~的一个实数根.

\item 设~$f$~是区间~$[a,b]$~上的连续函数, 且值域含于~$[a,b]$~, 证明: $f$~有不动点, 即存在~$\bar{x}\in[a,b]$~使~$f(\bar{x})=\bar{x}$.\\
{\heiti 证明}\\
考虑辅助函数~$F(x)=f(x)-x$. 由于~$f([a,b])\subseteq [a,b]$, 故~$a\leqslant f(a)\leqslant b$, $a\leqslant f(b)\leqslant b$. 于是~$f(a)-a\geqslant 0$, $f(b)-b\leqslant 0$. 如果上面两个不等式等号至少有一个成立, 则~$a$~或~$b$~就是~$f$~的不动点; 若不等号均不成立, 则~$F(a)>0$, $F(b)<0$, 在闭区间~$[a,b]$~上使用零值定理, 则存在~$\bar{x}\in(a,b)$~使得~$F(\bar{x})=0$, 即~$\bar{x}$~是~$f$~在~$[a,b]$~上的不动点.

\item 设~$f$~是区间~$[0,1]$~上的非负连续函数, 且~$f(0)=f(1)=0$. 证明: 对任意~$0<l<1$, 存在~$x_0\in[0,1-l]$~使~$f(x_0)=f(x_0+l)$.\\
{\heiti 证明}\\
考虑辅助函数~$F(x)=f(x)-f(x+l)$. $f$~是~$[0,1]$~上的非负连续函数, 则~$f(l)\geqslant0$, $f(1-l)\geqslant0$. 故~$f(0)-f(l)\leqslant 0$, $f(1-l)-f(1)\geqslant 0$. 若~$f(0)-f(l)=0$~或~$f(1-l)-f(1)=0$, 令~$x_0=0$~或~$x_0=1-l$~即可; 若~$f(0)-f(l)\neq 0$, $f(1-l)-f(1)\neq 0$, 则~$F(0)<0$, $F(1-l)>0$, 关于~$F$~在区间~$[0,1-l]$~上使用零值定理, 则存在~$x_0\in(0,1-l)$~使得~$F(x_0)=0$, 即~$f(x_0)=f(x_0+l)$.

\item 设~$I$~是一个区间, $f$~是~$I$~上的连续函数. 证明: 对~$I$~中的任意有限个点~$x_1, x_2, \cdots, x_n$, 必存在~$\xi\in I$~使
\[ f(\xi)=\dfrac{1}{n}[f(x_1)+f(x_2)+\cdots+f(x_n)]. \]
{\heiti 证明}\\
设~$f(x_M)=\max\limits_{1\leqslant k\leqslant n}f(x_k), f(x_m)=\min\limits_{1\leqslant k\leqslant n}f(x_k)$, $1\leqslant m,M\leqslant n$. 不妨设$x_M\leqslant x_m$, 则在闭区间~$[x_M,x_m]$~上(若~$x_m\leqslant x_M$~则在闭区间~$[x_m.x_M]$~上), 由于
\[f(x_m)\leqslant\dfrac{1}{n}[f(x_1)+f(x_2)+\cdots+f(x_n)]\leqslant f(x_M), \]
关于~$f$~使用介值定理, 则存在~$\xi\in[x_M,x_n]\subseteq I$~使得~$f(\xi)=\dfrac{1}{n}[f(x_1)+f(x_2)+\cdots+f(x_n)]$.

\item 设函数~$f$~在~$(-\infty,+\infty)$~上连续, 且~$\lim\limits_{|x|\to\infty}f(x)=+\infty$. 证明: 存在~$x_0\in\mathbb{R}$~使~$f(x_0)\leqslant f(x)$, $\forall x\in\mathbb{R}$.\\
{\heiti 证明}\\
由于~$\lim\limits_{|x|\to\infty}f(x)=+\infty$, 对于~$\forall c\in(-\infty,+\infty)$, 存在~$M>0$~使得当~$|x|>M$~时~$f(x)>f(c)$. 任取~$x_1,x_2\in{x: |x|>M}$~使得~$x_1<c<x_2$, 由于~$f(x)$~在~$[x_1,x_2]\subseteq(-\infty,+\infty)$~上连续, 故~$f$~在$[x_1,x_2]$~中存在最小值; 又由于~$c\in(x_1,x_2)$, 且~$f(c)<f(x_1), f(c)<f(x_2)$, 故最小值在~$(x_1,x_2)$~中取到, 设为~$f(x_0)$, 则对于~$x\in(-\infty, -M)\cup(M, +\infty)$, $f(x)>f(c)\leqslant f(x_0)$, 故~$f(x_0)$~是区间~$(-\infty, +\infty)$~上的最小值, 即
\[f(x_0)\leqslant f(x), \quad\forall x\in\mathbb{R}.\]

\item 设函数~$f$~在区间~$[a,b]$~上连续, 且存在~$0<\lambda<1$~使对任意~$x\in[a,b]$, 存在相应的~$y\in[a,b]$~使~$|f(y)|\leqslant\lambda|f(x)|$. 证明: $f$~在区间~$[a,b]$~上有零点.\\
{\heiti 证明}\\
对于$\forall x_0\in[a,b]$, 存在$x_1\in[a,b]$使得$|f(x_1)|\leqslant\lambda|f(x_0)|$; 同理存在$x_2\in[a,b]$使得$|f(x_2)|\leqslant\lambda|f(x_1)|$; 按照这一步骤下去, 应用数学归纳法, 或者在某有限步时得到了$x_i\in[a,b]$使得~$f(x_i)=0$, 或者得到了一个数列$\{f(x_n)\}, x_n\in[a,b], n=1,2,\cdots$: 一方面, 由于$x_n\in[a,b], n=1,2,\cdots$, 故$\{x_n\}$是有界数列, 存在收敛子列$\{x_{n_k}\}$使得$\lim\limits_{k\to\infty}x_{n_k}=\xi$, $\xi\in[a.b]$. 另一方面, $|f(x_n)|\leqslant \lambda|f(x_{n-1})|\leqslant\lambda^2|f(x_{n-2})|\leqslant\cdots\leqslant \lambda^n |f(x_0)|$. 于是对于任意的$\varepsilon>0$, 当$n>N=\left[\dfrac{\ln{\varepsilon/|f(x_0)}}{\ln{\lambda}}\right]+1$时, $|f(x_n)|\leqslant\lambda^n|f(x_0)<\varepsilon$, 即$\lim\limits_{n\to\infty}f(x_n)=0$, 从而$\lim\limits_{k\to\infty} f(x_{n_k})=0$. 由于$f$在区间$[a,b]$上连续, 特别地在$\xi\in[a,b]$一点连续, 于是$\lim\limits_{x\to\xi}f(x)=f(\xi)$. 由Heine定理, $f(\xi)=\lim\limits_{x\to\xi}=\lim\limits_{k\to\infty}f(x_{n_k})=0$, 即$\xi$是$f$在$[a,b]$上的零点.

\item 设函数~$f$~在开区间~$(a,b)$~上连续, 且有两列数$x_n, y_n\in(a,b) (n=1, 2, \cdots)$, 使~$\lim\limits_{n\to\infty}x_n=\lim\limits_{n\to\infty}y_n=a$, 且~$\lim\limits_{n\to\infty}f(x_n)=A, \lim\limits_{n\to\infty}f(y_n)=B$. 证明: 对~$A$~与~$B$~之间的任意数~$c$, 存在数列~$z_n\in(a,b)(n=1,2,\cdots)$, 使~$\lim\limits_{n\to\infty}z_n=a$~且~$\lim\limits_{n\to\infty}f(z_n)=c$.

\item 证明: 
	\begin{enumerate}[(1)]
	\item 如果~$f$~在区间~$I$~和~$J$~上都一致连续, 且~$I\cap J\neq\emptyset$, 则~$f$~也在~$I\cup J$~上一致连续;
	\item 设~$f$~在~$I$~上一致连续, $g$~在~$J$~上一致连续, 且~$f(I)\subseteq J$, 则~$g\circ f$~在~$I$~上一致连续.
	\end{enumerate}

\item 证明: 在~$(-\infty,+\infty)$~上连续的周期函数必在~$(-\infty,+\infty)$~上一致连续.

\item 设函数$f$在区间~$[a,+\infty)$~上连续, 且存在常数~$A$~使成立~$\lim\limits_{x\to+\infty}f(x)=A$. 证明: $f$~也在区间~$[a,+\infty)$~上一致连续.

\item 称函数~$f$~在区间~$I$~上{\bf 局部~$\mu$~阶H\"older连续}, 这里~$0<\mu\leqslant1$~是常数, 如果对每个~$x_0\in I$~都存在相应的~$\delta>0$~和~$C>0$, 使对任意~$x,y\in I\cap B_{\delta}(x_0)$~都有
\[ |f(x)-f(y)|\leqslant C|x-y|^{\mu}.\]
证明: 如果~$f$~在区间~$[a,b]$~上局部~$\mu$~阶H\"older连续, 则~$f$~也在此区间上{\bf 一致~$\mu$~阶H\"older连续}, 即存在常数~$C>0$~对于任意~$x,y\in I$~都成立
\[ |f(x)-f(y)|\leqslant C|x-y|^{\mu}.\]
\end{enumerate}

\section{一元函数微分学}
\subsection{导数的定义}
\begin{enumerate}[1.]
\item 根据导数的定义证明:
	\begin{enumerate}[(1)]
	\item $(\tan{x})'=\dfrac{1}{\cos^2{x}}$;	
	\item $(\cot{x})'=-\dfrac{1}{\sin^2{x}}$;
	\item $(\sqrt[n]{x})'=\dfrac{1}{n\sqrt[n]{n-1}}$;
	\item $(\arctan{x})'=\dfrac{1}{1+x^2}$;
	\item $(\arcsin{x})'=\dfrac{1}{\sqrt{1-x^2}} (|x|<1)$;
	\item $(a^x)'=a^x\ln{a} (a>0, a\neq1)$.
	\end{enumerate}
{\heiti 证明}
	\begin{flalign*}
		(\arcsin{x})'&=\lim\limits_{\Delta x\to0}\dfrac{\arcsin{(x+\Delta x)}-\arcsin{x}}{\Delta x}&\\
		&=\lim\limits_{\Delta x\to0}\dfrac{\arcsin{[(x+\Delta x)\sqrt{1-x^2}-x\sqrt{1-(x+\Delta x)^2}]}}{\Delta x}&\\
		&=\lim\limits_{\Delta x\to0}\dfrac{\arcsin{[(x+\Delta x)\sqrt{1-x^2}-x\sqrt{1-(x+\Delta x)^2}]}}{(x+\Delta x)\sqrt{1-x^2}-x\sqrt{1-(x+\Delta x)^2}}&\\
&\cdot\lim\limits_{\Delta x\to0}\dfrac{(x+\Delta x)^2(1-x^2)-x^2[1-(x+\Delta x)^2]}{x[(x+\Delta x)\sqrt{1-x^2}+x\sqrt{1-(x+\Delta x)^2}]}&\\
		&\text{作变换~$y=(x+\Delta x)\sqrt{1-x^2}-x\sqrt{1-(x+\Delta x)^2}$}&\\
		&=\lim\limits_{y\to0}\dfrac{\arcsin{y}}{y}\cdot\lim\limits_{\Delta x\to0} \dfrac{2x+\Delta x}{(x+\Delta x)\sqrt{1-x^2}+x\sqrt{1-(x+\Delta x)^2}}&\\
		&\text{作变换~
$z=\arcsin{y}$}&\\
		&=\lim\limits_{z\to0}\dfrac{z}{\sin{z}}\cdot \dfrac{2x}{2x\sqrt{1-x^2}}&\\
		&=\dfrac{1}{\sqrt{1-x^2}}&
	\end{flalign*}
\item 证明下列函数~$f(x)$~在点~$x=0$~可导, 并求~$f'(0)$:
	\begin{enumerate}[(1)]
	\item $f(x)=\begin{cases} x^2\sin{\dfrac{1}{x}}, &x\neq 0, \\ 0, &x=0;\end{cases}$
	\item $f(x)=\begin{cases} 2^{-\frac{1}{x}}, &x>0,\\ 0, &x\leqslant 0;\end{cases}$
	\item $f(x)=\begin{cases} x^2, &\text{当$x$是有理数}, \\ 0, &\text{当$x$是无理数}. \end{cases}$
	\end{enumerate}

\item 设~$a$~是正常数. 讨论下列函数在~$x=0$~处的连续性与可导性: 
	\begin{enumerate}[(1)]
	\item $f(x)=\begin{cases} |x|^{a-1}x, &x\neq 0,\\ 0 &x=0;\end{cases}$
	\item $f(x)=\begin{cases} |x|^a\sin{\dfrac{1}{x}}, &x\neq 0,\\ 0 &x=0;\end{cases}$
	\item $f(x)=\begin{cases} x^a, &x>0,\\0, &x\leqslant 0.\end{cases}$
	\end{enumerate}

\item 证明: 如果~$f(x)$~是偶函数, 且在点~$x=0$~可导, 则~$f'(0)=0$.

\item 证明: 如果~$f(x)$~在点~$x_0$~可导, 则对任意实数~$a,b$~都成立
\[ \lim\limits_{\Delta x\to 0} \dfrac{f(x_0+a\Delta x)-f(x_0+b\Delta x)}{\Delta x}=(a-b)f'(x_0).\]

\item 证明:
	\begin{enumerate}[(1)]
	\item 定义在区间~$(a,b)$~上的连续函数~$f$~在点~$x_0\in(a,b)$~可导的充要条件是函数
\[g(x)=\dfrac{f(x)-f(x_0)}{x-x_0}\]
在~$x_0$~点补充定义后是~$(a,b)$~上的连续函数;
	\item 可导的偶函数的导数是奇函数, 可导的奇函数的导数是偶函数;
	\item 可导的~$T$~周期函数的导数是~$T$~周期函数.
	\end{enumerate}

\item 设函数~$g$~在~$x=a$~点附近有定义并在该点连续且~$g(a)\neq 0$. 证明:
	\begin{enumerate}[(1)]
	\item 函数~$f(x)=(x-a)g(x)$~在点~$x=a$~可导;
	\item 函数~$f(x)=|x-a|g(x)$~在点~$x=a$~不可导, 但有左导数~$f'_{-}(a)$~和右导数~$f'_{+}(a)$.
	\end{enumerate}

\item 证明:
	\begin{enumerate}[(1)]
	\item 函数
	\[ f(x)=\begin{cases} x^2\left|\cos{\dfrac{\pi}{x}}\right|, &x\neq 0,\\ 0, &x=0.\end{cases} \]
在~$x=0$~点的任意邻域中都有不可导的点, 但它在该点可导;
	\item 函数
	\[ f(x)=\begin{cases} x\sin{\dfrac{\pi}{x}}, &x\neq 0,\\ 0, &x=0.\end{cases} \]
在点~$x=0$~连续, 但它在该点既无左导数, 又无右导数.
	\end{enumerate}

\item 设定义在区间~$[0,1)$~上的函数~$f$~在点~$x=0$~右连续, $f(0)=0$, 且对某常数~$a>1$成立
\[ \lim\limits_{x\to0^{+}}\dfrac{f(ax)-f(x)}{x}=c.\]
证明: $f$~在点~$x=0$~右可导, 且~$f'_{+}(0)=\dfrac{c}{a-1}$.

\item 设~$a_1, a_2, \cdots, a_n$~都是实数, 而
\[ f(x)=a_1\sin{x}+a_2\sin{2x}+\cdots+a_n\sin{nx}, \quad x\in\mathbb{R}. \]
假设已知对任意~$x\in\mathbb{R}$~都有~$|f(x)|\leqslant|\sin{x}|$, 证明: $|a_1+2a_2+\cdots+na_n|\leqslant 1$.\\
{\heiti 证明}\\
对任意~$x\in\mathbb{R}$~都有~$|f(x)|\leqslant|\sin{x}|$, 即~$-\sin{x}\leqslant f(x)\leqslant \sin{x}$, 注意到~$f(0)=\sin{0}=0$, 于是,
\[|\dfrac{f(x)-f(0)}{x}|\leqslant\dfrac{|f(x)-f(0)|}{x}\leqslant\dfrac{|\sin{x}-\sin{0}|}{x}=\dfrac{\sin{x}}{x}.\]
不等式两边令$x\to 0^{+}$, 则
\[|f'_{+}(0)|\leqslant 1.\]
\end{enumerate}

\subsection{函数的微分}
\begin{enumerate}
\item 设函数~$f$~在点~$x_0$~附近有定义, 并在该点可微. 又设函数~$g$~在~$y_0=f(x_0)$~附近有定义并在该点可微. 用微分的定义证明: 复合函数~$g\circ f$~在点~$x_0$~可微, 且~$(g\circ f)'(x_0)=g'(y_0)f'(x_0)$.\\
{\heiti 证明}\\
由~$f$~在点~$x_0$~可微及函数的微分和导数之间的关系, 有
\[ \Delta y=f'(x_0)\Delta x+o(\Delta x). \]
又设~$u=g(y)$, 由~$g$~在~$y_0=f(x_0)$~可微及函数的微分和导数之间的关系, 有
\[ \Delta u=g'(y_0)\Delta y+o(\Delta y). \]
于是有
\begin{equation*}
	\begin{split}
	\lim\limits_{\Delta x\to 0}\dfrac{\Delta u}{\Delta x}=&\lim\limits_{\Delta x\to 0} \dfrac{g'(y_0)\Delta y+o(\Delta y)}{\Delta x}\\	
	=&\lim\limits_{\Delta x\to 0}\dfrac{g'(y_0)[f'(x_0)\Delta x+o(\Delta x)]+o(\Delta y)}{\Delta x}\\
	=&\lim\limits_{\Delta x\to 0} \left[ g'(y_0)f'(x_0)+g'(y_0)\dfrac{o(\Delta x)}{\Delta x} \right]+\lim\limits_{\Delta x\to 0} \dfrac{o(\Delta y)}{\Delta y}\dfrac{\Delta y}{\Delta x} \\
	=&g'(y_0)f'(x_0)+\lim\limits_{\Delta y\to 0} \dfrac{o(\Delta y)}{\Delta y} \lim\limits_{\Delta x\to 0} \dfrac{\Delta y}{\Delta x}\\
	=&g'(y_0)f'(x_0)+0\cdot f'(x_0)\\
	=&g'(y_0)f'(x_0)
	\end{split}
\end{equation*}
这说明~$g\circ f$~在$x_0$可导从而可微, 又由
\[ \Delta u=(g\circ f)'(x_0) \Delta x +o(\Delta x), \]
知~$(g\circ f)'(x_0)=g'(y_0)f'(x_0)$. 

\item 设函数~$f$~在点~$x_0$~附近有定义, 并在该点可微. 又设~$\{x_n\}$~和~$\{y_n\}$~是~$f$~的定义域中的两个数列, 满足: 
	\begin{enumerate}[(1)]
	\item $x_n<x_0<y_n, n=1,2,\cdots,$;
	\item $\lim\limits_{n\to\infty}x_n=\lim\limits_{n\to\infty}y_n=x_0$.
	\end{enumerate}
证明: 
\[ \lim\limits_{n\to\infty}\dfrac{f(x_n)-f(y_n)}{x_n-y_n}=f'(x_0).\]
{\heiti 证明}\\
由$f$在$x_0$可微, 故在$x_0$左可导且右可导. 由Heine定理, 有
\[ f'(x_0)=f'_{-}(x_0)=\lim\limits_{x\to x_0^{-}}\dfrac{f(x)-f(x_0)}{x-x_0}=\lim\limits_{n\to\infty}\dfrac{f(x_n)-f(x_0)}{x_n-x_0}\]
和
\[ f'(x_0)=f'_{+}(x_0)=\lim\limits_{x\to x_0^{+}}\dfrac{f(x)-f(x_0)}{x-x_0}=\lim\limits_{n\to\infty}\dfrac{f(y_n)-f(x_0)}{y_n-x_0}.\]
于是有
\begin{equation*}
	\begin{split}
	\lim\limits_{n\to\infty}\dfrac{f(x_n)-f(y_n)}{x_n-y_n}=&\lim\limits_{n\to\infty}\dfrac{f(x_n)-f(x_0)}{x_n-y_n}-\lim\limits_{n\to\infty}\dfrac{f(y_n)-f(x_0)}{x_n-y_n}\\
	=&\lim\limits_{n\to\infty}\dfrac{f(x_n)-f(x_0)}{x_n-x_0}\dfrac{x_n-x_0}{x_n-y_n}-\lim\limits_{n\to\infty}\dfrac{f(y_n)-f(x_0)}{y_n-y_0}\dfrac{y_n-y_0}{x_n-y_n}\\
	=&\lim\limits_{x\to x_0^{-}}\dfrac{f(x)-f(x_0)}{x-x_0}\lim\limits_{n\to\infty}\dfrac{x_n-x_0}{x_n-y_n}-\lim\limits_{x\to x_0^{+}}\dfrac{f(x)-f(x_0)}{x-x_0}\lim\limits_{n\to\infty}\dfrac{y_n-y_0}{x_n-y_n}\\
	=&f'(x_0)\lim\limits_{n\to\infty}\left(\dfrac{x_n-x_0}{x_n-y_n}-\dfrac{y_n-x_0}{x_n-y_n}\right)\\
	=&f'(x_0)\lim\limits_{n\to\infty}\dfrac{x_n-y_n}{x_n-y_n}\\
	=&f'(x_0)
	\end{split}
\end{equation*}

\item 设~$f(x)$~在~$x_0$~可微, 且~$f(x_0)\neq 0$, $f'(x_0)\neq 0$. 再设
\[ af(x_0+\Delta x)+bf(x_0+2\Delta x)-f(x_0)=o(\Delta x), \quad \Delta x\to 0.\]
求~$a$~和~$b$.

\item 设~$f(x_0)=0$. 再设~$\phi(t)$~在~$t=0$~的一个邻域里有连续的导数且~$\phi(0)=x_0$, $\phi'(0)\neq 0$. 证明: 极限~$\lim\limits_{t\to 0}\dfrac{f(\phi(t))}{t}$~存在的充要条件是~$f(x)$~在点~$x_0$~可微.

\item 设函数~$f$~和~$g$~都在点~$x_0$~附近有定义并在该点可微, 且~$f(x_0)=g(x_0)\neq 0$. 求极限
\[ \lim\limits_{n\to\infty}\left(\dfrac{f\left(x_0+\dfrac{1}{n}\right)}{g\left(x_0+\dfrac{1}{n}\right)}\right)^n.\]

\item 证明: $\lim\limits_{x\to x_0}\dfrac{f(x)-a}{x-x_0}=b$~的充要条件是~$\lim\limits_{x\to x_0}\dfrac{\e^{f(x)}-\e^a}{x-x_0}=\e^ab$.

\item 设函数~$f$~在~$[0,1]$~上有定义, $f(0)=0$, 并在~$x=0$~有右导数. 证明:
\[ \lim\limits_{n\to\infty}\left[f\left(\dfrac{1}{n^2}\right)+f\left(\dfrac{2}{n^2}\right)+\cdots+f\left(\dfrac{n}{n^2}\right)\right]=\dfrac{1}{2}f'_{+}(0).\]
根据以上命题求下列极限:
	\begin{enumerate}[(1)]
	\item $\lim\limits_{n\to\infty}\left[\sin{\dfrac{1}{n^2}}+\sin{\dfrac{2}{n^2}}+\cdots+\sin{\dfrac{n}{n^2}}\right]$;
	\item $\lim\limits_{n\to\infty}\left(1+\dfrac{1}{n^2}\right)\left(1+\dfrac{2}{n^2}\right)\cdots\left(1+\dfrac{n}{n^2}\right)$;
	\item $\lim\limits_{n\to\infty}\cos{\dfrac{1}{n^2}}\cos{\dfrac{2}{n^2}}\cdots\cos{\dfrac{n}{n^2}}$.
	\end{enumerate}
{\heiti 证明}\\
\begin{equation*}
	\begin{split}
	&\lim\limits_{n\to\infty}\left[f\left(\dfrac{1}{n^2}\right)+f\left(\dfrac{2}{n^2}\right)+\cdots+f\left(\dfrac{n}{n^2}\right)\right]\\
	=&\lim\limits_{n\to\infty}\left[\dfrac{f\left(\dfrac{1}{n^2}\right)-f(0)}{\dfrac{1}{n^2}}\cdot\dfrac{1}{n^2}+\cdots+\dfrac{f\left(\dfrac{n}{n^2}\right)-f(0)}{\dfrac{n}{n^2}}\cdot\dfrac{n}{n^2}\right]\\
	=&\lim\limits_{n\to\infty}\dfrac{f\left(\dfrac{1}{n^2}\right)-f(0)}{\dfrac{1}{n^2}}\cdot\lim\limits_{n\to\infty}\dfrac{1}{n^2}+\cdots+\lim\limits_{n\to\infty}\dfrac{f\left(\dfrac{n}{n^2}\right)-f(0)}{\dfrac{n}{n^2}}\cdot\lim\limits_{n\to\infty}\dfrac{n}{n^2}\\
	=&\lim\limits_{x\to0^{+}}\dfrac{f(x)-f(0)}{x}\cdot\lim\limits_{n\to\infty}\left(\dfrac{1}{n^2}+\cdots+\dfrac{n}{n^2}\right)\\
	=&f'_{+}(0)\cdot\lim\limits_{n\to\infty}\dfrac{n(n+1)}{2n^2}\\
	=&\dfrac{1}{2}f'_{+}(0).
	\end{split}
\end{equation*}
于是有
	\begin{enumerate}[(1)]
	\item $\lim\limits_{n\to\infty}\left[\sin{\dfrac{1}{n^2}}+\sin{\dfrac{2}{n^2}}+\cdots+\sin{\dfrac{n}{n^2}}\right]=\dfrac{1}{2}\lim\limits_{x\to0^{+}}\dfrac{\sin{x}}{x}=\dfrac{1}{2}.$
	\item $\lim\limits_{n\to\infty}\ln\left(1+\dfrac{1}{n^2}\right)\cdots\left(1+\dfrac{n}{n^2}\right)=\lim\limits_{n\to\infty}\left[\ln\left(1+\dfrac{1}{n^2}\right)+\cdots+\ln\left(1+\dfrac{n}{n^2}\right)\right]=\dfrac{1}{2}.$\\
故$\lim\limits_{n\to\infty}\left(1+\dfrac{1}{n^2}\right)\left(1+\dfrac{2}{n^2}\right)\cdots\left(1+\dfrac{n}{n^2}\right)=\sqrt{\e}$.
	\item $\lim\limits_{n\to\infty}\ln\left(\cos{\dfrac{1}{n^2}}\cos{\dfrac{2}{n^2}}\cdots\cos{\dfrac{n}{n^2}}\right)=\dfrac{1}{2}(\ln(\cos x))'|_{x=0}=-\dfrac{1}{2}\tan{0}=0.$\\
故$\lim\limits_{n\to\infty}\cos{\dfrac{1}{n^2}}\cos{\dfrac{2}{n^2}}\cdots\cos{\dfrac{n}{n^2}}=1$.
	\end{enumerate}
\end{enumerate}

\clearpage
\subsection{微分中值定理}
\begin{enumerate}
\item 证明{\bf 广义Rolle定理}: 设函数~$f$~在有穷或无穷区间~$(a,b)$~上处处可微且成立
\[ \lim\limits_{x\to a^{+}}f(x)=\lim\limits_{x\to b^{-}}f(x).\]
这个等式的意思是或者等式两端的极限都存在且相等, 或者等式两端都是正无穷大或都是负无穷大. 则存在~$\xi\in(a,b)$~使得~$f'(\xi)=0$.

\item 对于~$n$~阶实系数多项式~$P(x)=a_0x^n+a_1x^{n-1}+\cdots+a_{n-1}x+a_n(a_0\neq 0)$, 证明: 
	\begin{enumerate}[(1)]
	\item $P(x)$~最多只有~$n$~个不同的实根;	
	\item 如果~$P(x)$~的~$n$~个根(重根按重数计算) 都是实数, 则其各阶导数~$P'(x), P''(x), \cdots, P^{(n)}(x)$~的根也都是实数.
	\end{enumerate}

\item 证明: 
	\begin{enumerate}[(1)]
	\item 方程~$ax^3+bx^2+cx=\dfrac{a}{4}+\dfrac{b}{3}+\dfrac{c}{2}$~在区间~$(0,1)$~上至少有一个根;
	\item 设~$a^2-3b<0$. 则方程~$x^3+ax^2+bx+c=0$~只有一个实根.
	\end{enumerate}

\item 对于方程~$1+x+\dfrac{x^2}{2}+\dfrac{x^3}{3}+\cdots+\dfrac{x^n}{n}=0$, 证明:
	\begin{enumerate}[(1)]
	\item 当~$n$~是奇数时只有一个实根;
	\item 当~$n$~是偶数时没有实根.
	\end{enumerate}

\item 证明{\bf Strum定理}: 设~$f(x)$~和~$g(x)$~都是区间~$I$~上的可微函数, 且~$f(x)g'(x)\neq f'(x)g(x), \forall x\in I$. 则在~$f(x)$~的任意两个零点之间都夹有$g(x)$的至少一个零点.

\item 设~$f(x)$~是可微函数, $a$~是常数. 证明: 
	\begin{enumerate}[(1)]
	\item 在~$f(x)$~的两个零点之间必有~$f'(x)+af(x)$~的一个零点;
	\item 在~$f(x)$~的两个正零点之间必有~$xf'(x)+af(x)$~的一个零点.
	\end{enumerate}

\item 设函数~$f$~在区间~$I$~上可微且导函数有界. 证明: 存在常数~$C>0$~使成立
\[|f(x)-f(y)|\leqslant C|x-y|, \quad \forall x,y\in I.\]

\item 设函数~$f(x), g(x), h(x)$~都在~$[a,b]$~上连续, 在~$(a,b)$~上可微. 证明:
	\begin{enumerate}[(1)]
	\item 存在~$\xi\in(a,b)$~使成立~$f'(\xi)f(a+b-\xi)=f(\xi)f'(a+b-\xi)$;
	\item 如果~$g'(x)\neq 0, \forall x\in (a,b)$, 则存在~$\xi\in(a,b)$~使成立
		\[\dfrac{f'(\xi)}{g'(\xi)}=\dfrac{f(\xi)-f(a)}{g(b)-g(\xi)};\]
	\item 存在~$\xi\in(a,b)$~使成立
		\[ \begin{vmatrix} f(a) & f(b) & f'(\xi)\\ g(a) & g(b) & g'(\xi)\\ h(a) & h(b) & h'(\xi) \end{vmatrix}=0.\]
	\end{enumerate}

\item 证明下列不等式: 
	\begin{enumerate}[(1)]
	\item $\dfrac{a-b}{1+a^2}<\arctan{a}-\arctan{b}<\dfrac{a-b}{1+b^2}(a>b>0)$;
	\item $\dfrac{a-b}{a}<\ln{a}{b}<\dfrac{a-b}{b}, \text{设~$a,b>0$}$;
	\item $pb^{p-1}(a-b)\leqslant a^p-b^p \leqslant pa^{p-1}(a-b), \text{设~$a,b>0$~且~$p>1$}$.
	\end{enumerate}

\item 设~$p>0$. 证明: 对任意自然数~$n$~成立不等式
\[\dfrac{n^{p+1}}{p+1}<1^p+2^p+\cdots+n^p<\dfrac{(n+1)^{p+1}}{p+1}.\]

\item 设~$m,n$~都是自然数且~$m>n$. 证明: 对任意~$x>0$~成立不等式
\[\dfrac{m}{n}\min\{1,x^{m-n}\}\leqslant \dfrac{1+x+\cdots+x^{m-1}}{1+x+\cdots+x^{n-1}}\leqslant \dfrac{m}{n}\max\{1,x^{m-n}\},\]
且等号成立当且仅当~$x=1$.

\item 证明: 
	\begin{enumerate}[(1)]
	\item 如果~$f(x)$~在~$(a,+\infty)$~上可导且~$f'(x)$~有界, 则~$f(x)$~在~$(a,+\infty)$~上一致连续;	
	\item 如果~$f(x)$~在~$(a,+\infty)$~上可导且~$\lim\limits_{x\to+\infty}|f'(x)|=+\infty$, 则~$f(x)$~在~$(a,+\infty)$~上不一致连续.
	\end{enumerate}

\item 设~$f(x)$~在~$(0,a]$~上连续, 在~$(0,\delta) (0<\delta\leqslant a)$~上可导, 且存在~$0<\mu<1$~和~$C>0$~使对任意~$x\in(0,\delta)$~有~$|f'(x)|\leqslant Cx^{-\mu}$. 证明: $f(x)$~在~$(0,a]$~上一致连续. 

\item 证明{\bf Darboux定理}: 设函数~$f$~在区间~$I$~上处处可微. 记~$A=\inf\limits_{x\in I}f'(x)$, $B=\sup\limits_{x\in I}f'(x)$ (当~$f'$~无下界时规定~$A=-\infty$, 当~$f'$~无上界时规定~$B=+\infty$). 则对任意~$A<c<B$, 必存在相应的~$\xi\in I$~使~$f'(\xi)=c$.

\item 设函数~$f$~在~$[a,b]$~上连续, 在~$(a,b)$~上有二阶导数. 证明: 存在~$\xi\in(a,b)$~使成立
\[f(a)+f(b)-2f\left(\dfrac{a+b}{2}\right)=\dfrac{(b-a)^2}{4}f''(\xi).\]
\end{enumerate}

\clearpage
\subsection{L'Hospital法则}
\begin{enumerate}
\item 设~$f(x)$~有二阶导数. 证明: 
\[\lim\limits_{h\to 0}\dfrac{f(x+h)+f(x-h)-2f(x)}{h^2}=f''(x).\]
{\heiti 证明}\\
等式左边的极限使用L'Hospital法则,
\begin{equation*}
	\begin{split}
	\lim\limits_{h\to 0}\dfrac{f(x+h)+f(x-h)-2f(x)}{h^2}=&\lim\limits_{h\to 0}\dfrac{f'(x+h)-f'(x-h)}{2h}\\
	=&\lim\limits_{h\to0}\left[\dfrac{f'(x+h)-f'(x)}{2h}+\dfrac{f'(x)-f'(x-h)}{2h}\right]\\
	=&\dfrac{1}{2}f''(x)+\dfrac{1}{2}f''(x)\\
	=&f''(x).
	\end{split}
\end{equation*}

\item 设~$f(x)$~在~$(a,+\infty)$~上可导, 且$\lim\limits_{x\to+\infty}[f'(x)+bf(x)]=c(b\neq0)$. 证明: $\lim\limits_{x\to+\infty}f(x)=\dfrac{c}{b}$.\\
{\heiti 证明}\\
\begin{equation*}
	\begin{split}
	\lim\limits_{x\to+\infty}f(x)=&\lim\limits_{x\to+\infty}\dfrac{\e^{bx}f(x)}{\e^{b}}\\
	=&\lim\limits_{x\to+\infty}\dfrac{\e^{bx}(f'(x)+bf(x))}{b\e^{bx}}\\
	=&\lim\limits_{x\to+\infty}\dfrac{f'(x)+bf(x)}{b}\\
	=&\dfrac{\lim\limits_{x\to+\infty}[f'(x)+bf(x)]}{b}\\
	=&\dfrac{c}{b}.
	\end{split}
\end{equation*}

\item 设函数~$f(x)$~在~$x=0$~附近有二阶导数且~$f''(0)\neq 0$. 对充分接近于零的~$x\neq 0$, 由Lagrange中值定理知存在~$\theta=\theta_x\in(0,1)$~使
\[f(x)-f(0)=f'(\theta x)x.\]
证明: $\lim\limits_{x\to 0}\theta=\dfrac{1}{2}$.\\
{\heiti 证明}\\
考虑辅助函数~$F$:
\[F(x)=\dfrac{f(x)-f(0)-f'(0)x}{x^2}.\]
一方面, 有
\[\lim\limits_{x\to 0}F(x)=\lim\limits_{x\to 0}\dfrac{f(x)-f(0)-f'(0)x}{x^2}=\lim\limits_{x\to 0}\dfrac{f'(\theta x)x-f'(0)x}{x^2}=\lim\limits_{x\to 0}\dfrac{f'(\theta x)-f'(0)}{x}.\]
注意到~$0<\theta<x$, 所以当~$x\to 0$~时, $\theta=\theta_x\to 0$. 于是
\[\lim\limits_{x\to 0} F(x)=\lim\limits_{x\to 0}\dfrac{f'(\theta x)-f'(x)}{x}=\lim\limits_{x\to 0}\theta \dfrac{f'(\theta x)-f'(0)}{\theta x}=f''(0)\cdot\lim\limits_{x\to0}\theta.\]
另一方面, 对~$\lim\limits_{x\to0}F(x)$~应用L'Hospital法则, 有
\[\lim\limits_{x\to 0} F(x)=\lim\limits_{x\to 0}\dfrac{f(x)-f(0)-f'(0)x}{x^2}=\lim\limits_{x\to 0}\dfrac{f'(x)-f'(0)}{2x}=\dfrac{1}{2}f''(0).\]
由于上面两式都是~$\lim\limits_{x\to 0} F(x)$, 由极限的唯一性有~$f''(0)\lim\limits_{x\to0}\theta=\dfrac{1}{2}f''(0)$, 注意到~$f''(0)\neq0$, 则有~$\lim\limits_{x\to 0}\theta =\dfrac{1}{2}$. 
\end{enumerate}

\clearpage                                                        
\subsection{利用导数判断两个函数相等}
\begin{enumerate}
\item 设函数~$f$~在区间~$I$~上满足以下条件: 存在常数~$\mu>1$~和~$C>0$~使成立
\[ |f(x)-f(y)|\leqslant C|x-y|^{\mu}, \quad \forall x,y\in I.\]
证明: $f$~在区间~$I$~上是常值函数.\\
{\heiti 证明}\\
对于任意的~$x_0\in I$,
\[\left|\dfrac{f(x)-f(x_0)}{x-x_0}\right|\leqslant \dfrac{|f(x)-f(x_0)|}{|x-x_0|}\leqslant C|x-x_0|^{\mu-1}, \quad\forall x\in I.\]
于是对于任意给定的~$\varepsilon>0$, 由于~$\mu>1$, 当~$|x-x_0|<\sqrt[\mu-1]{\dfrac{\varepsilon}{C}}$~时就有
\[\left|\dfrac{f(x)-f(x_0)}{x-x_0}\right|< C\sqrt[\mu-1]{\dfrac{\varepsilon}{C}}=\varepsilon.\]
即~$f'(x_0)=\lim\limits_{x\to x_0} \dfrac{f(x)-f(x_0)}{x-x_0}=0, \forall x_0\in I$, 于是~$f$~在区间~$I$~上是常值函数.

\item 设函数~$f$~和~$g$~在区间~$I$~上可微, $g(x)\neq 0, \forall x\in I$, 且成立
\[ \begin{vmatrix} f(x) & g(x)\\f'(x) & g'(x)\end{vmatrix}=0,\quad \forall x\in I.\]
证明: 存在常数~$c$~使~$f(x)=cg(x), \forall x\in I$.\\
{\heiti 证明}\\
$f$~和~$g$~在区间~$I$~上可微, $g(x)\neq 0, \forall x\in I$, 故~$\dfrac{f}{g}$~也在~$I$~上可微. 由导数的四则运算
\[ \left(\dfrac{f(x)}{g(x)}\right)'=\dfrac{f'(x)g(x)-f(x)g'(x)}{g^2(x)}=\dfrac{1}{g^2(x)}\begin{vmatrix} f(x) & g(x)\\f'(x) & g'(x)\end{vmatrix}=0,\quad \forall x\in I.\]
故~$\dfrac{f}{g}$~在区间~$I$~上是常函数, 即存在常数~$c$~使~$\dfrac{f(x)}{g(x)}=c, \forall x\in I$.

\item 设函数~$f$~在区间~$I$~上可微且导数是一常数: $f'(x)=a, \forall x\in I$. 证明: $f$在区间~$I$~上是一线性函数~$f(x)=ax+b, \forall x\in I$, 其中~$a, b$~是常数.\\
{\heiti 证明}\\
$f$~在区间~$I$~上可微, 故函数
\[F(x)=f(x)-ax, \forall x\in I,\]
也在~$I$~上可微, 并且~$F'(x)=(f(x)-ax)'=f'(x)-a=0, \forall x\in I$. 故~$F$~在区间~$I$~上是一常值函数, 即存在常数~$b$~使成立~$F(x)=b, \forall x\in I$, 也即~$f$~在~$I$~上是一线性函数.
 
\item 设~$I$~是正半轴上的一个区间, 函数~$f$~在~$I$~上可微且满足~$xf'(x)+af(x)=0, \forall x\in I$, 其中~$a$~是常数. 证明: 存在常数~$C$~使成立~$f(x)=Cx^{-a}, \forall x\in I$.\\
{\heiti 证明}\\
$f$~在~$I$~上可微, 故函数
\[F(x)=x^af(x), \forall x\in I,\]
也在~$I$~上可微, 并且~$F'(x)=(x^af(x))'=x^{a-1}(af(x)+xf'(x))=0, \forall x\in I$. 故~$F$~在区间~$I$~上是一常值函数, 即存在常数~$C$~使成立~$F(x)=x^af(x)=C, \forall x\in I$, 也即~$f(x)=Cx^{-a}, \forall x\in I$.

\item 设函数~$f$~在区间~$I$~上可微. 证明: 
	\begin{enumerate}[(1)]
	\item 如果~$f'(x)=b\e^{ax}, \forall x\in I$, 其中~$a, b$~是常数, 则~$f(x)=\dfrac{b}{a}\e^{ax}+c, \forall x\in I$, 其中~$c$~是常数;
	\item 如果~$f'(x)=af(x)+b, \forall x\in I$, 其中~$a, b$~是常数, 则~$f(x)=c\e^{ax}-\dfrac{b}{a}, \forall x\in I$, 其中~$c$~是常数.
	\end{enumerate}
{\heiti 证明}\\
分别取~$I$~上的辅助函数~$F$~和~$G$~为:
\[ F(x)=f(x)-\dfrac{b}{a}\e^{ax}, \quad G(x)=(f(x)+\dfrac{b}{a})\e^{-ax}.\]
由~$f$~在区间~$I$~上可微知, $F$~和~$G$~均在~$I$~上可微, 并能验证其导数各自均恒为零即得证.

\item 设函数~$f$~在~$[0,+\infty)$上连续, 在~$(0,+\infty)$~上可微, 且存在常数~$C>0$~使得~$|f'(x)|\leqslant C|f(x)|, \forall x>0$. 又设~$f(0)=0$. 证明: $f(x)=0, \forall x\geqslant 0$.

\item 设函数~$f$~在区间~$I$~上可微, 且存在常数~$a>0$~使得~$|f'(x)|\leqslant a, \forall x\in I$. 证明: 存在常数~$b>0$~使~$|f(x)|\leqslant a|x|+b, \forall x\in I$.
\end{enumerate}

\clearpage
\section{一元函数积分学(Riemann积分)}
\subsection{定积分的基本概念和性质}
\begin{enumerate}
\item 用定义证明以下函数在任意区间~$[a,b]$~上Riemann可积且:
	\begin{enumerate}[(1)]
	\item $\displaystyle\int_a^b x\ud{x}=\dfrac{1}{2}(b^2-a^2)$;
	\item $\displaystyle\int_a^b x^2\ud{x}=\dfrac{1}{3}(b^3-a^3)$;
	\item $\displaystyle\int_a^b \cos{x}\ud{x}=\sin{b}-\sin{a}$;
	\item $\displaystyle\int_a^b \sin{x}\ud{x}=\cos{a}-\cos{b}$.
	\end{enumerate}

\item 设~$0\leqslant a<b$, $m$~是正整数. 证明:
\[ \int_a^b x^m\ud{x}=\frac{1}{m+1}(b^{m+1}-a^{m+1}).\]

\item 设函数~$f$~在区间~$[a,b]$~上 H\"older 连续, 即存在常数~$0<\mu\leqslant 1$~和~$C>0$~使成立
\[ |f(x)-f(y)|\leqslant C|x-y|^{\mu}, \quad\forall x,y\in[a,b]. \]
又设存在在~$[a,b]$~上连续, 在~$(a,b)$~上可微的函数~$F$~使~$F'(x)=f(x), \forall x\in(a,b)$. 证明: $f$~在~$[a,b]$~上可积, 且
\[ \int_a^b f(x)\ud{x}=F(b)-F(a). \]
{\heiti 证明}\\
对于~$[a,b]$~的任意分割~$\Delta=\{x_0, x_1, \cdots, x_{n-1}, x_n\}$和相应的介点集~$\Xi$, 有
\[ \sum_{k=1}^nf(\xi_k)\Delta x_k-(F(b)-F(a)=\sum_{k=1}^nf(\xi_k)\Delta x_k-\sum_{k=1}^n(F(x_k)-F(x_k-1)).\]
对每个小区间~$[x_{k-1}, x_k], 1\leqslant k\leqslant n$, $F(x)$~在~$[x_{k-1}, x_k]$~上连续, 在~$(x_{k-1}, x_k)$~上可微, 应用Lagrange中值定理则存在~$\eta_k\in(x_{k-1}, x_k), k=1,2,\cdots,n$~使得
\begin{equation*}
	\begin{split}
		|\sum_{k=1}^nf(\xi_k)\Delta x_k-(f(b)-f(a))|=&|\sum_{k=1}^n f(\xi_k)\Delta x_k-\sum_{k=1}^n f(\eta_k)\Delta x_k|\\
		\leqslant&\sum_{k=1}^n|f(\xi_k)-f(\eta_k)|\Delta x_k|\\
		\leqslant&\sum_{k=1}^n C|\xi_k-\eta_k|^{\mu}\Delta x_i \\
		\leqslant&C||\Delta||^{\mu}(b-a).
	\end{split}
\end{equation*}
对于任意~$\varepsilon>0$~, 取~$\delta=\left(\dfrac{\varepsilon}{C(b-a)}\right)^{\frac{1}{\mu}}$, 则当~$||\Delta||<\delta$~时有
\[|\sum_{k=1}^nf(\xi_k)\Delta x_k-(f(b)-f(a))|\leqslant C||\Delta||^{\mu}(b-a)<\varepsilon.\]
即~$f(x)$~在~$[a,b]$~上可积, 且~$\displaystyle \int_a^b f(x)\ud{x}=F(b)-F(a)$.
		
\item 设~$f(x)$~在~$[a,b]$~上可积. 证明: $f(x-c)$~在~$[a+c,b+c]$~上可积, 且
\[ \int_{a+c}^{b+c} f(x-c)\ud{x}=\int_a^b f(x)\ud{x}.\]

\item 设~$f$~是以~$T>0$~为周期的周期函数, 且在~$[0,T]$~上可积. 证明: 对任意实数~$a<b$, $f$~也在~$[a,b]$~上可积, 且当~$b-a=nT+c$, 其中~$n$~是非负整数而~$0\leqslant c<T$~时, 有
\[\int_a^b f(x)\ud{x}=n\int_0^T f(x)\ud{x}+\int_a^{a+c} f(x)\ud{x}.\]
特别地, 
\[\int_a^{a+T} f(x)\ud{x}=\int_0^{T}f(x)\ud{x}.\]

\item 设~$f(x)$~在~$[a,b]$~上可积, 且积分值为~$I$. 在~$[a,b]$~上任意有限个点处改变函数~$f(x)$~的值, 得到一个新的函数为~$f^{*}(x)$. 证明~$f^{*}(x)$~也在~$[a,b]$~上也可积, 且积分值仍为~$I$.

\item 设~$f(x)$~在~$[0,+\infty)$~的任意有限子区间上可积, 且~$\lim\limits_{x\to+\infty} f(x)=c$. 证明:
\[ \lim\limits_{a\to+\infty} \dfrac{1}{a}\int_0^a f(x)\ud{x}=c.\]

\item 设~$f$~是以~$T>0$~为正周期的周期函数, 且在~$[0,T]$~上可积. 证明:
\[ \lim\limits_{a\to+\infty} \dfrac{1}{a}\int_0^a f(x)\ud{x}=\dfrac{1}{T}\int_0^T f(x)\ud{x}.\]

\item 已知当~$f(x)$~和~$g(x)$~都在~$[a,b]$~上可积时, $f(x)g(x)$~也在~$[a,b]$~上可积. 据此证明不等式
\[ \left(\int_a^b f(x)g(x)\ud{x}\right)^2\leqslant \left(\int_a^b |f(x)|^2\ud{x}\right) \left( \int_a^b |g(x)|^2\ud{x} \right). \]

\item 设~$f(x)$~和~$g(x)$~都在~$[a,b]$~上可积, $f(x)$~在~$[a,b]$~上单调递减, $g(x)$~满足~$0\leqslant g(x)\leqslant 1, \forall x\in[a,b]$. 令~$\displaystyle\theta=\int_a^b g(x)\ud{x}$. 证明:
	\begin{enumerate}[(1)]
	\item $\displaystyle\int_{b-\theta}^b [1-g(x)]\ud{x}=\int_a^{b-\theta} g(x)\ud{x}, \int_a^{a+\theta}[1-g(x)]\ud{x}=\int_{a+\theta}^b g(x)\ud{x}$;
	\item $\displaystyle\int_{b-\theta}^b f(x)\ud{x}\leqslant\int_a^b f(x)g(x)\ud{x}\leqslant\int_a^{a+\theta} f(x)\ud{x}$.
	\end{enumerate}
\end{enumerate}

\clearpage
\subsection{定积分的计算}
\begin{enumerate}
\item 利用定积分求极限:
	\begin{enumerate}[(1)]
	\item $\lim\limits_{n\to\infty}\left( \dfrac{1}{n+1}+\dfrac{1}{n+2}+\cdots+\dfrac{1}{2n} \right)$;
	\item $\lim\limits_{n\to\infty}\left( \dfrac{n}{n^2+1}+\dfrac{n}{n^2+2^2}+\cdots+\dfrac{n}{2n^2} \right)$;
	\item $\lim\limits_{n\to\infty}\dfrac{1}{n}\left( \sin{\dfrac{\pi}{n}}+\sin{\dfrac{2\pi}{n}}+\cdots+\sin{\dfrac{(n-1)\pi}{n}} \right)$;
	\item $\lim\limits_{n\to\infty}\dfrac{1}{n}\left( \sqrt{1+\dfrac{1}{n}}+\sqrt{1+\dfrac{2}{n}}+\cdots+\sqrt{1+\dfrac{n}{n}}\right)$;
	\item $\lim\limits_{n\to\infty}\dfrac{1^p+2^p+\cdots+n^p}{n^{p+1}} (p>0)$;
	\item $\lim\limits_{n\to\infty}\dfrac{1}{n}\sum\limits_{k=1}^n f\left(a+k\dfrac{b-a}{n}\right)$, $f$~是~$[a,b]$~上的连续函数.
	\end{enumerate}
{\heiti 证明}\\
	\begin{enumerate}[(1)]
	\item $\displaystyle\lim\limits_{n\to\infty}\left(\dfrac{1}{n+1}+\cdots+\dfrac{1}{2n}\right)=\lim\limits_{n\to\infty}\sum\limits_{k=1}^n\dfrac{1}{1+\dfrac{k}{n}}\cdot\dfrac{1}{n}=\int_0^1\dfrac{1}{1+x}\ud{x}=\ln{2}$;
	\item $\displaystyle\lim\limits_{n\to\infty}\left(\dfrac{n}{n^2+1}+\cdots+\dfrac{n}{2n^2} \right)=\lim\limits_{n\to\infty}\sum\limits_{k=1}^n\dfrac{1}{1+\left(\dfrac{k}{n}\right)^2}\cdot \dfrac{1}{n}=\int_0^1\dfrac{1}{1+x^2}\ud{x}=\dfrac{1}{2}\ln{2}$;
	\item $\displaystyle\lim\limits_{n\to\infty}\dfrac{1}{n}\left( \sin{\dfrac{\pi}{n}}+\cdots+\sin{\dfrac{(n-1)\pi}{n}} \right)=\lim\limits_{n\to\infty}\sum\limits_{k=1}{n}\sin{\dfrac{k\pi}{n}}\cdot\dfrac{1}{n}=\int_0^1\sin{\pi x}\ud{x}=\dfrac{2}{\pi}$;
	\item $\displaystyle\lim\limits_{n\to\infty}\dfrac{1}{n}\left( \sqrt{1+\dfrac{1}{n}}+\cdots+\sqrt{1+\dfrac{n}{n}}\right)=\lim\limits_{n\to\infty}\sum\limits_{k=1}^n\sqrt{1+\dfrac{k}{n}}\cdot\dfrac{1}{n}=\int_0^1\sqrt{1+x}\ud{x}=\dfrac{2(2\sqrt{2}-1)}{3}$;
	\item $\displaystyle\lim\limits_{n\to\infty}\dfrac{1^p+2^p+\cdots+n^p}{n^{p+1}}=\lim\limits_{n\to\infty}\sum\limits_{k=1}^n\left(\dfrac{k}{n}\right)^p\cdot\dfrac{1}{n}=\int_0^1 x^p\ud{x}=\dfrac{1}{p+1}$;
	\item $\displaystyle\lim\limits_{n\to\infty}\dfrac{1}{n}\sum\limits_{k=1}^n f\left(a+k\dfrac{b-a}{n}\right)=\int_a^b f(x)\ud{x}$.
	\end{enumerate}

\item 设~$f$~是连续函数. 证明:
	\begin{enumerate}[(1)]
	\item $\displaystyle\int_0^{\frac{\pi}{2}} f(\sin{x})\ud{x}=\int_0^{\frac{\pi}{2}} f(\cos{x})\ud{x}$;
	\item $\displaystyle\int_0^{\pi} xf(\sin{x})\ud{x}=\dfrac{\pi}{2}\int_0^{\pi}f(\sin{x})\ud{x}$;
	\item $\displaystyle\int_0^a x^3f(x^2)\ud{x}=\dfrac{1}{2}\int_0^{a^2}xf(x)\ud{x}\quad (a>0)$;
	\item $\displaystyle\int_1^a f\left(\dfrac{x^2}{a^2}+\dfrac{a^2}{x^2}\right)\ud{x}=a\int_1^a f\left(x^2+\dfrac{1}{x^2}\right)\dfrac{\ud{x}}{x^2}\quad(a>1)$.
	\end{enumerate}
{\heiti 证明}\\
	\begin{enumerate}[(1)]
		\item $\displaystyle\int_0^{\frac{\pi}{2}} f(\sin{x})\ud{x}=\int_0^{\frac{\pi}{2}} f\left[\sin{\left(\dfrac{\pi}{2}-x\right)}\right]\ud{\left(\dfrac{\pi}{2}-x\right)}=-\int_{\frac{\pi}{2}}^0f(\cos{x})=\int_0^{\frac{\pi}{2}} f(\cos{x})\ud{x}$;
		\item $\displaystyle\int_0^{\pi}xf(\sin{x})\ud{x}=\int_0^{\pi}(\pi-x)f[\sin{(\pi-x)}]\ud{(\pi-x)}=\pi\int_0^{\pi}f(\sin{x})\ud{x}-\int_0^{\pi}xf(\sin{x})\ud{x}$,\\
		故$\displaystyle\int_0^{\pi} xf(\sin{x})\ud{x}=\dfrac{\pi}{2}\int_0^{\pi}f(\sin{x})\ud{x}$;
		\item $\displaystyle\int_0^a x^3f(x^2)\ud{x}=\dfrac{1}{2}\int_0^a x^2f(x^2)\ud{x^2}=\dfrac{1}{2}\int_0^{a^2} xf(x)\ud{x}$;
		\item $\displaystyle\int_1^a f\left(\dfrac{x^2}{a^2}+\dfrac{a^2}{x^2}\right)\ud{x}=\int_1^a f\left(\dfrac{a^2}{x^2}+\dfrac{x^2}{a^2}\right)\ud{\left(\dfrac{a}{x}\right)}=a\int_1^a f\left(x^2+\dfrac{1}{x^2}\right)\dfrac{\ud{x}}{x^2}$.
	\end{enumerate}

\item 设~$f(x)$~是~$[a,b]$~上的凸函数. 证明下述{\bf Hadamard不等式}: 
\[f\left(\dfrac{a+b}{2}\right)\leqslant\dfrac{1}{b-a}\int_a^b f(x)\ud{x}\leqslant\dfrac{f(a)+f(b)}{2}.\]

\item 设~$f(x)$~在~$[a,b]$~上连续, 在~$(a,b)$~上二阶可导且二阶导函数有界. 证明:
\[\int_a^b |f(x)|\ud{x}\leqslant(b-a)\max_{a\leqslant x\leqslant b}|f(x)|+\dfrac{1}{6}(b-a)^3\sup_{a<x<b}|f''(x)|.\]

\item 设~$f(x)$~在~$[a,b]$~上有连续的二阶导数, 且~$f(a)=f(b)=0$. 证明:
	\begin{enumerate}[(1)]
	\item $\displaystyle \left|f(x)\cdot\dfrac{b-a}{(x-a)(b-x)}\right|\leqslant\int_a^b |f''(x)|\ud{x}$;
	\item $\displaystyle \max\limits_{a\leqslant x\leqslant b}|f(x)|\leqslant\dfrac{b-a}{4}\int_a^b|f''(x)|\ud{x}$.
	\end{enumerate}

\item 设~$f(x)$~是~$[0,1]$~上的连续函数, $f(0)=0$, 且~$\dfrac{f(x)}{x}$~在~$[0,1]$~上可积. 证明:
\[ \lim_{n\to\infty} n\int_0^1 f(x^n)\ud{x}=\int_0^1\dfrac{f(x)}{x}\ud{x}. \]
{\heiti 证明}\\
\begin{equation*}
	\begin{split}
	&\left|n\int_0^1 f(x^n)\ud{x}-\int_0^1 \dfrac{f(x)}{x}\ud{x}\right|\\
	=&\left|\int_0^1 \dfrac{f(x^n)\ud{x^n}}{x^n-1}-\int_0^1 \dfrac{f(x)}{x}\ud{x}\right|\\
	=&\left|\int_0^1 \dfrac{f(x)\sqrt[n]{x}}{x}\ud{x}-\int_0^1 \dfrac{f(x)}{x}\ud{x}\right|\\
	=&\left|\int_0^1 \dfrac{f(x)}{x}(\sqrt[n]{x}-1)\ud{x}\right|\\
	\leqslant& \int_0^1\left|\dfrac{f{x}}{x}\right||\sqrt[n]{x}-1|\ud{x}
	\end{split}
\end{equation*}
由于对于任何~$n>0$, $g(x)=\sqrt[n]{x}-1$~在~$[0,1]$~上连续, 故存在~$x_0\in[0,1]$~使得~$g(x_0)=\max_{0\leqslant x\leqslant 1}$. 又由于~$\dfrac{f(x)}{x}$~在~$[0,1]$~上可积, 于是~$\left|\dfrac{f(x)}{x}\right|$~也在~$[0,1]$~上可积. 由于$\lim\limits_{n\to\infty}\sqrt[n]{x}-1=0$, 故对于任意给定的$\varepsilon>0$, 存在$N>0$使得
\[|\sqrt[n]{x_0}-1|<\dfrac{\varepsilon}{\int_0^1\left|\dfrac{f(x)}{x}\right|\ud{x}}.\]
从而对于~$\varepsilon>0$~和~$N$, 有
\begin{equation*}
	\begin{split}
	&\left|n\int_0^1 f(x^n)\ud{x}-\int_0^1 \dfrac{f(x)}{x}\ud{x}\right|\\
	\leqslant& \int_0^1|\dfrac{f(x)}{x}|\left|\sqrt[n]{x}-1\right|\ud{x}\\
	\leqslant& |\sqrt[n]{x_0}-1|\int_0^1 \left|\dfrac{f(x)}{x}\right|\ud{x}\\
	<&\varepsilon.
	\end{split}
\end{equation*}
即$\displaystyle\lim\limits_{n\to\infty} n\int_0^1 f(x^n)\ud{x}=\int_0^1\dfrac{f(x)}{x}\ud{x}$.
\end{enumerate}

\clearpage
\subsection{连续函数的可积性}
\begin{enumerate}
\item 设~$f(x)$~是区间~$[a,b]$~上的非负连续函数, 且在此区间上不恒等于零. 证明:
\[ \int_a^b f(x)\ud{x}>0.\]
\begin{proof}
由于非负函数~$f$~在~$[a,b]$~上不恒等于零, 故存在~$x_0\in[a,b]$~使得~$f(x_0)\geqslant 0$, 又由于~$f$~在~$[a,b]$~上连续, 故存在~$\delta>0$~使得$[x_0-\delta,x_0+\delta]\subseteq [a,b]$~且~$\forall x\in [x_0-\delta,x_0+\delta], f(x)>0$. 于是有
\[ \int_a^b f(x)\ud{x} \geqslant \int_{x_0-\delta}^{x_0+\delta}f(x)\ud{x}=2\delta f(x)>0.\]
\end{proof}

\item 设~$f(x)$~在~$[a,b]$~上连续, 且存在在~$[a,b]$~上连续, 在~$(a,b)$~上可导的函数~$F$~使得~$F'(x)=f(x), \forall x\in(a,b)$. 不使用 Newton-Leibniz 公式直接证明: $f$在$[a,b]$上可积, 且
\[ \int_a^b f(x)\ud{x}=F(b)-F(a).\]
\begin{proof}
由微积分基本定理, $\displaystyle \int_a^x f(t)\ud{t}$~在~$[a,b]$~上连续可导, 且
\[\dfrac{\ud}{\ud{x}}\int_a^x f(t)\ud(t)=f(x).\]
于是$\displaystyle \int_a^xf(t)\ud{t}$~是~$f(x)$~在~$[a,b]$~上的一个原函数. 又因为~$F(x)$~也是~$f(x)$~在~$[a,b]$~上的原函数, 于是有
\[\int_a^xf(t)d(t)=F(x)+C, \quad\text{$C$~是常数.}\]
在上式中令~$x=a$有~$C=-F(a)$, 再令~$x=b$~就有
\[\int_a^bf(t)\ud{t}=\left.\int_a^xf(t)\ud{t}\right|_{x=b}=(F(x)-F(a))|_{x=b}=F(b)-F(a).\]
\end{proof}

\item 设~$f(x)$~在~$(a,b)$~上连续. 证明: 对任意~$[\alpha,\beta]\subseteq(a,b)$~成立
\[\lim_{h\to 0}\int_{\alpha}^{\beta}\dfrac{f(x+h)-f(x)}{h}\ud{x}=f(\beta)-f(\alpha).\]
\begin{proof}
\begin{equation*}
	\begin{split} 
		\int_{\alpha}^{\beta} \dfrac{f(x+h)-f(x)}{h}\ud{x}
		=&\dfrac{1}{h}\left[\int_{\alpha}^{\beta}f(x+h)\ud{x}-\int_{alpha}^{\beta}f(x)\ud{x}\right]\\
		=&\dfrac{1}{h}\left[\int_{\alpha+h}^{\beta+h}f(x)\ud{x}-\int_{\alpha}^{\beta} f(x)\ud{x}\right]\\
		=&\dfrac{1}{h}\left[\int_{\beta}^{\beta+h}f(x)\ud{x}-\int_{\alpha}^{\alpha+h}f(x)\ud{x}\right]
	\end{split}
\end{equation*}
由于函数~$f(x)$~在~$(a,b)$~上连续, 故~$\displaystyle \int_{\beta}^{x} f(t)\ud{t}$~和~$\displaystyle \int_{\alpha}^{x} f(t)\ud{t}$~分别在~$[\beta, \beta+h]$~和~$[\alpha,\alpha+h]$~上连续且可微. 分别应用Lagrange中值定理, 则存在~$\theta_1,\theta_2\in(0,1)$, 使得
\begin{equation*}
	\begin{split}
		 \int_{\alpha}^{\beta} \dfrac{f(x+h)-f(x)}{h}\ud{x}	
		=&\dfrac{1}{h}[f(\beta+\theta_1h)(\beta+h-\beta)-f(\alpha+\theta_2h)(\alpha+h-\alpha)]\\
		=&f(\beta+\theta_1h)-f(\alpha+\theta_2h)
		\to f(\beta)-f(\alpha), \text{当~$h\to 0$}.
	\end{split}
\end{equation*}
即~$\displaystyle \lim_{h\to 0}\int_{\alpha}^{\beta}\dfrac{f(x+h)-f(x)}{h}\ud{x}=f(\beta)-f(\alpha)$.
\end{proof}

\item 设~$f(x)$~在~$[-\pi,\pi]$~上可微且导函数在~$[-\pi,\pi]$~上可积. 证明:
\begin{equation*}
	\begin{split}
	&\lim_{n\to\infty} n\int_{-\pi}^{\pi} f(x)\sin{nx}\ud{x}=(-1)^{n-1}[f(\pi)-f(-\pi)],\\
	&\lim_{n\to\infty} n\int_{-\pi}^{\pi} f(x)\cos{nx}\ud{x}=0.
	\end{split}
\end{equation*}
\begin{proof}
	对每个自然数~$n$~有
	\begin{equation*}
		\begin{split}
			n\int_{-\pi}^{\pi} f(x)\sin{nx}\ud{x}=&\int_{-n\pi}^{n\pi}f\left(\dfrac{x}{n}\right)\sin{x}\ud{x}=\sum_{k=1}^{n}\int_{2(k-1)\pi-n\pi}^{2k\pi-n\pi}f\left(\dfrac{x}{n}\right)\sin{x}\ud{x}\\
			=&\sum_{k=1}^{n}\left[\int_{2(k-1)\pi-n\pi}^{(2k-1)\pi-n\pi}f\left(\dfrac{x}{n}\right)\sin{x}\ud{x}+\int_{(2k-1)\pi-n\pi}^{2k\pi-n\pi}f\left(\dfrac{x}{n}\right)\sin{x}\ud{x}\right]\\
			=&\sum_{k=1}^n\int_{2(k-1)\pi-n\pi}^{(2k-1)\pi-n\pi}\left[f\left(\dfrac{x}{n}\right)-f\left(\dfrac{x+\pi}{n}\right)\right]\sin{x}\ud{x}.
		\end{split}
	\end{equation*}
因为~$\sin{x}$~在区间~$[2(k-1)\pi-n\pi, (2k-1)\pi-n\pi]$~上不变号, 对等号右端的积分应用积分第一中值定理, 存在~$2(k-1)\pi-n\pi\leqslant\xi_k\leqslant(2k-1)\pi-n\pi$~使得
	\begin{equation*}
		\begin{split}
			n\int_{-\pi}^{\pi} f(x)\sin{nx}\ud{x}=&\sum_{k=1}^{n}\left[f\left(\dfrac{\xi_k}{n}\right)-f\left(\dfrac{\xi_k+\pi}{n}\right)\right]\int_{2(k-1)\pi-n\pi}^{(2k-1)\pi-n\pi}\sin{x}\ud{x}\\
			=&(-1)^n\sum_{k=1}^{n}\left[f\left(\dfrac{\xi_k}{n}\right)-f\left(\dfrac{\xi_k+\pi}{n}\right)\right]\int_{2(k-1)\pi}^{(2k-1)\pi}\sin{x}\ud{x}\\
			=&(-1)^n\cdot2\sum_{k=1}^{n}\left[f\left(\dfrac{\xi_k}{n}\right)-f\left(\dfrac{\xi_k+\pi}{n}\right)\right].
		\end{split}
	\end{equation*}
因为~$f(x)$~在~$[-\pi,\pi]$~可微, 由~Lagrange~中值定理知存在~$\dfrac{\xi_k}{n}<\eta_k<\dfrac{\xi_k+\pi}{n}$, 即~$\dfrac{2(k-1)\pi}{n}-\pi<\eta_k<\dfrac{2k\pi}{n}-\pi$, 使得
	\[n\int_{-\pi}^{\pi} f(x)\sin{nx}\ud{x}=(-1)^n\cdot2\cdot-\dfrac{\pi}{n}\sum_{k=1}^nf'(\eta_k)=(-1)^{n-1}\dfrac{2\pi}{n}\sum_{k=1}^nf'(\eta_k).\]
这是~$f'(x)$~在~$[-\pi,\pi]$~上关于~$n$~等分该区间所得分割的一个积分和. 由~$f'(x)$~在~$[-\pi,\pi]$~上可积就有
	\[\lim_{n\to\infty}n\int_{-\pi}^{\pi} f(x)\sin{nx}\ud{x}=(-1)^{n-1}\int_{-\pi}^{\pi}f'(x)\ud{x}=(-1)^{n-1}[f(\pi)-f(-\pi)].\]

同理有
	\begin{equation*}
		\begin{split}
			n\int_{-\pi}^{\pi} f(x)\cos{nx}\ud{x}=&\int_{-n\pi}^{n\pi}f\left(\dfrac{x}{n}\right)\cos{x}\ud{x}=\int_{-n\pi}^{\frac{\pi}{2}-n\pi}f\left(\dfrac{x}{n}\right)\cos{x}\ud{x}\\
			+&\int_{n\pi-\frac{\pi}{2}}^{n\pi}f\left(\dfrac{x}{n}\right)\cos{x}\ud{x}+\sum_{k=1}\int_{(2k-\frac{3}{2})\pi-n\pi}^{(2k+\frac{1}{2})\pi-n\pi} f\left(\dfrac{x}{n}\right)\cos{x}\ud{x}.
		\end{split}
	\end{equation*}
于是因为~$\cos{x}$~分别在~$[-n\pi,\dfrac{\pi}{2}-n\pi]$~和~$[-\dfrac{\pi}{2}+n\pi,n\pi]$~上不变号, 由积分第一中值定理知存在~$\delta\in[-n\pi,\dfrac{\pi}{2}-n\pi]$~和~$\sigma\in[-\dfrac{\pi}{2}+n\pi,n\pi]$, 当~$n\to\infty$~时, 根据两边夹法则和~$f$~在区间中可微从而连续以及~Heine~原理知~$\lim\limits_{n\to\infty} f(\delta)=f(-\pi), \lim\limits_{n\to\infty} f(\sigma)=f(\pi)$, 从而
	\begin{equation*}
		\begin{split}
			&\lim_{n\to\infty}\int_{-n\pi}^{\frac{\pi}{2}-n\pi}f\left(\dfrac{x}{n}\right)\cos{x}\ud{x}+\int_{n\pi-\frac{\pi}{2}}^{n\pi}f\left(\dfrac{x}{n}\right)\cos{x}\ud{x}\\
			=&\lim_{n\to\infty}f(\delta)\int_{-n\pi}^{\frac{\pi}{2}-n\pi} \cos{x}\ud{x}+f(\sigma)\int_{-\frac{\pi}{2}+n\pi}^{n\pi}\cos{x}\ud{x}\\
			=&(-1)^n[f(-\pi)-f(\pi)].
		\end{split}
	\end{equation*}
类似地也存在~$(2k-\dfrac{3}{2})\pi-n\pi\leqslant\xi_k\leqslant(2k-\dfrac{1}{2})\pi-n\pi$~和~$\dfrac{2(k-1)}{n}\pi-\pi<\dfrac{4k-3}{2n}\pi-\pi<\eta_k<\dfrac{4k-1}{2n}\pi-\pi<\dfrac{2k}{n}\pi-\pi$~使得
	\begin{equation*}
		\begin{split}
			\sum_{k=1}\int_{(2k-\frac{3}{2})\pi-n\pi}^{(2k+\frac{1}{2})\pi-n\pi} f\left(\dfrac{x}{n}\right)\cos{x}\ud{x}=&\sum_{k=1}^n\int_{(2k-\frac{3}{2})\pi-n\pi}^{(2k+\frac{1}{2})\pi-n\pi}\left[f\left(\dfrac{x}{n}\right)-f\left(\dfrac{x+\pi}{n}\right)\right]\cos{x}\ud{x}\\
			=&\sum_{k=1}^n\left[f\left(\dfrac{\xi_k}{n}\right)-f\left(\dfrac{\xi_k+\pi}{n}\right)\right]\int_{(2k-\frac{3}{2})\pi-n\pi}^{(2k+\frac{1}{2})\pi-n\pi}\cos{x}\ud{x}\\
			=&(-1)^{(n-1)}\cdot(-2)\sum_{k=1}^n\left[f\left(\dfrac{\xi_k}{n}\right)-f\left(\dfrac{\xi_k+\pi}{n}\right)\right]\\
			=&(-1)^n\dfrac{2\pi}{n}\sum_{k=1}f'(\eta_k)\\
			\to&(-1)^n\int_{-\pi}^{\pi}f'(x)\ud{x}(n\to\infty)\\
			=&(-1)^n[f(\pi)-f(-\pi)].
		\end{split}
	\end{equation*}
相加就有~$\displaystyle\lim\limits_{n\to\infty}n\int_{-\pi}^{\pi} f(x)\cos{nx}\ud{x}=0$.
\end{proof}

\item 设~$f(x)$~是~$[0,2\pi]$~上的连续函数. 证明
\[\lim_{n\to\infty}\int_0^{2\pi}f(x)|\sin{nx}|\ud{x}=\dfrac{2}{\pi}\int_0^{2\pi}f(x)\ud{x}.\]

\item 设~$f(x)$~在~$[a,b]$~上可微且导函数连续. 证明
\[\max_{a\leqslant x\leqslant b}|f(x)|\leqslant \left|\dfrac{1}{b-a}\int_a^b f(x)\ud{x}\right|+\int_a^b|f'(x)|\ud{x}.\]

\item 设~$f(x)$~在~$[a,b]$~上可积. 令
\[F(x)=\int_a^x f(t)\ud{t}, \quad \forall x\in[a,b].\]
证明:
	\begin{enumerate}[(1)]
	\item $F(x)$~在~$[a,b]$~上{\bf Lipschiz 连续}, 即存在常数~$C>0$~使成立
		\[ |F(x)-F(y)|\leqslant C|x-y|, \quad\forall x,y\in[a,b];\]
	\item 对任意使~$f(x)$~存在右极限~$f(x_0^{+})=\lim\limits_{x\to x_0^{+}}f(x)$~的点~$x_0\in[a,b)$, $F(x)$~在~$x_0$~点有右导数, 且~$F'_{+}(x_0)=f(x_0^{+})$; 对任意使~$f(x)$~存在左极限~$f(x_0^{-})=\lim\limits_{x\to x_0^{-}}f(x)$~的点~$x_0\in(a,b]$, $F(x)$~在~$x_0$~点有左导数, 且~$F'_{-}(x_0)=f(x_0^{-})$; 
	\item $F(x)$~在点~$x_0\in(a,b)$~可微当且仅当~$f(x)$~在点~$x_0$~连续或~$x_0$~是~$f(x)$~的可去间断点. 这时~$F'(x)=\lim\limits_{x\to x_0} f(x)$. 特别当~$f(x)$~在点~$x_0$~连续时有~$F'(x_0)=f(x_0)$.
	\end{enumerate}
\end{enumerate}

\clearpage
\subsection{函数的可积性理论}
\begin{enumerate}
\item 设~$f(x)$~在~$[a,b]$~上可积. 证明: 函数~$|f(x)|$~也在~$[a,b]$~上可积.\\
{\heiti 证明}\\
	对于~$[a,b]$~的任意分割~$\Delta$, 由~$f$~在~$[a,b]$~上可积可知, 对任意~$\varepsilon>0$, 存在~$\delta>0$使得当$||\Delta||<\delta$时有
\[ \sum_{k=1}^n \omega_k(f)\Delta x_k<\varepsilon.\]
注意到
\[ \omega_k(|f|)=\sup_{x,y\in[x_{k-1},x_k]}||f(x)|-|f(y)||\leqslant\sup_{x,y\in[x_{k-1},x_k]}|f(x)-f(y)|=\omega_k(f).\]
于是当~$||\Delta||<\delta$~时,
\[\sum_{k=1}^n \omega_k(|f|)\Delta x_k\leqslant \sum_{k=1}^n \omega_k(f)\Delta x_k<\varepsilon.\]
故~$|f|$~也在~$[a,b]$~上可积.

\item 
	\begin{enumerate}[(1)]
	\item 设~$f(x)$~和~$g(x)$~都在~$[a,b]$~上可积. 证明: 函数~$m(x)=\min\{f(x),g(x)\}$~和~$M(x)=\max\{f(x),g(x)\}$~也都在~$[a,b]$~上可积.\\
	{\heiti 证明}\\
	对于~$\forall x\in [a,b]$, 
	\begin{equation*}
		\begin{split} 
			M(x)=\max\{f(x),g(x)\}=\dfrac{f(x)+g(x)+|f(x)-g(x)|}{2},\\
			m(x)=\min\{f(x), g(x)\}=\dfrac{f(x)+g(x)-|f(x)-g(x)|}{2}.
		\end{split}
	\end{equation*}
	由于~$f$~和~$g$~都在~$[a,b]$~上可积, 故~$f+g$, $|f-g|$~以及~$\dfrac{f+g\pm|f-g|}{2}$~都在~$[a,b]$~上可积, 于是~$M(x)$~和~$m(x)$~都在~$[a,b]$~上可积.

	\item 对任意函数~$f(x)$, 令
		\[f_{+}(x)=\max\{f(x), 0\}, \quad f_{-}(x)=\min\{f(x),0\},\]
	他们分别叫做~$f(x)$~的{\bf 正部}和{\bf 负部}, 证明: ~$f(x)$~在~$[a,b]$~上可积的充要条件是~$f_{+}(x)$~和~$f_{-}(x)$~都在~$[a,b]$~上可积.\\
	{\heiti 证明}\\
	必要性. 当~$f$~在~$[a,b]$~上可积时, 由(1)可知~$f_{+}=\max\{f,0\}$~和~$f_{-}=\min\{f,0\}$~都在~$[a,b]$~上可积. 充分性. 设~$f_{+}$~和~$f_{-}$~都在~$[a,b]$~上可积, 则当~$f(x)>0$~时, $f_{+}(x)=f(x), f_{-}(x)=0$. 

	\item 设~$f(x)$~在~$[a,b]$~上可微, 且导函数~$f'(x)$~在~$[a,b]$~上可积. 证明: $f(x)$~可以写成两个单调递增的连续函数之差.\\
	{\heiti 证明}\\
	$f(x)$~在~$[a,b]$~上可微, 导函数~$f'(x)$~在~$[a,b]$~上可积, 故有
	\[f(x)=\int_a^x f'(t)\ud{t},\]
对于~$f'(x)$~有~$f'(x)=f'_{+}(x)+f'_{-}(x)=f'_{+}(x)-[-f'_{+}(x)]$. 其中$f'_{+}, f'_{-}$~是~$f'$~的正部和负部, 由上题可知~$f'_{+}(x)$, $-f'_{-}(x)$~均在~$[a,b]$~上非负可积, 于是
	\[f(x)=\int_a^x f'(t)\ud{t}=\int_a^x f'_{+}(t)\ud{t}-\int_a^x -f'_{-}(t)\ud{t}, \quad x\geqslant a. \]
是两个非负连续函数之差, 其中非负性由~$f'_{+}, f'_{-}$~的非负性立得, 连续性由可积函数的原函数~Lipschiz~连续从而连续即得.
	\end{enumerate}

\item 设~$\varphi(x)$~在~$[a,b]$~上连续, 值域含于~$[\alpha,\beta]$. 又设~$f$~是~$[\alpha,\beta]$~上的可积函数. 假设~$\varphi(x)$~在~$[a,b]$~上至多只有限次改变增减性, 且在任何区间上都不恒取常值. 证明复合函数~$f\circ \varphi$~是~$[a,b]$~上的可积函数.
\begin{proof}
	先证明当~$\varphi(x)$~在~$[a,b]$~上严格单调时, $f(\varphi(x))$~在~$[a,b]$~上可积.

	不妨假设~$\varphi(x)$~严格单增. 由于~$f$~在~$[\alpha, \beta]$~上可积, 对于任意的分割~$\pi: \alpha=\varphi_0<\varphi_1<\cdots<\varphi_{n-1}<\varphi_n=\beta$~和~$\varepsilon>0$, 存在~$\delta>0$~使得只要~$||\pi||=\max\limits_{1\leqslant k\leqslant n}\{\varphi_k-\varphi_{k-1}\}<\delta$~就有
	\[\sum_{k=1}^n\omega_k(f)\Delta\varphi_i<\varepsilon.\]
由于~$\varphi(x)$~在~$[a,b]$~上连续从而一致连续, 于是对于~$\delta>0$, 存在~$\xi>0$~使得对于任意~$x,y\in[a,b]$, 只要~$|x-y|<\xi$~就有~$|\varphi(x)-\varphi(y)|<\delta$. 对于~$[\alpha, \beta]$, 按照如下方法取一个相应的~$[a,b]$~的分割. 对于每个~$\varphi_k$, 由于~$\varphi(x)$~严格单调故有反函数, 取~$x_k=\varphi^{-1}(\varphi_k)$~, 并且由于~$\varphi(x)$~严格单增, $\varphi(a)=\alpha, \varphi(b)=\beta$, 如果这个分割的模大于~$\psi=\min\{\delta, \xi\}$, 则再添加一些分点(并按照顺序重新编号)形成分割~$\Delta: a=x_0<x_1<\cdots<x_{m-1}<x_m=b$, 于是~$||\Delta||\leqslant||\pi||<\delta$. 记分割~$\pi_{\Delta}=\{x\in\Delta|\varphi(x)\}$~的每个点为~$\varphi_k^{\Delta}$, 则
	\[\sum_{k=1}^m\omega_k(f\circ\varphi)\Delta x_k=\sum_{k=1}^m\sup_{x,y\in [x_{k-1},x_k]}|f(\varphi(x))-f(\varphi(y))|\Delta x_k=\sum_{k=1}^m\sup_{x,y\in [\varphi_{k-1}^{\Delta}, \varphi_k^{\Delta}]}|f(x)-f(y)|\Delta x_k.\]
即~$f\circ\varphi$~在~$[a,b]$~上可积.
\end{proof}

\item 证明: 函数~$f(x)$~在~$[a,b]$~上可积的充要条件是对任意给定的~$\varepsilon>0$, 存在区间~$[a,b]$~的一个分割~$\Delta: a=x_0<x_1<\cdots<x_{n-1}<x_n=b$, 使
\[\sum_{k=1}^n\omega_i(f)\Delta x_i<\varepsilon.\]
\begin{proof}
	必要性由~Daurbox~准则的~II~立得.\\
	充分性. 为了证明~$\lim\limits_{||\Delta||\to 0} \sum_k=1^n \omega_k(f)\Delta x_k=0$, 由于左边极限始终存在, 于是只需证明存在一个分割序列~$||\Delta_m||$~使得~$\lim\limits_{m\to\infty}||\Delta_m||=0$~且~$\lim\limits_{m\to\infty}\sum_k=1^{n_m}\omega_k^m(f)\Delta x_k^{(m)}=0$. 事实上, 取~$\Delta_1=\Delta$, 对于任意正整数~$m$, $\Delta_{m}$取为在~$\Delta_{m-1}$~的相邻的每对分点之间添加其中点形成的分割, 则显然$\lim\limits_{m\to\infty} ||\Delta_m||=\lim\limits_{m\to\infty}\dfrac{||\Delta||}{2^{m-1}}=0$.

	对于任意正整数~$m$, 由于$\omega_k^m(f)\leqslant \omega_{c_k}(f), k=1,2,\cdots, n^m, c_k=\lceil k/2^{m-1}\rceil$, 故有
	\[\sum_{k=1}^{n_m}\omega_k^m(f)\Delta x_k^{(m)}\leqslant \sum_{k=1}^{n}\omega_k(f)\Delta x_k<\varepsilon.\]
于是
	\[\lim_{m\to\infty}\sum_{k=1}^{n_m} \omega_k^m(f)\Delta x_k^{(m)}=0.\]
于是由~Daurbox~准则的~II~可知~$f(x)$~在~$[a,b]$~上可积.
\end{proof}

\item 设~$f(x)$~是定义在半开闭区间~$(a,b]$~上的函数, 它在该区间上有界, 且对任意充分小的~$\sigma>0$, $f(x)$~在~$[a+\sigma,b]$~上可积. 证明: 任意补充~$f(x)$~在区间端点处的值, 所得函数~$f^{*}(x)$~在~$[a,b]$~上可积, 且
\[\lim_{\sigma\to 0^{+}}\int_{a+\sigma}^b f(x)\ud{x}=\int_a^b f^{*}(x)\ud{x}.\]
\begin{proof}
	由于~$f(x)$~在~$(a,b]$~上有界, 任意补充端点的值得到~$f^{*}(x)$~也在~$[a,b]$~上有界, 设其为~$M$. 为证明~$f^{*}(x)$~可积, 对于任意的~$\varepsilon>0$, 取~$\sigma=\dfrac{\varepsilon}{4M+1}$, 由于~$f(x)$~在~$[a+\sigma, b]$~上可积, 故存在一个$[a+\sigma, b]$~的分割~$\Delta'$~使得~$\sum\limits_{k=1}^n\omega'_k(f)\Delta' x_k<\dfrac{\varepsilon}{2}$, 在~$\Delta'$~中添加一分点~$a$~并重新对全部分点编号, 得到分割~$\Delta: a=x_0<x_1<\cdots<x_n<x_{n+1}=b$, 这里有~$x_1=a+\sigma$, 则有
	\[\sum_{k=1}^{n+1} \omega_k(f) \Delta x_k=\omega_1(f)\sigma+\sum_{k=1}^{n} \omega'_k(f)\Delta' x_k\leqslant 2M\cdot\dfrac{\varepsilon}{4M+1}+\dfrac{\varepsilon}{2}<\varepsilon.\]
	故~$f^{*}(x)$~在~$[a,b]$~上可积.

	为证明~$\displaystyle \lim\limits_{\sigma\to 0^{+}}\int_{a+\sigma}^b f(x)\ud{x}=\int_a^b f^{*}(x)\ud{x}$, 对于任意的~$\eta>0$, 取~$\delta=\dfrac{\eta}{M}>0$, 则当~$0<\sigma<\delta$~时,
	\[\int_a^b f^{*}(x)\ud{x}-\int_{a+\sigma}^b f(x)\ud{x}=\int_a^{a+\sigma} f^{*}(x)\ud{x}\leqslant M\sigma<\eta,\]
即
	\[\lim_{\sigma\to 0^{+}}\int_{a+\sigma}^b f(x)\ud{x}=\int_a^b f^{*}(x)\ud{x}.\]
\end{proof}

\item 证明: 有界函数~$f(x)$~在~$[a,b]$~上可积的充要条件是对任意给定的~$\varepsilon>0$~和~$\varepsilon'>0$, 存在区间~$[a,b]$~的分割~$\Delta: a=x_0<x_1<\cdots<x_n=b$, 使得振幅~$\omega_i(f)\geqslant\varepsilon$~的那些小区间的长度总和
\[\sum_{\omega_i(f)\geqslant\varepsilon} \Delta x_i<\varepsilon'.\]
\begin{proof} 充分性. 对于任意给定的~$\xi>0$, 取~$\varepsilon'=\dfrac{\xi}{4M}, \varepsilon=\dfrac{\xi}{2(b-a)}$, 其中~$M$~是函数~$f(x)$在~$[a,b]$~上的界, 即~$|f(x)|\leqslant M, \forall x\in [a,b]$, 从而
\[ \sum_{i=1}^{n}\omega_i(f)\Delta x_i=\sum_{\omega_i(f)\geqslant\varepsilon} \omega_i(f)\Delta x_i+ \sum_{\omega_i(f)<\varepsilon} \omega_{i}(f)\Delta x_i\leqslant 2M\cdot\sum_{\omega_i(f)\geqslant\varepsilon}\Delta x_i+(b-a)\varepsilon<2M\varepsilon'+(b-a)\varepsilon=\xi.\]
即~$f(x)$~在~$[a,b]$~上可积.

必要性. 设~$f(x)$~在~$[a,b]$~上可积. 则对于~$\xi=\varepsilon\varepsilon'>0$, 存在分割~$\Delta$~使得
\[\varepsilon\sum_{\omega_i(f)\geqslant\varepsilon}\Delta x_i\leqslant\sum_{\omega_i(f)\geqslant\varepsilon} \omega_i(f)\Delta x_i\leqslant\sum_{i=1}^n\omega_i(f)\Delta x_i<\xi=\varepsilon\varepsilon'.\]
即~$\sum\limits_{\omega_i(f)\geqslant\varepsilon}\Delta x_i<\varepsilon'$.
\end{proof} 

\item 设~$f(x)$~在~$[a,b]$~上可积, 值域含于区间~$[\alpha,\beta]$. 又设~$g$~是~$[\alpha,\beta]$~上的连续函数. 证明: 复合函数~$g(f(x))$~也在~$[a,b]$~上可积.
\begin{proof}
	 $g$~在~$[\alpha,\beta]$~上连续从而一致连续, 对于任意给定的$\varepsilon>0$, 存在~$\delta>0$~使对于任意~$x,y\in[\alpha,\beta]$~只要满足~$|x-y|<\delta$, 就有~$|f(x)-f(y)|<\varepsilon$. 对于此~$\delta>0$, 由于~$f$~在~$[a,b]$~上可积, 故对于任意~$\xi>0$, 存在~$[a,b]$~的分割~$\Delta$~使成立
\[\sum_{\omega_i(f)\geqslant\delta} \Delta x_i <\xi.\]
于是
\[\sum_{\omega_i(g\circ f)\geqslant\varepsilon} \Delta x_i=(b-a)-\sum_{\omega_i(g\circ f)<\varepsilon}\Delta x_i\leqslant (b-a)-\sum_{\omega_i(f)<\delta} \Delta x_i=\sum_{\omega_i(f)\geqslant\delta} \Delta x_i<\xi.\]
即~$g(f(x))$~也在~$[a,b]$~上可积.
\end{proof}

\item 设定义在~$[a,b]$~上的函数~$f$~满足以下条件: 对任意给定的~$\varepsilon>0$, 存在~$[a,b]$~的有限个子区间, 其长度的总和小于~$\varepsilon$, 使~$f$~在~$[a,b]$~去掉这有限个子区间之后剩下的每个小区间上都连续, 证明: $f$~在~$[a,b]$~上可积.
\begin{proof}
	设~$f$~有界~$M$, 即~$|f(x)|\leqslant M, \forall x \in [a,b]$. 对于任意给定的~$\varepsilon>0$, 存在~$I_1, I_2, \cdots, I_m\subseteq[a,b]$~使得~$\sum\limits_{i=1}^m|I_i|<\dfrac{\varepsilon}{4M}$, 并且若记~$I=\bigcup\limits_{i=1}^{m} I_i$, 有~$f$~在~$[a,b]\backslash I$~上连续. 因此~$f$~在~$[b-a]\backslash I$ (是~$(m+1)$~个不连通子区间) 上可积, 从而存在关于这~$(m+1)$~个小区间上的分割~$\Delta_{1}, \Delta_{2}, \cdots, \Delta_{m+1}$, 使得
\[\sum_{k=1}^{l_n}\omega_{k}(f)\Delta x_k^{(l)}\leqslant\dfrac{\varepsilon}{2(m+1)}, \quad l=1,2,\cdots, m-1.\]
对于~$[a,b]$~做分割~$\bigcup_{k=1}^{m+1}\cup{a,b}$, 并将全部割点重新编号记为~$\Delta: a=x_0<x_1<\cdots<x_n=b$, 则
\[\sum_{k=1}^n\omega_k(f)\Delta x_k=\sum_{[x_{k-1},x_k]\subseteq I}\omega_k(f)\Delta x_k+\sum_{[x_{k-1},x_k]\not\subseteq I}\omega_k(f)\Delta x_k< 2M\cdot \dfrac{\varepsilon}{4M}+\dfrac{\varepsilon}{2(m+1)}\cdot(m+1)=\varepsilon.\]
故~$f$~在~$[a,b]$~上可积.
\end{proof}

\item 设~$f(x)$~在~$[a,b]$~上可积. 证明: $f(x)$~具有积分的连续性, 即对任意~$x\in[a,b]$~成立
\[\lim_{h\to 0}\int_a^b|f(x+h)-f(x)|\ud{x}=0.\]
这里假定~$x\notin [a,b]$~时~$f(x)=0$.
\begin{proof}
	$f(x)$~在~$[a,b]$~上可积, 故等式左边的积分有意义. 为证明等式, 将区间~$n$~等分, 取~$\xi_k=\dfrac{x_{k-1}+x_k}{2}, k=1,2,\cdots, n$. 由~$f$~可积性, 对于任意的~$\varepsilon>0$和~$[a,b]$的等分分割, 当~$\dfrac{b-a}{n}<\delta$~时有~$\sum\limits_{k=1}^n \omega_k(f)\Delta x_k<\dfrac{\varepsilon}{2}$. 由定积分定义, 对于$\varepsilon>0$, 存在~$\delta>0$~使得当~$\dfrac{b-a}{n}<\delta'$~时$\displaystyle |\sum\limits_{k=1}^n |f(\xi_k+h)-f(\xi_k)|\Delta x_k-\int_a^b|f(x+h)-f(x)|\ud{x}\leqslant \dfrac{\varepsilon}{2}$, 故当~$|h|<\dfrac{b-a}{2n}<\min\{\delta,\delta'\}$~时,
\[0\leqslant\int_a^b |f(x+h)-f(x)|\ud{x}<\sum_{k=1}^{n} |f(\xi_k+h)-f(\xi_k)|\Delta x_k+\dfrac{\varepsilon}{2}\leqslant \sum_{k=1}^{n}\omega_k(f)\Delta x_k+\dfrac{\varepsilon}{2}=\varepsilon.\]
故~$\displaystyle\lim_{h\to 0}\int_a^b|f(x+h)-f(x)|\ud{x}=0$.
\end{proof}
\end{enumerate}

\clearpage
\subsection{广义积分}

\clearpage
\section{无穷级数}
\clearpage
\section{函数级数}
\subsection{函数列的一致收敛}
\begin{enumerate}[1.]
\item 证明函数序列~$f_n(x)(n=1,2,\cdots)$~在区间~$I$~上一致收敛于函数~$f(x)$~的充要条件是存在收敛于零的正数列~$\varepsilon_n(n=1,2,\cdots)$~使成立
\[|f_n(x)-f(x)|\leqslant\varepsilon_n, \forall x\in I, n=1,2,\cdots.\]
\begin{proof}
 	必要性. 令~$\varepsilon_n=\sup_{x\in I}|f_n(x)-f(x)|$, 则对于任意的正整数~$n$~和~$x\in I$, $|f_n(x)-f(x)|\leqslant\varepsilon_n$. 由~$f_n$~一致收敛的充要条件, $\lim\limits_{n\to\infty}\varepsilon_n=\lim\limits_{n\to\infty}\sup_{x\in S}|f_n(x)-f(x)|=0$.
	
	充分性. 若存在这样的~$\{\varepsilon_n\}$, 则对于任意给定的~$\varepsilon>0$, 由~$\lim\limits_{n\to\infty}\varepsilon_{n}=0$~知存在正整数~$N$使对于~$n>N$~有~$\varepsilon_n<\varepsilon$, 从而对于~$n>N$~有
\[|f_n(x)-f(x)|\leqslant\varepsilon_n<\varepsilon, \forall x\in I.\]
即~$f_n$~在~$I$~上一致收敛于~$f$.
\end{proof}
 
\item 证明~$f_n(x)$~在~$I$~上不一致收敛于~$f(x)$~的充要条件是存在点列~$x_n\in I(n=1,2,\cdots)$~使当~$n\to\infty$~时, $f_n(x_n)-f(x_n)\not\to 0$.
\begin{proof}
	必要性. $f_n(x)$~在~$I$~上不一致收敛于~$f(x)$, 即存在~$\varepsilon>0$, 对于任意的~$N>0$, 存在~$n>N$~和~$x\in I$~使得~$|f_n(x)-f(x)|\geqslant \varepsilon$. 下面取一个~$I$~中的点列. \\
	取~$N_1=1$, 则存在~$n_1>1$~和~$x_{n_1}\in I$~使得~$|f_{n_1}(x_{n_1})-f(x_{n_1})|\geqslant \varepsilon$;\\
	取~$N_2=n_1$, 则存在~$n_2>n_1$~和~$x_{n_2}\in I$~使得~$|f_{n_2}(x_{n_2})-f(x_{n_2})|\geqslant \varepsilon$;\\
	... ...\\
	以此类推, 归纳地可以取一个单调递增数列~$\{n_k\}$. 对于~$n\neq n_k, k=1,2,\cdots$, 任取~$x_n\in I$, 由此可以得到一个~$I$~中数列~$\{x_n\}$~并且有子数列~$\{x_{n_k}\}$使得~$|f_{n_k}(x_{n_k})-f(x_{n_k})|\geqslant \varepsilon$, 从而~$f_n(x_n)-f(x_n)\not\to 0$.

	充分性. 若存在这样的~$\{x_n\}$, 则存在~$\varepsilon>0$, 对于任意的~$N>0$, 存在~$n>N$~使得~$|f_n(x_n)-f(x_n)|\geqslant\varepsilon$, 由于对于每个正整数~$n$~都有~$x_n\in I$, 即存在~$n>N$~和~$x\in I$~使得~$|f_n(x)-f(x)|\geqslant\varepsilon$(每个~$x_n$~都符合), 即~$f_n(x)$~在~$I$~上不一致收敛于~$f(x)$.
\end{proof}

\item 设~$\{f_n(x)\}$~和~$\{g_n(x)\}$~都在~$I$~上一致收敛, 且对每个~$n$, ~$f_n(x)$~和~$g_n(x)$~都是~$I$~上的有界函数. 证明: $\{f_n(x)g_n(x)\}$~在~$I$~上一致收敛.
\begin{proof}
	设~$\{f_n\}$~一致收敛到~$f$, $\{g_n\}$~一致收敛到~$g$. 故对于$\varepsilon=1$, 存在~$N_1$~当~$n>N_1$~时对于所有~$x\in I$~有~$|f(x)-f_n(x)|<1$. 取定~$n$~后由于~$f_n(x)$~在~$I$~上有界, 故存在~$M'>0$~使对于所有~$x\in I$~有~$|f_n(x)|\leqslant M'$, 从而对于所有~$x\in I$~有
\[ |f(x)|\leqslant|f(x)-f_n(x)|+|f_n(x)|<1+M'.\]
即~$f$~也是~$I$~上有界函数. 再次用~$n>N_1$~时对于所有~$x\in I$~有~$|f(x)-f_n(x)|\leqslant 1$, 则当~$n>N_1$~时, 
\[ |f_n(x)|\leqslant|f_n(x)-f(x)|+|f(x)|<1+1+M', \forall x\in I.\]
从而取~$M_f=\max\{M_1,M_2,\cdots, M_{N_1}, 2+M'\}$, 则~$M$~是全体~$\{f_n\}$~公共的上界. 同理存在~$M_g$~是全体~$\{g_n\}$~公共的上界. 对于任意的~$\varepsilon$, 由一致收敛的~Cauchy~准则, 存在~$N_2,N_3$~当~$m,n>N_2$~时对于所有~$x\in I$~有~$|f_m(x)-f_n(x)|\leqslant\dfrac{\varepsilon}{2M_g}$, 当~$m,n>N_3$~时对于所有的~$x\in I$~有~$|g_m(x)-g_n(x)|\leqslant\dfrac{\varepsilon}{2M_f}$. 取~$N=\max\{N_2,N_3\}$, 则当~$m,n>N$~时对于所有~$x\in I$~有
\[|f_m(x)g_m(x)-f_n(x)g_n(x)|\leqslant|f_m(x)-f_n(x)||g_m(x)|+|f_n(x)||g_m(x)-g_n(x)|<\dfrac{\varepsilon}{2M_g}M_g+M_f\dfrac{\varepsilon}{2M_f}=\varepsilon.\]
从而由一致收敛的~Cauchy~准则知~$\{f_ng_n\}$~在~$I$~上一致收敛.
\end{proof}

\item 设~$\{f_n(x)\}$~在区间~$I$~上一致收敛于~$f(x)$, 且对于每个~$n$, ~$f_n(x)$~都是~$I$~上的有界函数, 值域含于闭区间~$J$. 又设~$g(x)$~是~$J$~上的连续函数. 证明: $\{g(f_n(x))\}$~在~$I$~上一致收敛于函数~$g(f(x))$.
\begin{proof}
	对于任意给定的~$\varepsilon>0$, 由~$g$~在~$J$~上连续, 从而一致连续可知, 存在~$\delta>0$~使对于任意~$x,y\in J$, 只要~$|x-y|<\delta$, 就有~$|g(x)-g(y)|<\varepsilon$. 对于这个~$\delta>0$, 由~$f_n$~在~$I$~上一致收敛于~$f$~可知存在~$N$~当~$n>N$~时对于所有~$x\in I$~都有~$|f_n(x)-f(x)|\leqslant\delta$. 对于每个~$n$, ~$f_n(x)$~是~$I$~上的有界函数, 值域含于闭区间~$J=[a,b]$, 即对于每个~$n$,
\[a\leqslant f_n(x)\leqslant b, \forall x\in I.\]
由~$f_n$~一致收敛于~$f$, 从而~$f_n$~也逐点收敛到~$f$, 对于每个~$x\in I$, 在上式中令~$n\to\infty$, 则
\[a\leqslant f(x)\leqslant b, \forall x\in I.\]
即对于任意的~$n$, $f_n$~和~$f$在~$I$~上的值域都含于~$J$. 从而当~$n>N$~时对于所有~$x\in I$, $f_n(x), f(x)\in J$~且~$|f_n(x)-f(x)|<\delta$, 于是
\[|g(f_n(x))-g(f(x))|<\varepsilon.\]
即~$g_n(x)$~在~$I$~上一致收敛于~$g(f(x))$.
\end{proof}

\item 设存在收敛的正项级数~$\sum\limits_{n=1}^{\infty}M_n$~使成立
\[ |f_{n+1}(x)-f_n(x)|\leqslant M_n, \forall x\in I, n=1,2,\cdots.\]
证明: $f_n(x)$~在~$I$~上一致收敛.
\begin{proof}
	因为~$\sum\limits_{n=1}^{\infty}M_n$~收敛, 对于任意给定的~$\varepsilon>0$, 由级数收敛的~Cauchy~法则, 存在~$N>0$~使对于~$n>N$~和任意~$p$~有~$\sum_{k=1}^p M_{n+k}<\varepsilon$. 对于任意的~$m,n>N$ (不妨设~$m>n$, 否则交换~$m,n$) 有
\[ |f_m(x)-f_n(x)|\leqslant \sum_{k=n+1}^m |f_k-f_{k-1}(x)|\leqslant\sum_{k=n+1}^m M_k=\sum_{k=1}^{m-n} M_k<\varepsilon, \forall x\in I.\]
于是由一致收敛的~Cauchy~准则知, $f_n$~在~$I$~上一致收敛. 
\end{proof}

\item 设~$\{f_n(x)\}$~在有界闭区间~$I$~上逐点收敛于函数~$f(x)$, 且存在~$M>0$~和~$0<\alpha\leqslant1$~使成立
\[ |f_n(x)-f_n(y)|\leqslant M|x-y|^{\alpha}, \forall x,y \in I, n=1,2,\cdots.\]
证明: $f_n(x)$~在~$I$~上一致收敛于~$f(x)$.
\begin{proof}
	任取~$x_0\in I$. 对于任意给定的~$\varepsilon>0$, 取~$\delta<\sqrt[\alpha]{\dfrac{\varepsilon}{3M}}$, 则对于任意~$x\in(x_0-\delta,x_0+\delta)$,
\[|f_n(x)-f_n(x_0)|\leqslant M|x-x_0|^{\alpha}<\dfrac{\varepsilon}{3}.\]
在上面不等式中令~$n\to\infty$, 则有~$|f(x_0)-f(x)|\leqslant \dfrac{\varepsilon}{3}$. 由~$\lim\limits_{n\to\infty} f_n(x_0)=f(x_0)$~知存在~$N$~使得当~$n>N$~时有~$|f_n(x_0)-f(x_0)|\leqslant \dfrac{\varepsilon}{3}$. 于是当~$n>N$~时对于任意~$x\in(x_0-\delta,x_0+\delta)$~有
\[|f_n(x)-f(x)|\leqslant|f_n(x)-f_n(x_0)|+|f_n(x_0)-f(x_0)|+|f(x_0)-f(x)|<\dfrac{\varepsilon}{3}+\dfrac{\varepsilon}{3}+\dfrac{\varepsilon}{3}=\varepsilon.\]
即~$f_n$~在~$x_0$~附近局部一致收敛于~$f$. 由~$x_0$~任意性, ~$f_n$~在~$I$~上一致收敛于~$f$.
\end{proof}

\item 设~$\{f_n(x)\}$~是区间~$[a,b]$~上的一列单调函数, 且在~$[a,b]$~上逐点收敛于连续函数~$f(x)$. 证明: $f_n(x)$~在~$[a,b]$~上一致收敛于~$f(x)$.
\begin{proof}
任取~$x_0\in I$, 由于~$f$~在~$I$~上连续, 特别在~$x_0$~连续, 故存在~$\delta>0$~使对于满足~$|x-x_0|<\delta$~的~$x\in I$~均有~$|f(x_0)-f(x)|<\dfrac{\varepsilon}{3}$. 对于每个~$n$, 由于~$f_n(x)$~是$I$~上的单调函数, 特别在~$I_{x_0}=[x_0-\delta,x_0+\delta]\cap I$~上单调从而可积, 将~$I_{x_0}$~分为~$m$~份使得~$\dfrac{b-a}{m+1}<\delta<\dfrac{b-a}{m}$~且~$\sum\limits_{k=1}^m \omega_{k}(f_n)\Delta x_k<\dfrac{\delta\varepsilon}{3}$, 从而对于~$|x-x_0|<\delta$~的~$x\in I$~有
\[|f_n(x)-f_n(x_0)|\delta\leqslant\omega(f_n)\delta=\sum_{k=1}^m\omega_k(f_n)\delta<\dfrac{\delta\varepsilon}{3}.\]
即~$|f_n(x)-f_n(x_0)|<\dfrac{\varepsilon}{3}$. 由于~$f_n$~在~$I$~上逐点收敛于~$f$, 有~$\lim\limits_{n\to\infty} f_n(x_0)=f(x_0)$, 即存在~$N$~使对于~$n>N$~有~$|f_n(x_0)-f(x_0)|<\dfrac{\varepsilon}{3}$. 于是当~$n>N$~时对于~$|x-x_0|<\delta$~的~$x\in I$~有
\[|f_n(x)-f(x)|\leqslant|f_n(x)-f_n(x_0)|+|f_n(x_0)-f(x_0)|+|f(x_0)-f(x)|<\dfrac{\varepsilon}{3}+\dfrac{\varepsilon}{3}+\dfrac{\varepsilon}{3}=\varepsilon.\]
即~$f_n$~在~$x_0$~附近局部一致收敛于~$f$. 由~$x_0$~任意性, ~$f_n$~在~$I$~上一致收敛于~$f$.
\end{proof}

\item 设对每个正整数~$n$, 函数~$f_n(x)$~在区间~$I$~上有界. 又设当~$n\to\infty$~时, $f_n(x)$~在~$I$~上一致收敛于~$f(x)$. 证明:
	\begin{enumerate}[(1)]
		\item 极限函数~$f(x)$~在~$I$~上有界;
		\item 函数序列~$f_n(x)(n=1,2\cdots)$~在~$I$~上一致有界, 即存在~$M>0$~使对所有~$n$~都有
	\[|f_n(x)\leqslant M, \forall x\in I.\]
	\end{enumerate}
\begin{proof}
	对于$\varepsilon=1$, 由于~$f_n$~在~$I$~上一致收敛于~$f$, 存在~$N$~当~$n>N$~时对于所有~$x\in I$~有~$|f(x)-f_n(x)|<1$. 取定~$n$, 由于~$f_n(x)$~在~$I$~上有界, 故存在~$M'>0$~使对于所有~$x\in I$~有~$|f_n(x)|\leqslant M$, 从而对于所有~$x\in I$~有
\[ |f(x)|\leqslant|f(x)-f_n(x)|+|f_n(x)|<1+M'.\]
即~$f$~也是~$I$~上有界函数. 再次用~$f$~在~$I$~上一致收敛于~$f$, $n>N$~时对于所有~$x\in I$~有~$|f(x)-f_n(x)|\leqslant 1$, 则当~$n>N$~时, 
\[ |f_n(x)|\leqslant|f_n(x)-f(x)|+|f(x)|<1+1+M', \forall x\in I.\]
从而取~$M_f=\max\{M_1,M_2,\cdots, M_{N_1}, 2+M'\}$, 则对于所有~$n$~都有
\[|f_n(x)\leqslant M, \forall x\in I.\]
即~$f_n$~在~$I$~上一致有界.
\end{proof}

\item 设~$f(x)$~在~$[0,1]$~上~Riemann~可积, 且在左端点~$0$~处右连续. 证明:
\[ \lim\limits_{n\to\infty}\int_0^\frac{1}{n}\dfrac{nf(x)}{1+n^2x^2}\ud{x}=\dfrac{\pi}{4}f(0).\]
\end{enumerate}

\clearpage
\subsection{Weierstrass~逼近定理和~Arzel\`a-Ascoli~定理}
\begin{enumerate}[1.]
\item 设~$f(x)$~在~$[0,1]$~上连续, 且~$\displaystyle \int_0^1 f(x)x^n\ud{x}=0, n=0,1,2,\cdots$. 证明: $f(x)=0, \forall x\in[0,1]$.
\begin{proof}
	设~$f(x)\not\equiv0$, 即存在~$x_0\in[0,1]$~使得~$f(x_0)\neq 0$, 不妨设为~$f(x_0)>0$. 则由~$f(x)$~在~$[0,1]$~上连续知, 存在~$\delta>0$~使得任意~$x\in I=[x_0-\delta,x_0+\delta]\cap[0,1], f(x)x^n>0$. 
\[\int_0^1 f(x)x^n\ud{x}\geqslant\int_{I} f(x)x^n\ud{x}>0.\]
与~$\displaystyle \int_0^1 f(x)x^n\ud{x}>0$~矛盾, 故~$f(x)=0, \forall x\in[0,1]$.
\end{proof}
\end{enumerate}
\clearpage
%\section{多元数量函数微分学}
%\subsection{多远连续函数}
%\subsection{偏导数和全微分}
%\clearpage
\section{重积分}
\begin{enumerate}[1.]
\item 计算下列二重积分
	\begin{enumerate}[(1)]
	\item $\displaystyle\iint\limits_D xy(x+y)^2\ud{x}\ud{y}, D=[a,b]\times[c,d]$;
	\item $\displaystyle\iint\limits_D xy\e^{xy}\ud{x}\ud{y}, D=[0,a]\times[0,b]$;
	\item $\displaystyle\iint\limits_D \dfrac{x}{1+xy}\ud{x}\ud{y}, D=[0,1]\times[0,1]$;
	\end{enumerate}
\begin{proof}
	\begin{enumerate}[(1)]
	\item
	\item
	\item $\displaystyle\iint\limits_D \dfrac{x}{1+xy}\ud{x}\ud{y}=\int_0^1\ud{x}\int_0^1\dfrac{x}{1+xy}\ud{y}=\int_0^1\ln(1+x)\ud{x}=\ln{4}-1$.
	\item $\displaystyle\iint\limits_D \sin{x}\sin{y}\sin{(x-y)}\ud{x}\ud{y}=\iint\limits_D \sin^2{x}\sin{y}\cos{y}-\sin{x}\cos{x}\sin^2{y}\ud{x}\ud{y}\\
			=\int_0^{\frac{\pi}{2}}\ud{x}\int_0^{\frac{\pi}{2}}(\sin^2{x}\sin{y}\cos{y}-\sin{x}\cos{x}\sin^2{y})\ud{y}
			=\int_0^{\frac{\pi}{2}}\left(\dfrac{1}{2}\sin^2{x}-\dfrac{\pi}{4}\cos{x}\sin{x}\right)\ud{x}\\
			=-\dfrac{1}{4}\int_0^{\frac{\pi}{2}}\cos{2x}\ud{x}=0$
	\end{enumerate}
\end{proof}
\end{enumerate}
\end{document}