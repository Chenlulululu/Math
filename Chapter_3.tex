\ifx\all\undefined
\documentclass[a4paper,11pt,twoside]{ctexbook}
\usepackage[top=1in, bottom=1in, left=1in, right=1in]{geometry}
\usepackage{tablists}
\usepackage{amsmath}
\usepackage{amsfonts}
\usepackage{amsthm}
\usepackage{amssymb}
\usepackage{bm}
\usepackage{extarrows}
\usepackage{enumerate}
\usepackage{titlesec}
\usepackage{graphicx}
\newcommand{\ud}[1]{\mathrm{d}#1}
\newcommand{\e}{\mathrm e}
\DeclareMathOperator{\arccot}{arccot}
\DeclareMathOperator{\arcsec}{arcsec}
\DeclareMathOperator{\arccsc}{arccsc}
\DeclareMathOperator{\sech}{sech}
\DeclareMathOperator{\csch}{csch}
\DeclareMathOperator{\arsinh}{arsinh}
\DeclareMathOperator{\arcosh}{arcosh}
\DeclareMathOperator{\sgn}{sgn}
\pagestyle{empty}
\setlength\abovedisplayskip{3pt}
\setlength\belowdisplayskip{3pt}
\renewcommand{\proofname}{证明}
\begin{document}
\fi

\setcounter{chapter}{3}
\chapter{函数极限}
\section{函数极限的定义}
\subsection{思考题 pp. 98.}
\begin{enumerate}
	\item 以下几种叙述能否作为函数极限$\lim\limits_{x\to a} f(x)=A$的定义?
	      \begin{enumerate}[(1)]
		      \item $\forall\varepsilon>0, \exists \delta>0, \forall x\in O_{\delta}(a)-\{a\}$, 成立$|f(x)-A|\leqslant\varepsilon$;
		      \item $\forall\varepsilon>0, \exists \delta>0, \forall x\in O_{\delta}(a)-\{a\}$, 成立$|f(x)-A|<k\varepsilon$;
		      \item $\forall n\in\mathbf{N}_{+}, \exists \delta>0, \forall x\in O_{\delta}(a)-\{a\}$, 成立$|f(x)-A|<\dfrac{1}{n}$;
		      \item $\forall\varepsilon>0, \exists n, \forall x\in O_{\frac{1}{n}}(a)-{a}$, 成立$|f(x)-A|<\varepsilon$.
	      \end{enumerate}
	      \begin{proof}[解答]
		      均可以. 只需验证各项均与函数极限的定义等价即可.
		      \begin{enumerate}[(1)]
			      \item $\Rightarrow$$\forall\varepsilon>0, \dfrac{\varepsilon}{2}>0, \exists\delta>0, \forall x\in O_{\delta}(a)-\{a\}$,  成立$|f(x)-A|\leqslant\dfrac{\varepsilon}{2}<\varepsilon$.

			            $\Leftarrow$注意$\varepsilon\leqslant\varepsilon$即可.
			      \item 注意对于$\forall\varepsilon>0$, 有$k\varepsilon>0, \dfrac{\varepsilon}{k}>0$即可.
			      \item $\Rightarrow$由于$\lim\limits_{n\to\infty} \dfrac{1}{n}=0$, 故对于$\forall\varepsilon>0, \exists n_0\in\mathbf{N}_{+}$使得$\dfrac{1}{n}<\varepsilon$. 对于$n_0$, $\exists\delta>0, \forall x\in O_{\delta}(a)-\{a\}$, 成立$|f(x)-A|<\dfrac{1}{n_0}<\varepsilon$.

			            $\Leftarrow$取$\varepsilon=\dfrac{1}{n}$即可.
			      \item $\Rightarrow$取$\delta=\dfrac{1}{n}$即可.

			            $\Leftarrow$由于$\lim\limits_{n\to\infty} \dfrac{1}{n}=0$, 故对于$\forall\delta>0, \exists n_0\in\mathbf{N}_{+}$使得$\dfrac{1}{n}<\delta$. $\forall\varepsilon>0, \exists\delta>0, \forall x\in O_{\delta}(a)-\{a\}$, 成立$|f(x)-A|<\varepsilon$. 取合适的$n_0$使得$\dfrac{1}{n_0}<\delta$, 则对于$\forall x\in O_{\frac{1}{n}}(a)-\{a\}\subset O_{\delta}(a)-\{a\}$, 成立$|f(x)-A|<\varepsilon$. \qedhere
		      \end{enumerate}
	      \end{proof}
	\item 以下几种叙述能否作为函数极限$\lim\limits_{x\to a} f(x)=A$的定义?
	      \begin{enumerate}[(1)]
		      \item $\exists\delta>0, \forall\varepsilon>0, \forall x\in O_{\delta}(a)-\{a\}$, 成立$|f(x)-A|<\varepsilon$;
		      \item $\forall\delta>0, \exists\varepsilon>0, \forall x\in O_{\delta}(a)-\{a\}$, 成立$|f(x)-A|<\varepsilon$;
		      \item 当$x$充分靠近$a$时, $f(x)$越来越接近$A$.
	      \end{enumerate}
	      \begin{proof}[解答]
		      均可举出反例来说明与函数极限的定义不等价.
		      \begin{enumerate}[(1)]
			      \item 描述了函数$f(x)\equiv A, \forall x\in O_{\delta}(a)-\{a\}$;
			      \item 对于有界函数恒成立;
			      \item 对于函数$f(x)=x\sin{x}, A=0$, $f(x)\equiv A, \forall x\in O_{\delta}(a)-\{a\}$均不成立. \qedhere
		      \end{enumerate}
	      \end{proof}

	\item 用对偶法则给出: (1) ``$f(x)$在点$a$不收敛于$A$''的正面描述; (2) ``$f(x)$在点$a$处没有极限''的正面描述.
	      \begin{proof}[解答]
		      \begin{enumerate}[(1)]
			      \item $\exists\varepsilon>0, \forall \delta>0, \exists x\in O_{\delta}(a)-\{a\}$, 成立$|f(x)-A|\geqslant\varepsilon$.
			      \item $\forall A, \exists\varepsilon>0, \forall \delta>0, \exists x\in O_{\delta}(a)-\{a\}$, 成立$|f(x)-A|\geqslant\varepsilon$.
		      \end{enumerate}
	      \end{proof}

	\item 怎样用正面方式叙述以下否等性概念:
	      \begin{tabenum}[(1)]
		      \tabenumitem $\lim\limits_{x\to\infty} f(x)\neq A$;
		      \tabenumitem $\lim\limits_{x\to-\infty} f(x)\neq A$;\\
		      \tabenumitem $\lim\limits_{x\to a} f(x)\neq\infty$;
		      \tabenumitem $\lim\limits_{x\to a^{-}} f(x)\neq A$;\\
		      \tabenumitem $\lim\limits_{x\to a^{+}} f(x)\neq +\infty$.
	      \end{tabenum}
	      \begin{proof}[解答]
		      \begin{enumerate}[(1)]
			      \item $\exists\varepsilon>0, \forall M>0, \exists |x|>M$, 成立$|f(x)-A|\geqslant\varepsilon$;
			      \item $\exists\varepsilon>0, \forall M>0, \exists x<-M$, 成立$|f(x)-A|\geqslant\varepsilon$;
			      \item $\exists M>0, \forall\delta>0, \exists x\in O_{\delta}(a)-\{a\}$, 成立$|f(x)|\leqslant M$;
			      \item $\exists\varepsilon>0, \forall\delta>0, \exists a-\delta<x<a$, 成立$|f(x)-A|\geqslant\varepsilon$;
			      \item $\exists M>0, \forall\delta>0, \exists a<x<a+\delta$, 成立$f(x)\leqslant M$. \qedhere
		      \end{enumerate}
	      \end{proof}
\end{enumerate}

\subsection{练习题 pp. 102.}
以下各题要求按照函数极限的定义来做.
\begin{enumerate}
	\item 证明: $\lim\limits_{x\to 0} \dfrac{\sqrt{1+x}-\sqrt{1-x}}{x}=1$.
	      \begin{proof}
		      注意$\dfrac{\sqrt{1+x}-\sqrt{1-x}}{x}=\dfrac{2}{\sqrt{1-x}+\sqrt{1+x}}$, 对于$|x|<\dfrac{1}{4}$,
		      \[
			      \begin{split}
				      \left| \dfrac{\sqrt{1+x}-\sqrt{1-x}}{x}-1\right|=&\left|\dfrac{2}{\sqrt{1-x}+\sqrt{1+x}}-1\right|\\
				      \leqslant&\dfrac{1}{\sqrt{1+x}+\sqrt{1-x}}\cdot(|1-\sqrt{1+x}|+|1-\sqrt{1-x})\\
				      =&\dfrac{1}{\sqrt{1+x}+\sqrt{1-x}}\cdot\left(\dfrac{|x|}{1+\sqrt{1+x}}+\dfrac{|x|}{1+\sqrt{1-x}}\right)\\
				      \leqslant&\dfrac{2|x|}{\sqrt{1+x}+\sqrt{1-x}}\\
				      <&\dfrac{2\sqrt{3}}{3}|x|.
			      \end{split}
		      \]
		      对于任意给定的$\varepsilon>0$, 取$\delta=\min\left\{\dfrac{1}{4}, \dfrac{\sqrt{3}\varepsilon}{2}\varepsilon\right\}$, 则当$0<|x|<\delta$时,
		      \[
			      \left| \dfrac{\sqrt{1+x}-\sqrt{1-x}}{x}-1\right|<\dfrac{2\sqrt{3}}{3}|x|<\dfrac{2\sqrt{3}}{3}\dfrac{\sqrt{3}}{2}\varepsilon=\varepsilon. \qedhere
		      \]
	      \end{proof}

	\item 证明: $\lim\limits_{x\to 1} \dfrac{x^2+x-2}{x(x^2-3x+2)}=-3$.
	      \begin{proof}
		      注意当$x\neq 1$时, $\dfrac{x^2+x-2}{x(x^2-3x+2)}=\dfrac{(x+2)(x-1)}{x(x-1)(x-2)}=\dfrac{x+2}{x(x-2)}$, 故对于$0<|x|<\dfrac{1}{2}$,
		      \[
			      \begin{split}
				      \left|\dfrac{x^2+x-2}{x(x^2-3x+2)}+3\right|=&\left|\dfrac{x+2}{x(x-2)}+3\right|\\
				      =&\left|\dfrac{(3x-2)(x-1)}{x(x-2)}\right|\\
				      \leqslant& \left|\dfrac{3x-2}{x(x-2)}\right||x-1|\\
				      <&\dfrac{5}{2}\dfrac{1}{|x(x-2)|}|x-1|\\
				      <&\dfrac{5}{2}\cdot 2|x-1|\\
				      =&5|x-1|.
			      \end{split}
		      \]
		      对于任意给定的$\varepsilon>0$, 取$\delta=\min\left\{\dfrac{1}{2}, \dfrac{\varepsilon}{5}\right\}$, 则当$0<|x|<\delta$时,
		      \[
			      \left|\dfrac{x^2+x-2}{x(x^2-3x+2)}+3\right|<5|x-1|<5\cdot\dfrac{\varepsilon}{5}=\varepsilon. \qedhere
		      \]
	      \end{proof}

	\item 证明: $\lim\limits_{x\to+\infty} \dfrac{x+1}{x^2-x}=0$.
	      \begin{proof}
		      注意当$x\neq 1$时, $\dfrac{x+1}{x^2-x}=\dfrac{x+1}{x(x+1)}=\dfrac{1}{x}$. 故对于任意给定的$\varepsilon>0$, 取$M=\dfrac{1}{\varepsilon}$, 当$x>M$时,
		      \[
			      \left|\dfrac{x+1}{x^2-x}\right|=\dfrac{1}{x}<\varepsilon. \qedhere
		      \]
	      \end{proof}

	\item 当$a$取什么数值时, $\lim\limits_{x\to -1} \dfrac{x^3-ax^2-x+4}{x+1}$存在? 此时极限为何?
	      \begin{proof}[解答]
		      极限存在当且仅当$x^3-ax^2-x+4$含有一个$(x+1)$的因子, 即$x=-1$是$x^3-ax^2-x+4=0$的根. 故$-1-a+1+4=0$, 故$a=4$.
		      代入$a=4$, 有
		      \[
			      \lim\limits_{x\to -1} \dfrac{x^3-4x^2-x+4}{x+1}=\lim\limits_{x\to -1} (x-4)(x-1)=10. \qedhere
		      \]
	      \end{proof}

	\item 求$a,b$, 使$\lim\limits_{x\to 2} \dfrac{x^2+ax+b}{x^2-x-2}=2$.
	      \begin{proof}[解答]
		      注意$x^2-x-2=(x-2)(x+1)$, 故$\lim\limits_{x\to 2} \dfrac{x^2+ax+b}{x^2-x-2}=2$当且仅当$x=-4, x=2$是$x^2+ax+b=0$的两个根, 代入就有
		      \[
			      \begin{cases} 16-4a+b=0, \\ 4+2a+b=0. \end{cases} \Rightarrow \begin{cases} a=2,\\ b=-8. \end{cases} \qedhere
		      \]
	      \end{proof}

	\item 问: 使得$\lim\limits_{x\to 0^{+}} \dfrac{a+\sin{\dfrac{1}{x}}}{x}=\pm\infty$的参数$a$是什么?
	      \begin{proof}[解答]
		      当$a\geqslant 1$时, $\lim\limits_{x\to 0^{+}} \dfrac{a+\sin{\dfrac{1}{x}}}{x}=+\infty$;

		      当$a\leqslant -1$时, $\lim\limits_{x\to 0^{+}} \dfrac{a+\sin{\dfrac{1}{x}}}{x}=-\infty$;

		      当$-1<a<1$时, $\dfrac{a+\sin{\dfrac{1}{x}}}{x}$是符号不定的无穷大量. \qedhere
	      \end{proof}

	\item 证明: $\lim\limits_{x\to a} \ln{x}=\ln{a}$, 其中$a>0$.
	      \begin{proof}
		      对于任意给定的$\varepsilon>0$,
		      \[
			      \begin{split}
				      |\ln{x}-\ln{a}|<\varepsilon &\Leftrightarrow  \ln{a}-\varepsilon<\ln{x}<\ln{a}+\varepsilon \\
				      &\Leftrightarrow  a\e^{-\varepsilon}<x<a\e^{\varepsilon}\\
				      &\Leftrightarrow  a(\e^{-\varepsilon}-1)<x-a<a(\e^{\varepsilon}-1)\\
				      &\Leftrightarrow -a(1-\e^{-\varepsilon})<x-a<a(\e^{\varepsilon}-1).
			      \end{split}
		      \]
		      注意$a(1-\e^{-\varepsilon}), a(\e^{\varepsilon}-1)>0$, 故取$\delta=\min\{a(1-\e^{-\varepsilon}), a(\e^{\varepsilon}-1)\}$, 则当$0<|x-a|<\delta$时有$|\ln{x}-\ln{a}|<\varepsilon$. \qedhere
	      \end{proof}

	\item 证明: $\lim\limits_{x\to a} \e^{x}=\e^{a}$.
	      \begin{proof}
		      对于任意给定的$\varepsilon>0$,
		      \[
			      \begin{split}
				      |\e^{x}-\e^{a}|<\varepsilon &\Leftrightarrow \e^{a}-\varepsilon<\e^{x}<\e^{a}+\varepsilon\\
				      &\Leftrightarrow 1-\varepsilon\e^{-a}<\e^{x-a}<1+\varepsilon\e^{-a}\\
				      &\Leftrightarrow \ln(1-\varepsilon\e^{-a})<x-a<\ln(1+\varepsilon\e^{-a}).
			      \end{split}
		      \]
		      注意$-\ln(1-\varepsilon\e^{-a}),\ln(1+\varepsilon\e^{-a})>0$, 故取$\delta=\min\{-\ln(1-\varepsilon\e^{-a}),\ln(1+\varepsilon\e^{-a})\}$, 则当$0<|x-a|<\delta$时有$|\e^{x}-\e^{a}|<\varepsilon$. \qedhere
	      \end{proof}

	\item 证明$\lim\limits_{x\to 0} f(x)$与$\lim\limits_{x\to 0} f(x^3)$同时存在或不存在, 而当它们存在时必相等. \footnote{本题事实上就是在证明复合函数的极限定理.}
	      \begin{proof}
		      容易证明: $\lim\limits_{x\to 0} \sqrt[3]{x}=\lim\limits_{x\to 0} x^3=0$.
		      若$\lim\limits_{x\to 0} f(x)=A$, 则
		      \begin{enumerate}[(a)]
			      \item 对于任意给定的$\varepsilon>0$, 存在$\delta>0$, 使得当$0<|x|<\delta$时, $|f(x)-A|<\varepsilon$;
			      \item 对于任意给定的$\delta>0$, 存在$\eta>0$, 使得当$0<|x|<\eta$时, $|x^3|<\varepsilon$.
		      \end{enumerate}
		      注意$x\neq 0$时, $x^3\neq 0$, 故当$0<|x|<\eta$时, $0<|x^3|<\delta$, 从而当$0<|x|<\eta$时有
		      \[
			      |f(x^3)-A|<\varepsilon.
		      \]

		      若$\lim\limits_{x\to 0} f(x^3)=A$, 则
		      \begin{enumerate}[(a)]
			      \item 对于任意给定的$\varepsilon>0$, 存在$\delta>0$, 使得当$0<|x|<\delta$时, $|f(x^3)-A|<\varepsilon$;
			      \item 对于任意给定的$\delta>0$, 存在$\eta>0$, 使得当$0<|x|<\eta$时, $|\sqrt[3]{x}|<\delta$.
		      \end{enumerate}
		      注意$x\neq 0$时, $\sqrt[3]{x}\neq 0$, 故当$0<|x|<\eta$时, $0<|\sqrt[3]{x}|<\delta$, 从而当$0<|x|<\eta$时有
		      \[
			      |f(x)-A|=|f((\sqrt[3]{x})^3)-A|<\varepsilon.
		      \]
		      这就证明了$\lim\limits_{x\to 0} f(x)$与$\lim\limits_{x\to 0} f(x^3)$同时存在或不存在, 而当它们存在时必相等. \qedhere
	      \end{proof}

	\item 问$\lim\limits_{x\to 0} f(x)$与$\lim\limits_{x\to 0} f(x^2)$是否一定同时存在或不存在?
	      \begin{proof}[解答]
		      不一定. 主要问题在于$\sqrt{x^2}\not\equiv x$. 如取$f(x)=\sgn{x}$. \footnote{事实上类似于9., 可以证明有$\lim\limits_{x\to 0} f(x^2)\equiv\lim\limits_{x\to 0^{+}} f(x)$.}
	      \end{proof}

	\item 证明: 如下定义的Dirichlet函数
	      \[
		      D(x)=\begin{cases} 1, & \text{$x$是有理数}, \\ 0, & \text{$x$是无理数}.\end{cases}
	      \]
	      在每一点都没有极限.
	      \begin{proof}[方法一]
		      $\forall x_0\in(-\infty,+\infty)$, 若极限$\lim\limits_{x\to x_0} D(x)$存在, 则只有可能是$0$或$1$(否则对于任意$A\neq 0,1$, 显然无论$\delta$多么小都存在$x\in(x_0-\delta,x_0+\delta)-\{x_0\}$使得$|D(x)-A|>\dfrac{\min\{|1-A|,|A|\}}{2}$.).
		      对于任意$\delta>0$, 存在有理数$\xi\in(x_0-\delta,x_0+\delta)-\{x_0\}$使得$|D(\xi)-0|>\dfrac{1}{2}$; 存在无理数$\eta\in(x_0-\delta,x_0+\delta)-\{x_0\}$使得$|D(\eta)-1|>\dfrac{1}{2}$. 故$D(x)$在$\forall x_0\in(-\infty,+\infty)$都没有极限. \qedhere
	      \end{proof}

	      \begin{proof}[方法二]
		      $\forall x_0\in(-\infty,+\infty)$, 给定$\varepsilon<1$, 对于任意$\delta>0$, 存在有理数$\xi$和无理数$\eta\in(x_0-\delta,x_0+\delta)-\{x_0\}$使得$|D(\xi)-D(\eta)|=1>\varepsilon$. 由Cauchy收敛准则知$D(x)$在$\forall x_0\in(-\infty,+\infty)$都没有极限. \qedhere\qedhere
	      \end{proof}

	      \begin{proof}[方法三]
		      $\forall x_0\in(-\infty,+\infty)$, 由于有理数集$\mathbf{Q}$和无理数集$\mathbf{Q}^{C}$均在$\mathbf{R}$中稠密, 取有理数列$\{\xi_n\}$和无理数列$\{\eta\}$使得$\lim\limits_{n\to\infty} \xi_n=\lim\limits_{n\to\infty} \eta_n=x_0$. 但$\lim\limits_{n\to\infty} D(\xi_n)=1\neq 0=\lim\limits_{n\to\infty} \eta_n$. 由Heine定理知$D(x)$在$\forall x_0\in(-\infty,+\infty)$都没有极限. \qedhere
	      \end{proof}

	\item 试举出一个在区间$(-\infty,+\infty)$上定义的函数, 使得它在点$x=1$处有极限, 但在区间的其他点都没有极限. \footnote{同理, 若要求举例仅在$x=a$处可微的函数, 就是$(x-a)^2D(x)$.}
	      \begin{proof}[解答]
		      取$f(x)=(x-1)D(x)$. 对于$\forall x_0\neq 1$, 易知$f(x)$在$x_0$处没有极限; 对于$x_0=1$, $\forall\varepsilon>0$, 取$\delta=\varepsilon$, 则当$0<|x-1|<\delta$时,
		      \[
			      |f(x)|=|x-1||D(x)|\leqslant |x-1|<\delta=\varepsilon.
		      \]
		      即$\lim\limits_{x\to 1} f(x)=0$. \qedhere
	      \end{proof}

	\item 证明: 若$f$为周期函数, 且$\lim\limits_{x\to+\infty} f(x)=0$, 则$f(x)\equiv 0$.
	      \begin{proof}
		      反证法. 若$f(x)\not\equiv 0$, 即存在$x_0$使得$f(x_0)\neq 0$, 不妨设$f(x_0)>0$. 假设$T$是$f(x)$的周期, 即$f(x+T)\equiv f(x)$. 对于$\varepsilon=\dfrac{f(x_0)}{2}>0$, $\forall M>0$, 由Archimedes公理, 存在$n\in\mathbf{N}_{+}$使得$x_0+nT>M$, 故
		      \[
			      |f(x_0+nT)|=f(x_0)>\dfrac{f(x_0)}{2}=\varepsilon.
		      \]
		      即$\lim\limits_{x\to+\infty} f(x)\neq 0$. \qedhere
	      \end{proof}

	\item 证明: 任何非常值的周期函数不可能是有理分式函数.
	      \begin{proof}
		      反证法. 对于周期为$T$的有理分式函数$\dfrac{P(x)}{Q(x)}$的次数分类讨论:
		      \begin{enumerate}[(1)]
			      \item 若$\partial P(x)<\partial Q(x)$: $\lim\limits_{x\to+\infty} \dfrac{P(x)}{Q(x)}=0$, 由13.知$\dfrac{P(x)}{Q(x)}\equiv 0$;
			      \item 若$\partial P(x)=\partial Q(x)$: $\lim\limits_{x\to+\infty} \dfrac{P(x)}{Q(x)}=C$为一非零常数, 由13.知周期函数$\dfrac{P(x)}{Q(x)}-C\equiv 0$, 即$\dfrac{P(x)}{Q(x)}\equiv C$;
			      \item 若$\partial P(x)>\partial Q(x)$: $\lim\limits_{x\to+\infty} \dfrac{P(x)}{Q(x)}=\pm\infty$为一确定符号的无穷大量, 不妨设为正无穷大量. 对于$\forall x_0$, 取单调增加趋于$+\infty$的数列$\{x_0+nT\}$, 显然$\lim\limits_{n\to\infty} \dfrac{P(x_0+nT)}{Q(x_0+nT)}=\dfrac{P(x_0)}{Q(x_0)}<+\infty$. \qedhere
		      \end{enumerate}
	      \end{proof}
\end{enumerate}

\section{函数极限的基本性质}
\subsection{思考题 pp. 107.}
\begin{enumerate}
	\item 试就$\lim\limits_{x\to+\infty} f(x)=A$和$\lim\limits_{x\to a^{+}} f(x)=A$两类极限叙述极限的唯一性定理、局部有界性定理、局部保号性定理、比较定理、夹逼定理、Heine归结原理和Cauchy收敛准则.

	\item 回答下述有关四则极限运算法则方面的问题:
	      \begin{enumerate}[(1)]
		      \item 若$\lim\limits_{x\to a} [f(x)+g(x)]$存在, 则当$x$趋于$a$时在$f(x)$和$g(x)$的敛散性之间有何联系?
		      \item 若$\lim\limits_{x\to a} f(x)$存在, $\lim\limits_{x\to a} g(x)$不存在, 则$\lim\limits_{x\to a} f(x)g(x)$是否存在?
	      \end{enumerate}
	      \begin{proof}[解答]
		      \begin{enumerate}[(1)]
			      \item $f(x)$, $g(x)$同敛散. 利用四则运算可以证明, 若$f(x),g(x)$之一收敛, 则另一也必然收敛.
			      \item 不一定. 容易举出不存在例, 存在例可取$f(x)=x-a, g(x)=\dfrac{1}{x-a}$, 则$f(x)g(x)\equiv 1$. \qedhere
		      \end{enumerate}
	      \end{proof}

	\item 找出以下运算中的错误:
	      \begin{enumerate}[(1)]
		      \item $\lim\limits_{x\to 2} \dfrac{x-2}{\sin{\dfrac{1}{x-2}}}=\dfrac{\lim\limits_{x\to 2} (x-2)}{\lim\limits_{x\to 2} \sin{\dfrac{1}{x-2}}}=\dfrac{0}{\lim\limits_{x\to 2}\sin{\dfrac{1}{x-2}}}=0$.
		      \item $\lim\limits_{x\to\infty}\dfrac{\sin{x}}{x}=\lim\limits_{x\to\infty} \dfrac{1}{x}\cdot \lim\limits_{x\to\infty} \sin{x} =0\cdot\lim\limits_{x\to\infty} \sin{x}=0$.
	      \end{enumerate}
	      \begin{proof}
		      均在未验证极限存在就直接使用极限的四则运算: $\lim\limits_{x\to 2}\sin{\dfrac{1}{x-2}}$和$\lim\limits_{x\to\infty} \sin{x}$均不存在. \qedhere
	      \end{proof}

	\item 对于极限的加法运算法则做出两个证明: (1) 用函数极限定义; (2) 用Heine归结原理.
	      \begin{proof}[证明一]
		      由$\lim\limits_{x\to x_0} f(x)=A, \lim\limits_{x\to x_0} g(x)=B$知对于任意给定的$\varepsilon>0$, 存在$\delta_1>0$, 当$0<|x-x_0|<\delta_1$时有$|f(x)-A|<\dfrac{\varepsilon}{2}$; 存在$\delta_1>0$, 当$0<|x-x_0|<\delta_2$时有$|g(x)-B|<\dfrac{\varepsilon}{2}$. 故当$0<|x-x_0|<\min\{\delta_1,\delta_2\}$时,
		      \[
			      |(f(x)+g(x))-(A+B)|\leqslant |f(x)-A|+|g(x)-B|<\dfrac{\varepsilon}{2}+\dfrac{\varepsilon}{2}=\varepsilon.
		      \]
		      即$\lim\limits_{x\to x_0} [f(x)+g(x)]=A+B$. \qedhere
	      \end{proof}
	      \begin{proof}[证明二]
		      由$\lim\limits_{x\to x_0} f(x)=A, \lim\limits_{x\to x_0} g(x)=B$知对于任意$\{x_n\}$, 若$\lim\limits_{n\to\infty} x_n=x_0$, 则$\lim\limits_{n\to\infty} f(x_n)=A, \lim\limits_{n\to\infty} g(x_n)=B$. 由数列极限的加法运算法则, $\lim\limits_{n\to\infty} f(x_n)+g(x_n)=A+B$, 由Heine归结原理知$\lim\limits_{x\to x_0} f(x)+g(x)=A+B$. \qedhere
	      \end{proof}
\end{enumerate}

\subsection{练习题 pp. 109.}
\begin{enumerate}
	\item 证明:
	      \begin{tabenum}[(1)]
		      \tabenumitem $\lim\limits_{x\to+\infty} \dfrac{x^k}{a^x}=0 \ (a>1,k>0)$;
		      \tabenumitem $\lim\limits_{x\to+\infty} \dfrac{\ln{x}}{x^k}=0 \ (k>0)$;\\
		      \tabenumitem $\lim\limits_{x\to\infty} \sqrt[x]{a}=1 \ (a>0)$;
		      \tabenumitem $\lim\limits_{x\to+\infty} \sqrt[x]{x}=1$.
	      \end{tabenum}
	      \begin{proof}
		      \begin{enumerate}[(1)]
			      \item 注意$[x]\leqslant x<[x]+1$, 因此
			            \[
				            \dfrac{1}{a}\dfrac{[x]^k}{a^[x]}\leqslant\dfrac{x^k}{a^x}\leqslant a\dfrac{([x]+1)^k}{a^{[x]+1}}.
			            \]
			            注意$\lim\limits_{n\to\infty} \dfrac{n^k}{a^x}=0$, 即对于任意给定的$\varepsilon>0$, 存在$N\in\mathbf{N}_{+}$, 当$n>N$时有$\dfrac{n^k}{a^n}<\varepsilon$. 当$x>N+1$时, $[x]\geqslant N+1>N$. 故
			            \[
				            \dfrac{([x]+1)^k}{a^{[x]+1}}<\varepsilon, \dfrac{[x]^k}{a^{[x]}}<\varepsilon.
			            \]
			            从而$\lim\limits_{x\to+\infty} \dfrac{1}{a}\dfrac{[x]^k}{a^[x]}=\lim\limits_{x\to+\infty} a\dfrac{([x]+1)^k}{a^{[x]+1}}=0$, 故由夹逼准则知$\lim\limits_{x\to+\infty} \dfrac{x^k}{a^x}=0$.

			      \item 做变量替换$y=\ln{x}$,
			            \[
				            \lim\limits_{x\to+\infty} \dfrac{\ln{x}}{x^k}=\lim\limits_{y\to+\infty} \dfrac{y}{\e^{ky}}=\lim\limits_{y\to+\infty} \dfrac{1}{k}\dfrac{y}{\e^y}=0.
			            \]

			      \item $\lim\limits_{x\to\infty} \sqrt[x]{a}=\lim\limits_{x\to\infty} \e^{\dfrac{\ln{a}}{x}} =\e^0=1$.

			      \item $\lim\limits_{x\to+\infty} \sqrt[x]{x}=\lim\limits_{x\to+\infty} \e^{\dfrac{\ln{x}}{x}}=\e^0=1$. \qedhere
		      \end{enumerate}
	      \end{proof}

	\item 求$\lim\limits_{y\to+\infty} \dfrac{\sqrt{1+y^3}}{\sqrt{y^2+y^3}+y}$.
	      \begin{proof}
		      分子分母同时除以$y^{\frac{3}{2}}$,
		      \[
			      \lim\limits_{y\to+\infty} \dfrac{\sqrt{1+y^3}}{\sqrt{y^2+y^3}+y}=\lim\limits_{y\to+\infty} \dfrac{\sqrt{1+\dfrac{1}{y^3}}}{\sqrt{1+\dfrac{1}{y}}+\dfrac{1}{\sqrt{y}}}=1. \qedhere
		      \]
	      \end{proof}

	\item 求$\lim\limits_{x\to+\infty} \left(\dfrac{x^2-1}{x^2+1}\right)^{\dfrac{x-1}{x+2}}$.
	      \begin{proof}
		      \[
			      \lim\limits_{x\to+\infty} \left(\dfrac{x^2-1}{x^2+1}\right)^{\dfrac{x-1}{x+2}}=\left(\lim\limits_{x\to+\infty} \dfrac{x^2-1}{x^2+1}\right)^{\lim\limits_{x\to+\infty} \dfrac{x-1}{x+2}}=1. \qedhere
		      \]
	      \end{proof}

	\item 求$\lim\limits_{x\to 0} \dfrac{\sqrt[n]{1+x}-1}{x}$, 其中$n$为正整数.
	      \begin{proof}
		      \[
			      \lim\limits_{x\to 0} \dfrac{\sqrt[n]{1+x}-1}{x}=\lim\limits_{x\to 0}\dfrac{1}{(\sqrt[n]{1+x})^{n-1}+(\sqrt[n]{1+x})^{n-2}+\cdots+\sqrt[n]{1+x}+1}=\dfrac{1}{n}. \qedhere
		      \]
	      \end{proof}

	\item 求$\lim\limits_{x\to 0} \dfrac{f(x)}{x}=l, b\neq 0$, 求$\lim\limits_{x\to 0} \dfrac{f(bx)}{x}$.
	      \begin{proof}
		      \[
			      \lim\limits_{x\to 0} \dfrac{f(bx)}{x}=\lim\limits_{x\to 0} b\cdot\dfrac{f(bx)}{bx}\xlongequal{y=bx}\lim\limits_{y\to 0} b\cdot\dfrac{f(y)}{y}=bl. \qedhere
		      \]
	      \end{proof}

	\item 证明: $\lim\limits_{x\to 0} \dfrac{\sqrt{1+\sin{x}}-\sqrt{1-\sin{x}}}{\sin{x}}=1$.
	      \begin{proof}
		      \[
			      \lim\limits_{x\to 0} \dfrac{\sqrt{1+\sin{x}}-\sqrt{1-\sin{x}}}{\sin{x}}=\lim\limits_{x\to 0} \dfrac{2\sin{x}}{\sin{x}(\sqrt{1+\sin{x}}+\sqrt{1-\sin{x}})}=1. \qedhere
		      \]
	      \end{proof}

	\item 证明: 在区间$(a,+\infty)$上单调有界函数$f$一定存在极限$\lim\limits_{x\to+\infty} f(x)$.
	      \begin{proof}
		      对于任意单调增加的正无穷大数列$\{x_n\}\subset(a,+\infty)$, 由$f$在$(a,+\infty)$上单调有界, 知$\{f(x_n)\}$单调有界从而收敛. 由$\{x_n\}$的任意性, 从Heine归结原理可知极限$\lim\limits_{x\to+\infty} f(x)$存在. \qedhere
	      \end{proof}

	\item 设$f(x)$在区间$(a,b)$上为单调增加函数, 且存在一个数列$\{x_n\}\subset(a,b)$, 使得$\lim\limits_{n\to\infty} x_n=b, \lim\limits_{n\to\infty} f(x_n)=A$. 证明: (1) $f$在区间$(a,b)$上以$A$为上界; (2) $\lim\limits_{x\to b}=A$.
	      \begin{proof}
		      断言对于$\forall n\in\mathbf{N}_{+}, f(x_n)\leqslant A$. 否则存在$n_0\in\mathbf{N}_{+}$使得$f(x_{n_0})>A$, 对于$\forall n>n_0$, $f$在$(a,b)$上单调增加, 故$f(x_n)-A\geqslant f(x_{n_0})-A$, 与$\lim\limits_{n\to\infty} f(x_n)=A$矛盾. 对于$\forall x\in(a,b)$, $\varepsilon'=b-x>0$, 存在$N'\in\mathbf{N}_{+}$使得当$n>N'$时有$|x_n-b|<b-x$, 即$x_n>x$. 于是$f(x)\leqslant f(x_n)\leqslant A$, 即$f$在区间$(a,b)$上以$A$为上界.

		      对于任意给定的$\varepsilon>0$, 由于$\lim\limits_{n\to\infty} f(x_n)=A$, 存在$N\in\mathbf{N}_{+}$使得$|f(x_n)-A|<\varepsilon$, 即当$n>N$时$f(x_n)>A-\varepsilon$. 固定$n>N$, 令$\delta=\dfrac{b-x_n}{2}$, 则当$b-\delta<x<b$时, $x>x_n$, 由于$f$在$(a,b)$上单调增加以$A$为上界, 有
		      \[
			      A-\varepsilon<f(x_n)\leqslant f(x)<A<A+\varepsilon.
		      \]
		      即$\lim\limits_{x\to b}=A$. \qedhere
	      \end{proof}

	\item 设$\lim\limits_{x\to+\infty}f(x)=A>0$. 证明: 对每个$c\in(0,A)$, 存在$M>0$, 当$x>M$时, 成立$f(x)>c$. \\
	      (这是对于极限类型为$\lim\limits_{x\to+\infty} f(x)$的保号性定理.)
	      \begin{proof}
		      令$\varepsilon=A-c>0$, 由$\lim\limits_{x\to+\infty} f(x)=A$, 存在$M>0$使得当$x>M$时有$|f(x)-A|<\varepsilon=A-c$, 即当$x>M$时$f(x)>A+(c-A)=c$. \qedhere
	      \end{proof}

	\item 设$f(a^{-})<f(a^{+})$. 证明: 存在$\delta>0$, 当$x\in(a-\delta,a)$和$y\in(a,a+\delta)$时, 成立$f(x)<f(y)$.
	      \begin{proof}
		      对于$\varepsilon=\dfrac{f(a^{+})-f(a^{-})}{2}>0$, 存在$\delta_1>0$, 当$a-\delta_1<x<a$时, $|f(x)-f(a^{-})|<\varepsilon$, 即$f(x)<f(a^{-})+\varepsilon$; 存在$\delta_2>0$, 当$a<y<a+\delta_2$时, $|f(y)-f(a^{+})|<\varepsilon$, 即$f(y)>f(a^{+})-\varepsilon$. 令$\delta=\min\{\delta_1,\delta_2\}$则当$a-\delta<x<a<y<a+\delta$时, 有
		      \[
			      f(x)<f(a^{-})+\varepsilon=\dfrac{f(a^{-})+f(a^{+})}{2}=f(a^{+})-\varepsilon<f(y). \qedhere
		      \]
	      \end{proof}

	\item 试用Heine归结原理证明单调函数的单侧极限存在定理. \\
	      (这里先要将Heine归结原理 (命题4.2.3) 推广到单侧极限. 注意这时在条件中的数列可限于单侧数列.)
	      \begin{proof}
		      首先给出单侧极限时的Heine归结原理.
	      \end{proof}
\end{enumerate}

\section{两个重要极限}
\subsection{练习题 pp. 114.}
\begin{enumerate}
	\item 计算以下极限:
	      \begin{tabenum}[(1)]
		      \tabenumitem $\lim\limits_{x\to+\infty} \left(\dfrac{2}{\pi}\arctan{x}\right)^x$;
		      \tabenumitem $\lim\limits_{x\to\frac{\pi}{2}^{-}} (\sin{x})^{\tan{x}}$;\\
		      \tabenumitem $\lim\limits_{x\to\infty} \left(\dfrac{x^2-1}{x^2+1}\right)^{x^2}$;
		      \tabenumitem $\lim\limits_{x\to\frac{\pi}{2}^{-}} (\cos{x})^{\frac{\pi}{2}-x}$;\\
		      \tabenumitem $\lim\limits_{x\to 0} \dfrac{\sin{2x}-2\sin{x}}{x^3}$;
		      \tabenumitem $\lim\limits_{x\to 1} (1-x)\tan\left(\dfrac{\pi}{2}x\right)$.
	      \end{tabenum}

	      \begin{proof}[解答]
		      \begin{enumerate}[(1)]
			      \item $\begin{aligned}[t]
					            \lim\limits_{x\to+\infty} \left(\dfrac{2}{\pi}\arctan{x}\right)^x & =  \lim\limits_{x\to+\infty} \left[1+\left(\dfrac{2}{\pi}\arctan{x}-1\right)\right]^{\dfrac{x\cdot\left(\dfrac{2}{\pi}\arctan{x}-1\right)}{\dfrac{2}{\pi}\arctan{x}-1}} \\
					                                                                              & =                                                                   {\rm exp}\left[\lim\limits_{x\to+\infty} x\left(\dfrac{2}{\pi}\arctan{x}-1\right)\right]           \\
					                                                                              & =                                                                   \e^{-2/\pi}.
				            \end{aligned}$

			      \item $\begin{aligned}[t]
					            \lim\limits_{x\to\frac{\pi}{2}^{-}} (\sin{x})^{\tan{x}} & =\lim\limits_{x\to\frac{\pi}{2}^{-}} [1+(\sin{x}-1)]^{\dfrac{\tan{x}(\sin{x}-1)}{\sin{x}-1}} \\
					                                                                    & = {\rm exp}[\lim\limits_{x\to\frac{\pi}{2}^{-}} \tan{x}(\sin{x}-1)]                          \\
					                                                                    & = \e^0=1.
				            \end{aligned}$

			      \item $\begin{aligned}[t]
					            \lim\limits_{x\to\infty} \left(\dfrac{x^2-1}{x^2+1}\right)^{x^2} & = \lim\limits_{x\to\infty} \left(1-\dfrac{2}{x^2+1}\right)^{-\dfrac{x^2+1}{2}\dfrac{-2x^2}{x^2+1}} \\
					                                                                             & ={\rm exp}\left[\lim\limits_{x\to\infty} -\dfrac{2x^2}{x^2+1}\right]                               \\
					                                                                             & =\e^{-2}.
				            \end{aligned}$

			      \item $\begin{aligned}[t]
					            \lim\limits_{x\to\frac{\pi}{2}^{-}} (\cos{x})^{\frac{\pi}{2}-x} & = \lim\limits_{x\to\frac{\pi}{2}^{-}} [1+(\cos{x}-1)]^{\dfrac{(\cos{x}-1)(\frac{\pi}{2}-x)}{\cos{x}-1}} \\
					                                                                            & = {\rm exp}[(\cos{x}-1)(\frac{\pi}{2}-x)]                                                               \\
					                                                                            & = \e^0=1.
				            \end{aligned}$

			      \item $\lim\limits_{x\to 0} \dfrac{\sin{2x}-2\sin{x}}{x^3} = \lim\limits_{x\to 0}\dfrac{2\sin{x}}{x}\cdot\dfrac{\cos{x}-1}{x^2}=2\cdot -\dfrac{1}{2}=-1$.

			      \item $\begin{aligned}[t]
					            \lim\limits_{x\to 1} (1-x)\tan\left(\dfrac{\pi}{2}x\right) & \xlongequal{y=x-1} \lim\limits_{y\to 0} -y\tan\left(\dfrac{\pi}{2}(y+1)\right)             \\
					                                                                       & = \lim\limits_{y\to 0} y\cot\left(\dfrac{\pi}{2}x\right)                                   \\
					                                                                       & \xlongequal{z=\frac{\pi}{2}y} \lim\limits_{z\to 0} \dfrac{2}{\pi}\dfrac{z}{\sin{z}}\cos{z} \\
					                                                                       & =\dfrac{\pi}{2}\cdot 1\cdot 1= \dfrac{\pi}{2}.
				            \end{aligned}$
		      \end{enumerate}
	      \end{proof}

	\item 注意以下两个``不等式''并求出正确值:
	      \begin{tabenum}[(1)]
		      \tabenumitem $\lim\limits_{x\to+\infty} \dfrac{\sin{x}}{x}\neq 1$;
		      \tabenumitem $\lim\limits_{x\to+\infty} (1+x)^{\frac{1}{x}}\neq \e$.
	      \end{tabenum}
	      \begin{proof}[解答]
		      \begin{enumerate}[(1)]
			      \item $\lim\limits_{x\to+\infty} \dfrac{\sin{x}}{x}=0$;
			      \item $\lim\limits_{x\to+\infty} (1+x)^{\frac{1}{x}}=\lim\limits_{x\to+\infty} (1+x)^{\frac{1}{x+1}\cdot\frac{x+1}{x}}=1^1=1$. \qedhere
		      \end{enumerate}
	      \end{proof}

	\item 设$a>0, b>0$, 求极限$\lim\limits_{n\to\infty} \left(\dfrac{\sqrt[n]{a}+\sqrt[n]{b}}{2}\right)^n$. \\
	      (本题是数列极限问题, 但现在可以用函数极限知识来解决.)
	      \begin{proof}
			转化为函数极限$\lim\limits_{x\to 0} \left(\dfrac{a^x+b^x}{2}\right)^{\frac{1}{x}}$.
			\[
				\begin{split}
					\lim\limits_{x\to 0} \left(\dfrac{a^x+b^x}{2}\right)^{\frac{1}{x}}&=\lim\limits_{x\to 0} \left(1+\dfrac{a^x+b^x-2}{2}\right)^{\frac{2}{a^x+b^x-2}\cdot\frac{a^x+b^x-2}{2x}}\\
					&={\rm exp}\left(\lim\limits_{x\to 0}\dfrac{a^x+b^x-2}{2x}\right)\\
					&={\rm exp}\left(\dfrac{\ln{a}+\ln{b}}{2}\right)\\
					&=\sqrt{ab}. \qedhere
				\end{split}
			\]
	      \end{proof}
	\item 设$a_1, a_2, \cdots, a_n$为正数, $n\geqslant 2$, $f(x)=\left[\dfrac{a_1^x+a_2^x+\cdots+a_n^x}{n}\right]^{\frac{1}{x}}$, 求$\lim\limits_{x\to 0} f(x)$.
		\begin{proof}
			\[
				\begin{split}
					& \lim\limits_{x\to 0} \left[\dfrac{a_1^x+a_2^x+\cdots+a_n^x}{n}\right]^{\frac{1}{x}}\\
					=& \lim\limits_{x\to 0} \left(1+\dfrac{a_1^x+a_2^x+\cdots+a_n^x-n}{n}\right)^{\frac{n}{a_1^x+a_2^x+\cdots+a_n^x-n}\cdot\frac{a_1^x+a_2^x+\cdots+a_n^x-n}{2x}}\\
					=& {\rm exp}\left(\lim\limits_{x\to 0} \dfrac{a_1^x+a_2^x+\cdots+a_n^x-n}{2x}\right)\\
					=& {\rm exp}\left(\dfrac{\ln{a_1}+\ln{a_2}+\cdots+\ln{a_n}}{2}\right)\\
					=& \sqrt{a_1a_2\cdots a_n}. \qedhere
				\end{split}
			\]
		\end{proof}
		
	\item 计算极限$\lim\limits_{n\to\infty} \displaystyle\prod\limits_{k=1}^n \cos{\dfrac{x}{2^k}}$, 并证明Vi\`ete公式
	      \[
		      \dfrac{\pi}{2}=\dfrac{1}{ \sqrt{\dfrac{1}{2}} \cdot \sqrt{\dfrac{1}{2}+\dfrac{1}{2}\sqrt{\dfrac{1}{2}}} \cdot \sqrt{\dfrac{1}{2}+\dfrac{1}{2}\sqrt{\dfrac{1}{2}+\dfrac{1}{2}\sqrt{\dfrac{1}{2}}}}\cdots}.
	      \]
		(这是数学家Vi\`ete在1593年发表的. 它是数学史上第一次用无穷乘积来表示一个数, 同时也是对于圆周率$\pi$的认识上的重大突破.)
		\begin{proof}
			\[
				\begin{split}
				\prod\limits_{k=1}^n \cos{\dfrac{x}{2^k}}=
				\end{split}
			\]
		\end{proof}
\end{enumerate}

\ifx\all\undefined
\end{document}
\fi